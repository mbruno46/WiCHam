\section{The Effective Hamiltonian for $B\to X_{\lowercase{s}}\gamma$} 
         \label{sec:Heff:BXsgamma}
The effective hamiltonian for $B\to X_s\gamma$ at scales $\mu=O(\mb)$
is given by
\begin{equation}
\Heff(b\to s\gamma) = - \frac{G_F}{\sqrt{2}} V_{ts}^* V^{}_{tb}
\left[ \sum_{i=1}^6 C_i(\mu) Q_i(\mu) + C_{7\gamma}(\mu) Q_{7\gamma}(\mu)
+C_{8G}(\mu) Q_{8G}(\mu) \right]
\label{eq:HeffBXsgamma}
\end{equation}
where in view of $|V_{us}^*V_{ub}^{} / V_{ts}^* V_{tb}^{}| < 0.02$
we have neglected the term proportional to $V_{us}^*V_{ub}^{}$.

\subsection{Operators}
         \label{sec:Heff:BXsgamma:ops}
The complete list of operators is given as follows
\begin{eqnarray}
Q_1    & = & (\bar{s}_{i}  c_{j})_{V-A}
           (\bar{c}_{j}  b_{i})_{V-A}        \nn \\
Q_2    & = & (\bar{s} c)_{V-A}  (\bar{c} b)_{V-A}      \nn \\
Q_3    & = & (\bar{s} b)_{V-A}\sum_q(\bar{q}q)_{V-A}   \nn \\
Q_4    & = & (\bar{s}_{i}  b_{j})_{V-A} \sum_q (\bar{q}_{j}
          q_{i})_{V-A}     \nn \\
Q_5    & = & (\bar{s} b)_{V-A}\sum_q(\bar{q}q)_{V+A}
             \label{eq:BXsgamma:ops} \\
Q_6    & = & (\bar{s}_{i}  b_{j})_{V-A}
   \sum_q  (\bar{q}_{j}  q_{i})_{V+A}        \nn \\
Q_{7\gamma}    & = & \frac{e}{8\pi^2} \mb \bar{s}_i \sigma^{\mu\nu}
          (1+\gamma_5) b_i F_{\mu\nu}             \nn \\
Q_{8G}    & = & \frac{g}{8\pi^2} \mb \bar{s}_i \sigma^{\mu\nu}
   (1+\gamma_5)T^a_{ij} b_j G^a_{\mu\nu}  \nn
\end{eqnarray}
The current-current operators $Q_{1,2}$ and the QCD penguin operators
$Q_3,\ldots,Q_6$ have been already contained in the usual $\Delta B=1$
hamiltonian presented in section \ref{sec:HeffdF1:66:dB1}. The new
operators $Q_{7\gamma}$ and $Q_{8G}$ specific for $b\to s\gamma$ and
$b\to s g$ transitions carry the name of magnetic penguin operators.
They originate from the mass-insertion on the external b-quark line in
the QCD and QED penguin diagrams of fig.\ \ref{fig:oporig}\,(d),
respectively. In view of $\ms \ll \mb$ we do not include the corresponding
contributions from mass-insertions on the external s-quark line.

\subsection{Wilson Coefficients}
         \label{sec:Heff:BXsgamma:wc}
A very peculiar feature of the renormalization group analysis of the
set of operators in (\ref{eq:BXsgamma:ops}) is that the mixing under
(infinite) renormalization between the set $Q_1,\ldots,Q_6$ and the
operators $Q_{7\gamma},Q_{8G}$ vanishes at the one-loop level.
Consequently in order to calculate the coefficients $C_{7\gamma}(\mu)$
and $C_{8G}(\mu)$ in the leading logarithmic approximation, two-loop
calculations of ${\cal{O}}(e g^2_s)$ and ${\cal{O}}(g^3_s)$ are
necessary. The corresponding NLO analysis requires the evaluation of
the mixing in question at the three-loop level. In view of this new
feature it is useful to include additional couplings in the definition
of $Q_{7\gamma}$ and $Q_{8G}$ as done in (\ref{eq:BXsgamma:ops}).  In
this manner the entries in the anomalous dimension matrix representing
the mixing between $Q_1,\ldots,Q_6$ and $Q_{7\gamma},Q_{8G}$ at the
two-loop level are $O(g_s^2)$ and enter the anomalous dimension matrix
$\gamma^{(0)}_s$. Correspondingly the three-loop mixing between these
two sets of operators enters the matrix $\gamma^{(1)}_s$.

The mixing under renormalization in the sector $Q_{7\gamma},Q_{8G}$
proceeds in the usual manner and the corresponding entries in
$\gamma^{(0)}_s $ and $\gamma^{(1)}_s$ result from one-loop and 
two-loop calculations, respectively.

At present, the coefficients $C_{7\gamma}$ and $C_{8G}$ are only known
in the leading logarithmic approximation. Consequently we are in the
position to give here only their values in this approximation.
The work on NLO corrections to $C_{7\gamma}$ and $C_{8G}$ is in
progress and we will summarize below what is already known about
these corrections.

The peculiar feature of this decay mentioned above caused that the
first fully correct calculation of the leading  anomalous dimension
matrix has been obtained only in 1993 \cite{CFMRS:93}, \cite{CFRS:94}.
It is instructive to clarify this right at the beginning. We follow
here \cite{BMMP:94}.

The point is that the mixing between the sets $Q_1,\ldots,Q_6$ and
$Q_{7\gamma},Q_{8G}$ in $\gamma^{(0)}_s$ resulting from two-loop
diagrams is generally regularization scheme dependent. This is
certainly disturbing because the matrix $\gamma^{(0)}_s$, being the
first term in the expansion for $\gamma_s$, is usually scheme
independent.  There is a simple way to circumvent this difficulty
\cite{BMMP:94}.

As noticed in \cite{CFMRS:93}, \cite{CFRS:94} the regularization scheme
dependence of $\gamma^{(0)}_s$ in the case of $b\to s\gamma$ and
$b\to s g$ is signaled in the one-loop matrix elements of $Q_1,\ldots,Q_6$
for on-shell photons or gluons.  They vanish in any 4-dimensional
regularization scheme and in the HV scheme but some of them are
non-zero in the NDR scheme.  One has
\begin{equation}
\langle Q_i \rangle_{\rm one-loop}^\gamma =
y_i \, \langle Q_{7\gamma} \rangle_{\rm tree},
\qquad i=1,\ldots,6
\label{eq:defy}
\end{equation}
and
\begin{equation}
\langle Q_i\rangle_{\rm one-loop}^G =
z_i \, \langle Q_{8G} \rangle_{\rm tree},
\qquad i=1,\ldots,6.
\end{equation}

In the HV scheme all the $y_i$'s and $z_i$'s vanish, while in the NDR
scheme $\vec{y} = (0,0,0,0,-\frac{1}{3},-1)$ and $\vec{z} =
(0,0,0,0,1,0)$.  This regularization scheme dependence is canceled by a
corresponding regularization scheme dependence in $\gamma_s^{(0)}$
as first demonstrated in \cite{CFMRS:93}, \cite{CFRS:94}. It should be
stressed that the numbers $y_i$ and $z_i$ come from divergent, i.e.
purely short-distance parts of the one-loop integrals. So no reference
to the spectator-model or to any other model for the matrix elements is
necessary here.

In view of all this  it is convenient in the leading order to introduce
the so-called ``effective coefficients'' \cite{BMMP:94} for the
operators $Q_{7\gamma}$ and $Q_{8G}$ which are regularization scheme
independent. They are given as follows:
\begin{equation} \label{eq:defc7eff}
C^{(0)eff}_{7\gamma}(\mu) =
C^{(0)}_{7\gamma}(\mu) + \sum_{i=1}^6 y_i C^{(0)}_i(\mu).
\end{equation}
and 
\begin{equation}
C^{(0)eff}_{8G}(\mu) = C^{(0)}_{8G}(\mu) + \sum_{i=1}^6 z_i C^{(0)}_i(\mu)
\end{equation}
One can then introduce a scheme-independent vector
\begin{equation} 
\vec{C}^{(0)eff}(\mu) = \left( C^{(0)}_1(\mu),\ldots, C^{(0)}_6(\mu), 
C^{(0)eff}_{7\gamma}(\mu),C^{(0)eff}_{8G}(\mu) \right) \, .
\end{equation}
From the RGE for $\vec{C}^{(0)}(\mu)$ it is straightforward
to derive the RGE for $\vec{C}^{(0)eff}(\mu)$. It has the form
\begin{equation} \label{RGEeff}
\mu \frac{d}{d \mu} C^{(0)eff}_i(\mu) = 
\frac{\as}{4\pi} \gamma^{(0)eff}_{ji} C^{(0)eff}_j(\mu)
\end{equation}
where
\begin{equation} \label{def.geff}
\gamma^{(0)eff}_{ji} = \left\{ \begin{array}{ccl}
\gamma^{(0)}_{j7} +
\sum_{k=1}^6 y_k\gamma^{(0)}_{jk} -y_j\gamma^{(0)}_{77} -z_j\gamma^{(0)}_{87}
&\quad& $i=7$,\ $j=1,\ldots,6$ \\
\gamma^{(0)}_{j8} +
\sum_{k=1}^6 z_k\gamma^{(0)}_{jk} -z_j\gamma^{(0)}_{88}
&\quad& $i=8$,\ $j=1,\ldots,6$ \\
\gamma^{(0)}_{ji} &\quad& \mbox{otherwise.}
\end{array}
\right.
\end{equation}
The matrix $\gamma^{(0)eff}$ is a scheme-independent quantity.
It equals the matrix which one would directly obtain from two-loop
diagrams in the HV scheme.  In order to simplify the notation we will
omit the label ``eff'' in the expressions for the elements of this
effective one loop anomalous dimension matrix given below and keep it
only in the Wilson coefficients of the operators $Q_{7\gamma}$ and
$Q_{8G}$.

This discussion clarifies why it took so long to find the correct
leading anomalous dimension matrix for the $\bsg$ decay
\cite{CFMRS:93}, \cite{CFRS:94}. All previous calculations \cite{Grin},
\cite{cella:90a}, \cite{misiak:93}, \cite{Yao1}, \cite{Yao2} of the
leading order QCD corrections to $\bsg$ used the NDR scheme setting
unfortunately $z_i$ and $y_i$ to zero, or not including all operators
or making other mistakes.
The discrepancy between the calculation of \cite{grigjanis:88} (DRED
scheme) and \cite{Grin} (NDR scheme) has been clarified by
\cite{misiak:94b}.

\subsection{Renormalization Group Evolution and Anomalous Dimension Matrices}
         \label{sec:Heff:BXsgamma:RGE}
The coefficients $C_i(\mu)$ in (\ref{eq:HeffBXsgamma}) can be calculated
by using
\begin{equation}
\vec C(\mu)= U_5(\mu,\mw)\vec C(\mw)
\end{equation}
Here $ U_5(\mu,\mw)$ is the $8\times 8$ evolution matrix which is
given in general terms in \eqn{u0jj} with $\gamma$ being this
time an $8\times 8$ anomalous dimension matrix. In the leading order
$U_5(\mu,\mw)$ is to be replaced by $U_5^{(0)}(\mu,\mw)$ and
the initial conditions by $\vec C^{(0)}(\mw)$ given by \cite{Grin}
\begin{equation}\label{c2}
C^{(0)}_2(\mw) = 1                               
\end{equation}
\begin{equation}\label{c7}
C^{(0)}_{7\gamma} (\mw) = \frac{3 x_t^3-2 x_t^2}{4(x_t-1)^4}\ln x_t + 
   \frac{-8 x_t^3 - 5 x_t^2 + 7 x_t}{24(x_t-1)^3}
   \equiv -\frac{1}{2} D'_0(x_t)
\end{equation}
\begin{equation}\label{c8}
C^{(0)}_{8G}(\mw) = \frac{-3 x_t^2}{4(x_t-1)^4}\ln x_t +
   \frac{-x_t^3 + 5 x_t^2 + 2 x_t}{8(x_t-1)^3}                               
   \equiv -\frac{1}{2} E'_0(x_t)
\end{equation}
with all remaining coefficients being zero at $\mu=\mw$. The functions
$D'_0(x_t)$ and $E'_0(x_t)$ are sometimes used in the literature. The
$6 \times 6$ submatrix of $\gamma^{(0)}_s$ involving the operators
$Q_1,\ldots,Q_6$ is given in \eqn{eq:gs0Kpp}. Here we only give the
remaining non-vanishing entries of $\gamma^{(0)}_s$
\cite{CFMRS:93}, \cite{CFRS:94}.

Denoting for simplicity $\gamma_{ij} \equiv (\gamma_s)_{ij}$, the elements
$\gamma^{(0)}_{i7}$ with $i=1,\ldots,6$ are:
\begin{eqnarray}
\gamma^{(0)}_{17} = 0, &\qquad&  \gamma^{(0)}_{27} =
\frac{104}{27} C_F
\label{eq:g0127} \\
\gamma^{(0)}_{37} = -\frac{116}{27} C_F
 &\qquad&  \gamma^{(0)}_{47}  = \left(\frac{104}{27} u -\frac{58}{27}d
\right) C_F
\label{eq:g0347} \\
\gamma^{(0)}_{57} = \frac{8}{3} C_F &\qquad&
\gamma^{(0)}_{67} = \left( \frac{50}{27}d -\frac{112}{27}u \right) C_F
\label{eq:g0567}
\end{eqnarray}
The elements $\gamma^{(0)}_{i8}$ with $i=1,\ldots,6$ are:
\begin{eqnarray}
\gamma^{(0)}_{18} = 3, &\quad& \gamma^{(0)}_{28} =
\frac{11}{9} N-\frac{29}{9}\frac{1}{N}
\label{eq:g0128} \\
\gamma^{(0)}_{38} = \frac{22}{9} N-\frac{58}{9}\frac{1}{N}+3 f
 &\quad& \gamma^{(0)}_{48}  = 
6+\left(\frac{11}{9} N -\frac{29}{9}\frac{1}{N}\right) f
\label{eq:g0348} \\
\gamma^{(0)}_{58} = -2 N+\frac{4}{N} -3 f  &\quad&
\gamma^{(0)}_{68} = -4-\left( \frac{16}{9} N -
\frac{25}{9}\frac{1}{N}\right) f
\label{eq:g0568}
\end{eqnarray}

Finally the $2\times 2$ one-loop anomalous dimension matrix in the
sector $Q_{7\gamma},Q_{8G}$ is given by \cite{Grin}
\begin{eqnarray}
\gamma^{(0)}_{77} = 8 C_F
&\qquad&
\gamma^{(0)}_{78} = 0
\label{gammaB0} \\
\gamma^{(0)}_{87} = -\frac{8}{3} C_F
&\qquad&
\gamma^{(0)}_{88} = 16 C_F - 4 N
\nn
\end{eqnarray}

As we discussed above, the first correct calculation of the two-loop
mixing between $Q_1,\ldots,Q_6$ and $Q_{7\gamma}$, $Q_{8G}$ has been
presented in \cite{CFMRS:93}, \cite{CFRS:94} and confirmed subsequently
in \cite{CCRV:94a}, \cite{CCRV:94b}, \cite{misiak:94}.  In order to
extend these calculations beyond the leading order one would have to
calculate $\gamma_s^{(1)}$ (see (\ref{gg01})) and $O(\as)$ corrections to
the initial conditions in (\ref{c7}) and (\ref{c8}). We summarize below
the present status of this NLO calculation.

The $6\times 6$ two-loop submatrix of $\gamma^{(1)}_s$ involving
the operators $Q_1,\ldots,Q_6$ is given in eq. \eqn{eq:gs1ndrN3Kpp}.
The two-loop generalization of (\ref{gammaB0}) has been calculated only
last year \cite{misiakmuenz:95}. It is given for both NDR and HV
schemes as follows
\begin{eqnarray}
\gamma^{(1)}_{77} &=& 
   C_F \left(\frac{548}{9} N - 16 C_F - \frac{56}{9} f \right)
\nn \\
\gamma^{(1)}_{78} &=& 0
\label{gammaB1} \\
\gamma^{(1)}_{87} &=& 
   C_F \left(-\frac{404}{27} N +\frac{32}{3} C_F +\frac{56}{27} f \right)
\nn \\
\gamma^{(1)}_{88} &=& -\frac{458}{9} -\frac{12}{N^2}+ \frac{214}{9} N^2 +
   \frac{56}{9} \frac{f}{N} - \frac{13}{9} f N
\nn
\end{eqnarray}

The generalization of \eqn{eq:g0127}--\eqn{eq:g0568} to next-to-leading
order requires three loop calculations which have not been done yet.
The $O(\as)$ corrections to $C_{7\gamma}(\mw)$ and $C_{8G}(\mw)$ have
been considered in \cite{Yao1}.

\subsection{Results for the Wilson Coefficients}
         \label{sec:Heff:BXsgamma:wcres}
The leading order results for the Wilson Coefficients of all operators
entering the effective hamiltonian in (\ref{eq:HeffBXsgamma}) can be written
in an analytic form. They are \cite{BMMP:94}
\begin{eqnarray}
\label{coeffs}
C_j^{(0)}(\mu)    & = & \sum_{i=1}^8 k_{ji} \eta^{a_i}
  \qquad (j=1,\ldots,6)  \\
\label{C7eff}
C_{7\gamma}^{(0)eff}(\mu) & = & 
\eta^\frac{16}{23} C_{7\gamma}^{(0)}(\mw) + \frac{8}{3}
   \left(\eta^\frac{14}{23} - \eta^\frac{16}{23}\right) C_{8G}^{(0)}(\mw) +
    C_2^{(0)}(\mw)\sum_{i=1}^8 h_i \eta^{a_i},
\label{C7Geff}
\\
C_{8G}^{(0)eff}(\mu) & = & 
\eta^\frac{14}{23} C_{8G}^{(0)}(\mw) 
   + C_2^{(0)}(\mw) \sum_{i=1}^8 \bar h_i \eta^{a_i},
\end{eqnarray}
with
\begin{eqnarray}
\eta & = & \frac{\as(\mw)}{\as(\mu)}, 
\end{eqnarray}
and $C_{7\gamma}^{(0)}(\mw)$
and $ C_{8G}^{(0)}(\mw)$ given in (\ref{c7}) and (\ref{c8}),
respectively. The numbers $a_i$, $k_{ji}$, $h_i$ and $\bar h_i$ are
given in table \ref{tab:akh}.

\begin{table}[htb]
\caption[]{
\label{tab:akh}}
\begin{center}
\begin{tabular}{|r|r|r|r|r|r|r|r|r|}
$i$ & 1 & 2 & 3 & 4 & 5 & 6 & 7 & 8 \\
\hline
$a_i $&$ \frac{14}{23} $&$ \frac{16}{23} $&$ \frac{6}{23} $&$
-\frac{12}{23} $&$
0.4086 $&$ -0.4230 $&$ -0.8994 $&$ 0.1456 $\\
$k_{1i} $&$ 0 $&$ 0 $&$ \frac{1}{2} $&$ - \frac{1}{2} $&$
0 $&$ 0 $&$ 0 $&$ 0 $\\
$k_{2i} $&$ 0 $&$ 0 $&$ \frac{1}{2} $&$  \frac{1}{2} $&$
0 $&$ 0 $&$ 0 $&$ 0 $\\
$k_{3i} $&$ 0 $&$ 0 $&$ - \frac{1}{14} $&$  \frac{1}{6} $&$
0.0510 $&$ - 0.1403 $&$ - 0.0113 $&$ 0.0054 $\\
$k_{4i} $&$ 0 $&$ 0 $&$ - \frac{1}{14} $&$  - \frac{1}{6} $&$
0.0984 $&$ 0.1214 $&$ 0.0156 $&$ 0.0026 $\\
$k_{5i} $&$ 0 $&$ 0 $&$ 0 $&$  0 $&$
- 0.0397 $&$ 0.0117 $&$ - 0.0025 $&$ 0.0304 $\\
$k_{6i} $&$ 0 $&$ 0 $&$ 0 $&$  0 $&$
0.0335 $&$ 0.0239 $&$ - 0.0462 $&$ -0.0112 $\\
$h_i $&$ 2.2996 $&$ - 1.0880 $&$ - \frac{3}{7} $&$ -
\frac{1}{14} $&$ -0.6494 $&$ -0.0380 $&$ -0.0185 $&$ -0.0057 $\\
$\bar h_i $&$ 0.8623 $&$ 0 $&$ 0 $&$ 0
 $&$ -0.9135 $&$ 0.0873 $&$ -0.0571 $&$ 0.0209 $\\
\end{tabular}
\end{center}
\end{table}

\subsection{Numerical Analysis}
         \label{sec:Heff:BXsgamma:num}
The decay $B \to X_s \gamma$ is the only decay in our review for which
the complete NLO corrections are not available. In presenting the
numerical values for the Wilson coefficients a few remarks on the choice
of $\as$ should therefore be made. In the leading order the leading
order expression for $\as$ should be used. The question then is what to
use for $\Lambda_{\rm QCD}$ in this expression. In other decays for
which NLO corrections were available this was not important because LO
results were secondary. We have therefore simply inserted our standard
$\Lms$ values into the LO formula for $\as$. This procedure gives
$\as^{(5)}(\mz)=0.126, 0.136, 0.144$ for $\Lms^{(5)}=140\mev,
225\mev, 310\mev$, respectively. In view of these high values of
$\as^{(5)}(\mz)$ we will proceed here differently. Following
\cite{BMMP:94} we will use $\as^{(5)}(\mz)=0.110, 0.117, 0.124$ as
in our NLO calculations , but we will evolve $\as(\mu)$ to $\mu \approx
\ord(\mb)$ using the leading order expressions. In short, we will use
\begin{equation} 
\as(\mu) = \frac{\as(\mz)}{1 - \beta_0 \as(\mz)/2\pi \, \ln(\mz/\mu)} \, .
\label{eq:asmumz}
\end{equation} 
This discussion shows again the importance of the complete NLO
calculation for this decay.

Before starting the discussion of the numerical values for the
coefficients $C^{(0)eff}_{7\gamma}$ and $C^{(0)eff}_{8G}$, let us
illustrate the relative numerical importance of the three terms in
expression (\ref{C7eff}) for $C^{(0)eff}_{7\gamma}$. 

For instance, for $\mt = 170\gev$, $\mu = 5\gev$ and $\as^{(5)}(\mz)
=0.117$ one obtains
\begin{eqnarray}
C^{(0)eff}_{7\gamma}(\mu) &=&
0.698 \; C^{(0)}_{7\gamma}(\mw) +
0.086 \; C^{(0)}_{8G}(\mw) - 0.156 \; C^{(0)}_2(\mw)
\nn\\
 &=& 0.698 \; (-0.193) + 0.086 \; (-0.096) - 0.156 = -0.299 \, .
\label{eq:C7geffnum}
\end{eqnarray}

In the absence of QCD we would have $C^{(0)eff}_{7\gamma}(\mu) =
C^{(0)}_{7\gamma}(\mw)$ (in that case one has $\eta = 1$). Therefore, the
dominant term in the above expression (the one proportional to
$C^{(0)}_2(\mw)$) is the additive QCD correction that causes the
enormous QCD enhancement of the $\bsg$ rate \cite{Bert}, \cite{Desh}.
It originates solely from the two-loop diagrams. On the other hand, the
multiplicative QCD correction (the factor 0.698 above) tends to
suppress the rate, but fails in the competition with the additive
contributions.

In the case of $C^{(0)eff}_{8G}$ a similar enhancement is observed
\begin{eqnarray}
C^{(0)eff}_{8G}(\mu) &=&
0.730 \; C^{(0)}_{8G}(\mw) - 0.073 \; C^{(0)}_2(\mw)
\nn \\
 &=& 0.730 \; (-0.096) - 0.073 = -0.143 \, .
\label{eq:C8Geffnum}
\end{eqnarray}

In table \ref{tab:c78effnum} we give the values of
$C^{(0)eff}_{7\gamma}$ and $C^{(0)eff}_{8G}$ for different values of
$\mu$ and $\as^{(5)}(\mz)$. To this end \eqn{eq:asmumz} has been used.
A strong $\mu$-dependence of both coefficients is observed.  We will
return to this dependence in section \ref{sec:Heff:Bsgamma}.

\begin{table}[htb]
\caption[]{Wilson coefficients $C^{(0)eff}_{7\gamma}$ and $C^{(0)eff}_{8G}$
for $\mt = 170 \gev$ and various values of $\as^{(5)}(\mz)$ and $\mu$.
\label{tab:c78effnum}}
\begin{center}
\begin{tabular}{|c||c|c||c|c||c|c|}
& \multicolumn{2}{c||}{$\as^{(5)}(\mz) = 0.110$} &
  \multicolumn{2}{c||}{$\as^{(5)}(\mz) = 0.117$} &
  \multicolumn{2}{c| }{$\as^{(5)}(\mz) = 0.124$} \\
\hline
$\mu [\gev]$ & 
$C^{(0)eff}_{7\gamma}$ & $C^{(0)eff}_{8G}$ &
$C^{(0)eff}_{7\gamma}$ & $C^{(0)eff}_{8G}$ &
$C^{(0)eff}_{7\gamma}$ & $C^{(0)eff}_{8G}$ \\
\hline
 2.5 & --0.323 & --0.153 & --0.334 & --0.157 & --0.346 & --0.162 \\
 5.0 & --0.291 & --0.140 & --0.299 & --0.143 & --0.307 & --0.147 \\
 7.5 & --0.275 & --0.133 & --0.281 & --0.136 & --0.287 & --0.139 \\
10.0 & --0.263 & --0.129 & --0.268 & --0.131 & --0.274 & --0.133
\end{tabular}
\end{center}
\end{table}

\section{The Effective Hamiltonian for $B\to X_{\lowercase{s}}
         \lowercase{e}^+\lowercase{e}^-$}
         \label{sec:Heff:BXsee}
The effective hamiltonian for $B\to X_s e^+e^-$ at scales $\mu=O(\mb)$
is given by
\begin{eqnarray} \label{eq:Heff2atmu}
\Heff(b\to s e^+e^-) &=&
\Heff(b\to s\gamma)  - \\
& & \frac{G_F}{\sqrt{2}} V_{ts}^* V_{tb}^{} \left[ C_{9V}(\mu) Q_{9V}(\mu)+
C_{10A}(\mu) Q_{10A}(\mu) \right]
\nn
\end{eqnarray}
where again we have neglected the term proportional to $V_{us}^*V_{ub}^{}$
and $\Heff(b\to s\gamma)$ is given in (\ref{eq:HeffBXsgamma}).

\subsection{Operators}
         \label{sec:Heff:BXsee:ops}
In addition to the operators relevant for $B\to X_s\gamma$,
there are two new operators
\begin{equation}\label{Q9V}
Q_{9V}    = (\bar{s} b)_{V-A}  (\bar{e}e)_V         
\qquad
Q_{10A}  =  (\bar{s} b)_{V-A}  (\bar{e}e)_A
\end{equation}
where $V$ and $A$ refer to $\gamma_{\mu}$ and $ \gamma_{\mu}\gamma_5$,
respectively.

They originate in the $Z^0$- and $\gamma$-penguin diagrams
with external $\bar{e}e$ of fig.\ \ref{fig:oporig}\,(f) and from the
corresponding box diagrams.
\subsection{Wilson Coefficients}
         \label{sec:Heff:BXsee:wc}
The coefficient $C_{10A}(\mu)$ is given by
\begin{equation} \label{C10}
C_{10A}(\mw) =  \frac{\aem}{2\pi} \Ctilde_{10}(\mw), \qquad
\Ctilde_{10}(\mw) = - \frac{Y_0(x_t)}{\sin^2\Theta_W}
\end{equation}
with $Y_0(x)$ given in \eqn{eq:yz0}. Since $Q_{10A}$ does not renormalize
under QCD, its coefficient does not depend on $\mu\approx {\cal
O}(\mb)$. The only renormalization scale dependence in (\ref{C10})
enters through the definition of the top quark mass. We will return to
this issue in section \ref{sec:Heff:BXsee:nlo:num}.

The coefficient $C_{9V}(\mu)$ has been calculated over the last years
with increasing precision by several groups \cite{grinstein:89a},
\cite{GDSN:89}, \cite{cellaetal:91}, \cite{misiak:93} culminating in two
complete next-to-leading QCD calculations \cite{misiak:94},
\cite{burasmuenz:95} which agree with each other.

In order to calculate the coefficient $C_{9V}$ including
next-to-leading order corrections we have to perform in principle a
two-loop renormalization group analysis for the full set of operators
contributing to (\ref{eq:Heff2atmu}). However, $Q_{10A}$ is not
renormalized and the dimension five operators $Q_{7\gamma}$ and
$Q_{8G}$ have no impact on $C_{9V}$. Consequently only a set of seven
operators, $Q_1,\ldots,Q_6$ and $Q_{9V}$, has to be considered. This is
precisely the case of the decay $\kpiee$ discussed in \cite{burasetal:94a}
and in section \ref{sec:HeffKpe}, except for an appropriate change of quark
flavours and the fact that now $\mu\approx {\cal O}(\mb)$ instead of
$\mu\approx {\cal O}(1\gev)$. Since the NLO analysis of $\kpiee$ has
already been presented in section \ref{sec:HeffKpe} we will only give the
final result for $C_{9V}(\mu)$. Because of the one step evolution from
$\mu=\mw$ down to $\mu=\mb$ without any thresholds in between it is
possible to find an analytic formula for $C_{9V}(\mu)$. Defining
$\tilde C_{9}$ by
\begin{equation} \label{C9}
C_{9V}(\mu) = \frac{\aem}{2\pi} \Ctilde_9(\mu) 
\end{equation}
one finds \cite{burasmuenz:95} in the NDR scheme
\begin{equation}\label{C9tilde}
\Ctilde_9^{NDR}(\mu)  =  
P_0^{NDR} + \frac{Y_0(x_t)}{\sin^2\Theta_W} -4 Z_0(x_t) +
P_E E_0(x_t)
\end{equation}
with
\begin{eqnarray}
\label{P0NDR}
P_0^{NDR} & = & \frac{\pi}{\as(\mw)} (-0.1875+ \sum_{i=1}^8 p_i
\eta^{a_i+1}) \nn \\ 
          &   & + 1.2468 +  \sum_{i=1}^8 \eta^{a_i} \lbrack
r^{NDR}_i+s_i \eta \rbrack \\ 
\label{PE}
P_E & = & 0.1405 +\sum_{i=1}^8 q_i\eta^{a_i+1}  \, .
\end{eqnarray}
The functions $Y_0(x)$ and $Z_0(x)$ are defined by
\begin{equation}
Y_0(x) = C_0(x) - B_0(x)
\qquad
Z_0(x) = C_0(x) + \frac{1}{4} D_0(x)
\label{eq:yz0}
\end{equation}
with $B_0(x)$, $C_0(x)$ and $D_0(x)$ given in \eqn{eq:Bxt},
\eqn{eq:Cxt} and \eqn{eq:Dxt}, respectively. $E_0(x)$ is given in
eq.\ \eqn{eq:Ext}. The powers $a_i$ are the same as in table
\ref{tab:akh}.  The coefficients $p_i$, $r^{NDR}_i$, $s_i$, and $q_i$
can be found in table \ref{tab:prsq}.  $P_E$ is ${\cal O}(10^{-2})$ and
consequently the last term in (\ref{C9tilde}) can be neglected. We keep
it however in our numerical analysis. These results agree with
\cite{misiak:94}.

\begin{table}[htb]
\caption[]{
\label{tab:prsq}}
\begin{center}
\begin{tabular}{|r|r|r|r|r|r|r|r|r|}
$i$ & 1 & 2 & 3 & 4 & 5 & 6 & 7 & 8 \\
\hline
$p_i $&$ 0, $&$ 0, $&$ -\frac{80}{203}, $&$  \frac{8}{33}, $&$
0.0433 $&$  0.1384 $&$ 0.1648 $&$ - 0.0073 $\\
$r^{NDR}_{i} $&$ 0 $&$ 0 $&$ 0.8966 $&$ - 0.1960 $&$
- 0.2011 $&$ 0.1328 $&$ - 0.0292 $&$ - 0.1858 $\\
$s_i $&$ 0 $&$ 0 $&$ - 0.2009 $&$  -0.3579 $&$
0.0490 $&$ - 0.3616 $&$ -0.3554 $&$ 0.0072 $\\
$q_i $&$ 0 $&$ 0 $&$ 0 $&$  0 $&$
0.0318 $&$ 0.0918 $&$ - 0.2700 $&$ 0.0059 $\\
\svs
$r^{HV}_{i} $&$ 0 $&$ 0 $&$ -0.1193 $&$ 0.1003 $&$
- 0.0473 $&$ 0.2323 $&$ - 0.0133 $&$ - 0.1799 $
\end{tabular}
\end{center}
\end{table}

In the HV scheme only the coefficients $r_i$ are changed. 
They are given on the last line of table \ref{tab:prsq}.
Equivalently we can write
\begin{equation} \label{P0HV}
P_0^{k} = P_0^{NDR} + \xi_{k} \frac{4}{9} \left( 3 C_1^{(0)} +
C_2^{(0)} - C_3^{(0)} -3 C_4^{(0)} \right)
\end{equation}
with
\begin{equation} \label{xi}
\xi_k = \left\{
\begin{array}{rl}
0  &\quad k=\mbox{NDR} \\
-1 &\quad k=\mbox{HV}
\end{array} \, .
\right.
\end{equation}
We note that
\begin{eqnarray}
\label{sums1}
\sum_{i=1}^8 p_i = 0.1875, &\quad& \sum_{i=1}^8 q_i = -
0.1405, \\
\label{sums2}
\sum_{i=1}^8 (r_i^k + s_i) = - 1.2468 + \frac{4}{9} (1 +
\xi_k), &\quad& \sum_{i=1}^8 p_i (a_i + 1) = - \frac{16}{69}.
\end{eqnarray}
In this way for $\eta=1$ one finds $P_E=0$, $P_0^{NDR} = 4/9$ and
$P_0^{HV} = 0$ in accordance with the initial conditions at $\mu=\mw$.
 Moreover, the second relation in (\ref{sums2})
assures the correct large logarithm in $P_0^{NDR}$, i.e.\ $8/9\, \ln
(\mw/\mu)$. 

The special feature of $C_{9V}(\mu)$ compared to the coefficients
of the remaining operators contributing to $B\to X_s e^+e^-$ is the
large logarithm represented by $1/\as$ in $P_0$ in
(\ref{P0NDR}). Consequently the renormalization group improved
perturbation theory for $C_{9V}$ has the structure $ {\cal O}(1/\as) +
{\cal O}(1) + {\cal O}(\as)+ \ldots$, whereas the corresponding series
for the remaining coefficients is $ {\cal O}(1) + {\cal O}(\as)+
\ldots$\,. Therefore in order to find the next-to-leading ${\cal O}(1)$
term in the branching ratio for $B\to X_s e^+e^-$, the full two-loop
renormalization group analysis has to be performed in order to find
$C_{9V}$, but the coefficients of the remaining operators should be
taken in the leading logarithmic approximation. This is gratifying
because the coefficient of the magnetic operator $Q_{7\gamma}$ is known
only in the leading logarithmic approximation.

\subsection{Numerical Results}
         \label{sec:Heff:BXsee:num}
In our numerical analysis we will use the two-loop expression for
$\as$ and the parameters collected in the appendix. Our
presentation follows closely the one given in \cite{burasmuenz:95}. 

In table \ref{tab:p0C9} we show the constant $P_0$ in (\ref{P0NDR}) for
different $\mu$ and $\Lms$, in the leading order corresponding to the
first term in (\ref{P0NDR}) and for the NDR and HV schemes as given by
(\ref{P0NDR}) and (\ref{P0HV}), respectively. In table \ref{tab:BXsee:C9} we
show the corresponding values for $\Ctilde_9(\mu)$. To this end
we set $\mt= 170 \gev$. 

\begin{table}[htb]
\caption[]{The coefficient $P_0$ of $\widetilde C_9$ for various values
of $\Lms^{(5)}$ and $\mu$.
\label{tab:p0C9}}
\begin{center}
\begin{tabular}{|c||c|c|c||c|c|c||c|c|c|}
& \multicolumn{3}{c||}{$\Lms^{(5)} = 140 \mev$} &
  \multicolumn{3}{c||}{$\Lms^{(5)} = 225 \mev$} &
  \multicolumn{3}{c| }{$\Lms^{(5)} = 310 \mev$} \\
\hline
$\mu [\gev]$ & LO & NDR & HV & LO & NDR & HV & LO & NDR & HV \\
\hline
2.5 & 2.053 & 2.928 & 2.797 & 1.933 & 2.846 & 2.759 & 1.835 & 2.775 &
2.727 \\
5.0 & 1.852 & 2.625 & 2.404 & 1.788 & 2.591 & 2.395 & 1.736 & 2.562 &
2.388 \\
7.5 & 1.675 & 2.391 & 2.127 & 1.632 & 2.373 & 2.127 & 1.597 & 2.358 &
2.128 \\
10.0 & 1.526 & 2.204 & 1.912 & 1.494 & 2.194 & 1.917 & 1.469 & 2.185 &
1.921
\end{tabular}
\end{center}
\end{table}

\begin{table}[htb]
\caption[]{Wilson coefficient $\widetilde C_9$ for $\mt = 170 \gev$ and
various values of $\Lms^{(5)}$ and $\mu$.
\label{tab:BXsee:C9}}
\begin{center}
\begin{tabular}{|c||c|c|c||c|c|c||c|c|c|}
& \multicolumn{3}{c||}{$\Lms^{(5)} = 140 \mev$} &
  \multicolumn{3}{c||}{$\Lms^{(5)} = 225 \mev$} &
  \multicolumn{3}{c| }{$\Lms^{(5)} = 310 \mev$} \\
\hline
$\mu [\gev]$ & LO & NDR & HV & LO & NDR & HV & LO & NDR & HV \\
\hline
2.5 & 2.053 & 4.493 & 4.361 & 1.933 & 4.410 & 4.323 & 1.835 & 4.338 &
4.290 \\
5.0 & 1.852 & 4.191 & 3.970 & 1.788 & 4.156 & 3.961 & 1.736 & 4.127 &
3.954 \\
7.5 & 1.675 & 3.958 & 3.694 & 1.632 & 3.940 & 3.694 & 1.597 & 3.924 &
3.695 \\
10.0 & 1.526 & 3.772 & 3.480 & 1.494 & 3.761 & 3.485 & 1.469 & 3.752 &
3.488
\end{tabular}
\end{center}
\end{table}

\noindent
We observe:
\begin{itemize}
\item
The NLO corrections to $P_0$ enhance this constant relatively to the
LO result by roughly 45\% and 35\% in the NDR and HV schemes,
respectively. This enhancement is analogous to the one found in the
case of $\kpiee$.
\item
In calculating $P_0$ in the LO we have used $\as(\mu)$ at one-loop
level. Had we used the two-loop expression for $\as(\mu)$ we
would find for $\mu=5 \gev$ and $\Lms^{(5)} = 225 \mev$ the value $P_0^{LO}
\approx 1.98$. Consequently the NLO corrections would have smaller
impact. Ref.~\cite{grinstein:89a} including the next-to-leading term $4/9$
would find $P_0$ values roughly 20\% smaller than $P_0^{NDR}$ given in
table \ref{tab:p0C9}.
\item
It is tempting to compare $P_0$ in table \ref{tab:p0C9} with that found in
the absence of QCD corrections. In the limit $\as \to 0$ we
find $P_0^{NDR} = 8/9 \, \ln(\mw/\mu) + 4/9$ and $P_0^{HV} = 8/9\,
\ln(\mw/\mu)$ which for $\mu = 5 \gev$ give $P_0^{NDR} = 2.91$ and
$P_0^{HV} = 2.46$. Comparing these values with table~\ref{tab:p0C9} we
conclude that the QCD suppression of $P_0$ present in the leading order
approximation is considerably weakened in the NDR treatment of
$\gamma_5$ after the inclusion of NLO corrections. It is essentially
removed for $\mu > 5 \gev$ in the HV scheme.
\item
The NLO corrections to $\Ctilde_9$ which include also the
$\mt$-dependent contributions are large as seen in table
\ref{tab:BXsee:C9}. The results in HV and NDR schemes are by more than
a factor of two larger than the leading order result $\Ctilde_9 =
P_0^{LO}$ which consistently should not include $\mt$-contributions.
This demonstrates very clearly the necessity of NLO calculations which
allow a consistent inclusion of the important $\mt$-contributions. For
the same set of parameters the authors of \cite{grinstein:89a}
would find $\Ctilde_9$ to be smaller than $\Ctilde_9^{NDR}$ by
10--15\%.
\item
The $\Lms$ dependence of $\Ctilde_9$ is rather weak.  On the other
hand its $\mu$ dependence is sizable ($\sim 15\%$ in the range of $\mu$
considered) although smaller than in the case of the coefficients
$C_{7\gamma}$ and $C_{8G}$ given in table \ref{tab:c78effnum}.  We also
find that the $\mt$ dependence of $\Ctilde_9$ is rather weak. Varying
$\mt$ between $150 \gev$ and $190 \gev$ changes $\Ctilde_9$ by at most
10\%. This weak $\mt$ dependence of $\Ctilde_9$ originates in the
partial cancelation of $\mt$ dependences between $Y_0(x_t)$ and
$Z_0(x_t)$ in (\ref{C9tilde}) as already seen in the case of $\kpiee$
in fig.\ \ref{fig:kpiee:mty7VA}. Finally, the difference between
$\Ctilde_9^{NDR}$ and $\Ctilde_9^{HV}$ is small and amounts to roughly 5\%.
\item
The dominant $\mt$-dependence in this decay originates in the $\mt$
dependence of $\tilde C_{10}(\mw)$. In fact, as can be seen in section
\ref{sec:HeffKpe} $\tilde C_{10}(\mw)=2\pi y_{7A}/\aem$ with $y_{7A}$
present in $K_L \to \pi^0 e^+ e^-$. The $\mt$ dependence of $y_{7A}$ is
shown in fig.\ \ref{fig:kpiee:mty7VA}.
\end{itemize}
