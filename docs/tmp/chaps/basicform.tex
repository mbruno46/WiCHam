\section{Basic Formalism}
         \label{sec:basicform}
\subsection{Renormalization of QCD}
            \label{sec:basicform:renorm}
As already emphasized in the introduction, the effects of QCD play an
important role in the phenomenology of weak decays of hadrons. In fact
in the theoretical analysis of these decays the investigation of
QCD corrections is the most difficult and extensive part. In the
present subsection we shall therefore briefly recall basic features
of perturbative QCD and its renormalization. Thereby we will concentrate
on those aspects, that will be needed for the present review. We
will also take the opportunity to introduce for later reference the
expressions for the running coupling, the running mass and the
corresponding renormalization group functions.\\
The Lagrangian density of QCD reads
\begin{eqnarray}\label{lqcd}
{\cal L}_{QCD} & = &
-{1\over 4}(\partial_\mu A^a_\nu-\partial_\nu A^a_\mu)
(\partial^\mu A^{a\nu}-\partial^\nu A^{a\mu})-{1\over{2\xi}}
(\partial^\mu A^a_\mu)^2   \nonumber \\
&&{}+ \bar q(i\not\!\partial-m_q)q+\chi^{a\ast}\partial^\mu
        \partial_\mu \chi^a  \nonumber \\
&&{}-{g\over 2}f^{abc}(\partial_\mu A^a_\nu-\partial_\nu A^a_\mu)
 A^{b\mu}A^{c\nu} - {g^2\over 4}f^{abe}f^{cde}A^a_\mu A^b_\nu
 A^{c\mu}A^{d\nu} \nonumber \\
&&{}+ g \bar q_i T^a_{ij}\gamma^\mu q_j A^a_\mu +
 g f^{abc} (\partial^\mu\chi^{a\ast})\chi^b A^c_\mu
\end{eqnarray}
Here $q=(q_1, q_2, q_3)$ is the color triplet of quark flavor $q$,
$q$ = $u$, $d$, $s$, $c$, $b$, $t$. $g$ is the QCD coupling, $A^a_\mu$
the gluon field, $\chi^a$ the ghost field and $\xi$ the gauge
parameter. $T^a$, $f^{abc}$ ($a$, $b$, $c$ = 1,\ldots,8) are the
generators and structure constants of $SU(3)$, respectively. From
this Lagrangian one may read off the Feynman rules for QCD, e.g.
$i g T^a_{ij}\gamma^\mu$ for the quark-gluon vertex.\\
In order to deal with divergences that appear in quantum (loop)
corrections to Green functions, the theory has to be regularized
to have an explicit parametrization
of the singularities and subsequently renormalized to render
the Green functions finite. For these purposes we will employ:
\begin{itemize}
\item Dimensional regularization (DR) by continuation to $D=4-2\,\eps$
space-time dimensions \cite{bollini:72a}, \cite{bollini:72b},
\cite{thooft:72b}, \cite{ashmore:72}, \cite{cicutamontaldi:72}.
\item Subtraction of divergences in the minimal subtraction scheme $MS$
\cite{thooft:73} or the modified minimal subtraction scheme
($\overline{MS}$) \cite{bardeen:78}.
\end{itemize}
To eliminate the divergences one has to renormalize the fields and
parameters in the Lagrangian, in general through
\begin{equation}\label{zqcd}
\begin{array}{lclcl}
A^a_{0\mu}=Z^{1/2}_3 A^a_\mu &\qquad& q_0=Z^{1/2}_q q &\qquad&
\chi^a_0=\tilde Z^{1/2}_3 \chi^a \\
g_0=Z_g g\mu^\eps &\qquad& \xi_0=Z_3 \xi &\qquad& m_0=Z_m m
\end{array}
\end{equation}
The index ``0'' indicates unrenormalized quantities. The factors $Z$
are the renormalization constants. The scale $\mu$ has been introduced
to make $g$ dimensionless in $D=4-2\,\eps$ dimensions. Since we will
not consider Green functions with external ghosts, we will not need
the ghost field renormalization. We also do not need the gauge
parameter renormalization if we are dealing with gauge independent
quantities, as e.g. Wilson coefficient functions.\\
A straightforward way to implement renormalization is provided by
the coun\-ter\-term method. Thereby the parameters and fields in the
original Lagrangian, which are to be considered as unrenormalized
(bare) quantities, are reexpressed through renormalized ones by
means of \eqn{zqcd} from the very beginning. For instance, the quark
kinetic term becomes
\begin{equation}\label{ctm}
{\cal L}_F=\bar q_0 i\not\!\partial q_0-m_0\bar q_0 q_0\equiv
\bar q i\not\!\partial q-m\bar q q+(Z_q-1)\bar q i\not\!\partial q-
(Z_q Z_m-1)m\bar q q     \end{equation}
The advantage then is, that only renormalized quantities are present
in the Lagrangian. The counterterms ($\sim(Z-1)$), appearing in
addition, can be formally treated as interaction terms that contribute
to Green functions calculated in perturbation theory. The Feynman
rule for the counterterms in \eqn{ctm}, for example, reads
($p$ is the quark momentum)
\begin{equation}\label{ctex}
i(Z_q-1)\not\! p - i(Z_q Z_m-1) m   \end{equation}
The constants $Z_i$ are then determined such that they cancel the
divergences in the Green functions according to the chosen
renormalization scheme. In an analogous way all renormalization constants
can be fixed by considering the appropriate Green functions.
\\
Of central importance for the study of perturbative QCD effects are
the renormalization group equations, which govern the dependence of
renormalized parameters and Green functions on the renormalization
scale $\mu$. These differential equations are easily derived from the
definition \eqn{zqcd} by using the fact that bare quantities are
$\mu$-independent. In this way one finds that the renormalized
coupling $g(\mu)$ obeys \cite{gross:76}
\begin{equation}\label{rgbe}
{d\over d\ln\mu}g(\mu)=\beta(\eps, g(\mu))  \end{equation}
where
\begin{equation}\label{bete}
\beta(\eps, g)=-\eps g-g{1\over Z_g}{d Z_g\over d\ln\mu}\equiv
-\eps g+\beta(g)  \end{equation}
which defines the $\beta$ function. \eqn{rgbe} is valid in arbitrary
dimensions. In four dimensions $\beta(\eps, g)$ reduces to $\beta(g)$.
Similarly, the anomalous dimension of the mass, $\gamma_m$,
defined through
\begin{equation}\label{rggm}
{d m(\mu)\over d\ln\mu}=-\gamma_m(g) m(\mu)  \end{equation}
is given by
\begin{equation}\label{gamz} \gamma_m(g)={1\over Z_m}{d Z_m\over d\ln\mu}  \end{equation}
In the $MS$ $(\overline{MS})$-scheme, where just the pole terms in $\eps$ are
present in the renorma\-lization constants $Z_i$, these can be expanded
as follows
\begin{equation}\label{ziep}
Z_i=1+\sum^\infty_{k=1} {1\over \eps^k} Z_{i, k}(g)  \end{equation}
Using \eqn{rgbe}, \eqn{bete} one finds
\begin{equation}\label{zi1}
{1\over Z_i}{d Z_i\over d\ln\mu}=-2 g^2{\partial Z_{i, 1}(g)\over\partial g^2}\end{equation}
which allows a direct calculation of the renormalization group
functions from the $1/\eps$-pole part of the renormalization
constants. Along these lines one obtains at the two loop level,
required for next-to-leading order calculations,
\begin{equation}\label{bg01}
\beta(g)=-\beta_0{g^3\over 16\pi^2}-\beta_1{g^5\over (16\pi^2)^2}  \end{equation}
In terms of
\begin{equation}\label{aasg} \as\equiv{g^2\over 4\pi}  \end{equation}
we have
\begin{equation}\label{rga}
{d\as\over d\ln\mu}=-2\beta_0 {\as^2\over 4\pi}-2\beta_1
  {\as^3\over(4\pi)^2}  \end{equation}
Similarly, the two-loop expression for the quark mass anomalous
dimension can be written as
\begin{equation}\label{gama}
\gamma_m(\as)=\gamma_{m0}\aspi + \gamma_{m1}\left(\aspi\right)^2
\end{equation}
We also give the $1/\eps$-pole part $Z_{q, 1}$ of the quark field
renormalization constant $Z_1$ to $\ord(\as^2)$, which we will need
later on
\begin{equation}\label{zq1a} Z_{q, 1}=a_1\aspi + a_2\left(\aspi\right)^2
\end{equation}
The coefficients in eqs. \eqn{rga} -- \eqn{zq1a} read
\begin{equation}\label{b0b1}
\beta_0={{11N-2f}\over 3}\qquad
\beta_1={34\over 3}N^2-{10\over 3}Nf-2C_F f\qquad
C_F={{N^2-1}\over{2N}}\end{equation}
\begin{equation}\label{gm01} \gamma_{m0}=6C_F\qquad \gamma_{m1}=C_F\left(
     3C_F+{97\over 3}N-{10\over 3}f\right)  \end{equation}
\begin{equation}\label{a1a2} a_1=-C_F\qquad a_2=C_F\left(
     {3\over 4}C_F-{17\over 4}N+{1\over 2}f\right)  \end{equation}
$N$ is the number of colors, $f$ the number of quark flavors. The
coefficients are given in the $MS$ ($\overline{MS}$) scheme.
However, $\beta_0$, $\beta_1$, $\gamma_{m0}$ and $a_1$ are scheme
independent. The expressions for $a_1$ and $a_2$ in \eqn{a1a2} are
valid in Feynman gauge, $\xi=1$.\\
At two-loop order the solution of the renormalization group equation
\eqn{rga} for $\as(\mu)$ can always be written in the form
\begin{equation}\label{amu}
\as(\mu)={4\pi\over{\beta_0\ln{{\mu^2}\over{\Lambda^2}}}} \left[1-
 {{\beta_1}\over{\beta_0^2}}{{\ln\ln{{\mu^2}\over{\Lambda^2}}}\over
    {\ln{{\mu^2}\over{\Lambda^2}}}}\right] \end{equation}
Eq. \eqn{amu} gives the running coupling constant at NLO. $\as(\mu)$
vanishes as $\mu/\Lambda\to\infty$ due to asymptotic freedom. We
remark that, in accordance with the two-loop accuracy, \eqn{amu} is
valid up to terms of the order $\ord(1/\ln^3\mu^2/\Lambda^2)$.
For the purpose of counting orders in $1/\ln\mu^2/\Lambda^2$ the
double logarithmic expression $\ln\ln\mu^2/\Lambda^2$ may formally
be viewed as a constant. Note that an additional term
${\rm const.}/\ln^2\mu^2/\Lambda^2$, which is of the same order as
the next-to-leading correction term in \eqn{amu}, can always be
absorbed into a multiplicative redefinition of $\Lambda$. Hence the
choice of the form \eqn{amu} is possible without restriction, but one
should keep in mind that the definition of $\Lambda$ is related to this
particular choice. The introduction of the $\overline{MS}$ scheme and
the corresponding definition of $\Lambda_{\overline{MS}}$ and its
relation to $\Lambda_{MS}$ is discussed in
section~\ref{sec:basicform:wc:disc}.
\\
Finally we write down the two-loop expression for the running quark
mass in the $MS$ ($\overline{MS}$) scheme, which results from
integrating \eqn{rggm}
\begin{equation}\label{mmu}
m(\mu)=m(m)\left[{\as(\mu)\over\as(m)}\right]^{\gamma_{m0}\over 2\beta_0}
\left[1+\left({\gamma_{m1}\over 2\beta_0}-{\beta_1\gamma_{m0}\over
  2\beta^2_0}\right){\as(\mu)-\as(m)\over 4\pi}\right]  \end{equation}

\subsection{Operator Product Expansion in Weak Decays -- Preliminaries}
            \label{sec:basicform:prelim}
Weak decays of hadrons are mediated through the weak interactions of
their quark constituents, whose strong interactions, binding the
constituents into hadrons, are characterized by a typical hadronic
energy scale of the order of $1\gev$. Our goal is therefore to derive
an effective low energy theory describing the weak interactions of
quarks. The formal framework to achieve this is provided by the
operator product expansion (OPE). In order to introduce the main ideas
behind it, let us consider the simple example of
the quark level transition $c\to su\bar d$,
which is relevant for Cabibbo-allowed decays of D mesons.
Disregarding QCD effects for the moment, the tree-level W-exchange
amplitude for $c\to su\bar d$ is simply given by
\begin{eqnarray}\label{aope}
A&=&i{G_F\over\sqrt{2}}V^*_{cs}V_{ud}^{}{M^2_W\over k^2-M^2_W}
  (\bar sc)_{V-A}(\bar ud)_{V-A} \nonumber\\
 &=& -i{G_F\over\sqrt{2}}V^*_{cs}V_{ud}^{}
  (\bar sc)_{V-A}(\bar ud)_{V-A} + \ord({k^2\over M^2_W})
\end{eqnarray}
where $(V-A)$ refers to the Lorentz structure $\gamma_{\mu} (1-\gf)$.

Since $k$, the momentum transfer through the W propagator, is very
small as compared to the W mass $M_W$, terms of the order
$\ord(k^2/M^2_W)$ can safely be neg\-lected and the full amplitude $A$
can be approximated by the first term on the r.h.s. of \eqn{aope}.
Now this term may obviously be also obtained from an effective
hamiltonian defined by
\begin{equation}\label{hc0}
{\cal H}_{eff}={G_F\over\sqrt{2}}V^*_{cs}V_{ud}^{}
  (\bar sc)_{V-A}(\bar ud)_{V-A} + \ldots  \end{equation}
where the ellipsis denotes operators of higher dimensions, typically
involving derivative terms, which can in principle be chosen so
as to reproduce the terms of higher order in $k^2/M^2_W$ of the
full amplitude in \eqn{aope}. This exercise already provides us with
a simple example of an OPE. The product of two charged current
operators is expanded into a series of local operators,
whose contributions are weighted by effective coupling constants,
the Wilson coefficients.\\
A more formal basis for this procedure may be given by considering
the genera\-ting functional for Green functions in the path integral
formalism. The part of the generating functional relevant for the
present discussion is, up to an overall normalizing factor, given by
\begin{equation}\label{zw1}
Z_W\sim\int [dW^+][dW^-] \exp(i\int d^4x {\cal L}_W)  \end{equation}
where ${\cal L}_W$ is the Lagrangian density containing the kinetic
terms of the W boson field and its interaction with charged currents
\begin{eqnarray}\label{lwjw}
\lefteqn{{\cal L}_W=
-{1\over 2}(\partial_\mu W^+_\nu-\partial_\nu W^+_\mu)
 (\partial^\mu W^{-\nu}-\partial^\nu W^{-\mu})+M^2_W W^+_\mu W^{-\mu}}
\hspace{3cm} \nonumber\\
& & +{g_2\over 2\sqrt{2}}(J^+_\mu W^{+\mu}+J^-_\mu W^{-\mu})
\end{eqnarray}
\begin{equation}\label{jpn}
J^+_\mu=V_{pn} \bar p\gamma_\mu(1-\gf)n\qquad p=(u, c, t)
\quad n=(d, s, b)\qquad J^-_\mu=(J^+_\mu)^\dagger  \end{equation}
Since we are not interested in Green functions with external W lines,
we have not introduced external source terms for the W fields. In the
present argument we will furthermore choose the unitary gauge for
the W field for definiteness, however physical results do not depend
on this choice.\\
Introducing the operator
\begin{equation}\label{kxy}
K_{\mu\nu}(x, y)=\delta^{(4)}(x-y)\left(g_{\mu\nu}(\partial^2+
  M^2_W)-\partial_\mu\partial_\nu\right)   \end{equation}
we may, after discarding a total derivative in the W kinetic term,
rewrite \eqn{zw1} as
\begin{eqnarray}\label{zw2}    \lefteqn{
Z_W\sim\int [dW^+][dW^-] \exp\biggl[ i\int d^4x d^4y W^+_\mu(x)
K^{\mu\nu}(x, y) W^-_\nu(y)  } \hspace{3cm}\nonumber\\
& & {}+i{g_2\over 2\sqrt{2}}\int d^4x
 J^+_\mu W^{+\mu}+J^-_\mu W^{-\mu} \biggr]
\end{eqnarray}
The inverse of $K_{\mu\nu}$, denoted by $\Delta_{\mu\nu}$, and
defined through
\begin{equation}\label{kde1}
\int d^4y K_{\mu\nu}(x, y) \Delta^{\nu\lambda}(y, z)=
 g^{\ \lambda}_\mu \delta^{(4)}(x-z)  \end{equation}
is just the W propagator in the unitary gauge
\begin{equation}\label{dexy}
\Delta_{\mu\nu}(x, y)=\int{d^4k\over (2\pi)^4}\Delta_{\mu\nu}(k)
  e^{-i k(x-y)}  \end{equation}
\begin{equation}\label{dek}
\Delta_{\mu\nu}(k)={-1\over k^2-M^2_W}\left(g_{\mu\nu}-
  {k_\mu k_\nu\over M^2_W}\right)   \end{equation}
Performing the gaussian functional integration over $W^\pm(x)$ in
\eqn{zw2} explicitly, this expression simplifies to
\begin{equation}\label{zetw}
Z_W\sim\exp\left[ -i\int{g^2_2\over 8}J^-_\mu(x)\Delta^{\mu\nu}(x, y)
J^+_\nu(y) d^4x d^4y \right]   \end{equation}
This result implies a nonlocal action functional for the quarks
\begin{equation}\label{snl}
{\cal S}_{nl}=\int d^4x {\cal L}_{kin}-
{g^2_2\over 8}\int d^4x d^4y J^-_\mu(x)\Delta^{\mu\nu}(x, y)
J^+_\nu(y)    \end{equation}
where the first piece represents the quark kinetic terms and the second
their charged current interactions.\\
We can now formally expand this second, nonlocal term in powers of
$1/M^2_W$ to yield a series of local interaction operators of
dimensions that increase with the order in $1/M^2_W$. To lowest order
\begin{equation}\label{dloc}
\Delta^{\mu\nu}(x, y)\approx{g^{\mu\nu}\over M^2_W}\delta^{(4)}(x-y) \end{equation}
and the second term in \eqn{snl} becomes
\begin{equation}\label{jjx}
-{g^2_2\over 8M^2_W}\int d^4x J^-_\mu(x) J^{+\mu}(x)  \end{equation}
corresponding to the usual effective charged current interaction
Lagrangian
\begin{equation}\label{leff}
{\cal L}_{int,eff}=-{G_F\over\sqrt{2}} J^-_\mu J^{+\mu}(x)=-
{G_F\over\sqrt{2}}V^*_{pn}V_{p'n'}^{}(\bar np)_{V-A}
  (\bar p'n')_{V-A}
\end{equation}
which contains, among other terms, the leading contribution to
\eqn{hc0}.

The simple considerations we have presented so far already illustrate
several of the basic aspects of the general approach.
\begin{itemize}
\item Formally, the procedure to approximate the interaction term in
\eqn{snl} by \eqn{jjx} is an example of a short-distance OPE. The product
of the local operators $J^-_\mu(x)$ and $J^+_\nu(y)$, to be taken at
short-distances due to the convolution with the massive, short-range
W propagator $\Delta^{\mu\nu}(x, y)$ (compare \eqn{dloc}), is expanded
into a series of composite local operators, of which the leading term
is shown in \eqn{jjx}.
\item The dominant contributions in the short-distance expansion come
from the operators of lowest dimension. In our case these are
four-fermion operators of dimension six, whereas operators of higher
dimensions can usually be neglected in weak decays.
\item Note that, as far as the charged current weak interaction is
concerned, no approximation is involved yet in the nonlocal interaction
term in \eqn{snl}, except that we do not consider higher order weak
corrections or processes with external W boson states. Correspondingly,
the OPE series into which the nonlocal interaction is expanded, is
equivalent to the original theory, when considered to all orders in
$1/M^2_W$. In other words, the full series will reproduce the complete
Green functions for the charged current weak interactions of quarks.
The truncation of the operator series then yields a systematic
approximation scheme for low energy processes, neglecting contributions
suppressed by powers of $k^2/M^2_W$. In this way one is able to
construct low energy effective theories for weak decays.
\item In going from the full to the effective theory the W boson is
removed as an explicit, dynamical degree of freedom. This step is
often refered to as ``integrating out'' the W boson, a terminology
which is very obvious in the path integral language discussed above.
Alternatively one could of course use the canonical operator
formalism, where the W field instead of being intergrated out, gets
``contracted out'' through the application of Wick's theorem.
\item The effective local four-fermion interaction terms are a
modern version of the classic Fermi-theory of weak interactions.
\item An intuitive interpretation of the OPE formalism discussed so
far is, that from the point of view of low energy dynamics, the effects
of a short-range exchange force mediated by a heavy boson
approximately corresponds to a point interaction.
\item The presentation we have given illustrates furthermore, that
the approach of evaluating the relevant Green functions (or amplitudes)
directly in order to construct the OPE, as in \eqn{aope}, actually gives
the same result as the more formal technique employing path integrals.
While the latter can give some useful insight into the general aspects
of the method, the former is more convenient for practical calculations
and we will make use of it throughout the discussion to follow.
\item Up to now we have not talked about the strong interactions
among quarks, which have of course to be taken into account. They are
described by QCD and can at short-distances be calculated in
perturbation theory, due to the property of asymptotic freedom of QCD.
The corresponding gluon exchange contributions constitute quantum
corrections to the simplified picture sketched above, which can in this
sense be viewed as a classical approximation. We will describe the
incorporation of QCD corrections and related additional features
they imply for the OPE in the following section.
\end{itemize}

\subsection{OPE and Short Distance QCD Effects}
            \label{sec:basicform:ope}
We will now take up the discussion of QCD quantum corrections at
short-distances to the OPE for weak decays. A crucial point for this
enterprise is the property of asymptotic freedom of QCD. This allows
one to treat the short-distance corrections, that is the contribution
of hard gluons at energies of the order $\ord(M_W)$ down to hadronic
scales $\ge 1\gev$, in perturbation theory. In the following, we will
always restrict ourselves to the leading dimension six operators in
the OPE and omit the negligible contributions of higher dimensional
operators. Staying with our example of $c\to su\bar d$ transitions,
recall that we had for the amplitude without QCD
\begin{equation}\label{amp0}
A_0=-i{G_F\over\sqrt{2}}V^*_{cs}V_{ud}^{}
  (\bar s_ic_i)_{V-A}(\bar u_jd_j)_{V-A}
\end{equation}
where the summation over repeated color indices is understood.  This
result leads directly to the effective hamiltonian of \eqn{hc0} where the
color indices have been suppressed. If we now include QCD effects, the
effective hamiltonian, constructed to reproduce the low energy
approximation of the exact theory, is generalized to

\begin{equation}\label{hq12}
{\cal H}_{eff}={G_F\over\sqrt{2}}V^\ast_{cs}V_{ud}(C_1 Q_1+C_2 Q_2) \end{equation}
where
\begin{equation}\label{q1c} Q_1=(\bar s_ic_j)_{V-A}(\bar u_jd_i)_{V-A}   \end{equation}
\begin{equation}\label{q2c} Q_2=(\bar s_ic_i)_{V-A}(\bar u_jd_j)_{V-A}   \end{equation}
The essential features of this hamiltonian are:
\begin{itemize}
\item In addition to the original operator $Q_2$ (with index 2 for
historical reasons) a new operator $Q_1$ with the same flavor form
but different color structure is generated. This is because a gluon
linking the two color singlet weak current lines can ``mix'' the color
indices due to the following relation for the color charges $T^a_{ij}$
\begin{equation}\label{tata}
T^a_{ik}T^a_{jl}=-{1\over 2N}\delta_{ik}\delta_{jl}+{1\over 2}
\delta_{il}\delta_{jk}   \end{equation}
\item The Wilson coefficients $C_1$ and $C_2$, the coupling constants
for the interaction terms $Q_1$ and $Q_2$, become calculable nontrivial
functions of $\as$, $M_W$ and the renormalization scale $\mu$.
If QCD is neglected they have the trivial form $C_1=0$, $C_2=1$ and
\eqn{hq12} reduces to \eqn{hc0}.
\end{itemize}
In order to obtain the final result for the hamiltonian \eqn{hq12}, we
have to calculate the coefficients $C_{1, 2}$. These are determined
by the requirement that the amplitude $A$ in the full theory be
reproduced by the corresponding amplitude in the effective theory
\eqn{hq12}, thus
\begin{equation}\label{acq}
A=-i{G_F\over\sqrt{2}}V^*_{cs}V_{ud}^{}(C_1\langle Q_1\rangle +
C_2\langle Q_2\rangle)   \end{equation}
If we calculate the amplitude $A$ and, to the same order in $\as$,
the matrix elements of operators $\langle Q_1\rangle$,
$\langle Q_2\rangle$, we can obtain $C_1$ and $C_2$ via \eqn{acq}.
This procedure is called {\it matching} the full theory onto the
effective theory \eqn{hq12}.
\\
Here we use the term ``amplitude'' in the meaning of ``amputated Green
function''. Correspondingly operator matrix elements are -- within this
perturbative context -- amputated Green functions with operator
insertion. In a diagrammatic language these amputated Green functions
are given by Feynman graphs, but without gluonic self energy
corrections in external legs, like e.g. in figs.~\ref{fig:1loopful} and
~\ref{fig:1loopeff} for the full and effective theory, respectively.
In the present example penguin diagrams do not contribute due to the
flavor structure of the $c \to s u \bar d$ transition.

\begin{figure}[htb]
\vspace{0.10in}
\centerline{
\epsfysize=3in
\epsffile{ps/fulldiag.ps}
}
\vspace{0.08in}
\caption[]{One-loop current-current (a)--(c), penguin (d) and box (e)
diagrams in the full theory. For pure QCD corrections as considered in
this section and e.g.\ in \ref{sec:HeffdF1:66} the $\gamma$- and
$Z$-contributions in diagram (d) and the diagram (e) are absent.
Possible left-right or up-down reflected diagrams are not shown.
\label{fig:1loopful}}
\end{figure}

Evaluating the current-current diagrams of fig.~\ref{fig:1loopful}\,(a)--(c),
we find for the full amplitude $A$ to $\ord(\as)$
\begin{equation}\label{amp}
A=-i{G_F\over\sqrt{2}}V^\ast_{cs}V_{ud}\left[\left(1+2C_F \aspi
\ln{\mu^2\over -p^2}\right)S_2+{3\over N}\aspi\ln{M^2_W\over -p^2} S_2-
3\aspi\ln{M^2_W\over -p^2} S_1\right]   \end{equation}
Here we have introduced the spinor amplitudes
\begin{equation}\label{s1c} S_1=(\bar s_ic_j)_{V-A}(\bar u_jd_i)_{V-A}
\end{equation}
\begin{equation}\label{s2c} S_2=(\bar s_ic_i)_{V-A}(\bar u_jd_j)_{V-A}
\end{equation}
which are just the tree level matrix elements of $Q_1$ and $Q_2$.
We have employed the Feynman gauge ($\xi=1$) and taken all external
quark lines massless and carrying the off-shell momentum $p$.
Furthermore we have kept only logarithmic corrections
$\sim\as\cdot\log$ and discarded constant contributions of order
$\ord(\as)$, which corresponds to the leading log approximation.
The necessary renormalization of the quark fields in the $MS$-scheme
is already incorporated into \eqn{amp}. It has removed a $1/\eps$
singularity in the first term of \eqn{amp}, which therefore carries
an explicit $\mu$-dependence.

\begin{figure}[htb]
\vspace{0.10in}
\centerline{
\epsfysize=3in
\epsffile{ps/effdiag.ps}
}
\vspace{0.08in}
\caption[]{One loop current-current (a)--(c) and penguin (d) diagrams
contributing to the LO anomalous dimensions and matching conditions in
the effective theory. The 4-vertex ``$\otimes\,\,\otimes$'' denotes the
insertion of a 4-fermion operator $Q_i$. For pure QCD corrections as
considered in this section and e.g.\ in \ref{sec:HeffdF1:66} the
contributions from $\gamma$ in diagrams (d.1) and (d.2) are absent.
Again, possible left-right or up-down reflected diagrams are not
shown.
\label{fig:1loopeff}}
\end{figure}

Under the same conditions, the unrenormalized current-current matrix elements
of the operators $Q_1$ and $Q_2$ are from fig.~\ref{fig:1loopeff}\,(a)-(c)
found to be
\begin{eqnarray}\label{q10}
\lefteqn{\langle Q_1\rangle^{(0)}=} & & \\
& & \left(1+2C_F \aspi\left({1\over\eps}+\ln{\mu^2\over -p^2}
\right)\right)S_1+{3\over N}\aspi\left({1\over\eps}+\ln{\mu^2\over -p^2}\right)S_1-
3\aspi\left({1\over\eps}+\ln{\mu^2\over -p^2}\right) S_2  \nonumber
\end{eqnarray}
\begin{eqnarray}\label{q20}
\lefteqn{\langle Q_2\rangle^{(0)}=} & & \\
& & \left(1+2C_F \aspi\left({1\over\eps}+\ln{\mu^2\over -p^2}
\right)\right)S_2+{3\over N}\aspi\left({1\over\eps}+\ln{\mu^2\over -p^2}\right)S_2-
3\aspi\left({1\over\eps}+\ln{\mu^2\over -p^2}\right) S_1  \nonumber
\end{eqnarray}
Again, the divergences in the first terms are eliminated through field
renormalization. However, in contrast to the full amplitude, the
resulting expressions are still divergent. Therefore an additional
multiplicative renormalization, refered to as {\em operator renormalization},
is necessary:
\begin{equation}
Q_i^{(0)} = Z_{ij} Q_j
\label{AL}
\end{equation}
Since \eqn{q10} and \eqn{q20} each involve both $S_1$ and
$S_2$, the renormalization constant is in this case a $2\times 2$
matrix $Z$. The relation between the unrenormalized
($\langle Q_i\rangle^{(0)}$) and the renormalized amputated Green
functions ($\langle Q_i\rangle$) is then
\begin{equation}\label{q0zq}
\langle Q_i\rangle^{(0)}=Z^{-2}_q Z_{ij}\langle Q_j\rangle
\end{equation}
From \eqn{q10}, \eqn{q20} and \eqn{zq1a} we read off ($MS$-scheme)
\begin{equation}\label{zll} Z = 1+ \aspi {1\over\eps}
 \left(\begin{array}{cc}  3/N & -3 \\
                          -3 & 3/N
    \end{array}\right)   \end{equation}
It follows that the renormalized matrix elements $\langle Q_i\rangle$
are given by
\begin{equation}\label{q1re}
\langle Q_1\rangle=\left(1+2C_F \aspi\ln{\mu^2\over -p^2}
\right)S_1+{3\over N}\aspi\ln{\mu^2\over -p^2}S_1-
3\aspi\ln{\mu^2\over -p^2} S_2   \end{equation}
\begin{equation}\label{q2re}
\langle Q_2\rangle=\left(1+2C_F \aspi\ln{\mu^2\over -p^2}
\right)S_2+{3\over N}\aspi\ln{\mu^2\over -p^2}S_2-
3\aspi\ln{\mu^2\over -p^2} S_1   \end{equation}
Inserting $\langle Q_i\rangle$ into \eqn{acq} and comparing with \eqn{amp}
we derive
\begin{equation}\label{c12}
C_1=-3\aspi\ln{M^2_W\over\mu^2}   \qquad
C_2=1+{3\over N}\aspi\ln{M^2_W\over\mu^2}   \end{equation}
We would like to digress and add a comment on the renormalization of
the interaction terms in the effective theory. The commonly used
convention is to introduce via \eqn{q0zq} the renormalization constants
$Z_{ij}$, defined to absorb the divergences of the operator matrix
elements. It is however instructive to view this renormalization in
a slightly different, but of course equivalent way, corresponding to
the standard counter\-term method in perturbative renormalization.
Consider, as usual, the hamiltonian of the effective theory as the
starting point with fields and coupling constants as bare quantities,
which are renormalized according to ($q$=$s$, $c$, $u$, $d$)
\begin{equation}\label{q0z2}
q^{(0)}=Z^{1/2}_q q  \end{equation}
\begin{equation}\label{c0zc}  C^{(0)}_i=Z^c_{ij} C_j \end{equation}
Then the hamiltonian \eqn{hq12} is essentially (omitting the factor
${G_F\over\sqrt{2}}V^\ast_{cs}V_{ud}$)
\begin{equation}\label{ctcq}
C^{(0)}_iQ_i(q^{(0)})\equiv Z^2_qZ^c_{ij} C_jQ_i\equiv
C_iQ_i+(Z^2_q Z^c_{ij}-\delta_{ij})C_jQ_i  \end{equation}
that is, it can be written in terms of renormalized couplings and
fields ($C_iQ_i$), plus coun\-ter\-terms. The argument $q^{(0)}$ in the
first term in \eqn{ctcq} indicates that the interaction term $Q_i$ is
composed of bare fields. Calculating the amplitude with the
hamiltonian \eqn{ctcq}, which includes the counterterms, we get the
finite renormalized result
\begin{equation}\label{zqzc}
Z^2_qZ^c_{ij} C_j\langle Q_i\rangle^{(0)}=
C_j\langle Q_j\rangle     \end{equation}
Hence (compare \eqn{q0zq})
\begin{equation}\label{zcz}
Z^c_{ij}=Z^{-1}_{ji}  \end{equation}
In short, it is sometimes useful to keep in mind that one can think
of the ``operator renormalization'', which sounds like a new concept,
in terms of the completely equivalent, but customary, renormalization
of the coupling constants $C_i$, as in any field theory.

Now that we have presented in quite some detail the derivation of
the Wilson coefficients in \eqn{c12}, we shall discuss and interpret
the most important aspects of the short-distance expansion for
weak decays, which can be studied very transparently on the explicit
example we have given.
\begin{itemize}
\item First of all a further remark about the phenomenon of
operator mixing that we encountered in our example. This occurs
because gluonic corrections to the matrix element of the original
operator $Q_2$ are not just proportional to $Q_2$ itself, but involve
the additional structure $Q_1$ (and vice versa). Therefore, besides
a $Q_2$-counterterm, a counterterm $\sim Q_1$ is needed to renormalize
this matrix element -- the operators in question are said to mix
under renormalization. This however is nothing new in principle. It
is just an algebraic generalization of the usual concepts. Indeed,
if we introduce a different operator basis $Q_\pm=(Q_2\pm Q_1)/2$
(with coefficients $C_\pm=C_2\pm C_1$) the renormalization becomes
diagonal and matrix elements of $Q_+$ and $Q_-$ are renormalized
multiplicatively. In this new basis the OPE reads
\begin{equation}\label{apam}  A\equiv A_++A_- =
-i{G_F\over\sqrt{2}}V^\ast_{cs}V_{ud}(C_+\langle Q_+\rangle +
C_-\langle Q_-\rangle)   \end{equation}
where ($S_\pm=(S_2\pm S_1)/2$)
\begin{equation}\label{aspm}
A_\pm=-i{G_F\over\sqrt{2}}V^\ast_{cs}V_{ud}\left[\left(1+2C_F \aspi
\ln{\mu^2\over -p^2}\right)S_\pm+({3\over N}\mp 3)\aspi\ln{M^2_W\over -p^2} S_\pm
\right]   \end{equation}
and
\begin{equation}\label{qmpm}
\langle Q_\pm\rangle=\left(1+2C_F \aspi\ln{\mu^2\over -p^2}
\right)S_\pm+({3\over N}\mp 3)\aspi\ln{\mu^2\over -p^2} S_\pm  \end{equation}
\begin{equation}\label{cpm}
C_\pm=1+({3\over N}\mp 3)\aspi \ln{M^2_W\over\mu^2}   \end{equation}
\item In the calculation of the amplitude $A$ in \eqn{amp} and of the
matrix elements in \eqn{q10} and \eqn{q20} the off-shell momentum $p$ of
the external quark legs represents an infrared regulator. The
logarithmic infrared divergence of the gluon correction diagrams
(figs.~\ref{fig:1loopful}\,(a)--(c) and \ref{fig:1loopeff}\,(a)--(c))
as $p^2\to 0$ is evident from \eqn{amp}, \eqn{q10}
and \eqn{q20}.  A similar observation can be made for the $M_W$
dependence of the full amplitude $A$. We see that \eqn{amp} is
logarithmically divergent in the limit $M_W\to\infty$. This behaviour
is reflected in the ultraviolet divergences (persisting after field
renormalization) of the matrix elements \eqn{q10}, \eqn{q20} in the
effective theory, whose local interaction terms correspond to the weak
interactions in the infinite $M_W$ limit as they are just the leading
contribution of the $1/M_W$ operator product expansion.  This also
implies, that the characteristic logarithmic functional dependence of
the leading $\ord(\as)$ corrections is closely related to the
divergence structure of the effective theory, that is to the
renormalization constants $Z_{ij}$.
\item The most important feature of the OPE is that it provides a
factorization of short-distance (coefficients) and long-distance
(operator matrix elements) contributions. This is clearly exhibited
in our example. The dependence of the amplitude \eqn{amp} on $p^2$,
representing the long-distance structure of $A$, is fully contained
in the matrix elements of the local operators $Q_i$ \eqn{q1re}, \eqn{q2re},
whereas the Wilson coefficients $C_i$ in \eqn{c12} are free from this
dependence. Essentially, this factorization has the form
(see \eqn{aspm} -- \eqn{cpm})
\begin{equation}\label{fact}
(1+\as G \ln{M^2_W\over -p^2})\doteq
(1+\as G \ln{M^2_W\over\mu^2})\cdot
(1+\as G \ln{\mu^2\over -p^2})        \end{equation}
that is, amplitude = coefficient function $\times$ operator matrix
element. Hereby the logarithm on the l.h.s. is split according to
\begin{equation}\label{splt}
\ln{M^2_W\over -p^2}=\ln{M^2_W\over\mu^2}+ \ln{\mu^2\over -p^2}  \end{equation}
Since the logarithmic behaviour results from the integration over
some virtual loop momentum, we may -- roughly speaking -- rewrite
this as
\begin{equation}\label{pmuw}
\int^{M^2_W}_{-p^2}{d k^2\over k^2}=
\int^{M^2_W}_{\mu^2}{d k^2\over k^2} +
\int^{\mu^2}_{-p^2}{d k^2\over k^2}   \end{equation}
which illustrates that the coefficient contains the contributions
from large virtual momenta of the loop correction from scales
$\mu\approx 1\gev$ to $M_W$, whereas the low energy contributions are
separated into the matrix elements.\\
Of course, the latter can not be calculated in perturbation
theory for transitions between physical meson states. The point is,
that we have calculated the OPE for unphysical off-shell quark
external states only to extract the Wilson coefficients, which we
need to construct the effective hamiltonian \eqn{hq12}. For this
purpose the fact that we have considered an unphysical amplitude is
irrelevant since the coefficient functions do not depend on the
external states, but represent the short-distance structure of the
theory. Once we have extracted the coefficients and written down
the effective hamiltonian, the latter can be used -- at least in
principle -- to evaluate the physically interesting decay amplitudes
by means of some nonperturbative approach.
\item In interpreting the role of the scale $\mu$ we may distinguish
two different aspects. From the point of view of the effective
theory $\mu$ is just a renormalization scale, introduced in the
process of renormalizing the effective local interaction terms by
the dimensional method. On the other hand, from the point of view of
the full theory, $\mu$ acts as the scale at which the full
contribution is separated into a low energy and a high energy part,
as is evident from the above discussion. For this reason $\mu$ is
sometimes also called the {\it factorization scale}.
\item In our case the infrared structure of the amplitude is
characterized by the off-shell momentum $p$.
In general one could work with any other arbitrary momentum
configuration, on-shell or off-shell, with or without external
quark mass, with infrared divergences regulated by off-shell
momenta, quark masses, a fictitious gluon mass or by dimensional
regularization. In the case of off-shell momenta the amplitude is
furthermore dependent on the gauge parameter of the gluon field.
All these things belong to the infrared or long-distance structure of
the amplitude. Therefore the dependence on these choices is the same
for the full amplitude and for the operator matrix elements and
drops out in the coefficient functions. To check that this is really
the case for a particular choice is of crucial importance for
practical calculations. On the other hand one may use this freedom and
choose the treatment of external lines according to convenience or
taste. Sometimes it may however seem preferable to keep a slightly more
inconvenient dependence on external masses and/or gluon gauge in
order to have a useful check that this dependence does indeed cancel
out for the Wilson coefficients one is calculating.
\end{itemize}

\subsection{The Renormalization Group}
            \label{sec:basicform:rg}
\subsubsection{Basic Concepts}
               \label{sec:basicform:rg:basic}
So far we have computed the Wilson coefficient functions \eqn{cpm} in ordinary
perturbation theory. This, however, is not sufficient for the problem
at hand. The appropriate scale at which to normalize the hadronic
matrix elements of local operators is a low energy scale -- low
compared to the weak scale $M_W$ -- of a few $GeV$ typically.
In our example of charm decay $\mu=\ord(m_c)$. For such a low scale
$\mu$ the logarithm $\ln(M^2_W/\mu^2)$ multiplying $\as(\mu)$
in the expression \eqn{cpm} becomes large. Although $\as(\mu)$
by itself is a valid expansion parameter down to scales of
$\ord(1\gev)$, say, this is not longer true for the combination
$\as(\mu) \ln(M^2_W/\mu^2)$. In fact, for our example \eqn{cpm}
the first order correction term amounts for $\mu=1\gev$ to
65 -- 130\% although $\as/4\pi\approx 4\%$. The reason for
this breakdown of the naive perturbative expansion lies
ultimately in the appearance of largely disparate scales $M_W$ and
$\mu$ in the problem at hand. \\
This situation can be considerably improved by employing the
method of the renormalization group (RG). The renormalization group
is the group of transformations between different choices of the
renormalization scale $\mu$. The renormalization group equations
describe the change of renormalized quantities, Green functions
and parameters, with $\mu$ in a differential form. As we shall
illustrate below, solving these differential equations allows,
in the leading logarithmic approximation (LLA),
to sum up the terms $(\as\ln(M_W/\mu))^n$ to all orders $n$ ($n=0,
\ldots, \infty$) in perturbation theory. This leads to the RG improved
perturbation theory. Going one step beyond in this modified expansion,
to the next-to-leading logarithmic approximation (NLLA), the summation
is extended to all terms $\as(\as\ln(M_W/\mu))^n$, and so on.
In this context it is useful to consider $\as\ln(M_W/\mu)$ with
a large logarithm $\ln(M_W/\mu)$ as a quantity of order $\ord(1)$
\begin{equation}\label{alog}
\as \ln{M_W\over\mu}=\ord(1)\qquad  \mu\ll M_W   \end{equation}
Therefore the series in powers of $\as\ln(M_W/\mu)$ cannot be
truncated. Summed to all orders it yields again a contribution of
order $\ord(1)$. Correspondingly the next-to-leading logs
$\as(\as\ln(M_W/\mu)^n$ represent an $\ord(\as)$
perturbative correction to the leading term.\\
The renormalization group equation for the Wilson coefficient functions
follows from the fact, that the unrenormalized Wilson coefficients
$\vec C^{(0)}=Z_c \vec C$ ($\vec C^T=(C_1, C_2)$)
are $\mu$-independent. Defining the matrix
of anomalous dimensions $\gamma$ by
\begin{equation}\label{gazz} \gamma=Z^{-1}{d\over d\ln\mu}Z  \end{equation}
and recalling that $Z^T_c=Z^{-1}$, we obtain the renormalization group
equation
\begin{equation}\label{rgc}
{d\over d\ln\mu}\vec C(\mu)=\gamma^T(\as) \vec C(\mu)  \end{equation}
The solution of \eqn{rgc} may formally be written in terms of a
$\mu$-evolution matrix $U$ as
\begin{equation}\label{rgcu}
\vec C(\mu)=U(\mu, M_W) \vec C(M_W)  \end{equation}
From \eqn{zll} and \eqn{gazz} we have to first order in $\as$
\begin{equation}\label{g120} \gamma(\as)=\aspi \gamma^{(0)}=\aspi
 \left(\begin{array}{cc} -6/N & 6 \\
                          6 & -6/N
    \end{array}\right)   \end{equation}
or in the diagonal basis
\begin{equation}\label{gpm0}
\gamma_\pm(\as)=\aspi\gamma^{(0)}_\pm \qquad  \gamma^{(0)}_\pm=
   \pm 6{N\mp1\over N}  \end{equation}
Note that if we neglect QCD loop corrections completely, the
couplings $\vec C$ are independent of $\mu$. The nontrivial
$\mu$-dependence of $\vec C$ expressed in \eqn{rgc} is a genuine
quantum effect. It implies an anomalous scaling behaviour for the
dimensionless coefficients, i.e. one that is different from the
classical theory. For this reason the factor $\gamma$ is called
anomalous (scale) dimension (compare \eqn{rgc} with
${d\over d\ln\mu}\mu^n=n \mu^n$ for an n-dimensional $\mu$-dependent
term $\mu^n$).\\
Using \eqn{rga} the RG equation \eqn{rgc} is easily solved with the
result
\begin{equation}\label{cpmrg}
C_\pm(\mu)=\left[{\as(M_W)\over\as(\mu)}\right]^{\gamma^{(0)}_\pm\over
  2\beta_0} C_\pm(M_W)  \end{equation}
At a scale $\mu_W=M_W$ no large logarithms are present and $C_\pm(M_W)$
can therefore be calculated in ordinary perturbation theory.
From \eqn{cpm} we have to the order needed for the LLA
\begin{equation}\label{cmw1}  C_\pm(M_W)=1  \end{equation}
\eqn{cpmrg} and \eqn{cmw1} give the final result for the coefficients
in the leading log approximation of RG improved perturbation theory.
\\
At this point one should emphasize, that the choice of the high energy
matching scale $\mu_W=M_W$ is of course not unique. The only
requirement is that the choice of $\mu_W$ must not introduce large
logs $\ln(M_W/\mu_W)$ in order not to spoil the applicability of
the usual perturbation theory. Therefore $\mu_W$ should be of
order $\ord(M_W)$. The logarithmic correction in \eqn{cpm} is then
$\ord(\as)$ and is neglected in LLA. Then, still,
$C_\pm(\mu_W)=1$ and
\begin{equation}\label{cpmr2}
C_\pm(\mu)=\left[{\as(\mu_W)\over\as(\mu)}\right]^{\gamma^{(0)}_\pm\over
  2\beta_0} =
\left[{\as(M_W)\over \as(\mu)}\right]^{\gamma^{(0)}_\pm\over
  2\beta_0} (1+\ord(\as)) \end{equation}
A change of $\mu_W$ around the value of $M_W$ causes an ambiguity
of $\ord(\as)$ in the coefficient. This ambiguity represents a
theoretical uncertainty in the determination of $C_\pm(\mu)$. In
order to reduce it, it is necessary to go beyond the leading order.
At NLO the scale ambiguity is then reduced from $\ord(\as)$ to
$\ord(\as^2)$. We will come back to this point below. Presently, we
will set $\mu_W=M_W$, but it is important to keep the related
uncertainty in mind.\\
Taking into account the leading order solution of the RG equation
\eqn{rga} for the coupling, which can be expressed in the form
\begin{equation}\label{alls}
\as(m)={\as(\mu)\over 1+\beta_0{\as(\mu)\over 4\pi}\ln{m^2\over\mu^2}}  \end{equation}
we may rewrite \eqn{cpmrg} as
\begin{equation}\label{clls} C_\pm(\mu)=
\left({1\over 1+\beta_0{\as(\mu)\over 4\pi}\ln{M^2_W\over\mu^2}}\right)^{
   \gamma^{(0)}_\pm\over 2\beta_0}  \end{equation}
\eqn{clls} contains the logarithmic corrections
$\sim \as\ln(M^2_W/\mu^2)$ to all orders in $\as$. This shows very
clearly that the leading log corrections have been summed up to
all orders in perturbation theory by solving the RG equation. In
particular, if we again expand \eqn{clls} in powers of $\as$, keeping
the first term only we recover \eqn{cpm}.
This observation demonstrates, that the RG method allows to obtain
solutions, which go beyond the conventional perturbation theory.

Before concluding this subsection, we would like to introduce still
two generalizations of the approach developed so far, which will
appear in the general discussion below.

\subsubsection{Threshold Effects in LLA}
               \label{sec:basicform:rg:thold}
First we may generalize the renormalization group evolution from
$M_W$ down to $\mu\approx m_c$ to include the threshold effect of
heavy quarks like $b$ or $t$ as follows
\begin{equation}\label{cmub}
\vec C(\mu)=U^{(f=4)}(\mu, \mu_b)U^{(f=5)}(\mu_b, \mu_W)\vec C(\mu_W)\end{equation}
which is valid for the LLA. In our example of the $c\to su\bar d$
transition the top quark gives no contribution at all. Being
heavier (but comparable) in mass than the W, it is simply
removed from the theory along with the W-boson. In a first step the
coefficients at the initial scale $\mu_W\approx M_W$ are evolved down to
$\mu_b\approx m_b$
in an effective theory with five quark flavors ($f=5$). Then,
again in the spirit of the effective field theory technique, for
scales below $\mu_b$ also the bottom quark is removed as an explicit
degree of freedom from the effective theory, yielding a new effective
theory with only four ``active'' quark flavors left. The matching
corrections between both theories can be calculated in ordinary
perturbation theory at the scale $\mu_b$, since due to $\mu_b\approx m_b$
no large logs can occur in this procedure. For the same reason
matching corrections $(\ord(\as))$ can be neglected in LLA and the
coefficients at $\mu_b$, $\vec C(\mu_b)$, simply serve as the initial
values for the RG evolution in the four quark theory down to
$\mu\approx m_c$. In addition, continuity of the running coupling
across the threshold $\mu_b$ is imposed by the requirement
\begin{equation}\label{af45}
\alpha_{s, f=4}(\mu_b, \Lambda^{(4)}) =
\alpha_{s, f=5}(\mu_b, \Lambda^{(5)})
\end{equation}
which defines different QCD scales $\Lambda^{(f)}$ for each effective
theory.\\
Neglecting the b-threshold, as we did before \eqn{rgcu}, one may just
perform the full evolution from $\mu_W$ to $\mu$ in an effective
four flavor theory. It turns out that in some cases the difference
of these two approaches is even negligible.\\
We would like to add a comment on this effective field theory
technique. At the first sight the idea to ``remove by hand''
heavy degrees of freedom may look somewhat artificial. However it
appears quite natural when not viewed from the evolution from high
towards low energies but vice versa (which actually corrsponds to
the historical way). Suppose only the ``light'' quarks $u$, $d$, $s$,
$c$ were known. Then in the attempt to formulate a theory of their
weak interactions one would be lead to a generalized Fermi theory
with (effective) four quark coupling constants to be determined
somehow. Of course, we are in the lucky position to know the
underlying theory in the form of the Standard Model. Therefore we
can actually derive the coupling constants of the low energy effective
theory from ``first principles''. This is exactly what is achieved
technically by going through a series of effective theories, removing
heavy degrees of freedom successively, by means of a step-by-step
procedure.

\subsubsection{Penguin Operators}
               \label{sec:basicform:rg:pop}
A second, but very important issue is the generation of QCD penguin
operators \cite{vainshtein:77}. Consider for example the local
operator $(\bar s_iu_i)_{V-A}(\bar u_jd_j)_{V-A}$, which is directly
induced by W-boson exchange. In this case, additional QCD correction
diagrams, the penguin diagrams (d.1) and (d.2 ) with a gluon in
fig.\ \ref{fig:1loopeff}, contribute and as a consequence six operators
are involved in the mixing under renormalization instead of two. These
read
\begin{equation}\label{q16}
\begin{array}{rcl}
Q_1=(\bar s_iu_j)_{V-A}(\bar u_jd_i)_{V-A} \\
Q_2=(\bar s_iu_i)_{V-A}(\bar u_jd_j)_{V-A} \\
Q_3=(\bar s_id_i)_{V-A}\sum_q(\bar q_jq_j)_{V-A} \\
Q_4=(\bar s_id_j)_{V-A}\sum_q(\bar q_jq_i)_{V-A} \\
Q_5=(\bar s_id_i)_{V-A}\sum_q(\bar q_jq_j)_{V+A} \\
Q_6=(\bar s_id_j)_{V-A}\sum_q(\bar q_jq_i)_{V+A}
\end{array}
\end{equation}
The sum over $q$ runs over all quark flavors that exist in the
effective theory in question. The operators $Q_1$ and $Q_2$ are just
the ones we have encountered in subsection~\ref{sec:basicform:ope}, but with
the c-quark replaced by $u$. This modified flavor structure gives rise to
the gluon penguin type diagrams shown in fig.\ \ref{fig:1loopeff}\,(d).
Since the gluon coupling is of course flavor conserving, it is clear
that penguins cannot be generated from the operator $(\bar
sc)_{V-A}(\bar ud)_{V-A}$. The penguin graphs induce the new local
interaction vertices $Q_3,\ldots, Q_6$, which have the same quantum
numbers. Their structure is easily understood. The flavor content is
determined by the $(\bar sd)_{V-A}$ current in the upper part and by a
$\sum_q(\bar qq)_V$ vector current due to the gluon coupling in the
lower. This vector structure is for convenience decomposed into a
$(V-A)$ and a $(V+A)$ part. For each of these, two different color
forms arise due to the color structure of the exchanged gluon (see
\eqn{tata}). Together this yields the four operators $Q_3,\ldots,
Q_6$.\\
For all operators $Q_1,\ldots, Q_6$ all possible QCD corrections (that
is all amputated Green functions with insertion of $Q_i$) of the
current-current (fig.\ \ref{fig:1loopeff}\,(a)--(c)) as well as of the
penguin type (fig.\ \ref{fig:1loopeff}\,(d.1) and (d.2)) have to be
evaluated.  In this process no new operators are generated, so that
$Q_1,\ldots, Q_6$ form a complete set. They ``close under
renormalization''. In analogy to the case of
subsection~\ref{sec:basicform:ope} the divergent parts of these Green
functions determine, after field renormalization, the operator
renormalization constants, which in the present case form a $6\times 6$
matrix. The calculation of the corresponding anomalous dimension matrix
and the renormalization group analysis then proceeds in the usual way.
We will see that the inclusion of higher order electroweak interactions
requires the introduction of still more operators.

\subsection{Summary of Basic Formalism}
            \label{sec:basicform:summary}
We think that after this rather detailed discussion of the methods
required for the short-distance calculations in weak decays, it is
useful to give at this point a concise summary of the material
covered so far. At the same time this may serve as an outline of the
necessary procedure for practical calculations. Furthermore it will
also provide a starting point for the extension of the formalism
from the LLA considered until now to the NLLA to be presented
in the next subsection.

Ultimately our goal is the evaluation of weak decay amplitudes
involving hadrons in the framework of a low energy effective theory,
of the form
\begin{displaymath}
\langle {\cal H}_{eff}\rangle={G_F\over\sqrt{2}}V_{CKM}
\langle \vec Q^T(\mu)\rangle \vec C(\mu)
\end{displaymath}
The procedure for this calculation can be divided into the
following three steps.

\bigskip
\noindent
{\bf Step 1: Perturbation Theory}
\\
Calculation of Wilson coefficients $\vec C(\mu_W)$ at
$\mu_W\approx M_W$ to the desired order in $\as$. Since
logarithms of the form $\ln(\mu_W/M_W)$ are not large, this can be
performed in ordinary perturbation theory. It amounts to matching
the full theory onto a five quark effective theory.\\

\medskip
\noindent
{\bf Step 2: RG Improved Perturbation Theory}
\\
\begin{itemize}
\item Calculation of the anomalous dimensions of the operators
\item Solution of the renormalization group equation for $\vec{C}(\mu)$
\item Evolution of the coefficients from $\mu_W$ down to the
appropriate low energy scale $\mu$
\begin{displaymath}
\vec C(\mu)=U(\mu, \mu_W)\vec C(\mu_W)
\end{displaymath}
\end{itemize}

\medskip
\noindent
{\bf Step 3: Non-Perturbative Regime}
\\
Calculation of hadronic matrix elements $\langle\vec Q(\mu)\rangle$,
normalized at the appropriate low energy scale $\mu$, by means of
some non-perturbative method.

\bigskip
\noindent
Important issues in this procedure are:
\begin{itemize}
\item The OPE achieves a {\it factorization\/} of short- and long
distance contributions.
\begin{itemize}
\item
Correspondingly, in order to disentangle the short-distance from the
long-distance part and to extract $\vec C(\mu_W)$ in actual
calculations, a proper {\it matching\/} of the full onto the
{\it effective theory\/} has to be performed.
\item Similar comments apply to the matching of an effective theory
with $f$ quark flavors to a theory with $(f-1)$ flavors during the
RG evolution to lower scales.
\item Furthermore, factorization implies, that the $\mu$-dependence
and also the dependence on the renormalization scheme, which
appears beyond the leading order, cancel between $C_i$ and
$\langle Q_i\rangle$.
\item Since the top quark is integrated out along with the W, the
coefficients $\vec C(\mu_W)$ in general contain also the full dependence
on the top quark mass $m_t$.
\end{itemize}
\item A {\it summation of large logs\/} by means of the RG method is
necessary. More specifically, in the $n$-th order of
renormalization group improved perturbation theory the terms of the
form
\begin{displaymath}
\as^n(\mu)\left(\as(\mu)\ln{M_W\over\mu}\right)^k
\end{displaymath}
are summed to all orders in $k$ ($k$=0, 1, 2,$\ldots$). This approach
is justified as long as $\as(\mu)$ is small enough, which
requires that $\mu$ not be too low, typically not less than $1\gev$.
\end{itemize}

\subsection{Wilson Coefficients Beyond Leading Order}
            \label{sec:basicform:wc}
\subsubsection{The RG Formalism}
               \label{sec:basicform:wc:rgf}
We are now going to extend the renormalization group formalism for
the coefficient functions to the next-to-leading order level.
Subsequently we will discuss important aspects of the resulting
formulae, in particular the scale- and scheme dependences and their
cancellation.\\
To have something specific in mind, we may consider the calculation
for the $\dS$ effective hamiltonian for nonleptonic decays, which
without QCD effects and for low energy is given by
\begin{equation}\label{hds1}
{\cal H}^{\dS}_{eff}={G_F\over\sqrt{2}}V^\ast_{us}V_{ud}
  (\bar su)_{V-A}(\bar ud)_{V-A}   \end{equation}
At higher energies of course also the charm, bottom and top quark
have to be taken into account. The Feynman diagrams contributing to
$\ord(\as)$ corrections to this hamiltonian are shown in
fig.~\ref{fig:1loopful} and \ref{fig:1loopeff}.
Including current-current- as well as penguin type corrections, the
relevant operator basis consists of the six operators in \eqn{q16}.
\\
On the one hand, this particular case is very important by itself since
it provides the theoretical basis for a large variety of different
decay modes. On the other hand we will at this stage keep the
discussion fairly general, so that all important features of a general
validity are exhibited. In addition, the central formulae of this
subsection will be used at several places later on, if at times
extended or modified to match the specific cases in question. In
part two of this report we will give a more detailed discussion of
the hamiltonians relevant for various decays. Here, we would rather like
to concentrate on the presentation of the OPE and renormalization group
formalism.
\\
The effective hamiltonian for nonleptonic decays may be written in
general as
\begin{equation}\label{hqtc}
{\cal H}_{eff}={G_F\over\sqrt{2}}\sum_i C_i(\mu)Q_i(\mu)\equiv
  {G_F\over\sqrt{2}}\vec Q^T(\mu) \vec C(\mu)   \end{equation}
where the index $i$ runs over all contributing operators, in our
example $Q_1,\ldots, Q_6$ of \eqn{q16}. It is straightforward to
apply ${\cal H}_{eff}$ to D- and B-meson decays as well by changing
the quark flavors appropriately. For the time being we omit CKM
parameters, which can be reinserted later on. $\mu$ is some low
energy scale of the order $\ord(1\gev)$, $\ord(m_c)$ and $\ord(m_b)$
for K-, D- and B-meson decays, respectively. The argument $\mu$ of the
operators $Q_i(\mu)$ means, that their matrix elements are to be
normalized at scale $\mu$.\\
The Wilson coefficient functions are given by
\begin{equation}\label{cucw}
\vec C(\mu)=U(\mu, \mu_W)\vec C(\mu_W)   \end{equation}
The coefficients at the scale $\mu_W=\ord(M_W)$ can be evaluated in
perturbation theory. The evolution matrix $U$ then includes the
renormalization group improved perturbative contributions from the
scale $\mu_W$ down to $\mu$.\\
In the first step we determine $\vec C(\mu_W)$ from a comparison of
the amputated Green function with appropriate external lines in the
full theory with the corresponding amplitude in the effective theory.
At NLO we have to calculate to $\ord(\as)$, including
non-logarithmic, constant terms. The full amplitude results from
the current-current- and penguin type diagrams in fig.~\ref{fig:1loopful},
is finite after field renormalization and can be written as
\begin{equation}\label{aa01} A=
{G_F\over\sqrt{2}}\vec S^T(\vec A^{(0)}+{\as(\mu_W)\over 4\pi}\vec A^{(1)})\end{equation}
Here $\vec S$ denotes the tree level matrix elements of the
operators $\vec Q$.
In the effective theory \eqn{hqtc} the current-current- and penguin
corrections of fig.~\ref{fig:1loopeff} have to be calculated for all
the operators $Q_i$. In this case, besides the field renormalization, a
renormalization of operators is necessary
\begin{equation}\label{q0z3}
Z^2_q\langle\vec Q\rangle^{(0)}=Z\langle\vec Q\rangle  \end{equation}
where the matrix Z absorbes those divergences of the Green functions
with operator $\vec Q$ insertion, that are not removed by the field
renormalization. The renormalized matrix elements of the operators
can then to $\ord(\as)$ be written as
\begin{equation}\label{qars}
\langle\vec Q(\mu_W)\rangle=(1+{\as(\mu_W)\over 4\pi} r)\vec S  \end{equation}
and the amplitude in the effective theory to the same order becomes
\begin{equation}\label{aeff}  A_{eff}=
{G_F\over\sqrt{2}}\vec S^T(1+{\as(\mu_W)\over 4\pi} r^T) \vec C(\mu_W)\end{equation}
Equating \eqn{aa01} and \eqn{aeff} we obtain
\begin{equation}\label{cmuw} \vec C(\mu_W)=
\vec A^{(0)}+{\as(\mu_W)\over 4\pi}(\vec A^{(1)}-r^T\vec A^{(0)})  \end{equation}
In general $\vec A^{(1)}$ in \eqn{aa01} involves logarithms
$\ln(M^2_W/-p^2)$ where p denotes some global external momentum for the
amplitudes in fig.~\ref{fig:1loopful}.  On the other hand, the matrix
$r$ in \eqn{qars}, characterizing the radiative corrections to
$\langle\vec Q(\mu_W)\rangle$, includes $\ln(-p^2/\mu^2_W)$. As we have
seen in subsection \ref{sec:basicform:ope}, these logarithms combine to
$\ln(M^2_W/\mu^2_W)$ in the Wilson coefficient \eqn{cmuw}. For
$\mu_W=M_W$ this logarithm vanishes altogether. For $\mu_W=\ord(M_W)$
the expression $\ln(M^2_W/\mu^2_W)$ is a ``small logarithm'' and the
correction $\sim\as \ln(M^2_W/\mu^2_W)$, which could be neglected in
LLA, has to be kept in the perturbative calculation at NLO together
with constant pieces of order $\ord(\as)$.
\\
In the second step, the renormalization group equation for $\vec C$
\begin{equation}\label{rgcv}
{d\over d\ln\mu}\vec C(\mu)=\gamma^T(g)\vec C(\mu)   \end{equation}
has to be solved with boundary condition \eqn{cmuw}. The solution is
written with the help of the U-matrix as in \eqn{cucw}, where
$U(\mu, \mu_W)$ obeys the same equation as $\vec C(\mu)$ in \eqn{rgcv}.
The general solution is easily written down iteratively
\begin{equation}\label{uit}
U(\mu, m)=1+\int^{g(\mu)}_{g(m)}dg_1{\gamma^T(g_1)\over\beta(g_1)}+
\int^{g(\mu)}_{g(m)}dg_1\int^{g_1}_{g(m)}dg_2
{\gamma^T(g_1)\over\beta(g_1)}{\gamma^T(g_2)\over\beta(g_2)}+\ldots \end{equation}
which using $dg/d\ln\mu=\beta(g)$ is readily seen to solve the
renormalization group equation
\begin{equation}\label{rgu}
{d\over d\ln\mu}U(\mu, m)=\gamma^T(g)U(\mu, m)   \end{equation}
The series in \eqn{uit} can be more compactly expressed by introducing
the notion of g-ordering
\begin{equation}\label{utge}
U(\mu, m)=T_g \exp\int^{g(\mu)}_{g(m)}dg'{\gamma^T(g')\over\beta(g')}\end{equation}
where in the case $g(\mu)>g(m)$ the g-ordering operator $T_g$ is
defined through
\begin{equation}\label{tgdf}
T_g f(g_1)\ldots f(g_n)=\sum_{perm}
\Theta(g_{i_1}-g_{i_2})\Theta(g_{i_2}-g_{i_3})\ldots
\Theta(g_{i_{n-1}}-g_{i_n})f(g_{i_1})\ldots f(g_{i_n}) \end{equation}
and brings about an ordering of the factors $f(g_i)$ such that the
coupling constants increase from right to left. The sum in \eqn{tgdf} runs
over all permutations $\{i_1,\ldots, i_n\}$ of $\{1, 2,\ldots, n\}$.
The $T_g$ ordering is necessary since in general the anomalous
dimension matrices at different couplings do not commute beyond the
leading order, $[\gamma(g_1), \gamma(g_2)]\not= 0$.\\
At next-to-leading order we have to keep the first two terms in the
perturbative expansions for $\beta(g)$ (see \eqn{bg01}) and $\gamma(g)$
\begin{equation}\label{gg01}
\gamma(\as)=\gamma^{(0)}\aspi + \gamma^{(1)}\left(\aspi\right)^2
\end{equation}
To this order the evolution matrix $U(\mu, m)$ is given by
\cite{burasetal:92a}
\begin{equation}\label{u0jj}
U(\mu,m)=
(1+{\as(\mu)\over 4\pi} J) U^{(0)}(\mu,m) (1-{\as(m)\over 4\pi} J)
\end{equation}
$U^{(0)}$ is the evolution matrix in leading logarithmic approximation
and the matrix $J$ expresses the next-to-leading corrections to this
evolution. We have
\begin{equation}\label{u0vd} U^{(0)}(\mu,m)= V
\left({\left[{\as(m)\over\as(\mu)}\right]}^{{\vec\gamma^{(0)}\over 2\beta_0}}
   \right)_D V^{-1}   \end{equation}
where $V$ diagonalizes ${\gamma^{(0)T}}$
\begin{equation}\label{ga0d} \gamma^{(0)}_D=V^{-1} {\gamma^{(0)T}} V  \end{equation}
and $\vec\gamma^{(0)}$ is the vector containing the diagonal elements of
the diagonal matrix $\gamma^{(0)}_D$.\\
If we define
\begin{equation}\label{gvg1} G=V^{-1} {\gamma^{(1)T}} V   \end{equation}
and a matrix $H$ whose elements are
\begin{equation}\label{sij} H_{ij}=\delta_{ij}\gamma^{(0)}_i{\beta_1\over 2\beta^2_0}-
    {G_{ij}\over 2\beta_0+\gamma^{(0)}_i-\gamma^{(0)}_j}  \end{equation}
the matrix $J$ is given by
\begin{equation}\label{jvs} J=V H V^{-1}   \end{equation}
The fact that \eqn{u0jj} is indeed a solution of the RG equation \eqn{rgu}
to the order considered is straightforwardly verified by differentiation
with respect to $\ln\mu$. Combining now the initial values \eqn{cmuw}
with the evolution matrix \eqn{u0jj} we obtain
\begin{equation}\label{cjua} \vec C(\mu)=(1+{\as(\mu)\over 4\pi} J)U^{(0)}(\mu,\mu_W)
(\vec A^{(0)}+{\as(\mu_W)\over 4\pi}[\vec A^{(1)}-(r^T+J)\vec A^{(0)}])\end{equation}
Using \eqn{cjua} we can calculate for example the coefficients at a scale
$\mu=\mu_b=\ord(m_b)$, working in an effective five flavor theory, $f=5$.
If we have to evolve the coefficients to still lower values, we would
like to formulate a new effective theory for $\mu<\mu_b$ where now
also the b-quark is removed as an explicit degree of freedom. To
calculate the coefficients in this new four flavor theory at the
scale $\mu_b$, we have to determine the matching corrections at
this scale.\\
We follow the same principles as in the case of integrating out the
W-boson and require
\begin{equation}\label{qfcf}
\langle\vec Q_f(m)\rangle^T\vec C_f(m)=
\langle\vec Q_{f-1}(m)\rangle^T\vec C_{f-1}(m)  \end{equation}
in the general case of a change from an f-flavor to a (f--1)-flavor
theory at a scale $m$. The ``full amplitude'' on the l.h.s., which is
now in an f-flavor effective theory, is expanded into matrix elements
of the new (f--1)-flavor theory, multiplied by new Wilson coefficients
$\vec C_{f-1}$. From \eqn{qars}, determining the matrix elements of
operators to $\ord(\as)$, one finds
\begin{equation}\label{qfdr}
\langle\vec Q_f(m)\rangle=(1+{\as(m)\over 4\pi} \delta r)
\langle\vec Q_{f-1}(m)\rangle    \end{equation}
where
\begin{equation}\label{drrf} \delta r=r^{(f)}-r^{(f-1)}
\end{equation}
In \eqn{drrf} we have made explicit the dependence of the matrix $r$
on the number of quark flavors which enters in our example via the
penguin contributions. From \eqn{qfcf} and \eqn{qfdr} we find
\begin{equation}\label{cmcf}
\vec C_{f-1}(m)=M(m)\vec C_f(m)  \end{equation}
with
\begin{equation}\label{mdrt}  M(m)=1+{\as(m)\over 4\pi} \delta r^T  \end{equation}
The general renormalization group matrix $U$ in \eqn{u0jj}, now
evaluated for (f--1) flavors, can be used to evolve $\vec C_{f-1}(m)$
to lower values of the renormalization scale. It is clear that no
large logarithms can appear in \eqn{mdrt} and that therefore the
matching corrections, expressed in the matrix $M(m)$ can be
computed in usual perturbation theory. We note that this type of
matching corrections enters in a nontrivial way for the first time
at the NLO level. In the LLA $M\equiv 1$ and one can simply omit the
b-flavor components in the penguin operators when crossing the
b-threshold.\\
We also remark that the correction matrix $M$ introduces a small
discontinuity of the coefficients, regarded as functions of $\mu$,
at the matching scale $m$. This is however not surprising. In any
case the $\vec C(\mu)$ are not physical quantities and their
discontinuity precisely cancels the effect of removing the heavy
quark flavor from the operators, which evidently is a
``discontinuous'' step. Hence, physical amplitudes are not
affected and indeed the behaviour of $\vec C$ at the matching scale
ensures that the same physical result will be obtained,
whether we choose to calculate in the f-flavor- or in the (f--1)-flavor
theory for scales around the matching scale $m$.\\
To conclude we will write down how the typical final result for
the coefficient functions at $\mu\approx 1\gev$, appropriate for
K-decays, looks like, if we combine all the contributions discussed
above. Then we can write
\begin{equation}\label{cthr}
\vec C(\mu)=U_3(\mu,\mu_c)M(\mu_c)U_4(\mu_c,\mu_b)M(\mu_b)
U_5(\mu_b,\mu_W)\vec C(\mu_W)  \end{equation}
where $U_f$ is the evolution matrix for $f$ active flavors.
In the following discussion we will not always include the flavor
thresholds when writing the expression for the RG evolution. It is
clear, that they can be added in a straightforward fashion.

\subsubsection{The Calculation of the Anomalous Dimensions}
               \label{sec:basicform:wc:adm}
The matrix of anomalous dimensions is the most important ingredient
for the renormalization group calculation of the Wilson coefficient
functions. In the following we will summarize the essential steps of
its calculation.\\
Recall that the evaluation of the amputated Green functions with
insertion of the operators $\vec Q$ gives the relation
\begin{equation}\label{qzgf}
\langle\vec Q\rangle^{(0)}=Z^{-2}_q Z\langle\vec Q\rangle
\equiv Z_{GF}\langle\vec Q\rangle   \end{equation}
$\langle\vec Q\rangle^{(0)}$, $\langle\vec Q\rangle$ denote the
unrenormalized and renormalized Green functions, respectively.
$Z_q$ is the quark field renormalization constant and $Z$ is the
renormalization constant matrix of the operators $\vec Q$.\\
The anomalous dimensions are given by
\begin{equation}\label{gaz2}
\gamma(g)=Z^{-1}{d\over d\ln\mu}Z   \end{equation}
In the $MS$ (or $\overline{MS}$) scheme the renormalization constants
are chosen to absorb the pure pole divergences $1/\eps^k$
($D=4-2\eps$), but no finite parts. $Z$ can then be expanded in
inverse powers of $\eps$ as follows
\begin{equation}\label{zeps}
Z=1+\sum^\infty_{k=1}{1\over\eps^k}Z_k(g)  \end{equation}
Using the expression for the $\beta$-function \eqn{bete}, valid for
arbitrary $\eps$ we derive the useful formula \cite{floratosetal:77}
\begin{equation}\label{ggz1}
\gamma(g)=-2g^2{\partial Z_1(g)\over\partial g^2}
=-2\as{\partial Z_1(\as)\over\partial \as}  \end{equation}
Similarly to \eqn{zeps} we expand
\begin{equation}\label{zqep}
Z_q=1+\sum^\infty_{k=1}{1\over\eps^k}Z_{q,k}(g)  \end{equation}
\begin{equation}\label{zgfe}
Z_{GF}=1+\sum^\infty_{k=1}{1\over\eps^k}Z_{GF,k}(g)  \end{equation}
From the calculation of the unrenormalized Green functions \eqn{qzgf}
we immediately obtain $Z_{GF}$. What we need to compute $\gamma(g)$
is $Z_1(g)$ \eqn{ggz1}. From \eqn{qzgf}, \eqn{zeps}, \eqn{zqep}, \eqn{zgfe}
we find
\begin{equation}\label{zqgf}
Z_1=2Z_{q,1}+Z_{GF,1}  \end{equation}
At next-to-leading order we have from the $1/\eps$ poles of the
unrenormalized Green functions
\begin{equation}\label{zgf1}
Z_{GF,1}=b_1 \aspi + b_2 \left(\aspi\right)^2  \end{equation}
The corresponding expression for the well known factor $Z_{q,1}$
has been quoted in \eqn{zq1a}. Using \eqn{zq1a}, \eqn{ggz1},
\eqn{zqgf}, \eqn{zgf1} we finally obtain for the one- and two-loop
anomalous dimension matrices $\gamma^{(0)}$ and $\gamma^{(1)}$ in
\eqn{gg01}
\begin{equation}\label{a1b1}
\gamma^{(0)}_{ij}=-2[2a_1 \delta_{ij}+(b_1)_{ij}]  \end{equation}
\begin{equation}\label{a2b2}
\gamma^{(1)}_{ij}=-4[2a_2 \delta_{ij}+(b_2)_{ij}]  \end{equation}
\eqn{a1b1} and \eqn{a2b2} may be used as recipes to immediately extract
the anomalous dimensions from the divergent parts of the
unrenormalized Green functions.

\subsubsection{Renormalization Scheme Dependence}
               \label{sec:basicform:wc:rgdep}
A further issue, which becomes important at next-to-leading order is
the dependence of unphysical quantities, like the Wilson coefficients
and the anomalous dimensions, on the choice of the renormalization
scheme. This scheme dependence arises because the renormalization
prescription involves an arbitrariness in the finite parts to be
subtracted along with the ultraviolet singularities.  Two different
schemes are then related by a finite renormalization.  Considering the
quantities, which we encountered in
subsection~\ref{sec:basicform:wc:rgf}, the following of them are
independent of the renormalization scheme
\begin{equation}\label{rsi}
\beta_0,\quad\beta_1,\quad\gamma^{(0)},\quad\vec A^{(0)},\quad
\vec A^{(1)},\quad r^T+J,\quad\langle\vec Q\rangle^T\vec C  \end{equation}
whereas
\begin{equation}\label{rsd}
r,\quad\gamma^{(1)},\quad J,\quad\vec C,\quad\langle\vec Q\rangle \end{equation}
are scheme dependent.\\
In the framework of dimensional regularization one example of how
such a scheme dependence can occur is the treatment of $\gf$ in $D$
dimensions.
Possible choices are the ``naive dimensional regularization'' (NDR)
scheme with $\gf$ taken to be
anticommuting or the 't-Hooft--Veltman (HV) scheme
\cite{thooft:72}, \cite{breitenlohner:77} with
non-anticommuting $\gf$. Another example is the use of operators in
a color singlet or a non-singlet form, such as
\begin{equation}\label{qtwi}
Q_2=(\bar s_iu_i)_{V-A}(\bar u_jd_j)_{V-A}
\quad \hbox{\rm or}\quad
\tilde Q_2=(\bar s_id_j)_{V-A}(\bar u_ju_i)_{V-A} \end{equation}
where $i$, $j$ are color indices. In $D=4$ dimensions these operators
are equivalent since they are related by a Fierz transformation. In
the NDR scheme however these two choices yield different results
for $r$, $\gamma^{(1)}$ and $J$ and thus constitute two different
schemes, related by a nontrivial finite renormalization. On the
other hand, both choices give the same $r$, $\gamma^{(1)}$ and $J$
if the HV scheme is employed.\\
Let us now discuss the question of renormalization scheme dependences
in explicit terms in order to obtain an overview on how the
scheme dependences arise, how various quantities transform under a
change of the renormalization scheme and how the cancellation of
scheme dependences is guaranteed for physically relevant
quantities.\\
First of all, it is clear that the product
\begin{equation}\label{qtc}
\langle\vec Q(\mu)\rangle^T\vec C(\mu)  \end{equation}
representing the full amplitude, is independent of the
renormalization scheme chosen. This is simply due to the fact, that
it is precisely the factorization of the amplitude into Wilson coefficients
and matrix elements of operators by means of the operator product
expansion, which introduces the scheme dependence of $\vec C$ and
$\langle\vec Q\rangle$. In other words, the scheme dependence of
$\vec C$ and $\langle\vec Q\rangle$ represents the arbitrariness one
has in splitting the full amplitude into coefficients and matrix
elements and the scheme independence of the combined product \eqn{qtc}
is manifest in the construction of the operator product expansion.
\\
More explicitly, these quantities are in different schemes
(primed and unprimed) related by
\begin{equation}\label{rsqc}\langle\vec Q\rangle'=(1+\aspi s)\langle\vec Q\rangle\qquad
 \vec C'=(1-\aspi s^T)\vec C  \end{equation}
 where $s$ is a constant matrix. \eqn{rsqc} represents a finite
renormalization of $\vec C$ and $\langle\vec Q\rangle$.
From \eqn{qars} we immediately obtain
\begin{equation}\label{rprs}  r'=r+s  \end{equation}
Furthermore from
\begin{equation}\label{qtuc} \langle\vec Q(\mu)\rangle^T\vec C(\mu)\equiv
  \langle\vec Q(\mu)\rangle^T U(\mu,M_W) \vec C(M_W)  \end{equation}
we have
\begin{equation}\label{upus} U'(\mu,M_W)=
(1-{\as(\mu)\over 4\pi}s^T)U(\mu,M_W)(1+{\as(M_W)\over 4\pi}s^T)  \end{equation}
A comparison with \eqn{u0jj} yields
\begin{equation}\label{jpjs} J'=J-s^T \end{equation}
The renormalization constant matrix in the primed scheme, $Z'$,
follows from \eqn{rsqc} and \eqn{qzgf}
\begin{equation}\label{zpzs}  Z'=Z (1- \aspi s)  \end{equation}
Recalling the definition of the matrix of anomalous dimensions,
\eqn{gaz2} and \eqn{gg01}, we derive
\begin{equation}\label{gpgs}\gamma^{(0)\prime}=\gamma^{(0)} \qquad
 \gamma^{(1)\prime}=\gamma^{(1)}+[s,\gamma^{(0)}]+2\beta_0 s \end{equation}
With these general formulae at hand it is straightforward to
clarify the cancellation of scheme dependences in all particular
cases. Alternatively, they may be used to transform scheme
dependent quantities from one scheme to another, if desired, or to
check the compatibility of results obtained in different schemes.
\\
In particular we immediately verify from \eqn{rprs} and \eqn{jpjs} the
scheme independence of the matrix $r^T+J$. This means that in the
expression for $\vec C$ in \eqn{cjua} the factor on the right hand
side of $U^{(0)}$, related to the ``upper end'' of the evolution,
is independent of the renormalization scheme, as it must be. The
same is true for $U^{(0)}$. On the other hand $\vec C$ still
depends on the renormalization scheme through the matrix $J$ to the
left of $U^{(0)}$. As is evident from \eqn{rsqc},
this dependence is compensated for by the
corresponding scheme dependence of the matrix elements of operators
so that a physically meaningful result for the decay amplitudes
is obtained.
To ensure a proper cancellation of the scheme dependence the matrix
elements have to be evaluated in the same scheme
(renormalization, $\gf$, form of operators) as the coefficient
functions, which is a nontrivial task for the necessary
non-perturbative computations. In other words, beyond the leading
order the matching between short- and long-distance contributions has
to be performed properly not only with respect to the scale $\mu$,
but also with respect to the renormalization scheme employed.

\subsubsection{Discussion}
               \label{sec:basicform:wc:disc}
We will now specialize the presentation of the general formalism to the
case of a single operator (that is without mixing).  This situation is
e.g.\ relevant for the operators $Q_+$ and $Q_-$ with four different quark
flavors, which we encountered in section~\ref{sec:basicform:ope}. The
resulting simplifications are useful in order to display some more
details of the structure of the calculation and to discuss the most
salient features of the NLO analysis in a transparent way.
\\
In the case where only one single operator contributes, the
amplitude in the full theory (dynamical W-boson) may be written as
(see \eqn{aa01})
\begin{equation}\label{amp2}
A={G_F\over\sqrt{2}}(1+{\as(\mu_W)\over 4\pi}\left[-{\gamma^{(0)}\over 2}
\ln{M^2_W\over -p^2}+\tilde A^{(1)}\right]) S  \end{equation}
where we have made the logarithmic dependence on the W mass explicit.
In the effective theory the amplitude reads
\begin{eqnarray}\label{aef2}
A_{eff}&=&{G_F\over\sqrt{2}}C(\mu_W)\langle Q(\mu_W)\rangle \\
&=&{G_F\over\sqrt{2}}C(\mu_W)(1+{\as(\mu_W)\over 4\pi}\left[{\gamma^{(0)}\over 2}
 \left(\ln{-p^2\over\mu^2_W}+\gamma_E-\ln 4\pi\right)+\tilde r
 \right])S   \nonumber
\end{eqnarray}
The divergent pole term $1/\eps$ has been subtracted minimally. A
comparison of \eqn{amp2} and \eqn{aef2} yields the Wilson coefficient
\begin{equation}\label{cmw2}
 C(\mu_W)=1+{\as(\mu_W)\over 4\pi}\left[-{\gamma^{(0)}\over 2}
 \left(\ln{M^2_W\over\mu^2_W}+\gamma_E-\ln 4\pi\right)+B\right] \end{equation}
where
\begin{equation}\label{ba1r} B=\tilde A^{(1)}-\tilde r  \end{equation}
In the leading log approximation we had simply $C(\mu_W)=1$. By
contrast at NLO the $\ord(\as)$ correction has to be taken into
account in addition. This correction term exhibits the following
new features:
\begin{itemize}
\item The expression $\gamma_E-\ln 4\pi$, which is characteristic to
dimensional regularization appears. It is proportional to $\gamma^{(0)}$.
\item A constant term $B$ is present. $B$ depends on the
factorization scheme chosen.
\item An explicit logarithmic dependence on the matching scale
$\mu_W$ shows up.
\end{itemize}
We discuss these points one by one.\\
First, the term $\gamma_E-\ln 4\pi$ is characteristic for the $MS$
scheme. It can be eliminated by going from the $MS$- to the
$\overline{MS}$ scheme. This issue is well known in the literature.  We
find it however useful to briefly repeat the definition of the
$\overline{MS}$ scheme in the present context, since this is an
important point for NLO analyses.
\\
Consider the RG solution for the coefficient
\begin{eqnarray}\label{cja2}
 \lefteqn{C(\mu)=} & & \\
 & &(1+{\as(\mu)\over 4\pi}J)\left[{\as(\mu_W)\over \as(\mu)}\right]^{
  \gamma^{(0)}\over 2\beta_0}
 (1+{\as(\mu_W)\over 4\pi}\left[-{\gamma^{(0)}\over 2}
 \left(\ln{M^2_W\over\mu^2_W}+\gamma_E-\ln 4\pi\right)+B-J\right])
 \nonumber
\end{eqnarray}
This represents the solution for the $MS$ scheme. Therefore in
\eqn{cja2} $\as=\alpha_{s,MS}$. The redefinition of $\alpha_{s,MS}$
through
\begin{equation}\label{amsb}
\alpha_{s,MS}=\alpha_{s,\overline{MS}}\left(1+\beta_0(\gamma_E-\ln 4\pi)
 {\alpha_{s,\overline{MS}}\over 4\pi}\right)  \end{equation}
is a finite renormalization of the coupling,
which defines the $\overline{MS}$ scheme. Since
\begin{equation}\label{msbr}
[\alpha_{s,MS}(\mu_W)]^{\gamma^{(0)}\over 2\beta_0}\doteq
[\alpha_{s,\overline{MS}}(\mu_W)]^{\gamma^{(0)}\over 2\beta_0}
\left(1+{\gamma^{(0)}\over 2}(\gamma_E-\ln 4\pi)
{\alpha_{s,\overline{MS}}(\mu_W)\over 4\pi}\right)   \end{equation}
we see, that this transformation eliminates, to the order considered,
the $\gamma_E-\ln 4\pi$ term in \eqn{cja2}. At the lower end of the
evolution the same redefinition yields a factor
\begin{equation}\label{msbf}
1-{\gamma^{(0)}\over 2}(\gamma_E-\ln 4\pi)
{\alpha_{s,\overline{MS}}(\mu)\over 4\pi}   \end{equation}
which removes the corresponding factor from the matrix element
(see \eqn{aef2})
\begin{equation}\label{msbq}
\langle Q(\mu)\rangle_{MS}\equiv
\left(1+{\gamma^{(0)}\over 2}(\gamma_E-\ln 4\pi)
{\alpha_{s,\overline{MS}}(\mu)\over 4\pi}\right)
\langle Q(\mu)\rangle_{\overline{MS}}   \end{equation}
At the next-to-leading log level we are working, the transformation
\eqn{amsb} is equivalent to a redefinition of the scale $\Lambda$
according to
\begin{equation}\label{msbl}
\Lambda^2_{\overline{MS}}=4\pi e^{-\gamma_E}\Lambda^2_{MS}  \end{equation}
as one can verify with the help of \eqn{amu}. In practice, one can
just drop the ($\gamma_E-\ln 4\pi$) terms in \eqn{cja2}. Then $\as(\mu)$
and $\Lambda$ are to be taken in the $\overline{MS}$ scheme.
Throughout the present report it is always understood that the
transformation to $\overline{MS}$ has been performed. Then
\begin{equation}\label{msbc}
 C(\mu)=(1+{\as(\mu)\over 4\pi}J)\left[{\as(\mu_W)\over\as(\mu)}\right]^{
  \gamma^{(0)}\over 2\beta_0}
 (1+{\as(\mu_W)\over 4\pi}\left[-{\gamma^{(0)}\over 2}
 \ln{M^2_W\over\mu^2_W}+B-J\right]) \end{equation}
Second, from the issue of the $MS$ -- $\overline{MS}$ transformation,
or more generally an arbitrary redefinition of $\as$ (or $\Lambda$) one
should distinguish the renormalization scheme dependence due to the
ambiguity in the renormalization of the operator.  It suggests itself
to use for the latter the term ``factorization scheme dependence''.
This is the scheme dependence we have discussed in
section~\ref{sec:basicform:wc:rgdep}. A change in the factorization
scheme transforms $\gamma^{(1)}$, $B$ and $J$ as
\begin{equation}\label{gbjs}
{\gamma^{(1)}}'=\gamma^{(1)} + 2\beta_0 s
\qquad B'=B - s
\qquad J'=J - s
\end{equation}
where $s$ is a constant number. This follows from the formulae in
section~\ref{sec:basicform:wc:rgdep} and from the definition of $B$ in
\eqn{ba1r}. Note that in the case of a single operator the relation
between $\gamma^{(1)}$ and $J$ simplifies to
\begin{equation}\label{jg01}
J={1\over 2\beta_0}\left({\beta_1\over\beta_0}\gamma^{(0)}
  -\gamma^{(1)}\right)   \end{equation}
Obviously the scheme dependence cancels in the difference
$B-J$ in \eqn{msbc}.\\
Third, due to the explicit $\mu_W$ dependence in the $\ord(\as)$
correction term the coefficient function is, to the order considered,
independent of the precise value of the matching scale $\mu_W$, as it
must be. Indeed
\begin{equation}\label{cmud}
{d\over d\ln\mu_W}C(\mu)=\ord(\as^2)  \end{equation}
since
\begin{equation}\label{amud}
{d\over d\ln\mu_W}\as(\mu_W)=-2\beta_0{\as(\mu_W)^2\over 4\pi}+\ord(\as^3)\end{equation}
In the same way one can also convince oneself that the coefficient
function is independent of the heavy quark threshold scales, up to
terms of the neglected order.\\
Of course the dependence on the low energy scale $\mu$ remains
and has to be matched with the corresponding dependence of the
operator matrix element.\\
All the points we have mentioned here apply in an analogous
manner also to the case with operator mixing, only the algebra is
then slightly more complicated.

We would like to stress once again that it is only at the NLO level,
that these features enter the analysis in a nontrivial way, as should
be evident from the presentation we have given above.

\subsubsection{Evanescent Operators}
               \label{sec:basicform:wc:evanop}
Finally, we would like to mention the so called {\it evanescent}
operators. These are operators which exist in $D\not=4$ dimensions but
vanish in $D=4$. It has been stressed in \cite{burasweisz:90} that a
correct calculation of two-loop anomalous dimensions requires a proper
treatment of these operators. This discussion has been extended in
\cite{dugan:91} and further generalized in \cite{herrlichnierste:94}.
In view of a rather technical nature of this aspect we refer the
interested reader to the papers quoted above.
