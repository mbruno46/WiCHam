\section{The Decay $K_L\to\pi^0 \lowercase{e}^+\lowercase{e}^-$}
\label{sec:KLpee}

\subsection{General Remarks}
\label{sec:KLpee:General}

Let us next move on to discuss the rare decay $K_L\to\pi^0e^+e^-$.
Whereas in $K\to\pi\pi$ decays the CP violating contribution is
only a tiny part of the full amplitude and the direct CP violation
as we have just seen is expected to be at least by three orders of
magnitude smaller than the indirect CP violation, the corresponding
hierarchies are very different for $K_L\to\pi^0e^+e^-$. At lowest
order in electroweak interactions (one-loop photon penguin,
$Z^0$-penguin and W-box diagrams), this decay takes place only if
CP symmetry is violated. The CP conserving contribution to the
amplitude comes from a two photon exchange, which although of higher
order in $\alpha$ could in principle be sizable. Extensive studies
of several groups indicate however that the CP conserving part is
likely to be smaller than the CP violating contributions. We will be
more specific about this at the end of this section.  \\ The CP
violating part can again be divided into a direct and an indirect one.
The latter is given by the $K_S\to\pi^0e^+e^-$ amplitude times the CP
violating parameter $\varepsilon_K$.  The amplitude
$A(K_S\to\pi^0e^+e^-)$ can be written as \begin{equation}\label{akspee}
A(K_S\to\pi^0e^+e^-)=\langle\pi^0e^+e^-|{\cal H}_{eff}| K_S\rangle
\end{equation} where ${\cal H}_{eff}$ can be found in
(\ref{eq:HeffKpe}) with the operators $Q_1, \ldots, Q_6$ defined in
(\ref{eq:Kppbasis}), the operators $Q_{7V}$ and $Q_{7A}$ given by
\begin{equation}\label{q7v7a} Q_{7V}=(\bar sd)_{V-A}(\bar ee)_V  \qquad
Q_{7A}=(\bar sd)_{V-A}(\bar ee)_A \end{equation} and the Wilson
coefficients $z_i$ and $y_i$ calculated in section \ref{sec:HeffKpe}.
\\ Let us next note that the coefficients of $Q_{7V}$ and $Q_{7A}$ are
$\ord(\alpha)$, but their matrix elements $\langle\pi^0e^+e^-|
Q_{7V,A}| K_S\rangle$ are $\ord(1)$. In the case of $Q_i$ ($i=1,\ldots,
6$) the situation is reversed: the Wilson coefficients are $\ord(1)$,
but the matrix elements $\langle\pi^0e^+e^-| Q_i| K_S\rangle$ are
$\ord(\alpha)$. Consequently at $\ord(\alpha)$ all operators contribute
to $A(K_S\to\pi^0e^+e^-)$. However because $K_S\to\pi^0e^+e^-$ is CP
conserving, the coefficients $y_i$ multiplied by $\tau=\ord(\lambda^4)$
can be fully neglected and the operator $Q_{7A}$ drops out in this
approximation.  Now whereas $\langle\pi^0e^+e^-| Q_{7V}| K_S\rangle$
can be trivially calculated, this is not the case for
$\langle\pi^0e^+e^-| Q_i| K_S\rangle$ with $i=1,\ldots, 6$ which can
only be evaluated using non-perturbative methods. Moreover it is clear
from the short-distance analysis of section \ref{sec:HeffKpe} that the
inclusion of $Q_i$ in the estimate of $A(K_S\to\pi^0e^+e^-)$ cannot be
avoided. Indeed, whereas $\langle\pi^0e^+e^-| Q_{7V}| K_S\rangle$ is
independent of $\mu$ and the renormalization scheme, the coefficient
$z_{7V}$ shows very strong scheme and $\mu$-dependences.  They can only
be canceled by the contributions from the four-quark operators $Q_i$.
All this demonstrates that the estimate of the indirect CP violation in
$K_L\to\pi^0e^+e^-$ cannot be done very reliably at present. Some
estimates in the framework of chiral perturbation theory will be discussed
below. On the other hand, a much better assessment of the importance of
indirect CP violation in $K_L\to\pi^0e^+e^-$ will become possible after
a measurement of $B(K_S\to\pi^0e^+e^-)$.  \\ Fortunately the directly
CP violating contribution can be fully calculated as a function of
$m_t$, CKM parameters and the QCD coupling constant $\as$. There are
practically no theoretical uncertainties related to hadronic matrix
elements because $\langle\pi^0|(\bar sd)_{V-A}| K_L\rangle$ can be
extracted using isospin symmetry from the well measured decay
$K^+\to\pi^0e^+\nu$. In what follows, we will concentrate on this
contribution.

\subsection{Analytic Formula for $B(K_L\to\pi^0e^+e^-)_{dir}$}
\label{sec:KLpee:Analytic}

The directly CP violating contribution is governed by the coefficients
$y_i$ and consequently only the penguin operators $Q_3,\ldots, Q_6$,
$Q_{7V}$ and $Q_{7A}$ have to be considered. Since
$y_i=\ord(\as)$ for $i=3,\ldots, 6$, the contribution of
QCD penguins to $B(K_L\to\pi^0e^+e^-)_{dir}$ is really
$\ord(\alpha\as)$ to be compared with the $\ord(\alpha)$
contributions of $Q_{7V}$ and $Q_{7A}$. In deriving the final
formula we will therefore neglect the contributions of the
operators $Q_3,\ldots, Q_6$, i.e. we will assume that
\begin{equation}\label{qillq7v1}
\sum_{i=3}^6 y_i(\mu)\langle\pi^0e^+e^-|Q_i|K_L\rangle\ll
y_{7V}(\mu)\langle\pi^0e^+e^-|Q_{7V}|K_L\rangle
\end{equation}
This assumption is supported by the corresponding relation for the
quark-level matrix elements
\begin{equation}\label{qillq7v2}
\sum_{i=3}^6 y_i(\mu)\langle d e^+e^-|Q_i| s \rangle\ll
y_{7V}(\mu)\langle d e^+e^-|Q_{7V}| s \rangle
\end{equation}
that can be easily verified perturbatively.
\\
The neglect of the QCD penguin operators is compatible with the scheme
and $\mu$-independence of the resulting branching ratio.  Indeed
$y_{7A}$ does not depend on $\mu$ and the renormalization scheme at all
and the corresponding dependences in $y_{7V}$ are at the level of $\pm
1\%$ as discussed in section \ref{sec:HeffKpe:numres}.  Introducing the
numerical constant
\begin{equation}
\label{kappae}
\kappa_e=\frac{1}{V_{us}^2}\frac{\tau(K_L)}{\tau(K^+)}
\left( \frac{\aem}{2 \pi} \right)^2 B(K^+\to\pi^0e^+\nu)
= 6.3 \cdot 10^{-6}
\end{equation}
one then finds
\begin{equation}\label{bklpedir}
B(K_L\to\pi^0e^+e^-)_{dir}=\kappa_e ({\rm Im}\lambda_t)^2
\left(\tilde{y}_{7V}^2 + \tilde{y}_{7A}^2\right)
\end{equation}
where
\begin{equation}
y_i = \frac{\aem}{2 \pi} \tilde{y_i} \, .
\label{eq:ytilde}
\end{equation}
Using next the method of the penguin-box expansion
(section \ref{sec:PBE}) we can write similarly to \eqn{C9tilde} and
\eqn{C10}
\begin{equation}\label{y7vpbe}
\tilde{y}_{7V} =
P_0 + \frac{Y_0(x_t)}{\sin^2\Theta_W} - 4 Z_0(x_t)+ P_E E_0(x_t)
\end{equation}
\begin{equation}\label{y7apbe}
\tilde{y}_{7A}=-\frac{1}{\sin^2\Theta_W} Y_0(x_t)
\end{equation}
with $Y_0$, $Z_0$ and $E_0$ given in (\ref{yy0}), (\ref{eq:XYZ}) and
(\ref{eq:Ext}).  $P_E$ is $\ord(10^{-2})$ and consequently the last
term in (\ref{y7vpbe}) can be neglected. $P_0$ is given for different
values of $\mu$ and $\Lambda_{\overline{MS}}$ in table
\ref{tab:P0klpee}.  There we also show the leading order results and
the case without QCD corrections.

\begin{table}[htb]
\caption[]{PBE coefficient $P_0$ of $y_{7V}$ for various values of
$\Lms^{(4)}$ and $\mu$. In the absence of QCD $P_0=8/9\;\ln(M_W/m_c)
= 3.664$ holds universally.
\label{tab:P0klpee}}
\begin{center}
\begin{tabular}{|c|c||c|c|c|}
\multicolumn{2}{|c||}{}     &
\multicolumn{3}{c|}{$P_0$} \\
\hline
$\Lms^{(4)}\,[\mev]$ & $\mu\,[\gev]$ & {\rm LO} & {\rm NDR} & {\rm HV} \\
\hline
    & 0.8 & 2.073 &  3.159 &  3.110 \\
215 & 1.0 & 2.048 &  3.133 &  3.084 \\
    & 1.2 & 2.027 &  3.112 &  3.063 \\
\hline
    & 0.8 & 1.863 &  3.080 &  3.024 \\
325 & 1.0 & 1.834 &  3.053 &  2.996 \\
    & 1.2 & 1.811 &  3.028 &  2.970 \\
\hline
    & 0.8 & 1.672 &  2.976 &  2.914 \\
435 & 1.0 & 1.640 &  2.965 &  2.899 \\
    & 1.2 & 1.613 &  2.939 &  2.872
\end{tabular}
\end{center}
\end{table}

The analytic expressions in (\ref{y7vpbe}) and (\ref{y7apbe}) are
useful as they display not only the explicit $m_t$-dependence, but also
isolate the impact of leading and next-to-leading QCD effects. These
effects modify only the constants $P_0$ and $P_E$. As anticipated from
the results of section \ref{sec:HeffKpe:numres}, $P_0$ is strongly
enhanced relatively to the LO result. This enhancement amounts roughly
to a factor of $1.6\pm 0.1$. Partially this enhancement is however due
to the fact that for $\Lambda_{LO}=\Lambda_{\overline{MS}}$ the QCD
coupling constant in the leading order is $20-30\%$ larger than its
next-to-leading order value. Calculating $P_0$ in LO but with the full
$\as$ of (\ref{amu}) we have found that the enhancement then amounts to
a factor of $1.33\pm 0.06$. In any case the inclusion of NLO QCD
effects and a meaningful use of $\Lambda_{\overline{MS}}$ show that the
next-to-leading order effects weaken the QCD suppression of $y_{7V}$.
As seen in table \ref{tab:P0klpee}, the suppression of $P_0$ by QCD
corrections amounts to about $15\%$ in the complete next-to-leading
order calculation.

\subsection{Numerical Analysis}
\label{sec:KLpee:Numerical}

In fig.\ \ref{fig:kpiee:mty7VA} of section \ref{sec:HeffKpe:numres} we
have shown $|y_{7V}/\alpha|^2$ and $|y_{7A}/\alpha|^2$ as functions of
$m_t$ together with the leading order result for $|y_{7V}/\alpha|^2$
and the case without QCD corrections. From there it is obvious that the
dominant $m_t$-dependence of $B(K_L\to\pi^0e^+e^-)_{dir}$ originates
from the coefficient of the operator $Q_{7A}$. Another noteworthy
feature was that accidentally for $m_t\approx 175\gev$ one finds
$y_{7V}\approx y_{7A}$.

In fig.\ \ref{fig:mtbrimltAll} the ratio
$B(K_L\to\pi^0e^+e^-)_{dir}/({\rm Im}\lambda_t)^2$
is shown as a function of $m_t$. The enhancement of the directly
CP violating contribution through NLO corrections relatively to the
LO estimate is clearly visible on this plot. As we will see below,
due to large uncertainties present in ${\rm Im}\lambda_t$ this
enhancement cannot yet be fully appreciated phenomenologically.
\\
The very weak dependence on $\Lambda_{\overline{MS}}$ should be
contrasted with the very strong dependence found in the case of
$\varepsilon^\prime/\varepsilon$. Therefore, provided the other two
contributions to $K_L\to\pi^0e^+e^-$ can be shown to be small or can
be reliably calculated one day, the measurement of
$B(K_L\to\pi^0e^+e^-)$ should offer a good determination of
${\rm Im}\lambda_t$.

\begin{figure}[hbt]
\vspace{0.10in}
\centerline{
\epsfysize=5in
\rotate[r]{
\epsffile{ps/mtbrimltAll.ps}
}
}
\vspace{0.10in}
\caption[]{
$B(K_L \to \pi^0 e^+e^-)_{\rm dir}/(\IM\lambda_t)^2$
as a function of $\mt$ for various values of $\Lms^{(4)}$ at scale
$\mu =1.0\gev $.
\label{fig:mtbrimltAll}}
\end{figure}

Next we would like to comment on the possible uncertainties due to the
definition of $m_t$. At the level of accuracy at which we work we
cannot fully address this question yet. In order to be able to do it,
one needs to know the perturbative QCD corrections to $Y_0(x_t)$ and
$Z_0(x_t)$ and for consistency an additional order in the
renormalization group improved calculation of $P_0$.  Since the
$m_t$-dependence of $y_{7V}$ is rather moderate, the main concern in
this issue is the coefficient $y_{7A}$ whose $m_t$-dependence is fully
given by $Y(x_t)$. Fortunately the QCD corrected function $Y(x_t)$ is
known from the analysis of $K_L\to\mu^+\mu^-$ and can be directly used
here. As we will discuss in section \ref{sec:KLmm}, for $m_t=m_t(m_t)$
the QCD corrections to $Y_0(x_t)$ are around 2\%. On this basis we
believe that if $m_t=m_t(m_t)$ is chosen, the additional  QCD
corrections to $B(K_L\to\pi^0e^+e^-)_{dir}$ should be small.

Finally we give the predictions for the present and future sets of
input parameters as described in apppendix \ref{app:numinput}. It
should be  emphasized that the uncertainties in these predictions
result entirely from the CKM parameters. This situation will improve
considerably in the era of dedicated B-physics experiments in the
next decade, allowing a precise prediction for $B(K_L \to \pi^0
e^+e^-)_{\rm dir}$.

We find
\begin{equation}
B(K_L \to \pi^0 e^+ e^-)_{\rm dir}=
\left\{\begin{array}{ll}
(4.26 \pm 3.03) \cdot 10^{-12} & \mbox{no $x_d$ constraint} \\
(4.48 \pm 2.77) \cdot 10^{-12} & \mbox{with $x_d$ constraint} 
\end{array} \right.
\label{eq:BKLdirpresent}
\end{equation}
\begin{equation}
B(K_L \to \pi^0 e^+ e^-)_{\rm dir}=
\left\{\begin{array}{ll}
(3.71 \pm 1.61) \cdot 10^{-12} & \mbox{no $x_d$ constraint} \\
(4.32 \pm 0.96) \cdot 10^{-12} & \mbox{with $x_d$ constraint}
\end{array} \right.
\label{eq:BKLdirfuture}
\end{equation}
These results are compatible with those found in \cite{burasetal:94a},
\cite{donoghuegabbiani:95}, \cite{kohlerpaschos:95} with differences
originating in various choices of CKM parameters.


\subsection{The Indirectly CP Violating and CP Conserving Parts}
            \label{sec:KLpee:Comparison}
Now we want to compare the results obtained for the direct CP violating
part with the estimates made for the indirect CP-violating contribution
and the CP-conserving one. The most recent discussions have been
presented in \cite{cohenetal:93}, \cite{heiligerseghal:93},
\cite{donoghuegabbiani:95}, \cite{kohlerpaschos:95} where references to
earlier papers can be found.

The indirect CP violating amplitude is given by the
$K_S \to \pi^0 e^+ e^-$ amplitude times the CP parameter $\eps_K$.
Once $B(K_S \to \pi^0 e^+ e^-)$ has been accurately measured, it will
be possible to calculate this contribution precisely. Using chiral
perturbation theory it is however possible to get an estimate by
relating $K_S \to \pi^0 e^+ e^-$ to the $K^+ \to \pi^+ e^+ e^-$
transition \cite{eckeretal:87}, \cite{eckeretal:88}. To this end one
can write 

\begin{equation}
B(K_L \to \pi^0 e^+ e^-)_{indir}=B(K^+ \to \pi^+e^+e^-)
\frac{\tau(K_L)}{\tau(K^+)} |\eps_K|^2 r^2
\label{eq:BKLindir1}
\end{equation}
where
\begin{equation}
r^2=\frac{\Gamma(K_S \to \pi^0 e^+ e^-)}{\Gamma(K^+ \to \pi^+ e^+ e^-)}
\label{eq:r2}
\end{equation}
With $B(K^+ \to \pi^+e^+e^-)=(2.74\pm 0.23)\cdot 10^{-7}$
\cite{alliegro:92} and the most recent chiral perturbation theory
estimate $|r| \le 0.5 $ \cite{eckeretal:88}, \cite{brunoprades:93}
one has
\begin{equation}
B(K_L \to \pi^0 e^+ e^-)_{indir}=(5.9\pm 0.5)\cdot 10^{-12} r^2
 \le 1.6\cdot 10^{-12},
\label{eq:BKLindir2}
\end{equation}
i.e.\ a branching ratio more than a factor of 2 below the direct CP
violating contribution. \\
Yet as emphasized recently in \cite{donoghuegabbiani:95} and also in
\cite{heiligerseghal:93} the knowledge of $r$ is very uncertain at
present. In particular the estimate in \eqn{eq:BKLindir2} is based on
a relation between two non-perturbative parameters, which is rather
ad hoc and certainly not a consequence of chiral symmetry. As shown
in \cite{donoghuegabbiani:95} a small deviation from this relation
increases $r$ to values above unity so that $B(K_L \to \pi^0
e^+e^-)_{\rm indir}$ could be comparable or even large than 
$B(K_L \to \pi^0 e^+e^-)_{\rm dir}$. It appears then that this
enormous uncertainty in the indirectly CP violating part can only be 
removed by measuring the rate of $K_S \to \pi^0 e^+e^-$.

It should also be stressed, that in reality the CP indirect amplitude
may interfere with the vector part of the CP direct amplitude.  The
full CP violating amplitude can then be written following
\cite{dib1:89}, \cite{dib2:89} as follows
\begin{equation}
B(K_L \to \pi^0 e^+ e^-)_{CP}=| 2.43\cdot 10^{-6} r e^{i\pi/4}-
i\sqrt{\kappa_e} Im\lambda_t\tilde y_{7V}|^2+
\kappa_e (Im\lambda_t)^2\tilde y_{7A}^2
\label{eq:BKLCP}
\end{equation}

\begin{figure}[hbt]
\vspace{0.10in}
\centerline{
\epsfysize=4in
\rotate[r]{
\epsffile{ps/bklcp.ps}
}
}
\vspace{0.10in}
\caption[]{
$B(K_L \to \pi^0 e^+e^-)_{\rm CP}$ for $\mt=170\gev$,
$\Lms^{(4)}=325\mev$ and $\IM\lambda_t = 1.3 \cdot 10^{-4}$ as
a function of $r$.
\label{fig:BKLCP}}
\end{figure}

As an example we show in fig.\ \ref{fig:BKLCP} $B(K_L \to \pi^0
e^+e^-)_{\rm CP}$ for $\mt=170\gev$, $\Lms^{(4)}=325\mev$ and
$\IM\lambda_t = 1.3 \cdot 10^{-4}$ as a function of $r$.
We observe that whereas for $0 \le r \le 1$ the dependence of $B(K_L \to
\pi^0 e^+e^-)_{\rm CP}$ on $r$ is moderate, it is rather strong
otherwise and already for $r < -0.6$ values as high as $10^{-11}$ are
found.

The estimate of the CP conserving contribution is also difficult. We
refer the reader to \cite{cohenetal:93}, \cite{heiligerseghal:93} and
\cite{donoghuegabbiani:95} where further references to an extensive
literature on this subject can be found. The measurement of the
branching ratio
\begin{equation}
B(K_L \to \pi^0 \gamma\gamma) \leq
\left\{ \begin{array}{ll}
(1.7\pm 0.3) \cdot 10^{-6} & \hbox{\cite{barretal:92}} \\
(2.0\pm 1.0) \cdot 10^{-6} & \hbox{\cite{papadimitriou:91}}
\end{array} \right.
\label{eq:BKLexp}
\end{equation}
and of the shape of the $\gamma\gamma$ mass spectrum plays an important
role in this estimate. The most recent analyses give
\begin{equation}
B(K_L \to \pi^0 e^+ e^-)_{cons}\approx\left\{ \begin{array}{ll}
(0.3-1.8)\cdot 10^{-12} & \hbox{\cite{cohenetal:93}} \\
     4.0 \cdot 10^{-12} & \hbox{\cite{heiligerseghal:93}} \\
(5 \pm 5)\cdot 10^{-12} & \hbox{\cite{donoghuegabbiani:95}}
\end{array} \right.
\label{eq:BKLtheo}
\end{equation}
i.e.\ not necesarily below the CP violating contribution.
An improved estimate of this component is certainly desirable.
It should be noted that there is no interference in the rate between
the CP conserving and CP violating contributions so that the results
in fig.\ \ref{fig:BKLCP} and \eqn{eq:BKLtheo} can simply be added.

\subsection{Outlook} 
            \label{sec:KLpee:Outlook}
The results discussed above indicate that within the Standard Model
$B(K_L \to \pi^0 e^+ e^-)$ could be as high as $1\cdot 10^{-11}$.
Moreover the direct CP violating contribution is found to be important
and could even be dominant. Unfortunately the large uncertainties in
the remaining two contributions will probably not allow an easy
identification of the direct CP violation by measuring the branching
ratio only. The future measurements of $B(K_S \to \pi^0 e^+e^-)$ and
improvements in the estimate of the CP conserving part may of course
change this unsatisfactory situation. Alternatively the measurements
of the electron energy asymmetry \cite{heiligerseghal:93}, 
\cite{donoghuegabbiani:95} and the study of the time evolution of $K^0 \to
\pi^0 e^+e^-$ \cite{littenberg:89b}, \cite{donoghuegabbiani:95},
\cite{kohlerpaschos:95} could allow for a refined study of CP violation
in this decay.

The present experimental bounds
\begin{equation}
B(K_L \to \pi^0 e^+ e^-) \leq
\left\{ \begin{array}{ll}
4.3 \cdot 10^{-9} & \hbox{\cite{harrisetal:93}} \\
5.5 \cdot 10^{-9} & \hbox{\cite{ohletal:90}}
\end{array} \right.
\label{eq:BKLbounds}
\end{equation}
are still by three orders of magnitude away from the theoretical
expectations. Yet the prospects of getting the required sensitivity of
order $10^{-11}$--$10^{-12}$ in five years are encouraging
\cite{littenbergvalencia:93}, \cite{winsteinwolfenstein:93},
\cite{ritchiewojcicki:93}.
