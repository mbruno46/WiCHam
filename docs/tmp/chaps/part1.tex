\skipevenpage

{\Huge\bf
\noindent
Part One --

\bigskip
\bigskip
\bigskip

\noindent
The Basic Formalism
}

\vfil

\noindent
In this first part we will discuss the basic formalism behind radiative
corrections to weak decays.

In section \ref{sec:sewm} we recall those ingredients of the standard
$SU(3)\otimes SU(2)\otimes  U(1)$ model, which play an important role
in subsequent sections. In particular we recall the
Cabibbo-Kobayashi-Maskawa matrix in two useful parametrizations and we
briefly describe the unitarity triangle.

In section \ref{sec:basicform} we outline the basic formalism for the
calculation of QCD effects in weak decays.  Beginning with the idea of
effective field theories we introduce subsequently the techniques of
the operator product expansion and the renormalization group. These
important concepts are illustrated explicitly using the simple, but
phenomenologically relevant example of current-current operators, which
allows to demonstrate the procedure in a transparent way. The central
issue in this formalism is the computation of the Wilson coefficients
$C_i$ of local operators in the LO and NLO approximation. This
calculation involves the proper computation of $C_i$ at $\mu =
\ord(\mw)$ and the renormalization group evolution down to low energy
scales $\mu\ll \mw$ relevant for the weak decays considered. The latter
requires the evaluation of one-loop and two-loop anomalous dimensions
of $Q_i$ or more generally the anomalous dimension matrices, which
describe the mixing of these operators under renormalization.  We
outline the steps for a consistent calculation of the Wilson
coefficients $C_i$ and formulate recipes for the determination of the
anomalous dimensions of local operators.  In section
\ref{sec:basicform:wc} we give ``master formulae'' for the Wilson
coefficients $C_i$ , including NLO corrections.  Since these formulae
will be central for our review, we discuss their various properties in
some detail. In particular we address the $\mu$- and renormalization
scheme dependences and we show on general grounds how these dependences
are canceled by those present in the hadronic matrix elements.
