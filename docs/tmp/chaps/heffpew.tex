\section{The Effective $\Delta F=1$ Hamiltonian: Inclusion of
         Electroweak Penguin Operators}
         \label{sec:HeffdF1:1010}
Similarly to the creation of the penguin operators $Q_3,\ldots,Q_6$
through QCD corrections the inclusion of electroweak corrections, shown
in figs.\ \ref{fig:1loopful}\,(d) and (e), generates a set of new
operators, the so-called electroweak penguin operators. For the $\dS$
decay $\Kpipi$ they are usually denoted by $Q_7, \ldots,Q_{10}$. \\
This means that although now we will have to deal with technically more
involved issues like an extended operator basis or the possibility of
mixed QCD-QED contributions the underlying principles in performing the
RG evolution will closely resemble those used in section \ref{sec:HeffdF1:66}
for pure QCD. Obviously, the fundamental step has already been made
when going from current-current operators only in
section \ref{sec:HeffdF1:22}, to the inclusion of QCD penguins in
section \ref{sec:HeffdF1:66}. Hence, in this section we will wherever
possible only point out the differences between the pure $6 \times 6$
QCD and the combined $10 \times 10$ QCD-QED case.

The full $\dS$ effective hamiltonian for $\Kpipi$ at scales
$\mu < \mc$ reads including QCD and QED corrections\footnote{
In principle also operators $Q_{11} = \frac{g_{\rm s}}{16 \pi^2} \ms
\bar s \sigma_{\mu\nu} T^a G_a^{\mu\nu} (1-\gamma_5) d$ and $Q_{12} =
\frac{e e_d}{16 \pi^2} \ms \bar s \sigma_{\mu\nu} F^{\mu\nu}
(1-\gamma_5) d$ should be considered for $\Kpipi$. However, as shown in
\cite{bertolinietal:94} their numerical contribution is negligible.
Therefore $Q_{11}$ and $Q_{12}$ will not be included here for
$\Kpipi$.}
\begin{equation}
\Heff(\dS) = \frac{G_F}{\sqrt{2}} \V{us}^* \V{ud}^{} \sum_{i=1}^{10}
\left( z_i(\mu) + \tau \; y_i(\mu) \right) Q_i(\mu) \, ,
\label{eq:HeffdF1:1010}
\end{equation}
with $\tau=-\V{ts}^* \V{td}^{}/(\V{us}^* \V{ud}^{})$.
\subsection{Operators}
            \label{sec:HeffdF1:1010:op}
The basis of four-quark operators for the $\dS$ effective hamiltonian
in \eqn{eq:HeffdF1:1010} is given by $Q_1,\ldots,Q_6$ of
\eqn{eq:Kppbasis} and the electroweak penguin operators
\begin{eqnarray}
Q_{7} & = & \frac{3}{2} \left( \bar s d \right)_{\rm V-A}
         \sum_{q} e_{q} \left( \bar q q \right)_{\rm V+A}
\, , \nn \\
Q_{8} & = & \frac{3}{2} \left( \bar s_{i} d_{j} \right)_{\rm V-A}
         \sum_{q} e_{q} \left( \bar q_{j}  q_{i}\right)_{\rm V+A}
\, , \nn \\
Q_{9} & = & \frac{3}{2} \left( \bar s d \right)_{\rm V-A}
         \sum_{q} e_{q} \left( \bar q q \right)_{\rm V-A}
\, , \label{eq:dF1:1010basis} \\
Q_{10}& = & \frac{3}{2} \left( \bar s_{i} d_{j} \right)_{\rm V-A}
         \sum_{q} e_{q} \left( \bar q_{j}  q_{i}\right)_{\rm V-A}
\, . \nn
\end{eqnarray}
Here, $e_q$ denotes the quark electric charge reflecting the
electroweak origin of $Q_7,\ldots,Q_{10}$.  The basis
$Q_1,\ldots,Q_{10}$ closes under QCD and QED renormalization.  Finally,
for $\mb > \mu > \mc$ the operators $Q_1^c$ and $Q_2^c$ of
eq.\ \eqn{eq:KppQ12c} have to be included again similarly to the case of
pure QCD.

\subsection{Wilson Coefficients}
            \label{sec:HeffdF1:1010:wc}
As far as formulae for Wilson coefficients are concerned the
generalization of section \ref{sec:HeffdF1:66:wc} to the present case is
to a large extent straightforward. \\
First, due to the extended operator basis $\vec{v}(\mu)$ and
$\vec{z}(\mu)$ in eqs.\ \eqn{eq:WCv} and \eqn{eq:WCz} are now ten
dimensional column vectors. Furthermore, the substitution
\begin{displaymath}
\hU_f(m_1,m_2) \to \hU_f(m_1,m_2,\aem)
\end{displaymath}
has to be made in the RG evolution equations \eqn{eq:WCv}, \eqn{eq:WCz} and
\eqn{eq:zmc12}. Here $\hU_f(m_1,m_2,\aem)$ denotes the
full $10 \times 10$ QCD- QED RG evolution matrix for $f$ active
flavours. $\hU_f(m_1,m_2,\aem)$ will still be discussed in more detail
in subsection \ref{sec:HeffdF1:1010:rge}.

The extended initial values $\vC(\mw)$ including now $\ord(\aem)$
corrections and additional entries for $Q_7,\ldots,Q_{10}$ can be
obtained from the usual matching procedure between figs.\ \ref{fig:1loopful}
and \ref{fig:1loopeff}. They read in the NDR scheme \cite{burasetal:92d}
\begin{eqnarray}
C_1(\mw) &=&     \frac{11}{2} \; \frac{\as(\mw)}{4\pi} \, ,
\label{eq:CMw1} \\
C_2(\mw) &=& 1 - \frac{11}{6} \; \frac{\as(\mw)}{4\pi}
               - \frac{35}{18} \; \frac{\aem}{4\pi} \, ,
\label{eq:CMw2} \\
C_3(\mw) &=& -\frac{\as(\mw)}{24\pi} \widetilde{E}_0(x_t)
             +\frac{\aem}{6\pi} \frac{1}{\sin^2\theta_W}
             \left[ 2 B_0(x_t) + C_0(x_t) \right] \, , 
\label{eq:CMw3} \\
C_4(\mw) &=& \frac{\as(\mw)}{8\pi} \widetilde{E}_0(x_t) \, ,
\label{eq:CMw4} \\
C_5(\mw) &=& -\frac{\as(\mw)}{24\pi} \widetilde{E}_0(x_t) \, ,
\label{eq:CMw5} \\
C_6(\mw) &=& \frac{\as(\mw)}{8\pi} \widetilde{E}_0(x_t) \, ,
\label{eq:CMw6} \\
C_7(\mw) &=& \frac{\aem}{6\pi} \left[ 4 C_0(x_t) + \widetilde{D}_0(x_t)
\right]\, ,
\label{eq:CMw7} \\
C_8(\mw) &=& 0 \, ,
\label{eq:CMw8} \\
C_9(\mw) &=& \frac{\aem}{6\pi} \left[ 4 C_0(x_t) + \widetilde{D}_0(x_t) +
             \frac{1}{\sin^2\theta_W} (10 B_0(x_t) - 4 C_0(x_t)) \right] \, ,
\label{eq:CMw9} \\
C_{10}(\mw) &=& 0 \, ,
\label{eq:CMw10}
\end{eqnarray}
where
\begin{eqnarray}
B_0(x) &=& \frac{1}{4} \left[ \frac{x}{1-x} + \frac{x \ln x}{(x-1)^2}
\right]\, , \label{eq:Bxt} \\
C_0(x) &=& \frac{x}{8} \left[ \frac{x-6}{x-1} + \frac{3 x + 2}{(x-1)^2}
\ln x \right]\, ,
\label{eq:Cxt} \\
D_0(x) &=& -\frac{4}{9} \ln x + \frac{-19 x^3 + 25 x^2}{36 (x-1)^3} +
         \frac{x^2 (5 x^2 - 2 x - 6)}{18 (x-1)^4} \ln x \, ,
\label{eq:Dxt} \\
\widetilde{D}_0(x_t) &=& D_0(x_t) - \frac{4}{9} \, .
\label{eq:Dxttilde} 
\end{eqnarray}
$\widetilde{E}_0(x_t)$ and $x_t$ have already been defined in
eqs.\ \eqn{eq:Exttilde} and \eqn{eq:xt}, respectively.  Here $B_0(x)$
results from the evaluation of the box diagrams, $C_0(x)$ from the
$Z^0$-penguin, $D_0(x)$ from the photon penguin and $E_0(x)$ in
$\widetilde{E}_0(x_t)$ from the gluon penguin diagrams.

The initial values $\vec{C}(\mw)$ in the HV scheme can be found in
\cite{burasetal:92d}.

Finally, the generalization of \eqn{eq:zmc} to the $Q_1,\ldots,Q_{10}$
basis reads \cite{burasetal:92d}
\begin{equation}
\vec{z}^{\rm }(\mc) =
\left( \begin{array}{c}
z_1^{\rm }(\mc) \\ z_2^{\rm }(\mc) \\
-\as/(24\pi) F_{\rm s}^{\rm }(\mc) \\ \as/(8\pi) F_{\rm s}^{\rm }(\mc) \\ 
-\as/(24\pi) F_{\rm s}^{\rm }(\mc) \\ \as/(8\pi) F_{\rm s}^{\rm }(\mc) \\ 
\aem/(6\pi) F_{\rm e}^{\rm }(\mc) \\  0  \\
\aem/(6\pi) F_{\rm e}^{\rm }(\mc) \\  0
\end{array} \right) \, ,
\label{eq:zmc:1010}
\end{equation}
with $F_{\rm s}(\mc)$ given by \eqn{eq:Fsmc} and 
\begin{equation}
F_{\rm e}^{\rm }(\mc) =
-\frac{4}{9} \; \left( 3 z_1(\mc) + z_2(\mc) \right) \, .
\label{eq:Femc}
\end{equation}
In the HV scheme, in addition to $z_{1,2}$ differing from their NDR values, 
one has $F_{\rm s}^{\rm }(\mc) = F_{\rm e}^{\rm }(\mc)
= 0$ and, consequently, $z_i(\mc) = 0$ for $i\not=1,2$.

\subsection{Renormalization Group Evolution and Anomalous Dimension Matrices}
            \label{sec:HeffdF1:1010:rge}
Besides an extended operator basis the main difference between the pure
QCD case of section \ref{sec:HeffdF1:66} and the present case consists in
the additional presence of QED contributions to the RG evolution.  This
will make the actual formulae for the RG evolution matrices more
involved, however the underlying concepts developed in
sections \ref{sec:HeffdF1:22} and \ref{sec:HeffdF1:66} remain the same.

Similarly to \eqn{eq:UgeneralQCD} for pure QCD the general RG evolution
matrix $\hU(m_1,m_2,\aem)$ from scale $m_2$ down to $m_1 < m_2$ can be
written formally as \footnote{We neglect the running of the 
electromagnetic coupling $\aem$, which is a very good approximation
\cite{buchallaetal:90}.}
\begin{equation}
\hU(m_1,m_2,\aem) \equiv T_g \exp
\int_{g(m_2)}^{g(m_1)} \!\! dg' \; \frac{\hg^T(g'^2,\aem)}{\beta(g')} \, ,
\label{eq:Ugeneral}
\end{equation}
with $\hg(g^2,\aem)$ being now the full $10\times 10$ anomalous dimension
matrix including QCD and QED contributions.

For the case at hand $\hg(g^2,\aem)$ can be expanded in the following way
\begin{equation}
\hg(g^2,\aem) = \hg_{\rm s}(g^2) + \frac{\aem}{4\pi} \hG(g^2) + \ldots \, ,
\label{eq:gexpdF1:1010}
\end{equation}
with the pure $\as$-expansion of $\hg_{\rm s}(g^2)$ given in
\eqn{eq:gsexpKpp}. The term present due to QED corrections has the
expansion
\begin{equation}
\hG(g^2) = \gem + \frac{\as}{4\pi} \gse + \ldots  \, .
\label{eq:GexpdF1:1010}
\end{equation}

Using \eqn{eq:gexpdF1:1010}--\eqn{eq:GexpdF1:1010} the general RG
evolution matrix $\hU(m_1,m_2,\aem)$ of eq.~\eqn{eq:Ugeneral} may then
be decomposed as follows
\begin{equation}
\hU(m_1,m_2,\aem) =
\hU(m_1,m_2) + \frac{\aem}{4\pi} \hR(m_1,m_2) \, ,
\label{eq:UdF1:1010}
\end{equation}
Here $\hU(m_1,m_2)$ represents the pure QCD evolution already
encountered in section \ref{sec:HeffdF1:66} but now generalized to an
extended operator basis. $\hR(m_1,m_2)$ describes the additional
evolution in the presence of the electromagnetic interaction.
$\hU(m_1,m_2)$ sums the logarithms $(\as t)^n$ and $\as (\as t)^n$ with
$t=\ln(m_2^2/m_1^2)$, whereas $\hR(m_1,m_2)$ sums the logarithms $t (\as
t)^n$ and $(\as t)^n$. \\
The formula for $\hU(m_1,m_2)$ has already been given in
\eqn{eq:UQCDKpp}.  The leading order formula for $\hR(m_1,m_2)$ can be
found in \cite{buchallaetal:90} except that there a different
overall normalization (relative factor $-4\pi$ in $\hR$) has been
used.  Here we give the general expression for $\hR(m_1,m_2)$
\cite{burasetal:92d}
\begin{eqnarray}
\hR(m_1,m_2) &=& \int_{g(m_2)}^{g(m_1)} \!\! dg' \;
                 \frac{\hU(m_1,m') \, \hG^T(g') \, \hU(m',m_2)}{\beta(g')}
\label{eq:RdF1:1010} \\
        &\equiv& -\frac{2\pi}{\beta_0} \; \hV \; \left(
                 \hK^{(0)}(m_1,m_2) +
                 \frac{1}{4\pi} \sum_{i=1}^{3} \hK_i^{(1)}(m_1,m_2)
                 \right) \hV^{-1} \, , \nn
\end{eqnarray}
with $g' \equiv g(m')$.

The matrix kernels in \eqn{eq:RdF1:1010} are defined by
\begin{equation}
(\hK^{(0)}(m_1,m_2))_{ij} = \frac{\hM^{(0)}_{ij}}{a_i - a_j - 1}
\left[
\left( \frac{\as(m_2)}{\as(m_1)} \right)^{a_j} \frac{1}{\as(m_1)} -
\left( \frac{\as(m_2)}{\as(m_1)} \right)^{a_i} \frac{1}{\as(m_2)}
\right] \, ,
\label{eq:K0}
\end{equation}
\begin{equation}
\left( \hK_1^{(1)}(m_1,m_2) \right)_{ij} =
\left\{
\begin{array}{ll}
\frac{M^{(1)}_{ij}}{a_i - a_j}
\left[ \left( \frac{\as(m_2)}{\as(m_1)} \right)^{a_j} -
       \left( \frac{\as(m_2)}{\as(m_1)} \right)^{a_i} \right] & i \not= j \\
\svs
M^{(1)}_{ii} \left( \frac{\as(m_2)}{\as(m_1)} \right)^{a_i}
             \ln\frac{\as(m_1)}{\as(m_2)}             & i=j
\end{array} \, ,
\right.
\label{eq:K11}
\end{equation}
\begin{eqnarray}
\hK_2^{(1)}(m_1,m_2) & = &
-\,\as(m_2) \; \hK^{(0)}(m_1,m_2) \; H \, ,
\label{eq:K12} \\
\hK_3^{(1)}(m_1,m_2) & = &
\phantom{-}\,\as(m_1) \; H \; \hK^{(0)}(m_1,m_2)
\label{eq:K13}
\end{eqnarray}
with
\begin{eqnarray}
\hM^{(0)} &=& \hV^{-1} \; \gemt \; \hV \, ,
\nn \\
\hM^{(1)} &=&
\hV^{-1} \left( \gset - \frac{\beta_1}{\beta_0} \gemt +
                \left[ \gemt, \hJ \right] \right) \hV \, .
\label{eq:M0M1}
\end{eqnarray}
The matrix $H$ is defined in (\ref{sij}).

After this formal description we now give explicit expressions for the
$10 \times 10$ LO and NLO anomalous dimension matrices $\gs$, $\gem$,
$\gss$ and $\gse$. The values quoted for the NLO matrices are in the
NDR scheme \cite{burasetal:92b}, \cite{burasetal:92c},
\cite{ciuchini:93}.  The corresponding results for $\gss$ and $\gse$ in
the HV scheme can either be obtained by direct calculation or by using
the QCD/QED version of eq.\ \eqn{gpgs} given in \cite{burasetal:92c}.
They can be found in \cite{burasetal:92b}, \cite{burasetal:92c}
and \cite{ciuchini:92}, \cite{ciuchini:93}.

The $6 \times 6$ submatrices for $Q_1,\ldots,Q_6$ of the full LO and
NLO $10 \times 10$ QCD matrices $\gs$ and $\gss$ are identical to the
corresponding $6 \times 6$ matrices already given in
eqs.\ \eqn{eq:gs0Kpp} and \eqn{eq:gs1ndrN3Kpp}, respectively. Next,
$Q_1,\ldots,Q_6$ do not mix to $Q_7,\ldots,Q_{10}$ under QCD and hence
\begin{equation}
\left[ \gs \right]_{ij} = \left[ \gss \right]_{ij} = 0
\qquad
i=1,\ldots,6
\quad
j=7,\ldots,10 \, .
\label{eq:gs01ij}
\end{equation}
The remaining entries for rows 7--10 in $\gs$ \cite{bijnenswise:84} and
$\gss$ \cite{burasetal:92b}, \cite{ciuchini:93} are given in tables
\ref{tab:gs0} and \ref{tab:gs1}, respectively. There $u$ and $d$
($f=u+d$) denote the number of active up- and down-type quark
flavours.

\begin{table}[htb]
\caption[]{Rows 7--10 of the LO anomalous dimension matrix $\gs$.}
\begin{tabular}{|r|c|c|c|c|c|c|c|c|c|c|}
$(i,j)$ & 1 & 2 & 3 & 4 & 5 & 6 & 7 & 8 & 9 & 10 \\
\hline
$7 $&$ 0 $&$ 0 $&$ 0 $&$ 0 $&$ 0 $&$ 0 $&$ {6\over N} $&$ -6 $&$ 0 $&$ 0 $\\
\svs
$8 $&$ 0 $&$ 0 $&$ {{-2 \left( u-d/2 \right) }\over {3 N}} $&$ {{2 \left(\
  u-d/2 \right) }\over 3} $&$ {{-2 \left( u-d/2 \right)\
  }\over {3 N}} $&$ {{2 \left( u-d/2 \right) }\over 3} $&$ 0 $&$ {{-6\
  \left( -1 + {N^2} \right) }\over N} $&$ 0 $&$ 0 $\\ \svs
$9 $&$ 0 $&$ 0 $&$ {2\over {3 N}} $&$ -{2\over 3} $&$ {2\over {3 N}} $&$
-{2\over 3} $&$  0 $&$ 0 $&$\
  {{-6}\over N} $&$ 6 $\\ \svs
$10 $&$ 0 $&$ 0 $&$ {{-2 \left( u-d/2 \right) }\over {3 N}} $&$ {{2 \left(\
  u-d/2 \right) }\over 3} $&$ {{-2 \left( u-d/2 \right)\
  }\over {3 N}} $&$ {{2 \left( u-d/2 \right) }\over 3} $&$ 0 $&$ 0 $&$ 6\
  $&$ {{-6}\over N} $
\end{tabular}
\label{tab:gs0}
\end{table}

\begin{table}[htb]
\caption[]{Rows 7--10 of the NLO anomalous dimension matrix $\gss$ for $N=3$
and NDR.}
\begin{tabular}{|r|c|c|c|c|c|}
$(i,j)$ & 1 & 2 & 3 & 4 & 5 \\
\hline
$7 $&$ 0 $&$ 0 $&$ {{-61\,(u-d/2)}\over 9} $&$ {{-11\,(u-d/2)}\over 3} $&$
{{83\, (u-d/2)}\over 9} $\\ \svs
$8 $&$ 0 $&$ 0 $&$ {{-682\,(u-d/2)}\over {243}} $&$ {{106\,(u-d/2)}
\over {81}} $&$\ {{704\,(u-d/2)}\over {243}} $\\ \svs
$9 $&$ 0 $&$ 0 $&$ {{202}\over {243}} + {{73\,(u-d/2)}\over 9} $&$
 -{{1354}\over {81}} -\
  {{(u-d/2)}\over 3} $&$ {{1192}\over {243}} - {{71\,(u-d/2)}\over 9} $\\ \svs
$10 $&$ 0 $&$ 0 $&$ -{{79}\over 9} - {{106\,(u-d/2)}\over {243}} $&$
{7\over 3} +\
  {{826\,(u-d/2)}\over {81}} $&$ {{65}\over 9} - {{502\,(u-d/2)}\over\
  {243}} $
\end{tabular}

\begin{tabular}{|r|c|c|c|c|c|}
$(i,j)$ & 6 & 7 & 8 & 9 & 10 \\
\hline
$7 $&$ {{-11\,(u-d/2)}\over 3} $&$ {{71}\over 3} - {{22\,f}\over 9} $&$ -99 +\
  {{22\,f}\over 3} $&$ 0 $&$ 0 $\\ \svs
$8 $&$ {{736\,(u-d/2)}\over {81}} $&$ -{{225}\over 2} + 4\,f $&$ -{{1343}
\over 6} +\
  {{68\,f}\over 9} $&$ 0 $&$ 0 $\\ \svs
$9 $&$ -{{904}\over {81}} - {{(u-d/2)}\over 3} $&$ 0 $&$ 0
 $&$ -{{21}\over 2} -\
  {{2\,f}\over 9} $&$ {7\over 2} + {{2\,f}\over 3} $\\ \svs
$10 $&$ {7\over 3} + {{646\,(u-d/2)}\over {81}} $&$ 0 $&$ 0 $&$ {7\over 2} +
 {{2\,f}\over\
  3} $&$ -{{21}\over 2} - {{2\,f}\over 9} $
\end{tabular}
\label{tab:gs1}
\end{table}

The full $10 \times 10$ matrices $\gem$ \cite{lusignoli:89} and $\gse$
\cite{burasetal:92c}, \cite{ciuchini:93} can be found in tables \ref{tab:gem}
and \ref{tab:gse}, respectively.

\begin{table}[htb]
\caption[]{The LO anomalous dimension matrix $\gem$.}
\begin{tabular}{|r|c|c|c|c|c|c|c|c|c|c|}
$(i,j)$ & 1 & 2 & 3 & 4 & 5 & 6 & 7 & 8 & 9 & 10 \\
\hline
$1 $&$ -{8\over 3} $&$ 0 $&$ 0 $&$ 0 $&$ 0 $&$ 0 $&$ {{16\,N}\over {27}} $&$ 0 $&$  
{{16\,N}\over\
  {27}} $&$ 0 $\\ \svs
$2 $&$ 0 $&$ -{8\over 3} $&$ 0 $&$ 0 $&$ 0 $&$ 0 $&$ {{16}\over {27}} $&$ 0 $&$  
{{16}\over {27}} $&$ 0\
  $\\ \svs
$3 $&$ 0 $&$ 0 $&$ 0 $&$ 0 $&$ 0 $&$ 0 $&$ -{{16}\over {27}} + {{16\,N\,\left( u-d/2\
  \right) }\over {27}} $&$ 0 $&$ -{{88}\over {27}} + {{16\,N\,\left( 
  u-d/2 \right) }\over {27}} $&$ 0 $\\ \svs
$4 $&$ 0 $&$ 0 $&$ 0 $&$ 0 $&$ 0 $&$ 0 $&$ {{-16\,N}\over {27}} + {{16\,\left( u-d/2\
  \right) }\over {27}} $&$ 0 $&$ {{-16\,N}\over {27}} + {{16\,\left( 
  u-d/2 \right) }\over {27}} $&$ -{8\over 3} $\\ \svs
$5 $&$ 0 $&$ 0 $&$ 0 $&$ 0 $&$ 0 $&$ 0 $&$ {8\over 3} + {{16\,N\,\left( u-d/2\
  \right) }\over {27}} $&$ 0 $&$ {{16\,N\,\left( u-d/2 \right) }\over\
  {27}} $&$ 0 $\\ \svs
$6 $&$ 0 $&$ 0 $&$ 0 $&$ 0 $&$ 0 $&$ 0 $&$ {{16\,\left( u-d/2 \right) }\over {27}} $&$\
  {8\over 3} $&$ {{16\,\left( u-d/2 \right) }\over {27}} $&$ 0 $\\ \svs
$7 $&$ 0 $&$ 0 $&$ 0 $&$ 0 $&$ {4\over 3} $&$ 0 $&$ {4\over 3} + {{16\,N\,\left( u+d/4\
  \right) }\over {27}} $&$ 0 $&$ {{16\,N\,\left( u+d/4 \right) }\over\
  {27}} $&$ 0 $\\ \svs
$8 $&$ 0 $&$ 0 $&$ 0 $&$ 0 $&$ 0 $&$ {4\over 3} $&$ {{16\,\left( u+d/4 \right) }\over\
  {27}} $&$ {4\over 3} $&$ {{16\,\left( u+d/4 \right) }\over {27}} $&$ 0\
  $\\ \svs
$9 $&$ 0 $&$ 0 $&$ -{4\over 3} $&$ 0 $&$ 0 $&$ 0 $&$ {8\over {27}} + {{16\,N\,\left( 
  u+d/4 \right) }\over {27}} $&$ 0 $&$ -{{28}\over {27}} + {{16\,N\,\left( 
  u+d/4 \right) }\over {27}} $&$ 0 $\\ \svs
$10 $&$ 0 $&$ 0 $&$ 0 $&$ -{4\over 3} $&$ 0 $&$ 0 $&$ {{8\,N}\over {27}} + {{16\,\left( 
  u+d/4 \right) }\over {27}} $&$ 0 $&$ {{8\,N}\over {27}} + {{16\,\left( 
  u+d/4 \right) }\over {27}} $&$ -{4\over 3} $
\end{tabular}
\label{tab:gem}
\end{table}

\begin{table}[htb]
\caption[]{The NLO anomalous dimension matrix $\gse$ for $N=3$ and NDR.}
\begin{tabular}{|r|c|c|c|c|c|}
$(i,j)$ & 1 & 2 & 3 & 4 & 5 \\
\hline
$1 $&$ {{194}\over 9} $&$ -{2\over 3} $&$ -{{88}\over {243}} $&$ {{88}\over {81}} $&$\
  -{{88}\over {243}} $\\ \svs 
$2 $&$ {{25}\over 3} $&$ -{{49}\over 9} $&$ -{{556}\over {729}} $&$ {{556}\over {243}} $&$\
  -{{556}\over {729}} $\\ \svs 
$3 $&$ 0 $&$ 0 $&$ {{1690}\over {729}} - {{136 \left( u - d/2 \right) }\over {243}} $&$\
  -{{1690}\over {243}} + {{136 \left( u - d/2 \right) }\over {81}} $&$\
  {{232}\over {729}} - {{136 \left( u - d/2 \right) }\over {243}} $\\ \svs 
$4 $&$ 0 $&$ 0 $&$ -{{641}\over {243}} - {{388 u}\over {729}} + {{32 d}\over {729}} $&$\
  -{{655}\over {81}} + {{388 u}\over {243}} - {{32 d}\over {243}} $&$\
  {{88}\over {243}} - {{388 u}\over {729}} + {{32 d}\over {729}} $\\ \svs 
$5 $&$ 0 $&$ 0 $&$ {{-136 \left( u - d/2 \right) }\over {243}} $&$ {{136 \left( u - d/2\
  \right) }\over {81}} $&$ -2 - {{136 \left( u - d/2 \right) }\over {243}} $\\ \svs 
$6 $&$0 $&$ 0 $&$ {{-748 u}\over {729}} + {{212 d}\over {729}} $&$ {{748 u}\over {243}} -\
  {{212 d}\over {243}} $&$ 3 - {{748 u}\over {729}} + {{212 d}\over {729}} $\\ \svs 
$7 $&$ 0 $&$ 0 $&$ {{-136 \left( u + d/4 \right) }\over {243}} $&$ {{136 \left( u + d/4\
  \right) }\over {81}} $&$ -{{116}\over 9} - {{136 \left( u + d/4 \right)\
  }\over {243}} $\\ \svs 
$8 $&$ 0 $&$ 0 $&$ {{-748 u}\over {729}} - {{106 d}\over {729}} $&$ {{748 u}\over {243}} +\
  {{106 d}\over {243}} $&$ -1 - {{748 u}\over {729}} - {{106 d}\over {729}} $\\ \svs 
$9 $&$ 0 $&$ 0 $&$ {{7012}\over {729}} - {{136 \left( u + d/4 \right) }\over {243}} $&$\
  {{764}\over {243}} + {{136 \left( u + d/4 \right) }\over {81}} $&$\
  -{{116}\over {729}} - {{136 \left( u + d/4 \right) }\over {243}} $\\ \svs 
$10 $&$ 0 $&$ 0 $&$ {{1333}\over {243}} - {{388 u}\over {729}} - {{16 d}\over {729}} $&$\
  {{107}\over {81}} + {{388 u}\over {243}} + {{16 d}\over {243}} $&$\
  -{{44}\over {243}} - {{388 u}\over {729}} - {{16 d}\over {729}} $
\end{tabular}

\begin{tabular}{|r|c|c|c|c|c|}
$(i,j)$ & 6 & 7 & 8 & 9 & 10 \\
\hline
$1 $&$ {{88}\over {81}} $&$ {{152}\over {27}} $&$ {{40}\over 9} $&$ {{136}\over {27}} $&$\
  {{56}\over 9} $\\ \svs 
$2 $&$ {{556}\over {243}} $&$ -{{484}\over {729}} $&$ -{{124}\over {27}} $&$ -{{3148}\over\
  {729}} $&$ {{172}\over {27}} $\\ \svs 
$3 $&$ -{{232}\over {243}} + {{136 \left( u - d/2 \right) }\over {81}} $&$\
  {{3136}\over {729}} + {{104 \left( u - d/2 \right) }\over {27}} $&$\
  {{64}\over {27}} + {{88 \left( u - d/2 \right) }\over 9} $&$ {{20272}\over\
  {729}} + {{184 \left( u - d/2 \right) }\over {27}} $&$ -{{112}\over {27}} +\
  {{8 \left( u - d/2 \right) }\over 9} $\\ \svs 
$4 $&$ -{{88}\over {81}} + {{388 u}\over {243}} - {{32 d}\over {243}} $&$ -{{152}\over\
  {27}} + {{3140 u}\over {729}} + {{656 d}\over {729}} $&$ -{{40}\over 9} -\
  {{100 u}\over {27}} - {{16 d}\over {27}} $&$ {{170}\over {27}} + {{908\
  u}\over {729}} + {{1232 d}\over {729}} $&$ -{{14}\over 3} + {{148 u}\over\
  {27}} - {{80 d}\over {27}} $\\ \svs 
$5 $&$ 6 + {{136 \left( u - d/2 \right) }\over {81}} $&$ -{{232}\over 9} + {{104\
  \left( u - d/2 \right) }\over {27}} $&$ {{40}\over 3} + {{88 \left( u - d/2\
  \right) }\over 9} $&$ {{184 \left( u - d/2 \right) }\over {27}} $&$ {{8 \left(\
  u - d/2 \right) }\over 9} $\\ \svs 
$6 $&$ 7 + {{748 u}\over {243}} - {{212 d}\over {243}} $&$ -2 - {{5212 u}\over {729}}\
  + {{4832 d}\over {729}} $&$ {{182}\over 9} + {{188 u}\over {27}} - {{160\
  d}\over {27}} $&$ {{-2260 u}\over {729}} + {{2816 d}\over {729}} $&$ {{-140\
  u}\over {27}} + {{64 d}\over {27}} $\\ \svs 
$7 $&$ {{20}\over 3} + {{136 \left( u + d/4 \right) }\over {81}} $&$ -{{134}\over 9} +\
  {{104 \left( u + d/4 \right) }\over {27}} $&$ {{38}\over 3} + {{88 \left( u +\
  d/4 \right) }\over 9} $&$ {{184 \left( u + d/4 \right) }\over {27}} $&$ {{8\
  \left( u + d/4 \right) }\over 9} $\\ \svs 
$8 $&$ {{91}\over 9} + {{748 u}\over {243}} + {{106 d}\over {243}} $&$ 2 - {{5212\
  u}\over {729}} - {{2416 d}\over {729}} $&$ {{154}\over 9} + {{188 u}\over\
  {27}} + {{80 d}\over {27}} $&$ {{-2260 u}\over {729}} - {{1408 d}\over {729}}\
  $&$ {{-140 u}\over {27}} - {{32 d}\over {27}} $\\ \svs 
$9 $&$ {{116}\over {243}} + {{136 \left( u + d/4 \right) }\over {81}} $&$\
  -{{1568}\over {729}} + {{104 \left( u + d/4 \right) }\over {27}} $&$\
  -{{32}\over {27}} + {{88 \left( u + d/4 \right) }\over 9} $&$ {{5578}\over\
  {729}} + {{184 \left( u + d/4 \right) }\over {27}} $&$ {{38}\over {27}} + {{8\
  \left( u + d/4 \right) }\over 9} $\\ \svs 
$10 $&$ {{44}\over {81}} + {{388 u}\over {243}} + {{16 d}\over {243}} $&$ {{76}\over\
  {27}} + {{3140 u}\over {729}} - {{328 d}\over {729}} $&$ {{20}\over 9} -\
  {{100 u}\over {27}} + {{8 d}\over {27}} $&$ {{140}\over {27}} + {{908 u}\over\
  {729}} - {{616 d}\over {729}} $&$ -{{28}\over 9} + {{148 u}\over {27}} + {{40\
  d}\over {27}} $
\end{tabular}
\label{tab:gse}
\end{table}

\subsection{Quark Threshold Matching Matrix}
            \label{sec:HeffdF1:1010:Mm}
Extending the matching matrix $\hM(m)$ of \eqn{eq:MmKpp} to the
simultaneous presence of QCD and QED corrections yields
\begin{equation}
\hM(m) = 1 + \frac{\as(m)}{4\pi} \; \vardrs^T +
         \frac{\aem}{4\pi} \; \vardre^T \, .
\label{eq:MmdF1:1010}
\end{equation}
At scale $\mu=\mb$ the matrices $\vardrs$ and $\vardre$ read
\begin{equation}\nonumber
\vardrs^T=\frac{5}{18} P \, (0,0,0,-2,0,-2,0,1,0,1)
\end{equation}
\begin{equation}
\vardre^T=\frac{10}{81} \bar{P} \, (0,0,6,2,6,2,-3,-1,-3,-1)
\label{eq:drsembdF1:1010}
\end{equation}
and at $\mu=\mc$
\begin{equation}\nonumber
\vardrs^T=-\frac{5}{9} P \, (0,0,0,1,0,1,0,1,0,1)
\end{equation}
\begin{equation}
\vardre^T=-\frac{40}{81} \bar{P} \, (0,0,3,1,3,1,3,1,3,1)
\label{eq:drsemcdF1:1010}
\end{equation}
with eq.\ \eqn{eq:PdrsKpp} generalized to
\begin{eqnarray}
P^T &=& (0,0,-\frac{1}{3},1,-\frac{1}{3},1,0,0,0,0) \, ,
\label{eq:Pdrs:10} \\
\bar{P}^T &=& (0,0,0,0,0,0,1,0,1,0) \, .
\label{eq:Pbardre:10}
\end{eqnarray}

\subsection{Numerical Results for the $\Kpipi$ Wilson Coefficients}
            \label{sec:HeffdF1:1010:numres}
\begin{table}[htb]
\caption[]{$\dS$ Wilson coefficients at $\mu=1\gev$ for $\mt=170\gev$.
% $|z_3|,\ldots,|z_{10}|$ are numerically much smaller than $|z_{1,2}|$
% because their RG evolution only starts at $\mu=\mc$.
$y_1 = y_2 \equiv 0$.
\label{tab:wc10smu1}}
\begin{center}
\begin{tabular}{|c|c|c|c||c|c|c||c|c|c|}
& \multicolumn{3}{c||}{$\Lms^{(4)}=215\mev$} &
  \multicolumn{3}{c||}{$\Lms^{(4)}=325\mev$} &
  \multicolumn{3}{c| }{$\Lms^{(4)}=435\mev$} \\
\hline
Scheme & LO & NDR & HV & LO & 
NDR & HV & LO & NDR & HV \\
\hline
$z_1$ & --0.607 & --0.409 & --0.494 & --0.748 & 
--0.509 & --0.640 & --0.907 & --0.625 & --0.841 \\
$z_2$ & 1.333 & 1.212 & 1.267 & 1.433 & 
1.278 & 1.371 & 1.552 & 1.361 & 1.525 \\
\hline
$z_3$ & 0.003 & 0.008 & 0.004 & 0.004 & 
0.013 & 0.007 & 0.006 & 0.023 & 0.015 \\
$z_4$ & --0.008 & --0.022 & --0.010 & --0.012 & 
--0.035 & --0.017 & --0.017 & --0.058 & --0.029 \\
$z_5$ & 0.003 & 0.006 & 0.003 & 0.004 & 
0.008 & 0.004 & 0.005 & 0.009 & 0.005 \\
$z_6$ & --0.009 & --0.022 & --0.009 & --0.013 & 
--0.035 & --0.014 & --0.018 & --0.059 & --0.025 \\
\hline
$z_7/\aem$ & 0.004 & 0.003 & --0.003 & 0.008 & 
0.011 & --0.002 & 0.011 & 0.021 & --0.001 \\
$z_8/\aem$ & 0 & 0.008 & 0.006 & 0.001 & 
0.014 & 0.010 & 0.001 & 0.027 & 0.017 \\
$z_9/\aem$ & 0.005 & 0.007 & 0 & 0.008 & 
0.018 & 0.005 & 0.012 & 0.034 & 0.011 \\
$z_{10}/\aem$ & 0 & --0.005 & --0.006 & --0.001 & 
--0.008 & --0.010 & --0.001 & --0.014 & --0.017 \\
\hline
$y_3$ & 0.030 & 0.025 & 0.028 & 0.038 & 
0.032 & 0.037 & 0.047 & 0.042 & 0.050 \\
$y_4$ & --0.052 & --0.048 & --0.050 & --0.061 & 
--0.058 & --0.061 & --0.071 & --0.068 & --0.074 \\
$y_5$ & 0.012 & 0.005 & 0.013 & 0.013 & 
--0.001 & 0.016 & 0.014 & --0.013 & 0.021 \\
$y_6$ & --0.085 & --0.078 & --0.071 & --0.113 & 
--0.111 & --0.097 & --0.148 & --0.169 & --0.139 \\
\hline
$y_7/\aem$ & 0.027 & --0.033 & --0.032 & 0.036 & 
--0.032 & --0.030 & 0.043 & --0.031 & --0.027 \\
$y_8/\aem$ & 0.114 & 0.121 & 0.133 & 0.158 & 
0.173 & 0.188 & 0.216 & 0.254 & 0.275 \\
$y_9/\aem$ & --1.491 & --1.479 & --1.480 & --1.585 & 
--1.576 & --1.577 & --1.700 & --1.718 & --1.722 \\
$y_{10}/\aem$ & 0.650 & 0.540 & 0.547 & 0.800 & 
0.690 & 0.699 & 0.968 & 0.892 & 0.906 \\
\end{tabular}
\end{center}
\end{table}


\begin{table}[htb]
\caption[]{$\dS$ Wilson coefficients at $\mu=\mc=1.3\gev$ for
$\mt=170\gev$ and $f=3$ effective flavours.
$|z_3|,\ldots,|z_{10}|$ are numerically irrelevant relative to
$|z_{1,2}|$. $y_1 = y_2 \equiv 0$.
\label{tab:wc10smu13}}
\begin{center}
\begin{tabular}{|c|c|c|c||c|c|c||c|c|c|}
& \multicolumn{3}{c||}{$\Lms^{(4)}=215\mev$} &
  \multicolumn{3}{c||}{$\Lms^{(4)}=325\mev$} &
  \multicolumn{3}{c| }{$\Lms^{(4)}=435\mev$} \\
\hline
Scheme & LO & NDR & HV & LO & 
NDR & HV & LO & NDR & HV \\
\hline
$z_1$ & --0.521 & --0.346 & --0.413 & --0.625 & 
--0.415 & --0.507 & --0.732 & --0.490 & --0.617 \\
$z_2$ & 1.275 & 1.172 & 1.214 & 1.345 & 
1.216 & 1.276 & 1.420 & 1.265 & 1.354 \\
\hline
$y_3$ & 0.027 & 0.023 & 0.025 & 0.034 & 
0.029 & 0.033 & 0.041 & 0.036 & 0.042 \\
$y_4$ & --0.051 & --0.048 & --0.049 & --0.061 & 
--0.057 & --0.060 & --0.070 & --0.068 & --0.072 \\
$y_5$ & 0.013 & 0.007 & 0.014 & 0.015 & 
0.005 & 0.016 & 0.017 & 0.001 & 0.018 \\
$y_6$ & --0.076 & --0.068 & --0.063 & --0.096 & 
--0.089 & --0.081 & --0.120 & --0.118 & --0.103 \\
\hline
$y_7/\aem$ & 0.030 & --0.031 & --0.031 & 0.039 & 
--0.030 & --0.028 & 0.048 & --0.029 & --0.026 \\
$y_8/\aem$ & 0.092 & 0.103 & 0.112 & 0.121 & 
0.136 & 0.145 & 0.155 & 0.179 & 0.189 \\
$y_9/\aem$ & --1.428 & --1.423 & --1.423 & --1.490 & 
--1.479 & --1.479 & --1.559 & --1.548 & --1.549 \\
$y_{10}/\aem$ & 0.558 & 0.451 & 0.457 & 0.668 & 
0.547 & 0.553 & 0.781 & 0.656 & 0.664 \\
\end{tabular}
\end{center}
\end{table}


\begin{table}[htb]
\caption[]{$\dS$ Wilson coefficients at $\mu=2\gev$ for
$\mt=170\gev$. For $\mu > \mc$ the GIM mechanism gives $z_i \equiv 0$,
$i=3,\ldots,10$. $y_1 = y_2 \equiv 0$.
\label{tab:wc10smu2}}
\begin{center}
\begin{tabular}{|c|c|c|c||c|c|c||c|c|c|}
& \multicolumn{3}{c||}{$\Lms^{(4)}=215\mev$} &
  \multicolumn{3}{c||}{$\Lms^{(4)}=325\mev$} &
  \multicolumn{3}{c| }{$\Lms^{(4)}=435\mev$} \\
\hline
Scheme & LO & NDR & HV & LO & 
NDR & HV & LO & NDR & HV \\
\hline
$z_1$ & --0.413 & --0.268 & --0.320 & --0.480 & 
--0.310 & --0.376 & --0.544 & --0.352 & --0.432 \\
$z_2$ & 1.206 & 1.127 & 1.157 & 1.248 & 
1.151 & 1.191 & 1.290 & 1.176 & 1.227 \\
\hline
$y_3$ & 0.021 & 0.020 & 0.019 & 0.025 & 
0.024 & 0.023 & 0.028 & 0.028 & 0.027 \\
$y_4$ & --0.041 & --0.046 & --0.040 & --0.047 & 
--0.055 & --0.046 & --0.053 & --0.063 & --0.053 \\
$y_5$ & 0.011 & 0.010 & 0.012 & 0.012 & 
0.011 & 0.013 & 0.014 & 0.011 & 0.015 \\
$y_6$ & --0.056 & --0.058 & --0.047 & --0.068 & 
--0.071 & --0.057 & --0.079 & --0.086 & --0.068 \\
\hline
$y_7/\aem$ & 0.031 & --0.023 & --0.020 & 0.037 & 
--0.019 & --0.020 & 0.042 & --0.016 & --0.019 \\
$y_8/\aem$ & 0.068 & 0.076 & 0.084 & 0.084 & 
0.094 & 0.102 & 0.101 & 0.113 & 0.121 \\
$y_9/\aem$ & --1.357 & --1.361 & --1.357 & --1.393 & 
--1.389 & --1.389 & --1.430 & --1.419 & --1.423 \\
$y_{10}/\aem$ & 0.442 & 0.356 & 0.360 & 0.513 & 
0.414 & 0.419 & 0.581 & 0.472 & 0.477 \\
\end{tabular}
\end{center}
\end{table}

Tables \ref{tab:wc10smu1}--\ref{tab:wc10smu2} give the $\dS$ Wilson
coefficients for $Q_1,\ldots,Q_{10}$ in the mixed case of QCD and QED.
\\
The coefficients for the current-current and QCD penguin operators
$Q_1,\ldots,Q_6$ are only very weakly affected by the extension of the
operator basis to the electroweak penguin operators $Q_7,\ldots,Q_{10}$.
Therefore the discussion for $Q_1,\ldots,Q_6$ given in connection with
tables \ref{tab:wc6smu1}--\ref{tab:wc6smu2} for the case of pure
QCD basically still holds and will not be repeated here.
\\
For the remaining coefficients of $Q_7,\ldots,Q_{10}$ one finds a
moderate scheme dependence for $y_7$, $y_9$ and $y_{10}$, but a $\ord(9\%)$
one for $y_8$. The notable feature of $|y_6|$ being larger in NDR than in
HV still holds, but is now confronted with an exactly opposite
dependence for the other important $\dS$ Wilson coefficient $y_8$ which
is in addition enhanced over its LO value.
\\
The particular dependence of $y_6$ and $y_8$ with respect to scheme,
LO/NLO and $\mt$ (see below) should be kept in mind for the later
discussion of $\epe$ in section \ref{sec:nloepe}.
\\
We also note that in the range of $\mt$ considered here, $y_7$ is very
small, $y_9$ is essentially unaffected by NLO QCD corrections and
$y_{10}$ is suppressed for $\mu \ge \mc$. It should also be stressed
that $|y_9|$  and $|y_{10}|$ are substantially larger than $|y_8|$
although, as we will see in the analysis of $\epe$, the operator $Q_8$
is more important than $Q_9$ and $Q_{10}$ for this ratio.
\\
Next, one infers from tables \ref{tab:wc10smu1}--\ref{tab:wc10smu2} 
that also in the mixed QCD/QED case the Wilson coefficients show a
strong dependence on $\Lms$.
\\
In contrast to the coefficients $y_3,\ldots,y_6$ for QCD penguins,
$y_7,\ldots,y_{10}$ for the electroweak penguins show a sizeable
$\mt$ dependence in the range $\mt = (170 \pm 15)\gev$. With
in/decreasing $\mt$ there is a relative variation of $\ord(\pm 19\%)$
and $\ord(\pm 10\%)$ for the absolute values of $y_8$ and $y_{9,10}$,
respectively.
This is illustrated further in figs.\ \ref{fig:dF1:mty78} and
\ref{fig:dF1:mty910} where the $\mt$ dependence of these coefficients is
shown explicitly. This strong $\mt$-dependence originates in the
$Z^0$-penguin diagrams.
The $\mt$-dependence of $y_9$ and $y_{10}$ can be conveniently
parametrized by a linear function to an accuracy better than $0.5\,\%$.
Details of this $\mt$-parametriziation can be found in table
\ref{tab:linfity910}.
\\
Finally, in tables \ref{tab:wc10smu1}--\ref{tab:wc10smu2} one
observes again the usual feature of decreasing Wilson coefficients
with increasing scale $\mu$.

\begin{table}[htb]
\caption[]{Coefficients in linear $\mt$-parametriziation $y_i/\aem = a
+ b \cdot (\mt/170\gev)$ of Wilson coefficients $y_9/\aem$ and
$y_{10}/\aem$ at scale $\mu=\mc$ for $\Lms^{(4)}=325\mev$.
\label{tab:linfity910}}
\begin{center}
\begin{tabular}{|c||c|c||c|c|}
\hline
 & \multicolumn{2}{|c||}{$y_9/\aem$} &
   \multicolumn{2}{|c| }{$y_{10}/\aem$} \\
\hline
 & a & b & a & b \\
\hline
LO  & 0.189 & --1.682 & --0.111 & 0.780 \\
NDR & 0.129 & --1.611 & --0.128 & 0.676 \\
HV  & 0.129 & --1.611 & --0.121 & 0.676 \\
\hline
\end{tabular}
\end{center}
\end{table}

\begin{figure}[htb]
\vspace{0.10in}
\centerline{
\epsfysize=7in
\rotate[r]{
\epsffile{ps/mty78.ps}
}}
\vspace{0.08in}
\caption[]{
Wilson coefficients $y_{7}(\mc)/\aem$ and $y_{8}(\mc)/\aem$ as
functions of $\mt$ for $\Lms^{(4)}=325\mev$.
\label{fig:dF1:mty78}}
\end{figure}
\begin{figure}[htb]
\vspace{0.10in}
\centerline{
\epsfysize=7in
\rotate[r]{
\epsffile{ps/mty910.ps}
}}
\vspace{0.08in}
\caption[]{
Wilson coefficients $y_{9}(\mc)/\aem$ and $y_{10}(\mc)/\aem$ as a
function of $\mt$ for $\Lms^{(4)}=325\mev$.
\label{fig:dF1:mty910}}
\end{figure}

\subsection{The $\dB$ Effective Hamiltonian Including Electroweak
            Penguins}
            \label{sec:HeffdF1:1010:dB1}
Finally we present in this section the Wilson coefficient functions of
the $\Delta B=1$, $\Delta C=0$ hamiltonian, including the effects of
electroweak penguin contributions \cite{burasetal:92d}.  These effects
play a role in certain penguin-induced B meson decays as discussed in
\cite{fleischer:94a}, \cite{fleischer:94b}, \cite{deshpandeetal:94},
\cite{deshpandehe:94}.

The generalization of the $\Delta B=1$, $\Delta C=0$ hamiltonian in pure
QCD (\ref{eq:HeffdB1:66}) to incorporate also electroweak penguin
operators is straightforward. One obtains
\begin{eqnarray} 
\Heff(\dB) &=& \frac{G_F}{\sqrt{2}} \bigl\{
   \xi_c \, \left[ C_1(\mu) Q_1^c(\mu) + C_2(\mu) Q_2^c(\mu) \right] +
   \xi_u \, \left[ C_1(\mu) Q_1^u(\mu) + C_2(\mu) Q_2^u(\mu) \right] 
\nn \\
 & & - \xi_t \, \sum_{i=3}^{10} C_i(\mu) Q_i(\mu)
\bigr\} \, .
\label{eq:HeffdB1:1010}
\end{eqnarray} 
where the operator basis now includes the electroweak penguin
operators
\begin{eqnarray}
Q_{7} & = & \frac{3}{2} \left( \bar b d \right)_{\rm V-A}
         \sum_{q} e_{q} \left( \bar q q \right)_{\rm V+A}
\, , \nn \\
Q_{8} & = & \frac{3}{2} \left( \bar b_{i} d_{j} \right)_{\rm V-A}
         \sum_{q} e_{q} \left( \bar q_{j}  q_{i}\right)_{\rm V+A}
\, , \nn \\
Q_{9} & = & \frac{3}{2} \left( \bar b d \right)_{\rm V-A}
         \sum_{q} e_{q} \left( \bar q q \right)_{\rm V-A}
\, , \label{eq:dB1:1010basis} \\
Q_{10}& = & \frac{3}{2} \left( \bar b_{i} d_{j} \right)_{\rm V-A}
         \sum_{q} e_{q} \left( \bar q_{j}  q_{i}\right)_{\rm V-A}
\nn
\end{eqnarray}
in addition to (\ref{eq:dB1basis}). The Wilson coefficients at $\mu=m_b$
read
\begin{equation}
\vC(\mb) = \hU_5(\mb,\mw,\aem) \, \vC(\mw) \, .
\label{eq:HeffdB1:1010:wc}
\end{equation}
where $\hU_5$ is the $10\times 10$ evolution matrix of
(\ref{eq:UdF1:1010}) for $f=5$ flavors. The $\vC(\mw)$ are given in
(\ref{eq:CMw1}) -- (\ref{eq:CMw10}) in the NDR scheme.

\subsection{Numerical Results for the $\dB$ Wilson Coefficients}
            \label{sec:HeffdF1:1010:dB1num}
\begin{table}[htb]
\caption[]{$\dB$ Wilson coefficients at $\mu=\overline{m}_{\rm b}(\mb)=
4.40\gev$ for $\mt=170\gev$.
\label{tab:wc10b}}
\begin{center}
\begin{tabular}{|c|c|c|c||c|c|c||c|c|c|}
& \multicolumn{3}{c||}{$\Lms^{(5)}=140\mev$} &
  \multicolumn{3}{c||}{$\Lms^{(5)}=225\mev$} &
  \multicolumn{3}{c| }{$\Lms^{(5)}=310\mev$} \\
\hline
Scheme & LO & NDR & HV & LO & 
NDR & HV & LO & NDR & HV \\
\hline
$C_1$ & --0.273 & --0.165 & --0.202 & --0.308 & 
--0.185 & --0.228 & --0.339 & --0.203 & --0.251 \\
$C_2$ & 1.125 & 1.072 & 1.091 & 1.144 & 
1.082 & 1.105 & 1.161 & 1.092 & 1.117 \\
\hline
$C_3$ & 0.013 & 0.013 & 0.012 & 0.014 & 
0.014 & 0.013 & 0.016 & 0.016 & 0.015 \\
$C_4$ & --0.027 & --0.031 & --0.026 & --0.030 & 
--0.035 & --0.029 & --0.033 & --0.039 & --0.033 \\
$C_5$ & 0.008 & 0.008 & 0.008 & 0.009 & 
0.009 & 0.009 & 0.009 & 0.009 & 0.010 \\
$C_6$ & --0.033 & --0.036 & --0.029 & --0.038 & 
--0.041 & --0.033 & --0.043 & --0.046 & --0.037 \\
\hline
$C_7/\aem$ & 0.042 & --0.003 & 0.006 & 0.045 & 
--0.002 & 0.005 & 0.047 & --0.001 & 0.005 \\
$C_8/\aem$ & 0.041 & 0.047 & 0.052 & 0.048 & 
0.054 & 0.060 & 0.054 & 0.061 & 0.067 \\
$C_9/\aem$ & --1.264 & --1.279 & --1.269 & --1.280 & 
--1.292 & --1.283 & --1.294 & --1.303 & --1.296 \\
$C_{10}/\aem$ & 0.291 & 0.234 & 0.237 & 0.328 & 
0.263 & 0.266 & 0.360 & 0.288 & 0.291 \\
\end{tabular}
\end{center}
\end{table}

Table \ref{tab:wc10b} lists the $\dB$ Wilson coefficients for
$Q_1^{u,c},Q_2^{u,c},Q_3,\ldots,Q_{10}$ in the mixed case of QCD and
QED.
\\
Similarly to the $\dS$ case the coefficients for the current-current and
QCD penguin operators $Q_1,\ldots,Q_6$ are only very weakly affected by
the extension of the operator basis to the electroweak penguin
operators $Q_7,\ldots,Q_{10}$. Therefore the discussion of
$C_1,\ldots,C_6$ in connection with table \ref{tab:wc6b} is also valid
for the present case.
\\
Here we therefore restrict our discussion to the coefficients
$C_7,\ldots,C_{10}$ of the operators $Q_7,\ldots,Q_{10}$ in the
extended basis.
\\
The coefficients $C_7,\ldots,C_{10}$ show a visible dependence on the
scheme, $\Lms$ and LO/NLO. However, this dependence is less pronounced
for the coefficient $C_9$ than it is for $C_{7,8,10}$. This is
noteworthy since in $B$-meson decays $C_9$ usually resides in the
dominant electroweak penguin contribution \cite{fleischer:94a},
\cite{fleischer:94b}, \cite{deshpandeetal:94}, \cite{deshpandehe:94}.
\\
In contrast to $C_1,\ldots,C_6$ the additional coefficients
$C_7,\ldots,C_{10}$ show a non negligible $\mt$ dependence in the range
$\mt = (170 \pm 15)\gev$. With in/decreasing $\mt$ there is similarly to
the $\dS$ case a relative variation of $\ord(\pm 19\%)$ and $\ord(\pm
10\%)$ for the absolute values of $C_8$ and $C_{9,10}$, respectively.

Since the coefficients $C_9$ and $C_{10}$ play an important role in
B decays we show in fig.\ \ref{fig:dF1:mtC910} their $\mt$ dependence
explicitly. Again the $\mt$-dependence can be parametrized by a linear
function to an accuracy better than $0.5\,\%$. Details of the
$\mt$-parametriziation are given in table \ref{tab:linfitC910}.

\begin{table}[htb]
\caption[]{Coefficients in linear $\mt$-parametriziation $C_i/\aem = a
+ b \cdot (\mt/170\gev)$ of Wilson coefficients $C_9/\aem$ and
$C_{10}/\aem$ at scale $\mu=\mb=4.4\gev$ for $\Lms^{(5)}=225\mev$.
\label{tab:linfitC910}}
\begin{center}
\begin{tabular}{|c||c|c||c|c|}
\hline
 & \multicolumn{2}{|c||}{$C_9/\aem$} &
   \multicolumn{2}{|c| }{$C_{10}/\aem$} \\
\hline
 & a & b & a & b \\
\hline
LO  & 0.152 & --1.434 & --0.056 & 0.385 \\
NDR & 0.109 & --1.403 & --0.065 & 0.328 \\
HV  & 0.117 & --1.403 & --0.062 & 0.328 \\
\hline
\end{tabular}
\end{center}
\end{table}
\begin{figure}[htb]
\vspace{0.10in}
\centerline{
\epsfysize=7in
\rotate[r]{
\epsffile{ps/mtC910.ps}
}}
\vspace{0.08in}
\caption[]{
Wilson coefficients $C_{9}/\aem$ and $C_{10}/\aem$ at $\mu=
\overline{m}_{\rm b}(\mb)= 4.40\gev$ as a function of $\mt$ for
$\Lms^{(5)}=225\mev$.
\label{fig:dF1:mtC910}}
\end{figure}
