\section{The Decays $B\to X\nu\bar\nu$ and $B\to\mu^+\mu^-$}
\label{sec:BXnnBmm}
\subsection{General Remarks}
\label{sec:BXnnBmm:General}
The rare decays $B\to X_{s}\nu\bar\nu$, $B\to X_{d}\nu\bar\nu$ and
$B_{s}\to\mu^+\mu^-$, $B_{d}\to\mu^+\mu^-$ are fully dominated by
internal top quark contributions. The relevant effective hamiltonians
are given in (\ref{hxnu}) and (\ref{hyll}) respectively.  Only the top
functions $X(x_t)$ and $Y(x_t)$ enter these expressions and the
uncertainties due to $m_c$ and $\Lambda_{\overline{MS}}$ affecting
\kpnn and \klmm are absent here.  Consequently these two decays are
theoretically very clean. In particular the residual renormalization
scale dependence of the relevant branching ratios, though sizable in
leading order, can essentially be neglected after the inclusion of
next-to-leading order corrections. On the other hand a measurement of
these rare B decays, in particular of $B\to X_{s}\nu\bar\nu$ and $B\to
X_{d}\nu\bar\nu$ , is experimentally very challenging. In addition, as
we will see below, $B(B_{s}\to\mu^+\mu^-)$ and $B(B_{d}\to\mu^+\mu^-)$
is subject to the uncertainties in the values of the B meson decay
constants $F_{B_{s}}$ and $F_{B_{d}}$, which hopefully will be removed
one day.

\subsection{The Decays $B\to X_{s}\nu\bar\nu$ and $B\to X_{d}\nu\bar\nu$}
\label{sec:BXnnBmm:BXnn}
The branching fraction for $B\to X_s\nu\bar\nu$ is given by
\begin{equation}\label{bbxnn}
\frac{B(B\to X_s\nu\bar\nu)}{B(B\to X_c e\bar\nu)}=
\frac{3 \alpha^2}{4\pi^2\sin^4\Theta_W}
\frac{|V_{ts}|^2}{|V_{cb}|^2}\frac{X^2(x_t)}{f(z)}
\frac{\bar\eta}{\kappa(z)}
\end{equation}
Here $f(z)$, $z=m_c/m_b$ is the phase-space factor for $B\to X_c
e\bar\nu$ defined already in \eqn{g} and $\kappa(z)$ is the
corresponding QCD correction \cite{CM:78} given in \eqn{eq:kappaz}. The
factor $\bar\eta$ represents the QCD correction to the matrix element
of the $b\to s\nu\bar\nu$ transition due to virtual and bremsstrahlung
contributions and is given by the well known expression
\begin{equation}\label{etabar}
\bar\eta=\kappa(0)=
1+\frac{2\alpha_s(m_b)}{3\pi}\left(\frac{25}{4}-\pi^2\right)
\approx 0.83
\end{equation}
For the numerical analysis we will use $\Lambda^{(5)}_{QCD}=225\mev$,
(\ref{alsinbr}), $|V_{ts}|=|V_{cb}|$, $m_t(m_t)=170\gev$, $B(B\to X_c
e\bar\nu)=0.104$, $f(z)=0.49$ and $\kappa(z)=0.88$, keeping in mind
the QCD uncertainties in $B\to X_c e\bar\nu$ discussed in section
\ref{sec:InclB}.

Varying $\mu_t$ as in (\ref{muctnum}) we find that the ambiguity
\begin{equation}\label{lobxn}
3.82\cdot 10^{-5}\leq B(B\to X_s\nu\bar\nu)\leq 4.65\cdot 10^{-5}
\end{equation}
present in the leading order is reduced to
\begin{equation}\label{nlobxn}
3.99\cdot 10^{-5}\leq B(B\to X_s\nu\bar\nu)\leq 4.09\cdot 10^{-5}
\end{equation}
after the inclusion of QCD corrections \cite{buchallaburas:93b}.

It should be noted that $B(B \to X_s \nu\bar\nu)$ as given in
\eqn{bbxnn} is in view of $|V_{ts}/V_{cb}|^2 \approx 0.95 \pm 0.03$
essentially independent of the CKM parameters and the main uncertainty
resides in the value of $\mt$. Setting all parameters as given above and
in appendix \ref{app:numinput}, and using \eqn{xxappr} we have
\begin{equation}
B(B \to X_s \nu\bar\nu) = 4.1 \cdot 10^{-5} \,
\frac{|V_{ts}|^2}{|V_{cb}|^2} \,
\left[ \frac{\mt(\mt)}{170\gev} \right]^{2.30} \, .
\label{eq:bxsnnnum}
\end{equation}
In view of a new interest in this decay \cite{grossmanetal:95} we quote
the Standard Model expectation for $B(B \to X_s \nu\bar\nu)$ based on
the input parameters collected in the appendix \ref{app:numinput}. We
find
\begin{equation}
3.1 \cdot 10^{-5} \le B(B \to X_s \nu\bar\nu) \le 4.9 \cdot 10^{-5}
\label{eq:bxsnnnum2}
\end{equation}
for the ``present day'' uncertainties in the input parameters and
\begin{equation}
3.6 \cdot 10^{-5} \le B(B \to X_s \nu\bar\nu) \le 4.2 \cdot 10^{-5}
\label{eq:bxsnnnum3}
\end{equation}
for our ``future'' scenario. 

In the case of $B\to X_d\nu\bar\nu$ one has to replace $V_{ts}$ by
$V_{td}$ which results in a decrease of the branching ratio by
roughly an order of magnitude.

\subsection{The Decays $B_{s}\to\mu^+\mu^-$ and $B_{d}\to\mu^+\mu^-$}
\label{sec:BXnnBmm:Bmm}
The branching ratio for $B_s\to l^+l^-$ is given by \cite{buchallaburas:93b}
\begin{equation}\label{bbll}
B(B_s\to l^+l^-)=\tau(B_s)\frac{G^2_F}{\pi}
\left(\frac{\alpha}{4\pi\sin^2\Theta_W}\right)^2 F^2_{B_s}m^2_l m_{B_s}
\sqrt{1-4\frac{m^2_l}{m^2_{B_s}}} |V^\ast_{tb}V_{ts}|^2 Y^2(x_t)
\end{equation}
where $B_s$ denotes the flavor eigenstate $(\bar bs)$ and $F_{B_s}$ is
the corresponding decay constant (normalized as $F_\pi=131\mev$). Using
(\ref{alsinbr}), (\ref{yeta}) and \eqn{eq:approxSXYZE} we find in the
case of $B_s\to\mu^+\mu^-$
\begin{equation}\label{bbmmnum}
B(B_s\to\mu^+\mu^-)=4.18\cdot 10^{-9}\left[\frac{\tau(B_s)}{1.6 ps}\right]
\left[\frac{F_{B_s}}{230\mev}\right]^2 
\left[\frac{|V_{ts}|}{0.040}\right]^2 
\left[\frac{m_t(m_t)}{170\gev}\right]^{3.12}
\end{equation}
which approximates the next-to-leading order result.
\\
Taking the central values for $\tau(B_s)$, $F_{B_s}$, $|V_{ts}|$ and
$m_t(m_t)$ and varying $\mu_t$ as in (\ref{muctnum}) we find that the
uncertainty
\begin{equation}\label{lobmm}
3.44\cdot 10^{-9}\leq B(B_s\to\mu^+\mu^-)\leq 4.50\cdot 10^{-9}
\end{equation}
present in the leading order is reduced to
\begin{equation}\label{nlobmm}
4.05\cdot 10^{-9}\leq B(B_s\to\mu^+\mu^-)\leq 4.14\cdot 10^{-9}
\end{equation}
when the QCD corrections are included.
This feature is once more illustrated in fig.\ \ref{fig:bmmmut}.

\begin{figure}[hbt]
\vspace{0.10in}
\centerline{
\epsfysize=5in
\rotate[r]{
\epsffile{ps/bmmmut.ps}
} }
\vspace{0.08in}
\caption[]{
The $\mu_t$-dependence of $B(B_s \to \mu^+\mu^-) [10^{-9}]$
with (solid curve) and without (dashed curve) $\ord(\as)$
corrections for fixed parameter values as described in the text.
\label{fig:bmmmut}}
\end{figure}

Finally, we quote the standard model expectation 
for $B(B_s\to\mu^+\mu^-)$ based on the
input parameters collected in the Appendix. We find
\begin{equation}\label{bsmm1}
1.7\cdot 10^{-9}\leq B(B_s\to\mu^+\mu^-)\leq 8.4\cdot 10^{-9}
\end{equation}
using present day uncertainties in the parameters and
$F_{B_s}=230\pm 40\mev$. With reduced errors for the input
quantities, corresponding to our second scenario as defined in 
Appendix \ref{app:numinput}, 
and taking $F_{B_s}=230\pm 10\mev$ this range would
shrink to
\begin{equation}\label{bsmm2}
3.1\cdot 10^{-9}\leq B(B_s\to\mu^+\mu^-)\leq 5.0\cdot 10^{-9}
\end{equation}

For the case of $B_d\to\mu^+\mu^-$ similar formulae hold with obvious
replacements of labels $(s\to d)$. Provided the decay constants
$F_{B_s}$ and $F_{B_d}$ will have been calculated reliably by
non-perturbative methods or measured in leading leptonic decays one
day, the rare processes $B_{s}\to\mu^+\mu^-$ and $B_{d}\to\mu^+\mu^-$
should offer clean determinations of $|V_{ts}|$ and $|V_{td}|$. The
accuracy of the related analysis will profit considerably from the
reduction of theoretical ambiguity achieved through the inclusion of
short-distance QCD effects. In particular $B(B_s\to\mu^+\mu^-)$, which
is expected to be ${\cal O}(4\cdot 10^{-9})$, should be attainable at 
hadronic machines such as HERA-B, Tevatron and LHC.
