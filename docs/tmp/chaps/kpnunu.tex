\section{The Decays \kpnn and \klpnn}
\label{sec:Kpnn}
\subsection{General Remarks on \kpnn}
\label{sec:Kpnn:GeneralKp}
The rare decay \kpnn is one of the theoretically cleanest decays.
As such it is very well suited for the determination of CKM
parameters, in particular of the element $V_{td}$. \kpnn is CP
conserving and receives contributions from both internal top and
charm exchanges. The inclusion of next-to-leading QCD corrections
incorporated in the effective hamiltonian in (\ref{hkpn}) and
discussed in detail in section \ref{sec:HeffRareKB:kpnn} reduces
considerably the theoretical uncertainties due to the choice of the
renormalization scales present in the leading order expressions.
We will illustrate this below. Since in addition the relevant hadronic
matrix element of the weak current $(\bar sd)_{V-A}$ can be measured
in the leading decay $K^+\to\pi^0e^+\nu$, the resulting theoretical
expression for $B(\kpn)$ is only a function of the CKM parameters, the
QCD scale $\Lambda_{\overline{MS}}$ and the quark masses $m_t$ and
$m_c$. The long-distance contributions to \kpnn have been found to be
very small: a few percent of the charm contribution to the amplitude at
most, which is safely negligible \cite{reinsehgal:89},
\cite{hagelinlittenberg:89} and \cite{luwise:94}.

Conventionally the branching fraction $B(K^+\to\pi^+\nu\bar\nu)$ is
related to the experimentally well known quantity
$B(K^+\to\pi^0e^+\nu)$ using isospin symmetry. Corrections to this
approximation have recently been studied in \cite{marcianoparsa:95}.
The breaking of isospin is due to quark mass effects and electroweak
radiative corrections.  In the case of $K^+\to\pi^+\nu\bar\nu$ these
effects result in a decrease of the branching ratio by $10\%$. The
corresponding corrections in $K_L\to\pi^0\nu\bar\nu$ lead to a $5.6\%$
reduction of $B(K_L\to\pi^0\nu\bar\nu)$. We have checked the analysis
of \cite{marcianoparsa:95} and agree with their findings.  Once
calculated, the inclusion of these effects is straightforward as they
only amount to an overall factor for the branching ratio and do not
affect the short-distance structure of $K\to\pi\nu\bar\nu$.  We shall
neglect the isospin violating corrections in the following chapters,
where the focus is primarily on the short-distance physics. The effects
are however incorporated in the final prediction quoted in our summary
table in section \ref{sec:summary}.

In the following we shall concentrate on a discussion of $K^+ \to \pi^+
\nu\bar\nu$ within the framework of the standard model. The impact of
various scenarios of new physics on this decay has been considered for
instance in \cite{bigigabbiani:91}.

\subsection{Master Formulae for \kpnn}
\label{sec:Kpnn:MasterKp}
Using the effective hamiltonian (\ref{hkpn}) and summing over the three
neutrino flavors one finds
\begin{equation}\label{bkpn}
B(\kpn)=\kappa_+\cdot\left[\left({\imlt\over\lambda^5}X(x_t)\right)^2+
\left({\relc\over\lambda}P_0(X)+{\relt\over\lambda^5}X(x_t)\right)^2
\right]
\end{equation}
\begin{equation}\label{kapp}
\kappa_+={3\alpha^2 B(K^+\to\pi^0e^+\nu)\over 2\pi^2\sin^4\Theta_W}
 \lambda^8=4.57\cdot 10^{-11}
\end{equation}
where we have used
\begin{equation}\label{alsinbr}
\alpha=\frac{1}{129}\qquad \sin^2\Theta_W=0.23 \qquad
B(K^+\to\pi^0e^+\nu)=4.82\cdot 10^{-2}
\end{equation}
Here $\lambda_i=V^\ast_{is}V_{id}$ with $\lambda_c$ being
real to a very high accuracy. The function $X$ of (\ref{xx})
can also be written as
\begin{equation}\label{xeta}
X(x)=\eta_X\cdot X_0(x) \qquad\quad \eta_X=0.985
\end{equation}
where $\eta_X$ summarizes the NLO corrections discussed in section
\ref{sec:HeffRareKB:kpnn}. With $m_t\equiv m_t(m_t)$ the QCD factor $\eta_X$
is practically independent of $m_t$ and $\Lambda_{\overline{MS}}$.
Next
\begin{equation}\label{p0k}
P_0(X)=\frac{1}{\lambda^4}\left[\frac{2}{3} X^e_{NL}+\frac{1}{3}
 X^\tau_{NL}\right]
\end{equation}
with the numerical values for $X_{NL}^l$ given in table \ref{tab:xnlnum}.
The corresponding values for $P_0(X)$ as a function of
$\Lambda_{\overline{MS}}$ and $m_c\equiv m_c(m_c)$ are collected in
table \ref{tab:P0Kplus}.
We remark that a negligibly small term $\sim(X_{NL}^e-X_{NL}^\tau)^2$
($\sim 0.2\%$ effect on the branching ratio)
has been discarded in formula (\ref{bkpn}).

\begin{table}[htb]
\caption[]{The function $P_0(X)$ for various $\Lms^{(4)}$ and $m_c$.
\label{tab:P0Kplus}}
\begin{center}
\begin{tabular}{|c|c|c|c|}
&\multicolumn{3}{c|}{$P_0(X)$}\\
\hline
$\Lms^{(4)}$ $\backslash$ $m_c$ & $1.25\gev$ & $1.30\gev$ & $1.35\gev$  \\
\hline
$215\mev$ & 0.402 & 0.436 & 0.472 \\
$325\mev$ & 0.366 & 0.400 & 0.435 \\
$435\mev$ & 0.325 & 0.359 & 0.393 
\end{tabular}
\end{center}
\end{table}

Using the improved Wolfenstein parametrization and the approximate
formulae (\ref{2.51}) -- (\ref{2.53}) we can next write
\begin{equation}\label{108}
B(K^{+} \to \pi^{+} \nu \bar\nu) = 4.57 \cdot 10^{-11} A^4 X^2(x_t)
\frac{1}{\sigma} \left[ (\sigma \bar\eta)^2 +
\left(\varrho_0 - \bar\varrho \right)^2 \right]
\end{equation}
where
\begin{equation}\label{109}
\sigma = \left( \frac{1}{1- \frac{\lambda^2}{2}} \right)^2
\end{equation}

The measured value of B($K^{+} \to \pi^{+} \nu \bar\nu$) then
determines  an ellipse in the $(\bar\varrho,\bar\eta)$ plane  centered at
$(\varrho_0,0)$ with \cite{burasetal:94b}
%
\begin{equation}\label{110}
\varrho_0 = 1 + \frac{P_0(X)}{A^2 X(x_t)}
\end{equation}
%
and having the squared axes
%
\begin{equation}\label{110a}
\bar\varrho_1^2 = r^2_0 \qquad \bar\eta_1^2 = \left( \frac{r_0}{\sigma}
\right)^2
\end{equation}
%
where
%
\begin{equation}\label{111}
r^2_0 = \frac{1}{A^4 X^2(x_t)} \left[
\frac{\sigma \cdot BR(K^{+} \to \pi^{+} \nu \bar\nu)}{4.57 \cdot 10^{-11}} \right]
\end{equation}
%
The departure of $\varrho_0$ from unity measures the relative importance
of the internal charm contributions.

The ellipse defined by $r_0$, $\varrho_0$ and $\sigma$ given above
intersects with the circle (\ref{2.94}).  This allows to determine
$\bar\varrho$ and $\bar\eta$  with 
\begin{equation}\label{113}
\bar\varrho = \frac{1}{1-\sigma^2} \left( \varrho_0 - \sqrt{\sigma^2
\varrho_0^2 +(1-\sigma^2)(r_0^2-\sigma^2 R_b^2)} \right) \qquad
\bar\eta = \sqrt{R_b^2 -\bar\varrho^2}
\end{equation}
%
and consequently
%
\begin{equation}\label{113aa}
R_t^2 = 1+R_b^2 - 2 \bar\varrho
\end{equation}
%
where $\bar\eta$ is assumed to be positive.

In the leading order of the Wolfenstein parametrization
%
\begin{equation}\label{113ab}
\sigma \to 1 \qquad \bar\eta \to \eta \qquad \bar\varrho \to \varrho
\end{equation}
%
and $B(K^+ \to \pi^+ \nu \bar\nu)$ determines a circle in the
$(\varrho,\eta)$ plane centered at $(\varrho_0,0)$ and having the radius
$r_0$ of (\ref{111}) with $\sigma =1$. Formulae (\ref{113}) and
(\ref{113aa}) then simplify to \cite{buchallaburas:94}
%
\begin{equation}\label{113a}
R_t^2 = 1 + R_b^2 + \frac{r_0^2 - R_b^2}{\varrho_0} - \varrho_0 \qquad
\varrho = \frac{1}{2} \left( \varrho_0 + \frac{R_b^2 - r_0^2}{\varrho_0}
\right)
\end{equation}
Given $\bar\varrho$ and $\bar\eta$ one can determine $V_{td}$:
\begin{equation}\label{vtdrhoeta}
V_{td}=A \lambda^3(1-\bar\varrho-i\bar\eta)\qquad
|V_{td}|=A \lambda^3 R_t
\end{equation}
Before proceeding to the numerical analysis a few remarks are in
order:
\begin{itemize}
\item
The determination of $|V_{td}|$ and of the unitarity triangle requires
the knowledge of $V_{cb}$ (or $A$) and of $|V_{ub}/V_{cb}|$. Both
values are subject to theoretical uncertainties present in the existing
analyses of tree level decays. Whereas the dependence on
$|V_{ub}/V_{cb}|$ is rather weak, the very strong dependence of
$B(\kpn)$ on $A$ or $V_{cb}$ makes a precise prediction for this
branching ratio difficult at present. We will return to this below.
\item
The dependence of $B(\kpn)$ on $m_t$ is also strong. However $m_t$
should be known already in this decade within $\pm 5\%$ and
consequently the uncertainty in $m_t$ will soon be less serious for
$B(\kpn)$ than the corresponding uncertainty in $V_{cb}$.
\item
Once $\varrho$ and $\eta$ are known precisely from CP asymmetries in
B decays, some of the uncertainties present in (\ref{108}) related
to $|V_{ub}/V_{cb}|$ (but not to $V_{cb}$) will be removed.
\item
A very clean determination of $\sin 2\beta$ without essentially
any dependence on $m_t$ and $V_{cb}$ can be made by combining
$B(\kpn)$ with $B(\klpn)$ discussed below. We will present an
analysis of this type in section \ref{sec:Kpnn:sin2b}.
\end{itemize}

\subsection{Numerical Analysis of \kpnn}
\label{sec:Kpnn:NumericalKp}
\subsubsection{Renormalization Scale Uncertainties}
\label{sec:Kpnn:NumericalKp:RSU}
We will now investigate the uncertainties in $X(x_t)$, $X_{NL}$,
$B(\kpn)$, $|V_{td}|$ and in the determination of the unitarity
triangle related to the choice of the renormalization scales $\mu_t$
and $\mu_c$ (see section \ref{sec:HeffRareKB:kpnn}). To this end we
will fix the remaining parameters as follows
\begin{equation}\label{mcmtnum}
m_c\equiv m_c(m_c)=1.3\gev \qquad m_t\equiv m_t(m_t)=170\gev
\end{equation}
\begin{equation}\label{vcbubnum}
V_{cb}=0.040 \qquad |V_{ub}/V_{cb}|=0.08
\end{equation}
In the case of $B(\kpn)$ we need the values of both $\bar\varrho$
and $\bar\eta$. Therefore in this case we will work with
\begin{equation}\label{rhetnum}
\bar\varrho=0 \qquad\quad  \bar\eta=0.36
\end{equation}
rather than with $|V_{ub}/V_{cb}|$. Finally we will set
$\Lambda_{\overline{MS}}^{(4)}=0.325\gev$ and
$\Lambda_{\overline{MS}}^{(5)}=0.225\gev$ for the charm part and top
part, respectively.
We then vary the scales $\mu_c$ and $\mu_t$, entering $m_c(\mu_c)$
and $m_t(\mu_t)$ respectively, in the ranges
\begin{equation}\label{muctnum}
1\gev\leq\mu_c\leq 3\gev \qquad 100\gev\leq\mu_t\leq 300\gev
\end{equation}

\begin{figure}[hbt]
\vspace{0.10in}
\centerline{
\epsfysize=5in
\rotate[r]{
\epsffile{ps/mucX.ps}
} }
\vspace{0.08in}
\caption[]{
Charm quark function $X_{NL}$ (for $m_l=0$) compared to the leading-log
result $X_L$ and the case without QCD as functions of $\mu_c$.
\label{fig:kpnunu:XNL}}
\end{figure}

In fig.\ \ref{fig:kpnunu:XNL} we show the charm function $X_{NL}$ (for
$m_l=0$) compared to the leading-log result $X_L$ and the case without
QCD as functions of $\mu_c$. We observe the following features:
\begin{itemize}
\item
The residual slope of $X_{NL}$ is considerably reduced in
comparison to $X_L$, which exhibits a quite substantial dependence
on the unphysical scale $\mu_c$. The variation
of $X$ (defined as $(X(1\gev)-X(3\gev))/X(m_c)$)
is 24.5\% in NLLA compared to 56.6\% in LLA.
\item
The suppression of the
uncorrected function through QCD effects is somewhat less pronounced
in NLLA.
\item
The next-to-leading effects amount to a $\sim 10\%$ correction relative
to $X_L$ at $\mu=m_c$. However the size of this correction strongly
depends on $\mu$ due to the scale ambiguity of the leading order
result. This means that the question of how large the next-to-leading
effects compared to the LLA really are cannot be answered uniquely.
Therefore the relevant result is actually the reduction of the
$\mu$-dependence in NLLA .
\end{itemize}

\begin{figure}[hbt]
\vspace{0.10in}
\centerline{
\epsfysize=5in
\rotate[r]{
\epsffile{ps/mutX.ps}
} }
\vspace{0.08in}
\caption[]{
Top quark function $X(x_t)$ as a function of $\mu_t$ for fixed
$\mt(\mt)=170\gev$ with (solid curve) and without (dashed curve)
$\ord(\as)$ corrections.
\label{fig:kpnunu:X}}
\end{figure}

\noindent
In fig.\ \ref{fig:kpnunu:X} we show the analogous results for the top
function $X(x_t)$ as a function of $\mu_t$. We observe:
\begin{itemize}
\item
Due to $\mu_t\gg\mu_c$ the scale dependences in the top function
are substantially smaller than in the case of charm.
Note in particular how the yet appreciable scale dependence of $X_0$
gets flattened out almost perfectly when the $\ord(\as)$
effects are taken into account. The total variation of $X(x_t)$
with $100\gev\leq\mu_t\leq 300\gev$ is around 1\% in NLLA compared
to 10\% in LLA.
\item
As already stated above after (\ref{xeta}), with the choice
$\mu_t=m_t$ the NLO correction is very small. It is substantially
larger for $\mu_t$ very different from $m_t$.
\end{itemize}
Using (\ref{bkpn}) and varying $\mu_{c, t}$ in the ranges
(\ref{muctnum}) we find that for the above choice of the remaining
parameters the uncertainty in $B(\kpn)$
\begin{equation}\label{varbkpnLO}
0.76\cdot 10^{-10}\leq B(\kpn)\leq 1.20\cdot 10^{-10}
\end{equation}
present in the leading order is reduced to
\begin{equation}\label{varbkpnNLO}
0.88\cdot 10^{-10}\leq B(\kpn)\leq 1.02\cdot 10^{-10}
\end{equation}
after including NLO corrections. Similarly we obtain
\begin{equation}\label{varvtdLO}
8.24\cdot 10^{-3}\leq |V_{td}|\leq 10.97\cdot 10^{-3} \qquad {\rm LLA}
\end{equation}
\begin{equation}\label{varvtdNLO}
9.23\cdot 10^{-3}\leq |V_{td}|\leq 10.10\cdot 10^{-3}  \qquad {\rm NLLA}
\end{equation}
where we have set $B(\kpn)=1\cdot 10^{-10}$. We observe that including
the full next-to-leading corrections reduces the uncertainty in the
determination of $|V_{td}|$ from $\pm 14\%$ (LLA) to $\pm 4.6\%$ (NLLA)
in the present example. The main bulk of this theoretical error stems
from the charm sector. Indeed, keeping $\mu_c=m_c$ fixed and varying
only $\mu_t$, the uncertainties in the determination of $|V_{td}|$
would shrink to $\pm 4.7\%$ (LLA) and $\pm 0.6\%$ (NLLA).
Similar comments apply to $B(\kpn)$ where, as seen in
(\ref{varbkpnLO}) and (\ref{varbkpnNLO}), the theoretical uncertainty
due to $\mu_{c,t}$ is reduced from $\pm 22\%$ (LLA) to $\pm 7\%$ (NLLA).

\begin{figure}[hbt]
\vspace{0.10in}
\centerline{
\epsfysize=5in
\rotate[r]{
\epsffile{ps/kpnrhoeta.ps}
} }
\vspace{0.08in}
\caption[]{
The theoretical uncertainties in the determination of the unitarity
triangle (UT) in the $(\bar\varrho, \bar\eta)$ plane from
$B(K^+\to\pi^+\nu\bar\nu)$. With fixed input parameters the vertex of
the UT has to lie on a circle around the origin with radius $R_b$.  A
variation of the scales $\mu_c$, $\mu_t$ within  $1\gev \le \mu_c \le 3\gev$
and $100\gev \le \mu_t \le 300\gev$ then yields the indicated ranges in LLA
(full) and NLLA (reduced). We show the cases $R_b=0.25, 0.36, 0.47$.
\label{fig:kpnunu:rhoetabar}}
\end{figure}

Finally in fig.\ \ref{fig:kpnunu:rhoetabar} we show the position of the
point ($\bar\varrho$, $\bar\eta$) which determines the unitarity
triangle.  To this end we have fixed all parameters as stated above
except for $R_b$, for which we have chosen three representative
numbers, $R_b=0.25$, $0.36$, $0.47$. The full and the reduced ranges
represent LLA and NLLA respectively. The impact of the inclusion of NLO
corrections on the accuracy of determining the unitarity triangle is
clearly visible.

\subsubsection{Expectations for $B(\kpn)$}
\label{sec:Kpnn:NumericalKp:EfB}
The purely theoretical uncertainties discussed so far should be
distinguished from the uncertainties coming from the input parameters
such as $m_t$, $V_{cb}$, $|V_{ub}/V_{cb}|$
etc.. As we will see the latter uncertainties are still rather large to
date. Consequently the progress achieved by the NLO calculations
\cite{buchallaburas:94} cannot yet be fully exploited
phenomenologically at present.  However the determination of the
relevant parameters should improve in the future. Once the precision in
the input parameters will have attained the desired level, the gain in
accuracy of the theoretical prediction for \kpnn in NLLA by a factor of
more than 3 compared to the LLA will become very important.

Using our standard set of input parameters specified in appendix
\ref{app:numinput} and the constraints implied by the analysis of
$\varepsilon_K$ and $B_d-\bar B_d$ mixing as described in section
\ref{sec:epsBBUT}, we find for the \kpnn branching fraction
the range
\begin{equation}\label{kpnn1}
0.6\cdot 10^{-10}\leq B(\kpn)\leq 1.5\cdot 10^{-10}
\end{equation}
Eq. (\ref{kpnn1}) represents the current standard model expectation for
$B(\kpn)$ (neglecting small isospin breaking corrections). To obtain
this estimate we have allowed for a variation of the parameters $m_t$,
$|V_{cb}|$, $|V_{ub}/V_{cb}|$, $B_K$, $F^2_B B_B$, $x_d$ within their
uncertainties as summarized in appendix \ref{app:numinput}. The
uncertainties in $m_c$ and $\Lambda_{\overline{MS}}$, on the other
hand, are small in comparison and have been neglected in this context.
The above range would be reduced to \begin{equation}\label{kpnn2}
0.8\cdot 10^{-10}\leq B(\kpn)\leq 1.0\cdot 10^{-10} \end{equation} if
the uncertainties in the input parameters could be decreased as assumed
by our ``future'' scenario in appendix \ref{app:numinput}.

It should be remarked that the $x_d$-constraint, excluding a part of
the second quadrant for the CKM phase $\delta$, plays an essentail role
in obtaining the upper bounds given above, without essentially any
effect on the lower bounds. Without the $x_d$-constraint the upper
bounds in \eqn{kpnn1} and \eqn{kpnn2} are relaxed to $2.3 \cdot
10^{-10}$ and $1.6 \cdot 10^{-10}$, respectively.

\subsection{General Remarks on \klpnn}
            \label{sec:Kpnn:GeneralKL}
The rare decay \klpnn is even cleaner than $\kpn$. It proceeds almost
entirely through direct CP violation \cite{littenberg:89} and
is completely dominated by short-distance loop diagrams with top quark
exchanges. In fact the $m_t$-dependence of $B(\klpn)$ is again
described by $X(x_t)$.  Since the charm contribution can be fully
neglected also the theoretical uncertainties present in \kpnn due to
$m_c$, $\mu_c$ and $\Lambda_{\overline{MS}}$ are absent here. For this
reason \klpnn is very well suited for the determination of CKM
parameters, in particular the Wolfenstein parameter $\eta$.

\subsection{Master Formulae for \klpnn}
\label{sec:Kpnn:MasterKL}
Using the effective hamiltonian (\ref{hxnu}) and summing over three
neutrino flavors one finds
\begin{equation}\label{bklpn}
B(K_L\to\pi^0\nu\bar\nu)=\kappa_L\cdot\left({\imlt\over\lambda^5}X(x_t)
  \right)^2
\end{equation}
\begin{equation}\label{kapl}
\kappa_L=\kappa_+ {\tau(K_L)\over\tau(K^+)}=1.91\cdot 10^{-10}
\end{equation}
with $\kappa_+$ given in (\ref{kapp}). Using the Wolfenstein
parametrization we can rewrite (\ref{bklpn}) as
\begin{equation}\label{bklpnwol1}
B(\klpn)=1.91\cdot 10^{-10} \eta^2 A^4 X^2(x_t)
\end{equation}
or
\begin{equation}\label{bklpnwol2}
B(\klpn)=3.48\cdot 10^{-5} \eta^2 |V_{cb}|^4 X^2(x_t)
\end{equation}
A few remarks are in order:
\begin{itemize}
\item
The determination of $\eta$ using $B(\klpn)$ requires the knowledge
of $V_{cb}$ and $m_t$. The very strong dependence on $V_{cb}$ or $A$
makes a precise prediction for this branching ratio difficult at
present.
\item
It has been pointed out \cite{buras:94b} that the strong
dependence of $B(\klpn)$ on $V_{cb}$, together with the clean nature of
this decay, can be used to determine this element without any hadronic
uncertainties. To this end $\eta$ and $m_t$ have to be known with
sufficient precision in addition to $B(\klpn)$. $\eta$ should be
measured accurately in CP asymmetries in $B$ decays and the value of
$m_t$ known to better than $\pm 5\gev$ from TEVATRON and future LHC
experiments. Inverting (\ref{bklpnwol2}) and using a very accurate
approximation for $X(x_t)$ (valid for $\mt = \overline{m}_{\rm t}(\mt)$)
as given by \eqn{xeta} and \eqn{eq:approxSXYZE}
\begin{equation}\label{xxappr}
X(x_t)=0.65\cdot x_t^{0.575}
\end{equation}
one finds
\begin{equation}\label{vcbklpn}
V_{cb}=39.3\cdot 10^{-3} \sqrt{\frac{0.39}{\eta}}
\left[\frac{170\gev}{m_t}\right]^{0.575}
\left[\frac{B(\klpn)}{3\cdot 10^{-11}}\right]^{1/4}
\end{equation}
We note that the weak dependence of $V_{cb}$ on $B(\klpn)$ allows
to achieve a high precision for this CKM element even when $B(\klpn)$
is known with only relatively moderate accuracy, e.g. 10--15\%.
Needless to say that any measurement of $B(\klpn)$ is extremely
challenging. A numerical analysis of (\ref{vcbklpn}) can be found in
\cite{buras:94b}.
\end{itemize}

\subsection{Numerical Analysis of \klpnn}
\label{sec:Kpnn:NumericalKL}
\subsubsection{Renormalization Scale Uncertainties}
\label{sec:Kpnn:NumericalKL:RSU}
The scale ambiguities present in the function $X(x_t)$ have already been
discussed in connection with $\kpn$. After the inclusion of NLO
corrections they are so small that they can be neglected for all
practical purposes. Effectively they could also be taken into
account by introducing an additional error $\Delta m_t\leq\pm 1\gev$.
At the level of $B(\klpn)$ the ambiguity in the choice of $\mu_t$ is
reduced from $\pm 10\%$ (LLA) down to $\pm 1\%$ (NLLA), which
considerably increases the predictive power of the theory. Varying
$\mu_t$ according to (\ref{muctnum}) and using the input parameters
of section \ref{sec:Kpnn:NumericalKp} we find that the uncertainty
in $B(\klpn)$
\begin{equation}\label{varbklpnLO}
2.68\cdot 10^{-11}\leq B(\klpn)\leq 3.26\cdot 10^{-11}
\end{equation}
present in the leading order is reduced to
\begin{equation}\label{varbklpnNLO}
2.80\cdot 10^{-11}\leq B(\klpn)\leq 2.88\cdot 10^{-11}
\end{equation}
after including NLO corrections. This means that the theoretical
uncertainty in the determination of $\eta$ amounts to only $\pm 0.7\%$
in NLLA which is safely negligible.
The reduction of the scale ambiguity for $B(\klpn)$ is further 
illustrated in fig.\ \ref{fig:klpnmut}.

\begin{figure}[hbt]
\vspace{0.10in}
\centerline{
\epsfysize=5in
\rotate[r]{
\epsffile{ps/klpnmut.ps}
} }
\vspace{0.08in}
\caption[]{
The $\mu_t$-dependence of $B(K_L \to \pi^0\nu\bar\nu)/10^{-11}$ with
(solid curve) and without (dashed curve) $\ord(\as)$ corrections for
$\mt(\mt)=170\gev$, $|V_{cb}|=0.04$ and $\bar\eta=0.36$.
\label{fig:klpnmut}}
\end{figure}

\subsubsection{Expectations for $B(\klpn)$}
\label{sec:Kpnn:NumericalKL:EfB}
From an analysis of $B(\klpn)$ similar to the one described for
\kpnn in section \ref{sec:Kpnn:NumericalKp:EfB} we obtain the
standard model expectation
\begin{equation}\label{klpn1}
1.1\cdot 10^{-11}\leq B(\klpn)\leq 5.0\cdot 10^{-11}
\end{equation}
corresponding to present day errors in the relevant input
parameters. This would change into
\begin{equation}\label{klpn2}
2.2\cdot 10^{-11}\leq B(\klpn)\leq 3.6\cdot 10^{-11}
\end{equation}
if the parameter uncertainties would decrease as anticipated by our 
``future'' scenario defined in appendix \ref{app:numinput}.

\subsection{Unitarity Triangle from $K\to\pi\nu\bar\nu$}
\label{sec:Kpnn:Triangle}
The measurement of $B(\kpn)$ and $B(\klpn)$ can determine the
unitarity triangle completely provided $m_t$ and $V_{cb}$ are known.
Using these two branching ratios simultaneously allows to eliminate
$|V_{ub}/V_{cb}|$ from the analysis which removes considerable
uncertainty. Indeed it is evident from (\ref{bkpn}) and
(\ref{bklpn}) that, given $B(\kpn)$ and $B(\klpn)$, one can extract
both $\imlt$ and $\relt$. We get
\begin{equation}\label{imre}
\imlt=\lambda^5{\sqrt{B_2}\over X(x_t)}\qquad
\relt=-\lambda^5{{\relc\over\lambda}P_0(X)+\sqrt{B_1-B_2}\over X(x_t)}
\end{equation}
where we have defined the ``reduced'' branching ratios
\begin{equation}\label{b1b2}
B_1={B(\kpn)\over 4.57\cdot 10^{-11}}\qquad
B_2={B(\klpn)\over 1.91\cdot 10^{-10}}
\end{equation}
Using next the expressions for $\imlt$, $\relt$ and $\relc$ given
in (\ref{2.51}) -- (\ref{2.53}) we find
\begin{equation}\label{rhetb}
\bar\varrho=1+{P_0(X)-\sqrt{\sigma(B_1-B_2)}\over A^2 X(x_t)}\qquad
\bar\eta={\sqrt{B_2}\over\sqrt{\sigma} A^2 X(x_t)}
\end{equation}
with $\sigma$ defined in (\ref{109}). An exact treatment of the CKM
matrix shows that the formulae (\ref{rhetb}) are rather precise
\cite{buchallaburas:94c}. The error in $\bar\eta$ is below 0.1\% and
$\bar\varrho$ may deviate from the exact expression by at most
$\Delta\bar\varrho=0.02$ with essentially negligible error for
$0\leq\bar\varrho\leq 0.25$.
\\
As an illustrative example, let us consider the following scenario.
We assume that the branching ratios are known to within $\pm 10\%$
\begin{equation}\label{bkpkl}
B(\kpn)=(1.0\pm 0.1)\cdot 10^{-10}\qquad
B(\klpn)=(2.5\pm 0.25)\cdot 10^{-11}
\end{equation}
Next we take ($m_i\equiv m_i(m_i)$)
\begin{equation}\label{mtcv}
m_t=(170\pm 5)\gev\quad m_c=(1.30\pm 0.05)\gev\quad
V_{cb}=0.040\pm 0.001
\end{equation}
where the quoted errors are quite reasonable if one keeps in mind
that it will take at least ten years to achieve the accuracy
assumed in (\ref{bkpkl}).
Finally, we use
\begin{equation}\label{lamuc}
\Lms^{(4)}=(200 - 350)\mev \qquad \mu_c=(1-3)\gev
\end{equation}
where $\mu_c$ is the renormalization scale present in the analysis of
the charm contribution. Its variation gives an indication of the
theoretical uncertainty involved in the calculation.  In comparison to
this error we neglect the effect of varying $\mu_W=\ord(M_W)$, the high
energy matching scale at which the W boson is integrated out, as well
as the very small scale dependence of the top quark contribution.  As
reference parameters we use the central values in (\ref{bkpkl}) and
(\ref{mtcv}) and $\Lms^{(4)} = 300\mev$, $\mu_c=m_c$.  The results that
would be obtained in such a scenario for $\bar\eta$, $|V_{td}|$ and
$\bar\varrho$ are collected in table \ref{tab:utkpnn}.

\begin{table}[htb]
\caption[]{$\bar\eta$, $|V_{td}|$ and $\bar\varrho$ determined from
\kpnn and \klpnn for the scenario described in the text together with
the uncertainties related to various parameters.
\label{tab:utkpnn}}
\begin{center}
\begin{tabular}{|c||c||c|c|c|c||c|}
&&$\Delta(BR)$&$\Delta(m_t, V_{cb})$&$\Delta(m_c,\Lms^{(4)})
$&$\Delta(\mu_c)$&$\Delta_{total}$\\ \hline
$\bar\eta$&$0.33$&$\pm 0.02$&$\pm 0.03$&$\pm 0.00$&$\pm 0.00$&
$\pm 0.05$\\ \hline
$|V_{td}|/10^{-3}$&$9.3$&$\pm 0.6$&$\pm 0.6$&$\pm 0.5$&$\pm 0.4$&
$\pm 2.1$\\ \hline
$\bar\varrho$&$0.00$&$\pm 0.08$&$\pm 0.09$&$\pm 0.06$&$\pm 0.04$&
$\pm 0.27$
\end{tabular}
\end{center}
\end{table}

There we have also displayed separately the associated, symmetrized
errors ($\Delta$) coming from the uncertainties in the branching ratios,
$m_t$ and $V_{cb}$, $m_c$ and $\Lms^{(4)}$, $\mu_c$,
as well as the total uncertainty.
\\
We observe that respectable determinations of $\bar\eta$ and
$|V_{td}|$ can be obtained. On the other hand the determination of
$\bar\varrho$ is rather poor. We also note that a sizable part of the
total uncertainty results in each case from the strong dependence of
both branching ratios on $m_t$ and $V_{cb}$. There is however one
important quantity for which the strong dependence of $B(\kpn)$ and
$B(\klpn)$ on $m_t$ and $V_{cb}$ does not matter at all.

\subsection{$\sin 2\beta$ from $K\to\pi\nu\bar\nu$}
\label{sec:Kpnn:sin2b}
Using (\ref{rhetb}) one finds \cite{buchallaburas:94c}
\begin{equation}\label{sin}
r_s=r_s(B_1, B_2)\equiv{1-\bar\varrho\over\bar\eta}=\cot\beta \qquad
\sin 2\beta=\frac{2 r_s}{1+r^2_s}
\end{equation}
with
\begin{equation}\label{cbb}
r_s(B_1, B_2)=\sqrt{\sigma}{\sqrt{\sigma(B_1-B_2)}-P_0(X)\over\sqrt{B_2}}
\end{equation}
Thus within the approximation of (\ref{rhetb}) $\sin 2\beta$ is
independent of $V_{cb}$ (or $A$) and $m_t$. An exact treatment of
the CKM matrix confirms this finding to a high accuracy. The
dependence on $V_{cb}$ and $m_t$ enters only at order
$\ord(\lambda^2)$ and as a numerical analysis shows this
dependence can be fully neglected.
\\
It should be stressed that $\sin 2\beta$ determined this way depends
only on two measurable branching ratios and on the function
$P_0(X)$ which is completely calculable in perturbation theory.
Consequently this determination is free from any hadronic
uncertainties and its accuracy can be estimated with a high degree
of confidence. To this end we use the input given in
(\ref{bkpkl}) -- (\ref{lamuc}) to find
\begin{equation}\label{sin2bnum}
\sin 2\beta=0.60\pm 0.06\pm 0.03\pm 0.02
\end{equation}
where the first error comes from $B(\kpn)$ and $B(\klpn)$, the second
from $m_c$ and $\Lambda_{\overline{MS}}$ and the last one from the
uncertainty due to $\mu_c$. We note that the largest partial
uncertainty results from the branching ratios themselves. It can be
probably reduced with time as is the case with the $\pm 0.03$
uncertainty related to $\Lambda_{\overline{MS}}$ and $m_c$.  Note that
the theoretical uncertainty represented by $\Delta(\mu_c)$, which
ultimately limits the accuracy of the analysis, is small.  This
reflects the clean nature of the $K\to\pi\nu\bar\nu$ decays. However
the small uncertainty of $\pm 0.02$ is only achieved by including
next-to-leading order QCD corrections.  In the leading logarithmic
approximation the corresponding error would amount to $\pm 0.05$,
larger than the one coming from $m_c$ and $\Lambda_{\overline{MS}}$.
\\
The accuracy to which $\sin 2\beta$ can be obtained from
$K\to\pi\nu\bar\nu$ is, in our example, comparable to the one expected
in determining $\sin 2\beta$ from CP asymmetries in B decays prior to
LHC experiments.  In this case $\sin 2\beta$ is determined best by
measuring the time integrated CP violating asymmetry in $B^0_d\to\psi
K_S$ which is given by
\begin{eqnarray}
A_{CP}(\psi K_S) &=& \frac{
 \int^\infty_0\left[\Gamma(B\to\psi K_S)-\Gamma(\bar B\to\psi K_S)\right] dt}
{\int^\infty_0\left[\Gamma(B\to\psi K_S)+\Gamma(\bar B\to\psi K_S)\right] dt}
\nn \\
 &=& -\sin 2\beta {x_d\over 1+x^2_d}
\label{acp}
\end{eqnarray}
where $x_d=\Delta m/\Gamma$ gives the size of $B^0_d-\bar B^0_d$
mixing. Combining (\ref{sin}) and (\ref{acp}) we obtain an
interesting connection between rare K decays and B physics
\begin{equation}\label{kbcon}
{2 r_s(B_1,B_2)\over 1+r^2_s(B_1,B_2)}=-A_{CP}(\psi K_S){1+x^2_d\over x_d}
\end{equation}
which must be satisfied in the Standard Model. We stress that except
for $P_0(X)$ given in table \ref{tab:P0Kplus} all quantities in
(\ref{kbcon}) can be directly measured in experiment and that this
relationship is essentially independent of $m_t$ and $V_{cb}$.
