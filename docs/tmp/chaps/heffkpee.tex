\section{The Effective Hamiltonian for $\Kpiee$}
         \label{sec:HeffKpe}
The $\dS$ effective hamiltonian for $\Kpiee$ at scales $\mu < \mc$ is given by
\begin{equation}
\Heff(\dS) = \frac{G_F}{\sqrt{2}} \V{us}^* \V{ud}^{} \left[ \; \sum_{i=1}^{6,7V}
             \left( z_i(\mu) + \tau \; y_i(\mu) \right) Q_i(\mu)
             + \tau \; y_{7A}(\mw) \; Q_{7A}(\mw) \; \right]
\label{eq:HeffKpe}
\end{equation}
with
\begin{equation}
\tau = - \frac{ \V{ts}^* \V{td}^{} }{ \V{us}^* \V{ud}^{} } \, .
\label{eq:tauagain}
\end{equation}

\subsection{Operators}
            \label{sec:HeffKpe:op}
In \eqn{eq:HeffKpe} $Q_{1,2}$
denote the $\dS$ current-current and $Q_3,\ldots,Q_6$ the QCD penguin
operators of eq.~\eqn{eq:Kppbasis}. For scales $\mu > \mc$ again the
current-current operators $Q^c_{1,2}$ of eq.~\eqn{eq:KppQ12c} have to
be taken into account.

The new operators specific to the decay $\Kpiee$ are
\begin{eqnarray}
Q_{7V} &=& (\bar{s} d)_{\rm V-A} (\bar{e} e)_{\rm V} \, ,
\label{eq:Q7VKpe} \\
Q_{7A} &=& (\bar{s} d)_{\rm V-A} (\bar{e} e)_{\rm A} \, .
\label{eq:Q7AKpe}
\end{eqnarray}
They originate through the $\gamma$- and $Z^0$-penguin and box diagrams of
fig.\ \ref{fig:1loopful}.

It is convenient to introduce the auxiliary operator
\begin{equation}
Q'_{7V} = (\aem/\as) \; (\bar{s} d)_{\rm V-A} (\bar{e}e)_{\rm V}
\label{eq:Q7VprimeKpe}
\end{equation}
and work for the renormalization group analysis in the basis
$Q_1,\ldots,Q_6$, $Q'_{7V}$. The factor $\aem/\as$ in the definition of
$Q'_{7V}$ implies that in this new basis the anomalous
dimension matrix $\hg$ will be a function of $\as$ alone. At the end of the
renormalization group analysis, this factor will be put back into the
Wilson coefficient $C_{7V}(\mu)$ of the operator $Q_{7V}$ in
eq.~\eqn{eq:Q7VKpe}. There is no need to include a similar factor in
$Q_{7A}$ as this operator does not mix under renormalization with the
remaining operators. Since $Q_{7A}$ has no anomalous dimension its Wilson
coeffcient is $\mu$-independent.

In principle one can think of including the electroweak four-quark
penguin operators $Q_7,\ldots,Q_{10}$ of eq.~\eqn{eq:dF1:1010basis} in
$\Heff$ of \eqn{eq:HeffKpe}. However, their Wilson coefficients and
matrix elements for the decay $\Kpiee$ are both of order $\ord(\aem)$
implying that these operators eventually would enter the amplitude
$A(\Kpiee)$ at $\ord(\aem^2)$.  To the order considered here this
contribution is thus negligible. This should be distinguished from the
case of $\Kpipi$ discussed in section~\ref{sec:HeffdF1:1010}. There, in
spite of being suppressed by $\aem/\as$ relative to QCD penguin
operators, the electroweak penguin operators have to be included in the
analysis because of the additional enhancement factor $\RE A_0/\RE A_2
\simeq 22$ present in the formula for $\epe$ (see section
\ref{sec:nloepe}). Such an enhancement factor is not present in the
$\Kpiee$ case and the electroweak penguin operators can be safely
neglected. \\
Concerning the Wilson coefficients, the electroweak four-quark penguin
operators would also affect through mixing under renormalization the
coefficients $C_3,\ldots,C_6$ at $\ord(\aem)$ and $C_{7V}$ at
$\ord(\aem^2)$.  Since the corresponding matrix elements are
$\ord(\aem)$ and $\ord(1)$, respectively, we again obtain a negligible
$\ord(\aem^2)$ effect in $A(\Kpiee)$. \\
In summary, the electroweak four-quark penguin operators $Q_7, \ldots,
Q_{10}$ can safely be neglected in the following discussion of
$\Heff(\dS)$ for $\Kpiee$.

We also neglect the ``magnetic moment'' operators.These operators,
being of dimension five, do not influence the Wilson coefficients of
the operators $Q_1,\ldots,Q_6$, $Q_{7V}$ and $Q_{7A}$. Since their
contributions to $\Kpiee$ are suppressed by an additional factor $\ms$,
they appear strictly speaking at higher order in chiral perturbation
theory. On the other hand the magnetic moment type operators play a
crucial role in $b \to s \gamma$ and $b \to d \gamma$ transitions as
discussed in sections \ref{sec:Heff:BXsgamma} and
\ref{sec:Heff:Bsgamma}. They also have to be kept in the decay $B\to
X_se^+e^-$.

\subsection{Wilson Coefficients}
            \label{sec:HeffKpe:wc}
Eqs.~\eqn{eq:WCy}--\eqn{eq:WCz} remain valid in the case of $\Kpiee$ with
$\hU_{f}(m_1,m_2)$ and $\hM(m_i)$ now denoting $7 \times 7$ matrices in
the $Q_1,\ldots,Q_6$, $Q'_{7V}$ basis. The Wilson coefficients are
given by seven-dimensional column vectors $\vec{z}(\mu)$ and
$\vec{v}(\mu)$ having components $(z_1,\ldots,z_6,z'_{7V})$ and
$(v_1,\ldots,v_6,v'_{7V})$, respectively. Here
\begin{equation}
v'_{7V}(\mu) = \frac{\as(\mu)}{\aem} \; v_{7V}(\mu) \, ,
\qquad\qquad
z'_{7V}(\mu) = \frac{\as(\mu)}{\aem} \; z_{7V}(\mu)
\end{equation}
are the rescaled Wilson coefficients of the auxiliary operator $Q'_{7V}$
used in the renormalization group evolution.

The initial conditions $C_1(\mw),\ldots,C_6(\mw)$, $z_1(\mw)$, $z_2(\mw)$
and $z_1(\mc),\ldots,z_6(\mc)$ for the four-quark operators $Q_1,\ldots,Q_6$
are readily obtained from eqs.~\eqn{eq:CMw1QCD}--\eqn{eq:CMw6QCD},
\eqn{eq:zMw12} and \eqn{eq:zmc}.

The corresponding initial conditions for the remaining operators
$Q'_{7V}$ and $Q_{7A}$ specific to $\Kpiee$ are then given in the NDR
scheme by
\begin{equation}
C'_{7V}(\mw) = \frac{\as(\mw)}{2 \pi}
\left[ \frac{C_0(x_t)-B_0(x_t)}{\sin^2\theta_W} -\tilde{D}_0(x_t) -
4 C_0(x_t) \right]
\label{eq:C7VprimeMw}
\end{equation}
and
\begin{equation}
C_{7A}(\mw) = 
y_{7A}(\mw) = \frac{\aem}{2 \pi} \frac{B_0(x_t)-C_0(x_t)}{\sin^2\theta_W}.
\label{eq:C7AMw}
\end{equation}
In order to find $z'_{7V}(\mc)$ which results from the diagrams of
fig.\ \ref{fig:1loopeff}, we simply have to rescale the NDR result for
$z_7(\mc)$ in eq.~\eqn{eq:zmc:1010} by a factor of $-3 \as/\aem$. This
yields
\begin{equation}
z'_{7V}(\mc) = -\frac{\as(\mc)}{2 \pi} F_{\rm e}(\mc) \, .
\label{eq:z7primemc}
\end{equation}

\subsection{Renormalization Group Evolution and Anomalous Dimension Matrices}
            \label{sec:HeffKpe:rge}
Working in the rescaled basis $Q_1,\ldots,Q_6$, $Q'_{7V}$ the anomalous
dimension matrix $\hg$ has per construction a pure $\ord(\as)$ expansion
\begin{equation}
\hg = \frac{\as}{4\pi}       \gamma^{(0)} +
      \frac{\as^2}{(4\pi)^2} \gamma^{(1)} + \ldots
\, ,
\label{eq:gsexpKpe}
\end{equation}
where $\hg^{(0)}$ and $\hg^{(1)}$ are $7 \times 7$ matrices. The
evolution matrices $\hU_{f}(m_1,m_2)$ in eqs.~\eqn{eq:WCv} and
\eqn{eq:WCz} are for the present case simply given by \eqn{eq:UQCDKpp}
and \eqn{u0vd}--\eqn{jvs}.

The $6 \times 6$ submatrix of $\gamma^{(0)}$ involving the
operators $Q_1,\ldots,Q_6$ has already been given in eq.~\eqn{eq:gs0Kpp}.
Here we only give the remaining entries of $\gamma^{(0)}$ related to
the additional presence of the operator $Q'_{7V}$
\begin{equation}
\begin{array}{lclclcl}
\gamma^{(0)}_{17} &=& -\frac{16}{9} N
&\qquad&
\gamma^{(0)}_{27} &=& -\frac{16}{9}
\svs \\
\gamma^{(0)}_{37} &=& -\frac{16}{9} N \left(u-\frac{d}{2}-\frac{1}{N} \right)
&\qquad&
\gamma^{(0)}_{47} &=& -\frac{16}{9} \left( u -\frac{d}{2} - N \right)
\svs \\
\gamma^{(0)}_{57} &=& -\frac{16}{9} N \left( u -\frac{d}{2} \right)
&\qquad&
\gamma^{(0)}_{67} &=& -\frac{16}{9} \left( u -\frac{d}{2} \right)
\svs \\
\gamma^{(0)}_{77} &=& -2 \beta_0 = -\frac{22}{3} N + \frac{4}{3} f
&\qquad&
\gamma^{(0)}_{7i} &=& 0 \qquad i=1,\ldots,6
\svs
\end{array}
\label{eq:g0col7kpe}
\end{equation}
where $N$ denotes the number of colours. These elements have been first
calculated in \cite{gilman:80} except that $\gamma^{(0)}_{37}$ and
$\gamma^{(0)}_{47}$ have been corrected in \cite{eegpicek:88},
\cite{flynn:89b}.

The $6 \times 6$ submatrix of $\gamma^{(1)}$ involving the
operators $Q_1,\ldots,Q_6$ has already been presented as $\gss$ in
eq.~\eqn{eq:gs1ndrN3Kpp} and
the seventh column of $\gamma^{(1)}$ is given as follows in the
NDR scheme \cite{burasetal:94a}
\begin{eqnarray}
\gamma^{(1)}_{17} &=& \frac{8}{3} \left( 1 - N^2 \right) \, ,
\nn \\
\gamma^{(1)}_{27} &=& \frac{200}{81} \left( N - \frac{1}{N} \right) \, ,
\nn \\
\gamma^{(1)}_{37} &=& \frac{8}{3} \left( u - \frac{d}{2} \right) \left( 1
    - N^2 \right) + \frac{464}{81} \left( \frac{1}{N} - N \right) \, ,
\nn \\
\gamma^{(1)}_{47} &=& \left( u \frac{280}{81} + d \frac{64}{81} \right)
    \left( \frac{1}{N} - N \right) + \frac{8}{3} \left( N^2 - 1 \right) \, ,
\label{eq:g1col7kpe} \\
\gamma^{(1)}_{57} &=& \frac{8}{3} \left( u - \frac{d}{2} \right)
                      \left( 1 - N^2 \right) \, ,
\nn \\
\gamma^{(1)}_{67} &=& \left( u \frac{440}{81} - d \frac{424}{81} \right)
                      \left( N - \frac{1}{N} \right) \, ,
\nn \\
\gamma^{(1)}_{77} &=& -2 \beta_1 = -\frac{68}{3} N^2 + \frac{20}{3} N f +
                      4 C_F f
\nn \\
\gamma^{(1)}_{7i} &=& 0 \qquad i=1,\ldots,6
\nn
\end{eqnarray}
where $C_F = (N^2 -1)/(2 N)$. The corresponding results in the HV
scheme can be found in \cite{burasetal:94a}.

\subsection{Quark Threshold Matching Matrix}
            \label{sec:HeffKpe:Mm}
For the case of $\Kpiee$ the matching matrix $M(m)$ has in the
basis $Q_1,\ldots,Q_6,Q'_{7V}$ the form
\begin{equation}
\hM(m) = 1 + \frac{\as(m)}{4 \pi} \; \vardrs^T
\label{eq:MmKpe}
\end{equation}
where $1$ and $\vardrs^T$ are $7 \times 7$ matrices and $m$ is
the scale of the quark threshold.

The $6 \times 6$ submatrix of $\hM(m)$ involving $Q_1,\ldots,Q_6$ has been
given in eq.~\eqn{eq:drsmbcKpp}. The remaining entries of $\vardrs$ can
be found from the matrix $\vardre$ given in eqs.\ \eqn{eq:drsembdF1:1010}
and \eqn{eq:drsemcdF1:1010} by making a simple rescaling by $-3 \;
\as/\aem$ as in the case of $z_7(\mc)$.

In summary, for the quark threshold $m = \mb$ the matrix $\vardrs$ reads
\begin{equation}
\vardrs =
\left(
\begin{array}{ccccccc}
0 & 0 & 0 & 0 & 0 & 0 & 0 \\
0 & 0 & 0 & 0 & 0 & 0 & 0 \\
0 & 0 & 0 & 0 & 0 & 0 & -\frac{20}{9} \\
0 & 0 & \frac{5}{27} & -\frac{5}{9} & \frac{5}{27} & -\frac{5}{9} &
  -\frac{20}{27} \\
0 & 0 & 0 & 0 & 0 & 0 & -\frac{20}{9} \\
0 & 0 & \frac{5}{27} & -\frac{5}{9} & \frac{5}{27} & -\frac{5}{9} &
  -\frac{20}{27} \\
0 & 0 & 0 & 0 & 0 & 0 & 0
\end{array}
\right) \, .
\label{eq:drsKpe}
\end{equation}
For $m=\mc$ the seventh column of $\vardrs$ in \eqn{eq:drsKpe} has
to be multiplied by $-2$.

\subsection{Numerical Results for the $\Kpiee$ Wilson Coefficients}
            \label{sec:HeffKpe:numres}
\begin{table}[htb]
\caption[]{$K_{\rm L} \rightarrow \pi^0\,e^+\,e^-$ Wilson coefficients
for $Q_{7V,A}$ at $\mu=1\gev$ for $\mt=170\gev$. The corresponding
coefficients for $Q_1,\ldots,Q_6$ can be found in table \ref{tab:wc6smu1}
of section \ref{sec:HeffdF1:66}.
\label{tab:wckpemu1}}
\begin{center}
\begin{tabular}{|c|c|c|c||c|c|c||c|c|c|}
& \multicolumn{3}{c||}{$\Lms^{(4)}=215\mev$} &
  \multicolumn{3}{c||}{$\Lms^{(4)}=325\mev$} &
  \multicolumn{3}{c| }{$\Lms^{(4)}=435\mev$} \\
\hline
Scheme & LO & NDR & HV & LO & 
NDR & HV & LO & NDR & HV \\
\hline
$z_{7V}/\aem$ & --0.014 & --0.015 & 0.005 & --0.024 & 
--0.046 & --0.003 & --0.035 & --0.084 & --0.011 \\
\hline
$y_{7V}/\aem$ & 0.575 & 0.747 & 0.740 & 0.540 & 
0.735 & 0.725 & 0.509 & 0.720 & 0.710 \\
$y_{7A}/\aem$ & --0.700 & --0.700 & --0.700 & --0.700 & 
--0.700 & --0.700 & --0.700 & --0.700 & --0.700 \\
\end{tabular}
\end{center}
\end{table}

In the case of $K_L \to \pi^0 e^+ e^-$, due to
$\gamma^{(0)}_{7i}=\gamma^{(1)}_{7i}=0$, $i=1,\ldots,6$ in
eq.\ \eqn{eq:g0col7kpe} and \eqn{eq:g1col7kpe}, respectively, the RG
evolution of $Q_1,\ldots,Q_6$ is completely unaffected by the additional
presence of the operator $Q_{7V}$. The $K_L \to \pi^0 e^+ e^-$ Wilson
coefficients $z_i$ and $y_i$, $i=1,\ldots,6$ at scale $\mu=1\gev$
can therefore be found in table \ref{tab:wc6smu1} of section
\ref{sec:HeffdF1:66}.
\\
The $K_L \to \pi^0 e^+ e^-$ Wilson coefficients for the remaining operators
$Q_{7V}$ and $Q_{7A}$ are given in table \ref{tab:wckpemu1}. Some
insight in the analytic structure of $y_{7V}$ will be gained by studying
the analogous decay $B \to X_s e^+ e^-$ in section \ref{sec:Heff:BXsee}
and also in section \ref{sec:KLpee} where the phenomenology of $K_L \to
\pi^0 e^+ e^-$ will be presented.

\begin{table}[htb]
\caption[]{$K_{\rm L} \to \pi^0 e^+ e^-$ Wilson coefficients
$z_{7V}/\aem$ and  $y_{7V}/\aem$ for $\mt=170\gev$ and various values
of $\mu$.
\label{tab:kpemuzy7}}
\begin{center}
\begin{tabular}{|c|c|c|c||c|c|c||c|c|c|}
& \multicolumn{3}{c||}{$\Lms^{(4)}=215\mev$} &
  \multicolumn{3}{c||}{$\Lms=^{(4)}325\mev$} &
  \multicolumn{3}{c| }{$\Lms=^{(4)}435\mev$} \\
\hline
Scheme & LO & NDR & HV & LO & NDR & HV & LO & NDR & HV \\
\hline
$\mu\,[\gev]$ & \multicolumn{9}{c|}{$z_{7V}/\aem$} \\
\hline
0.8 & --0.031 & --0.029 & 0.004 & --0.053 &
      --0.081 & --0.012 & --0.077 & --0.149 & --0.023 \\
1.0 & --0.014 & --0.015 & 0.005 & --0.024 &
      --0.046 & --0.003 & --0.035 & --0.084 & --0.011 \\
1.2 & --0.004 & --0.009 & 0.002 & --0.006 &
      --0.029 &       0 & --0.009 & --0.051 & --0.002 \\
\hline
$\mu\,[\gev]$ & \multicolumn{9}{c|}{$y_{7V}/\aem$} \\
\hline
0.8 & 0.578 & 0.751 & 0.744 & 0.545 & 0.739 & 0.730 & 0.514 & 0.722 & 0.712 \\
1.0 & 0.575 & 0.747 & 0.740 & 0.540 & 0.735 & 0.725 & 0.509 & 0.720 & 0.710 \\
1.2 & 0.571 & 0.744 & 0.736 & 0.537 & 0.731 & 0.721 & 0.505 & 0.716 & 0.706
\end{tabular}
\end{center}
\end{table}

In table \ref{tab:kpemuzy7} we show the $\mu$-dependence of
$z_{7V}/\aem$ and $y_{7V}/\aem$.  We find a pronounced scheme and
$\mu$-dependence for $z_{7V}$.  This signals that these dependences
have to be carefully addressed in the calculation of the CP conserving
part in the $K_L \to \pi^0 e^+ e^-$ amplitude. On the other hand, the
scheme and $\mu$-dependences for $y_{7V}$ are below $\ord(1.5\%)$.
\\
Similarly, $z_{7V}$ shows a strong dependence on the choice of the QCD
scale $\Lms$ while this dependence is small or absent for $y_{7V}$ and
$y_{7A}$, respectively.
\\
Finally, as seen from eq.\ \eqn{eq:z7primemc} $z_{7V}$ is independent
of $\mt$. However, with in/decreasing $\mt$ in the range $\mt = (170
\pm 15)\gev$ there is a relative variation of $\ord(\pm 3\%)$ and
$\ord(\pm 14\%)$ for the absolute values of $y_{7V}$ and $y_{7A}$,
respectively.
This is illustrated further in fig.\ \ref{fig:kpiee:mty7VA} and table
\ref{tab:kpey7VAmt} where the $\mt$ dependence of these coefficients is
shown explicitly. Accidentally for $m_t\approx 175\gev$ one finds
$|y_{7V}| \approx |y_{7A}|$.  Most importantly the impact of NLO
corrections is to enhance the Wilson coefficient $y_{7V}$ by roughly
$25\%$. As we will see in section \ref{sec:KLpee} this implies an
enhancement of the direct CP violation in $\Kpiee$.

\begin{figure}[hbt]
\vspace{0.10in}
\centerline{
\epsfysize=5in
\rotate[r]{
\epsffile{ps/mty7VA.ps}
} }
\vspace{0.08in}
\caption[]{
Wilson coefficients $|y_{7V}/\aem|^2$ and $|y_{7A}/\aem|^2$ as a
function of $\mt$ for $\Lms^{(4)}=325\mev$ at scale $\mu=1.0\gev$.
\label{fig:kpiee:mty7VA}}
\end{figure}

\begin{table}[htb]
\caption[]{$K_{\rm L} \to \pi^0 e^+ e^-$ Wilson coefficients
$y_{7V}/\aem$ and $y_{7A}/\aem$ for $\mu=1.0\gev$ and various values of
$\mt$.
\label{tab:kpey7VAmt}}
\begin{center}
\begin{tabular}{|c|c|c|c||c|c|c||c|c|c||c|}
& \multicolumn{9}{c||}{$y_{7V}/\aem$} & $y_{7A}/\aem$ \\
\hline
& \multicolumn{3}{c||}{$\Lms^{(4)}=215\mev$} &
  \multicolumn{3}{c||}{$\Lms^{(4)}=325\mev$} &
  \multicolumn{3}{c||}{$\Lms^{(4)}=435\mev$} & \\
\hline
$\mt [\gev]$ & LO & NDR & HV & LO & NDR & HV & LO & NDR & HV & \\
\hline
150 & 0.546 & 0.719 & 0.711 & 0.512 & 
0.706 & 0.697 & 0.481 & 0.692 & 0.681 & --0.576 \\
160 & 0.560 & 0.733 & 0.726 & 0.526 & 
0.721 & 0.711 & 0.495 & 0.706 & 0.696 & --0.637 \\
170 & 0.575 & 0.747 & 0.740 & 0.540 & 
0.735 & 0.725 & 0.509 & 0.720 & 0.710 & --0.700 \\
180 & 0.588 & 0.761 & 0.753 & 0.554 & 
0.748 & 0.739 & 0.523 & 0.734 & 0.723 & --0.765 \\
190 & 0.601 & 0.774 & 0.766 & 0.567 & 
0.761 & 0.752 & 0.536 & 0.747 & 0.736 & --0.833 \\
200 & 0.614 & 0.786 & 0.779 & 0.580 & 
0.774 & 0.764 & 0.549 & 0.760 & 0.749 & --0.902 \\
\end{tabular}
\end{center}
\end{table}
