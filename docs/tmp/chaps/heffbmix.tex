\section{The Effective Hamiltonian for $B^0-\bar B^0$ Mixing}
\label{sec:HeffBBbar}

\subsection{General Structure}
\label{sec:HeffBBbar:General}
Due to the particular hierarchy of the CKM matrix elements only the
top sector can contribute significantly to $B^0-\bar B^0$ mixing.
The charm sector and the mixed top-charm contributions are
entirely negligible here, in contrast to the $K^0-\bar K^0$ case,
which considerably simplifies the analysis.

Refering to  our earlier presentation of the top sector for $\Delta
S=2$ transitions in section \ref{sec:HeffKKbar:eta2} we can immediately
write down the effective $\Delta B=2$ hamiltonian. Performing the RG
evolution only down to scales $\mu_b=\ord(m_b)$ and making the
necessary replacements ($s\to b$) we get, in analogy to (\ref{hds2})
\cite{burasjaminweisz:90}
\begin{equation}\label{hdb2}
{\cal H}^{\Delta B=2}_{eff}=\frac{G^2_F}{16\pi^2}M^2_W
 \left(V^\ast_{tb}V_{td}\right)^2 \eta_{2B}
 S_0(x_t) \left[\as(\mu_b)\right]^{-6/23}\left[
  1 + \frac{\as(\mu_b)}{4\pi} J_5\right]  Q + h. c.
\end{equation}
where here
\begin{equation}\label{qbdbd}
Q=(\bar bd)_{V-A}(\bar bd)_{V-A}
\end{equation}
and
\begin{eqnarray}\label{eta2b}
&&\eta_{2B}=\left[\as(\mu_t)\right]^{6/23}\times
\\
&& \times \left[1+\frac{\as(\mu_t)}{4\pi} \left(
 \frac{S_1(x_t)}{S_0(x_t)}+B_t - J_5+\frac{\gamma^{(0)}}{2}
 \ln\frac{\mu^2_t}{M^2_W}+\gamma_{m0}
 \frac{\partial\ln S_0(x_t)}{\partial\ln x_t}\ln\frac{\mu^2_t}{M^2_W}
\right)\right] \nonumber
\end{eqnarray}
The definitions of the various quantities in (\ref{eta2b}) can
be found in section \ref{sec:HeffKKbar:eta2}.
Several important aspects of $\eta_2$ in the kaon system have also
been discussed in this section. Similar comments apply to the
present case of $\eta_{2B}$.
Here we would still like to supplement this discussion by writing 
down the formula for $\eta_{2B}$ in the limiting case $\mt\gg \mw$,
\begin{eqnarray}\label{eta2bas}
&&\eta_{2B}=\left[\alpha_s(\mu_t)\right]^{6/23}\times
\\
&& \times \left[1+\frac{\alpha_s(\mu_t)}{4\pi} \left(
\frac{\gamma^{(0)}}{2}\ln\frac{\mu^2_t}{m^2_t}+
\gamma_{m0}\ln\frac{\mu^2_t}{m^2_t}+11-\frac{20}{9}\pi^2+B_t-J_5+
\ord\left(\frac{M^2_W}{m^2_t}\right) \right)\right] \nonumber
\end{eqnarray}
This expression clarifies the structure of the RG evolution in the
limit $\mt\gg \mw$. It also suggests that the 
renormalization scale is most naturally to be taken as
$\mu_t=\ord(\mt)$ rather than $\mu_t=\ord(\mw)$,
both in the definition of the top quark mass and as the initial
scale of the RG evolution. Formula (\ref{eta2bas}) also holds,
with obvious modifications, for the $\eta_2$ factor in the
kaon system, which has been discussed in sec. \ref{sec:HeffKKbar:eta2}.

We finally mention that in the literature the $\mu_b$-dependent
factors in (\ref{hdb2}) are sometimes not attributed to the
matrix elements of $Q$, as implied by (\ref{hdb2}), but absorbed
into the definition of the QCD correction factor
\begin{equation}\label{e2bbar}
\bar\eta_{2B}=\eta_{2B} \left[\as(\mu_b)\right]^{-6/23}\left[
  1 + \frac{\as(\mu_b)}{4\pi} J_5\right]
\end{equation}
Whichever definition is employed, it is important to remember this
difference and to evaluate the hadronic matrix element consistently.
Note that, in contrast to $\eta_{2B}$, $\bar\eta_{2B}$ is scale
and scheme dependent.

\subsection{Numerical Results}
\label{sec:HeffBBbar:Num}
The correction factor $\eta_{2B}$ describes the short-distance
QCD effects in the theoretical expression for $B^0-\bar B^0$
mixing.
Due to the arbitrariness that exists in dividing
the physical amplitude into short-distance contribution and
hadronic matrix element, the short-distance QCD factor is strictly
speaking an unphysical quantity and hence definition dependent.
The $B$-factor, parametrizing the hadronic matrix element, has to
match the convention used for $\eta_{2B}$. With the definition of
$\eta_{2B}$ employed in this article and given explicitly in the previous
section, the appropriate $B$-factor to be used is the so-called
scheme independent bag-parameter $B_B$ as defined in eq.
(\ref{eq:BBrenorm}), where $\mu=\mu_b={\cal O}(m_b)$.
We remark, that the factor $\eta_{2B}$ is identical for
$B_d-\bar B_d$ and $B_s-\bar B_s$ mixing. The effects of $SU(3)$
breaking enter only the hadronic matrix elements. This feature
is a consequence of the factorization of short-distance and
long-distance contributions inherent to the operator product expansion.
For further
comments see also the discussion of the analogous case of
short-distance QCD factors in the neutral kaon system in section
\ref{sec:HeffKKbar:Num:Rem}.
\\
In the following we summarize the main results of a numerical analysis
of $\eta_{2B}$. The factor $\eta_{2B}$ is analogous to $\eta_2$
entering the top contribution to $K^0-\bar K^0$ mixing and both
quantities share many important features.
\\
The value of $\eta_{2B}$ for
$\Lms^{(4)}=0.325 GeV$, $\mt(\mt)=170 GeV$ and
with $\mu_t$ set equal to $\mt(\mt)$ reads at NLO
\begin{equation}\label{eta2bnum}
\eta_{2B}=0.551
\end{equation}
This can be compared with $\eta^{LO}_{2B}=0.580$, obtained, using
the same input, in the leading logarithmic approximation.
In the latter case the product
$\eta^{LO}_{2B}(\mu_t)\cdot S(x_t(\mu_t))$ is, however, affected
by a residual scale ambiguity of $\pm 9\%$ (for
$100 GeV\leq\mu_t\leq 300 GeV$). This uncertainty is reduced to the
negligible amount of $\pm 0.3\%$ in the complete NLO expression of
$\eta_{2B}(\mu_t)\cdot S(x_t(\mu_t))$, corresponding to an increase
in accuracy by a factor of 25. The sensitivity to the unphysical
scale $\mu_t$ in leading and next-to-leading order is illustrated
in fig. \ref{fig:eta2mut}.

\begin{figure}[htb]
\vspace{0.15in}
\centerline{
\epsfysize=4.5in
\rotate[r]{
\epsffile{ps/eta2mut.ps}
}}
\vspace{0.15in}
\caption[]{
Scale $\mu_t$ dependence of $\eta_{2B}(\mu_t) S_0(x_t(\mu_t))$ in LO and
NLO.
The quantity $\eta_{2B}(\mu_t) S_0(x_t(\mu_t))$ enters the theoretical
expression for $\Delta m_B$, describing $B^0-\bar B^0$ mixing. It is
independent of the precise value of the renormalization scale $\mu_t$
up to terms of the neglected order in $\as$. The remaining
sensitivity represents an unavoidable theoretical uncertainty.
This ambiguity is shown here for the leading order (dashed) and
the next-to-leading order calculation (solid).
\label{fig:eta2mut}}
\end{figure}

In addition the number shown in (\ref{eta2bnum}) is also very stable
against changes in the input parameters. Taking $\Lms^{(4)}=(0.325\pm
0.110) GeV$ and $\mt(\mt)=(170\pm 15)GeV$ results in a variation of
$\eta_{2B}$ by $\pm 1.3\%$ and $\pm 0.3\%$, respectively.

It is clear from this discussion, that the short-distance QCD effects
in $B^0-\bar B^0$ mixing are very well under control, once NLO
corrections have been properly included, and the remaining
uncertainties are extremely small. The effective hamiltonian given in
(\ref{hdb2}) therefore provides a solid foundation for the
incorporation of non-perturbative effects, to be determined from
lattice gauge theory, and for further phenomenological investigations
related to $B^0-\bar B^0$ mixing phenomena.
