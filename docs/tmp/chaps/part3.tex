\skipevenpage

{\Huge\bf
\noindent
Part Three --

\bigskip
\bigskip
\bigskip

\noindent
The Phenomenology of Weak Decays
}

\vfil

\noindent
The third part of our review presents the phenomenological picture
of weak decays beyond the leading logarithmic approximation.

There is essentially a one-to-one correspondence between the sections
in the second and in the third part of this review. Part three
uses heavily the results derived in part two. In spite of this, the
third part is meant to be essentially self-contained and can be followed
without difficulties by those readers who only scanned the material
of the second part and read section \ref{sec:sewm}.

The phenomenological part of our review is organized as follows.  We
begin with a few comments on the input parameters in section
\ref{sec:inputparams}. Next, as an application of the NLO corrections
in the current-current sector, we summarize the present status of the
tree level inclusive B-decays, in particular the theoretical status of
the semi-leptonic branching ratio.  The issue of exclusive two-body
non-leptonic decays and the question of factorization will not be
discussed here. The numerical values of the related factors $a_i$ for
various renormalization schemes can be found in \cite{buras:94}.
\\
The main part of the phenomenology begins in section \ref{sec:epsBBUT}
where we update the "standard" analysis of the unitarity triangle based
on the indirect CP violation in $K\to\pi\pi$ (the parameter
$\varepsilon_K$) and the $B^0_d-\bar B^0_d$ mixing described by $x_d$.
We incorporate in this analysis the most recent values of $m_t$,
$V_{ub}/V_{cb}$, $V_{cb}$, $B_K$ and $F_B$.  In addition to the
analysis of the unitarity triangle we determine several quantities of
interest. These results will be used frequently in subsequent sections.
\\
In section \ref{sec:nloepe} we present $\varepsilon'/\epsilon$ beyond
leading logarithms, summarizing and updating the extensive analysis
presented in \cite{burasetal:92d}. $\varepsilon'$ measures the size of
the direct CP violation in $K\to \pi\pi$ and its accurate estimate is
an important but very difficult task. In section \ref{sec:mki12} we
discuss briefly the $K_L-K_S$ mass difference and the $\Delta I=1/2$
rule. Next, in section \ref{sec:KLpee} we present an update for 
$K_L\to\pi^0 e^+e^-$.
\\
Next in sections \ref{sec:Heff:Bsgamma} and \ref{sec:Heff:BXsee:nlo} we
consider $B\to X_s\gamma$ and $B\to X_s e^+e^-$, respectively.  $B\to
X_s\gamma$ is known only in the LO approximation. However, in view of
its importance we summarize the leading order formulae and show the
standard model prediction compared with the CLEO II findings. We also
summarize the present status of NLO calculations for this decay. The
NLO calculations for $B\to X_s e^+ e^-$ have been completed and we give
a brief account of these results.
\\
Sections \ref{sec:Kpnn}--\ref{sec:BXnnBmm} discuss  $K\to
\pi\nu\bar\nu$, $K_L\to \mu^+\mu^-$ and rare B-decays ($B\to
X_s\nu\bar\nu$, $B\to l^+l^-$).  Except for $K_L\to \mu^+\mu^-$, all
these decays have only very small hadronic uncertainties and the
dominant theoretical errors are related to various renormalization
scale ambiguities. We demonstrate that these uncertainties are
considerably reduced by including NLO corrections, which will improve
the determination of the CKM matrix in forthcoming experiments. Using
the results of section \ref{sec:epsBBUT}, we also give updated standard
model predictions for these decays.
