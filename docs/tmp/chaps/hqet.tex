\section{Heavy Quark Effective Theory Beyond Leading Logs}
\label{sec:HQET}

\subsection{General Remarks}
\label{sec:HQET:General}

Since its advent in 1989 heavy quark effective theory (HQET) has
developped into an elaborate and well-established formalism, providing a
systematic framework for the treatment of hadrons containing a heavy quark.
HQET represents a static approximation for
the heavy quark, covariantly formulated in the language of an effective
field theory.
It allows to extract the dependence of hadronic matrix elements on the
heavy quark mass and to exploit the simplifications that arise in QCD in
the static limit.
\\
There are several excellent reviews on this subject \cite{neubert:94},
\cite{georgi:91}, \cite{grinstein:91}, \cite{isgurwise:92},
\cite{mannel:93} and we do not attempt here to cover the details of
this extended field. However, we would like to emphasize the close
parallels in the general formalism employed to calculate perturbative
QCD effects for the effective weak hamiltonians we have been discussing
in this review and in the context of HQET. In particular we will
concentrate on results that have been obtained in HQET beyond the
leading logarithmic approximation in QCD perturbation theory. Such
calculations have been done mainly for bilinear current operators
involving heavy quark fields, which have important applications in
semileptonic decays of heavy hadrons.  For the purpose of illustration
we will focus on the simplest case of heavy-light currents as an
important example.  Furthermore, while existing reviews concentrate on
semileptonic decays and current operators, we will also include results
obtained for nonleptonic transitions and summarize the calculation of
NLO QCD corrections to $B^0-\bar B^0$ mixing in HQET \cite{flynnetal:91},
\cite{gimenez:93}. These latter papers generalize the leading-log
results first obtained in \cite{voloshinshifman:87}, \cite{politzerwise:88a},
\cite{politzerwise:88b}.
\\
Throughout we will restrict ourselves to the leading order in HQET and
not address the question of $1/m$ corrections. For a discussion of this
topic we refer the reader to the literature, in particular the above
mentioned review articles.

\subsection{Basic Concepts}
\label{sec:HQET:Basic}

Let us briefly recall the most important basic concepts
underlying the idea of HQET. 
\\
The Lagrangian describing a quark field $\Psi$ with mass $m$ and its QCD
interactions with gluons reads
\begin{equation}\label{lful}
{\cal L}=\bar\Psi i\not\!\! D\Psi-m\bar\Psi\Psi
\end{equation}
where $D_\mu=\partial_\mu-i g T^a A^a_\mu$ is the gauge-covariant
derivative. If $\Psi$ is a heavy quark, i.e. its mass is large compared to
the QCD scale, $\Lambda_{QCD}/m\ll 1$, it acts approximately like a
static color source and its QCD interactions simplify.
A heavy quark inside a hadron moving with velocity $v$ has approximately
the same velocity. Thus its momentum can be written as $p=m v+k$, where $k$
is a small residual momentum of the order of $\Lambda_{QCD}$ and
subject to changes of the same order through soft QCD interactions.
To implement this approximation, the quark field $\Psi$ is decomposed into
\begin{equation}\label{qehv}
\Psi(x)=e^{-i m v\cdot x}\left(h_v(x)+H_v(x)\right)
\end{equation}
with $h_v$ and $H_v$ defined by
\begin{equation}\label{hvpl}
h_v(x)=e^{imv\cdot x}\frac{1+\not\! v}{2} \Psi(x)
\end{equation}
\begin{equation}\label{hvmi}
H_v(x)=e^{imv\cdot x}\frac{1-\not\! v}{2} \Psi(x)
\end{equation}
To be specific we consider here the case of a hadron containing a heavy
quark, as opposed to a heavy antiquark. In order to describe a heavy
antiquark, the definitions (\ref{hvpl}) and (\ref{hvmi}) are replaced by
\begin{equation}\label{ahvp}
h^{(-)}_v(x)=e^{-imv\cdot x}\frac{1-\not\! v}{2} \Psi(x)
\end{equation}
\begin{equation}\label{ahvm}
H^{(-)}_v(x)=e^{-imv\cdot x}\frac{1+\not\! v}{2} \Psi(x)
\end{equation}
Consequently, for a heavy antiquark, one only needs to substitute $v\to -v$ in the  
expressions given below for the case of a heavy quark.
\\
In the rest frame of the heavy quark $h_v$ and $H_v$ correspond to the
upper and lower components of $\Psi$, respectively. In general, for
$m\to\infty$ $h_v$ and $H_v$ represent the "large" and "small" components
of $\Psi$. In fact, the equations of motion of QCD imply that $H_v$ is
suppressed by a factor $\Lambda_{QCD}/m$ in comparison to $h_v$.
The inclusion of an explicit exponential factor $\exp(-i mv\cdot x)$ in
(\ref{qehv}) ensures that the momentum associated with the field $h_v$
is only a small residual momentum of order $\Lambda_{QCD}$.
\\
Now an effective theory for $h_v$ is constructed by eliminating the
small component field $H_v$ from explicitly appearing in the
description of the heavy quark. On the classical level this can be done
by using the equations of motion or, equivalently, by directly
integrating out the $H_v$ degrees of freedom in the context of a path
integral formulation \cite{manneletal:92}.  The effective Lagrangian
one obtains from (\ref{lful}) along these lines is given by
($D^\mu_\perp=D^\mu-v^\mu v\cdot D$)
\begin{equation}\label{lhqt}
{\cal L}_{eff,tot}=\bar h_v iv\cdot D h_v+\bar h_v i\not\!\! D_\perp
\frac{1}{iv\cdot D+2m-i\varepsilon}i\not\!\! D_\perp h_v
\end{equation}
The first term in (\ref{lhqt})
\begin{equation}\label{lehq}
{\cal L}_{eff}=\bar h_v (i v^\mu\partial_\mu+g v^\mu T^a A^a_\mu)h_v
\end{equation}
represents the Lagrangian of HQET to lowest order in $1/m$ and will be 
sufficient for our purposes. The second, nonlocal contribution in
(\ref{lhqt}) can be expanded into a series of local, higher dimension
operators, carrying coefficients with increasing powers of $1/m$. To first
order it yields the correction due to the residual heavy quark kinetic
energy and the chromo-magnetic interaction term, coupling the heavy
quark spin to the gluon field. 
\\
From (\ref{lehq}) one can obtain the Feynman rules of HQET, the
propagator of the effective field $h_v$
\begin{equation}\label{hqpr}
\frac{i}{v\cdot k} \frac{1+\not\! v}{2}
\end{equation}
and the $\bar h_v$-$h_v$-gluon vertex, $igv^\mu T^a$. The explicit
factor $(1+\not\! v)/2$ in (\ref{hqpr}) arises because the effective
field $h_v$ is a constrained spinor, satisfying $\not\! v h_v\equiv h_v$,
as is obvious from (\ref{hvpl}). The velocity $v_\mu$ is a constant in the
effective theory and plays the role of a label for the effective field $h_v$.
In principle, a different field $h_v$ has to be considered for every
velocity $v$.
\\
The Lagrangian in (\ref{lehq}) exhibits the crucial features of HQET:
The quark-gluon coupling is independent of the quark's spin degrees of
freedom and the Lagrangian is independent of the heavy quark flavor,
since the heavy quark mass has been eliminated. This observation forms
the basis for the spin-flavor symmetry of HQET \cite{isgurwise:89},
\cite{isgurwise:90}, which gives rise to important simplifications in
the strong interactions of heavy quarks and allows to establish
relations among the form factors of different heavy hadron matrix
elements. The heavy quark symmetries are broken by $1/m$-contributions
as well as radiative corrections.
\\
So far our discussion has been limited to the QCD interactions of the
heavy quark. Weak interactions introduce new operators into the theory,
which may be current operators, bilinear in quark fields, or four-quark
operators, relevant for semileptonic and nonleptonic transitions,
respectively. Such operators form the basic ingredients to be studied in
weak decay phenomenology. They can as well be expanded in $1/m$ and
incorporated into the framework of HQET. For example a heavy-light current
operator $\bar q\Gamma \Psi$ (evaluated at the origin, $x=0$), can be
written (\ref{qehv})
\begin{equation}\label{qgqh}
\bar q\Gamma \Psi=\bar q\Gamma h_v+{\cal O}(\frac{1}{m})
\end{equation}
to lowest order in HQET.
\\
Up to now we have restricted our discussion to the classical level. In
addition, of course, quantum radiative corrections have to be
included. They will for example modify relations such as (\ref{qgqh}).
Technically their effects are taken into account by performing the
appropriate matching calculations, relating operators in the effective
theory to the corresponding operators in the full theory to the
required order in renormalization group improved QCD perturbation theory.
The procedure is very similar to the calculation of the usual effective hamiltonians
for weak decays. The basic difference consists in the heavy degrees of
freedom that are being integrated out in the matching process. In the
general case of effective weak hamiltonians the heavy field to be 
removed as a dynamical variable is the W boson, whereas it is the lower
component heavy quark field $H_v$ in the case of HQET. This similarity
will become obvious from our presentation below.
\\
At this point some comment might be in order concerning the 
relationship of the HQET formalism to the general weak effective
hamiltonians discussed primarily in this review, in particular those
relevant for b-physics.
\\
The effective hamiltonians for $\Delta B=1, 2$ nonleptonic transitions
are the relevant hamiltonians for scales $\mu={\cal O}(m_b)$, which are
appropriate for B hadron decays, and their Wilson coefficients
incorporate the QCD short distance dynamics between scales of ${\cal
O}(M_W)$ and ${\cal O}(m_b)$. As already mentioned at the end of
section \ref{sec:HeffdF1:22} it is therefore not necessary to invoke
HQET. The physics below $\mu={\cal O}(m_b)$ is completely contained
within the relevant hadronic matrix elements.  On the other hand, HQET
may be useful in certain cases, like e.g.  $B^0-\bar B^0$ mixing, to
gain additional insight into the structure of the hadronic matrix
elements for scales below $m_b$, but still large compared to
$\Lambda_{QCD}$. These scales are still perturbative and the related
contributions can be extracted analytically within HQET.  In
particular, this procedure makes the dependence of the matrix element
on the heavy quark mass explicit, as we will see on examples below.
Furthermore, this approach can be useful e.g. in connection with
lattice calculations of hadronic matrix elements, which are easier to
perform in the static limit for b-quarks, i.e.\ employing HQET
\cite{sachrajda:92}. The simplifications obtained are however at the
expense of the approximation due to the expansion in $1/m$.
\\
The most important application of HQET has been to the analysis of
exclusive semileptonic transitions involving heavy quarks, where this
formalism allows to exploit the consequences of heavy quark symmetry to
relate formfactors and provides a basis for systematic corrections to the
$m\to\infty$ limit. This area of weak decay phenomenology has been already
reviewed in detail \cite{neubert:94}, \cite{georgi:91}, \cite{grinstein:91},
\cite{isgurwise:92}, \cite{mannel:93} and will not be covered
in the present article.

\subsection{Heavy-Light Currents}
\label{sec:HQET:Currents}
As an example of a next-to-leading QCD calculation within the context of
HQET, we will now discuss the case of a weak current, composed of one
heavy and one light quark field, to leading order in the $1/m$ expansion.
For definitness we consider the axial vector heavy-light current, whose
matrix elements determine the decay constants of pseudoscalar mesons
containing a single heavy quark, like $f_B$ and $f_D$.
\\
The axial vector current operator in the full theory is given by
\begin{equation}\label{aqq}
A=\bar q\gamma_\mu\gamma_5 \Psi
\end{equation}
where $\Psi$ is the heavy and $q$ the light quark field. In HQET this operator
can be expanded in the following way
\begin{equation}\label{ac12}
A=C_1(\mu)\tilde A_1+C_2(\mu)\tilde A_2+{\cal O}(\frac{1}{m})
\end{equation}
where the operator basis in the effective theory reads
\begin{equation}\label{at12}
\tilde A_1=\bar q\gamma_\mu\gamma_5 h_v \qquad\qquad
\tilde A_2=\bar q v_\mu\gamma_5 h_v
\end{equation}
with the heavy quark effective field $h_v$ defined in (\ref{hvpl}).
The use of the expansion (\ref{ac12}) is to make the dependence of the
matrix elements of $A$ on the heavy quark mass $m$ explicit. The dependence
on this mass is two-fold. First, there is a power dependence, which is 
manifest in the heavy quark expansion in powers of $1/m$. From this
series only the lowest order term is shown in (\ref{ac12}). Second, there is
a logarithmic dependence on $m$ due to QCD radiative corrections, which
can be calculated in perturbation theory. This dependence is factorized
into the coefficient functions $C_1$, $C_2$ in much the same way as the 
logarithmic dependence of nonleptonic weak decay amplitudes on the W boson 
mass is factorized into the Wilson coefficients of the usual weak
hamiltonians. Since the dynamics of HQET is, by construction, independent 
of $m$, no further $m$ dependence is present in the matrix
elements of the effective theory operators $\tilde A_{1,2}$, except for
trivial factors of $m$ related to the normalization of meson states.
Consequently the $m$ dependence of (\ref{ac12}) is determined explicitly. 

We remark that in general the meson states in HQET to be used for the
r.h.s. of (\ref{ac12}) differ from the meson states in the full theory
to be used to sandwich the operator $A$ on the l.h.s.. For the leading
order in $1/m$ we are working in this distinction is irrelevant, however.
\\
An important point is that the operators $\tilde A_{1,2}$ in the effective
theory have anomalous dimensions, although the operator $A$ in the full
theory, being an axial vector current operator, does not. As a 
consequence matrix elements of $\tilde A_{1,2}$ will depend on the
renormalization scale and scheme. This dependence is canceled however
through a corresponding dependence of the coefficients so that the physical
matrix elements of $A$ will be scale and scheme independent as they must be.
The existence of anomalous dimensions for the effective theory operators
merely reflects the logarithmic dependence on the heavy quark mass $m$
due to QCD effects. This dependence results in logarithmic divergences in
the limit $m\to\infty$, corresponding to the effective theory, which
require additional infinite renormalizations not present in full QCD.
Obviously the situation is completely analogous to the case of constructing
effective weak hamiltonians through integrating out the W boson, which we
have described in detail in section \ref{sec:basicform}. In fact, the
extraction of the coefficient functions by factorizing long and short
distance contributions to quark level amplitudes and the renormalization
group treatment follow exactly the same principles.
\\
The Wilson coefficients at the high matching scale $\mu_h={\cal O}(m)$,
the initial condition to the RG evolution, can be calculated in
ordinary perturbation theory with the result (NDR scheme)
\begin{equation}\label{c1mh}
C_1(\mu_h)=1+\frac{\alpha_s}{4\pi}\left(\gamma^{(0)}_{hl}\ln\frac{\mu_h}{m}
+B_1\right)
\end{equation}
\begin{equation}\label{c2mh}
C_2(\mu_h)=\frac{\alpha_s}{4\pi} B_2
\end{equation}
with
\begin{equation}\label{b12h}
B_1=-4 C_F\qquad\qquad B_2=-2 C_F
\end{equation}
and $\gamma^{(0)}_{hl}$ given in (\ref{g0hl}) below. $C_F$ is the QCD color
factor $(N^2-1)/(2N)$. We remark that the coefficient of the new
operator $\tilde A_2$, generated at ${\cal O}(\alpha_s)$, is finite without
requiring renormalization. As a consequence no explicit scale dependence
appears in (\ref{c2mh}) and $B_2$ is a scheme independent constant.
For the same reason $\tilde A_1$ and $\tilde A_2$ do not mix under
renormalization, but renormalize only multiplicatively.
The anomalous dimension of the effective heavy quark currents is independent
of the Dirac structure. It is the same for $\tilde A_1$ and $\tilde A_2$
and reads at two-loop order
\begin{equation}\label{g01h}
\gamma_{hl}=\gamma^{(0)}_{hl}\frac{\alpha_s}{4\pi}+\gamma^{(1)}_{hl}
\left(\frac{\alpha_s}{4\pi}\right)^2
\end{equation}
where
\begin{equation}\label{g0hl}
\gamma^{(0)}_{hl}=-3 C_F
\end{equation}
\begin{eqnarray}\label{g1hl}
\gamma^{(1)}_{hl} &=& \left(-\frac{49}{6}+\frac{2}{3}\pi^2\right)N C_F+
 \left(\frac{5}{2}-\frac{8}{3}\pi^2\right) C^2_F+\frac{5}{3}C_F f=
 \nonumber\\
&=& -\frac{254}{9}-\frac{56}{27}\pi^2+\frac{20}{9}f\qquad\qquad {\rm (NDR)}
\end{eqnarray}
$N$ ($f$) denotes the number of colors (flavors). The anomalous
dimension $\gamma^{(0)}_{hl}$ has been first calculated by
\cite{voloshinshifman:87} and \cite{politzerwise:88a},
\cite{politzerwise:88b}. The generalization to next-to-leading order
has been performed in \cite{jimuslof:91} and
\cite{broadhurstgrozin:91}.
\\
The RG equations are readily solved to obtain the coefficients at a lower
but still perturbative scale $\mu$, where, formally,
$\mu\ll\mu_h={\cal O}(m)$. Using the results of section
\ref{sec:basicform:wc} we have
\begin{equation}\label{c1hl}
C_1(\mu)=\left(1+\frac{\alpha_s(\mu)}{4\pi}J_{hl}\right)
\left[\frac{\alpha_s(\mu_h)}{\alpha_s(\mu)}\right]^
{d_{hl}}\left(1+\frac{\alpha_s(\mu_h)}{4\pi}
\left[\gamma^{(0)}_{hl}\ln\frac{\mu_h}{m}+B_1-J_{hl}\right]\right)
\end{equation}
\begin{equation}\label{c2hl}
C_2(\mu)=\left[\frac{\alpha_s(\mu_h)}{\alpha_s(\mu)}\right]^
{d_{hl}}\frac{\alpha_s(\mu_h)}{4\pi}B_2
\end{equation}
with
\begin{equation}\label{djhl}
d_{hl}=\frac{\gamma^{(0)}_{hl}}{2\beta_0}\qquad\qquad
J_{hl}=\frac{d_{hl}}{\beta_0}\beta_1-\frac{\gamma^{(1)}_{hl}}{2\beta_0}
\end{equation}
We remark that the corresponding formulae for the vector current can
be simply obtained from the above expressions by letting $\gamma_5\to 1$
and changing the sign of $B_2$.
\\
In addition to the case of heavy-light currents considered here, the
NLO corrections have also been calculated for flavor-conserving and
flavor-changing heavy-heavy currents of the type $\bar\Psi\Gamma\Psi$
and $\bar\Psi_1\Gamma\Psi_2$ respectively, where $\Psi$, $\Psi_{1,2}$
are heavy quark fields ($\Gamma=\gamma_\mu$, $\gamma_\mu\gamma_5$). In
these cases the anomalous dimensions become velocity dependent.
Additional complications arise in the analysis of flavor changing
heavy-heavy currents due to the presence of two distinct heavy mass
scales. For a detailed presentation see \cite{neubert:94} and
references cited therein.

\subsection{The Pseudoscalar Decay Constant in the Static Limit}
\label{sec:HQET:DecCons}
An important application of the results summarized in the last
section is the calculation of the short distance QCD effects, from
scales between $\mu_h={\cal O}(m)$ and the low scale $\mu={\cal O}(1 GeV)$,
for the decay constants $f_P$ of pseudoscalar heavy mesons. Using only
the leading term in the expansion (\ref{ac12}), omitting all $1/m$
power corrections, corresponds to the static limit for $f_P$, which plays
some role in lattice studies. As already mentioned we will restrict
ourselves here to this limit. We should remark however, that
non-negligible power corrections are known to exist for the realistic case 
of B or D meson decay constants \cite{sachrajda:92}.
\\
The decay constant $f_P$ is defined through
\begin{equation}\label{apme}
\langle 0|A|P\rangle =-i f_P m_P v_\mu
\end{equation}
where the pseudoscalar meson state $|P\rangle$ is normalized in the
conventional way ($\langle P|P\rangle=2 E V$). The matrix elements of
$\tilde A_{1,2}$ are related via heavy quark symmetry and are given by
\begin{equation}\label{atme}
\langle 0|\tilde A_1|P\rangle =-\langle 0|\tilde A_2|P\rangle =
  -i \tilde f(\mu) \sqrt{m_P} v_\mu
\end{equation}
Apart from the explicit mass factor $\sqrt{m_P}$, which is merely due to
the normalization of $|P\rangle$, these matrix elements are independent
of the heavy quark mass. The "reduced" decay constant $\tilde f(\mu)$ is
therefore $m$-independent. It does however depend on the 
renormalization scale and scheme chosen. The computation of
$\tilde f(\mu)$ is a nonperturbative problem involving strong dynamics
below scale $\mu$. Using (\ref{ac12}), (\ref{c1hl}), (\ref{c2hl}),
(\ref{apme}) and (\ref{atme}) we obtain
\begin{equation}\label{fpft}
f_P=\frac{\tilde f(\mu)}{\sqrt{m_P}}\left(1+\frac{\alpha_s(\mu)}{4\pi}J_{hl}
\right)\left[\frac{\alpha_s(\mu_h)}{\alpha_s(\mu)}\right]^{d_{hl}}
\left(1+\frac{\alpha_s(\mu_h)}{4\pi}
\left[\gamma^{(0)}_{hl}\ln\frac{\mu_h}{m}+B_1-J_{hl}-B_2\right]\right)
\end{equation}
The dependence of the coefficient function on the renormalization scheme
through $J_{hl}$ in the second factor in (\ref{fpft}), and its dependence on
$\mu$ cancel the corresponding dependences in the hadronic quantity
$\tilde f(\mu)$ to the considered order in $\alpha_s$. The last factor
in (\ref{fpft}) is scheme independent. Furthermore the cancellation of
the dependence on $\mu_h$ to the required order can be seen explicitly.
Note also the leading scaling behaviour $f_P\sim 1/\sqrt{m_P}$, which
is manifest in (\ref{fpft}).
\\
Although $\tilde f(\mu)$ cannot be calculated without nonperturbative
input, its independence of the heavy quark mass $m$ implies that $\tilde f$
will drop out in the ratio of $f_B$ over $f_D$, if charm is treated as
a heavy quark. One thus obtains
\begin{eqnarray}\label{fbfd}
\lefteqn{\frac{f_B}{f_D} = \sqrt{\frac{m_D}{m_B}}
\left[\frac{\alpha_s(\mu_b)}{\alpha_s(\mu_c)}\right]^{d_{hl}}
 \left( 1 +\frac{\alpha_s(\mu_b)-\alpha_s(\mu_c)}{4\pi}
\left(B_1-J_{hl}-B_2\right)+ \right.} \hspace{2cm} \nonumber\\
& & \hspace{3cm} + \left. \frac{\alpha_s(\mu_b)}{4\pi}
\gamma^{(0)}_{hl}\ln\frac{\mu_b}{m_b}-\frac{\alpha_s(\mu_c)}{4\pi}
\gamma^{(0)}_{hl}\ln\frac{\mu_c}{m_c}\right) 
\end{eqnarray}
The QCD factor on the right hand side of (\ref{fbfd}) amounts to $1.14$
for $m_b=4.8 GeV$, $m_c=1.4GeV$ and $\Lambda_{\overline{MS}}=0.2 GeV$
if we set $\mu_i=m_i$, $i=b,c$. If we allow for a variation of the
renormalization scales as $2/3\leq\mu_i/m_i\leq 2$, this factor lies
within a range of $1.12$ to $1.16$. This is to be compared with the
leading log approximation, where the central value reads $1.12$ with a
variation from $1.10$ to $1.15$. Note that due to cancellations in the
ratio $f_B/f_D$ the scale ambiguity is not much larger in LLA than in
NLLA. However the next-to-leading order QCD effects further enhance
$f_B/f_D$ independently of the renormalization scheme. 

\subsection{$\Delta B=2$ Transitions in the Static Limit}
\label{sec:HQET:DelB2}
In section \ref{sec:HeffBBbar} we have described the effective
hamiltonian for $B^0-\bar B^0$ mixing. The calculation of the mixing
amplitude requires in particular the evaluation of the matrix element
$\langle \bar B^0|Q|B^0\rangle\equiv\langle Q\rangle$ of the operator
\begin{equation}\label{qbd}
Q=(\bar bd)_{V-A}(\bar bd)_{V-A}
\end{equation}
in addition to the short-distance Wilson coefficient. Coefficient
function and operator matrix element are to be evaluated at a common
renormalization scale, $\mu_b={\cal O}(m_b)$, say. In contrast to the 
determination of the Wilson coefficient, the computation of the hadronic
matrix element involves nonperturbative long-distance contributions.
Ultimately this problem should be solved using lattice QCD. However,
the b quark is rather heavy and it is therefore difficult to
incorporate it as a fully dynamical field in the context of a
lattice regularization approach. On the other hand QCD effects from
scales below $\mu_b={\cal O}(m_b)$ down to $\sim 1GeV$ are still
accessible to a perturbative treatment. HQET provides the tool to
calculate these contributions. At the same time it allows one to
extract the dependence of $\langle \bar B^0|Q|B^0\rangle$ on the bottom
mass $m_b$ explicitly, albeit at the prize of the further
approximation introduced by the expansion in inverse powers of $m_b$.
\\
In a first step the operator $Q$ in (\ref{qbd}) is expressed as a
linear combination of HQET operators by matching the "full" to the
effective theory at a scale $\mu_b={\cal O}(m_b)$
\begin{equation}\label{qqtm}
\langle Q(\mu_b)\rangle=\left(1+\frac{\alpha_s(\mu_b)}{4\pi}
\left[(\tilde\gamma^{(0)}-\gamma^{(0)})\ln\frac{\mu_b}{m_b}+\tilde B-B
\right]\right)\langle\tilde Q(\mu_b)\rangle+
\frac{\alpha_s(\mu_b)}{4\pi}\tilde B_s \langle\tilde Q_s(\mu_b)\rangle
\end{equation}
Here 
\begin{equation}\label{qqst}
\tilde Q=2(\bar hd)_{V-A}(\bar h^{(-)}d)_{V-A}\qquad\qquad
\tilde Q_s=2(\bar hd)_{S-P}(\bar h^{(-)}d)_{S-P}
\end{equation}
($(\bar hd)_{S-P}\equiv\bar h(1-\gamma_5)d$)
are the necessary operators in HQET relevant for the case of a
$B^0\to\bar B^0$ transition. The field $\bar h$ creates a heavy quark,
while $\bar h^{(-)}$ annihilates a heavy antiquark. Since the
effective theory field $\bar h$ ($\bar h^{(-)}$) cannot, unlike the
full theory field $\bar b$ in $Q$, at the same time annihilate (create)
the heavy antiquark (heavy quark), explicit factors of two have to appear
in (\ref{qqst}). Similarly to the case of the heavy-light current
discussed in the previous section a new operator $\tilde Q_s$ with
scalar-pseudoscalar structure is generated. Its coefficient is finite
and hence no operator mixing under infinite renormalization occurs
between $\tilde Q$ and $\tilde Q_s$.
\\
In a second step, the matrix element $\langle\tilde Q(\mu_b)\rangle$
at the high scale $\mu_b$ has to be expressed through the matrix
element of $\tilde Q$ evaluated at a lower scale $\mu\sim 1 GeV$,
which is relevant for a nonperturbative calculation, for example
using lattice gauge theory. This relation can be obtained through the
usual renormalization group evolution and reads in NLLA
\begin{equation}\label{qtrg}
\langle\tilde Q(\mu_b)\rangle=
\left[\frac{\alpha_s(\mu_b)}{\alpha_s(\mu)}\right]^{\tilde d}
\left(1+\frac{\alpha_s(\mu)-\alpha_s(\mu_b)}{4\pi}\tilde J\right)
\langle\tilde Q(\mu)\rangle
\end{equation}
where
\begin{equation}\label{djt}
\tilde d=\frac{\tilde\gamma^{(0)}}{2\beta_0}\qquad\qquad
\tilde J=\frac{\tilde d}{\beta_0}\beta_1-\frac{\tilde\gamma^{(1)}}{2\beta_0}
\end{equation}
with the beta-function coefficients $\beta_0$ and $\beta_1$ given in
(\ref{b0b1}). The calculation of the one-loop anomalous dimension
$\tilde\gamma^{(0)}$ of the HQET operator $\tilde Q$, required for the
leading log approximation to (\ref{qtrg}), has been first performed in
\cite{voloshinshifman:87} and \cite{politzerwise:88a},
\cite{politzerwise:88b}. The computation of the two-loop anomalous
dimension $\tilde\gamma^{(1)}$ is due to \cite{gimenez:93}. Finally,
the next-to-leading order matching condition (\ref{qqtm}) has been
determined in \cite{flynnetal:91}.  In the following we summarize the
results obtained in these papers.

The scheme dependent next-to-leading order quantities $B$, $\tilde B$
and $\tilde\gamma^{(1)}$ refer to the NDR scheme with anticommuting
$\gamma_5$ and the usual subtraction of evanescent terms as defined
in \cite{burasweisz:90}. For $N=3$ colors we then have
\begin{equation}\label{ggt}
\tilde\gamma^{(0)}=-8\qquad\qquad\gamma^{(0)}=4
\end{equation}
\begin{equation}\label{bbt}
\tilde B-B=-14\qquad B=\frac{11}{3}\qquad \tilde B_s=-8
\end{equation}
\begin{equation}\label{gt1}
\tilde\gamma^{(1)}=-\frac{808}{9}-\frac{52}{27}\pi^2+\frac{64}{9}f
\end{equation}
where $f$ is the number of active flavors.  

At this point we would like to make the following comments.
\begin{itemize}
\item
The logarithmic term in (\ref{qqtm}) reflects the ${\cal O}(\alpha_s)$
scale dependence of the matrix elements of $Q$ and
$\tilde Q$. Accordingly its coefficient is given by
the difference in the one-loop anomalous dimensions of these operators,
$\gamma^{(0)}$ and $\tilde\gamma^{(0)}$.
\item
The one-loop anomalous dimension of the effective theory operator
$\tilde Q$, $\tilde\gamma^{(0)}$, is exactly twice as large as the
one-loop anomalous dimension of the heavy-light current discussed in
section \ref{sec:HQET:Currents} (see eq. (\ref{g0hl}). Therefore the
scale dependence of $\langle\tilde Q\rangle$ below $\mu_b$ is entirely
contained in the scale dependence of the decay constant squared
$\tilde f^2(\mu)$. This implies the well known result that in leading log
approximation the parameter $B_B$ has no perturbative scale
dependence in the static theory below $\mu_b$. As the result of
\cite{gimenez:93} for $\tilde\gamma^{(1)}$ shows, this somewhat accidental
cancellation is not valid beyond the one-loop level.
\item
The matching condition (\ref{qqtm}) contains besides the logarithm
a scheme dependent constant term in the relation between
$\langle Q\rangle$ and $\langle\tilde Q\rangle$. We have written this
coefficient in the form $\tilde B-B$ in order to make the
cancellation of scheme dependences, to be discussed below, more
transparent. Here $B$ is identical to $B_+$ introduced in (\ref{B8})
and characterizes the scheme dependence of $\langle Q\rangle$ 
(see also sections \ref{sec:HeffKKbar} and \ref{sec:HeffBBbar}).
\item
The quantity $\tilde\gamma^{(1)}$ has been originally calculated in
\cite{gimenez:93} using dimensional reduction (DRED) instead of NDR as
renormalization scheme. However, $\tilde B$ turns out to be the same in
DRED and NDR, implying that also $\tilde\gamma^{(1)}$ coincides in
these schemes \cite{gimenez:93}.
\end{itemize}

Finally we may put together (\ref{qqtm}) and (\ref{qtrg}), omitting
for the moment the scheme independent, constant correction due to
$\tilde Q_s$, to obtain 
\begin{equation}\label{qqtl}
\langle Q(\mu_b)\rangle=
\left[\frac{\alpha_s(\mu_b)}{\alpha_s(\mu)}\right]^{\tilde d}
\left(1+\frac{\alpha_s(\mu_b)}{4\pi}
\left[(\tilde\gamma^{(0)}-\gamma^{(0)})\ln\frac{\mu_b}{m_b}+\tilde B-B
-\tilde J\right]+\frac{\alpha_s(\mu)}{4\pi}\tilde J\right)
\langle\tilde Q(\mu)\rangle
\end{equation}
This relation serves to express the $B^0-\bar B^0$ matrix element of the
operator $Q$, evaluated at a scale $\mu_b={\cal O}(m_b)$, which is the
relevant scale for the effective hamiltonian of section 
\ref{sec:HeffBBbar}, in terms of the static theory matrix element
$\langle\tilde Q(\mu)\rangle$ normalized at a low scale $\mu\sim 1 GeV$.
The latter is more readily accessible to a nonperturbative lattice
calculation than the full theory matrix element $\langle Q(\mu_b)\rangle$.
Note that (\ref{qqtl}) as it stands is valid in the continuum theory.
In order to use lattice results one still has to perform an ${\cal
O}(\alpha_s)$ matching of $\tilde Q$ to its lattice counterpart. This
step however does not require any further renormalization group
improvement since by means of (\ref{qqtl}) $\tilde Q$ is already
normalized at the appropriate low scale $\mu$.  The continuum --
lattice theory matching was determined in \cite{flynnetal:91} and is
also discussed in \cite{gimenez:93}.
\\
Of course, the right hand side in (\ref{qqtl}) gives only the leading
contribution in the $1/m$ expansion of the full matrix element $\langle
Q(\mu_b)\rangle$ (apart from $\tilde Q_s$). Going beyond this
approximation would require the consideration of several new operators,
which appear at the next order in $1/m$.  These contributions have been
studied in \cite{kilianmannel:93} in the leading logarithmic
approximation. On the other hand (\ref{qqtl}), while restricted to the
static limit, includes and resums all leading and next-to-leading
logarithmic corrections between the scales $\mu_b={\cal O}(m_b)$ and
$\mu\sim 1GeV$ in the relation among $Q$ and $\tilde Q$. It is
interesting to consider the scale and scheme dependences present in
(\ref{qqtl}). The dependence on $\mu$ in the first factor on the r.h.s.
of (\ref{qqtl}) is canceled by the $\mu$-dependence of $\langle\tilde
Q(\mu)\rangle$. The dependence on $\mu_b$ of this factor is canceled by
the explicit $\ln\mu_b$ term proportional to $\tilde\gamma^{(0)}$.
Hence the only scale dependence remaining on the r.h.s., to the
considered order ${\cal O}(\alpha_s)$, is the one
$\sim\alpha_s(\mu_b)\gamma^{(0)}\ln\mu_b$. This is precisely the scale
dependence of the full theory matrix element on the l.h.s., which is
required to cancel the corresponding dependence of the Wilson
coefficient. Similarly the term $\sim\alpha_s(\mu_b)B$ represents the
correct scheme dependence of $\langle Q(\mu_b)\rangle$, while the
scheme dependence of $\alpha_s(\mu)\tilde J$ cancels with the scheme
dependence of $\langle\tilde Q(\mu)\rangle$ and the difference $\tilde
B-\tilde J$ is scheme independent by itself.  This discussion
demonstrates explicitly that the transition from full QCD to HQET can
be made at an arbitrary scale $\mu_b={\cal O}(m_b)$, as we have already
emphasized above.
\\
Finally we would like to remark that since the logarithm $\ln\mu_b/\mu$
is not really very large in the present case, one might take the
attitude of neglecting higher order resummations of logarithmic terms
altogether and restricting oneself to the ${\cal O}(\alpha_s)$
corrections alone. Then (\ref{qqtm}) would be already the final result,
as it was used in \cite{flynnetal:91}. This approximation is fully
consistent from a theoretical point of view. Yet it is useful to have
the more complete expression (\ref{qqtl}) at hand. Of course, as
indicated above, the finite ${\cal O}(\alpha_s)$ correction due to the
matrix element of $\tilde Q_s$ in (\ref{qqtm}) must still be added to
the r.h.s. of (\ref{qqtl}).  However, to complete the NLO
renormalization group calculation also the leading logarithmic
corrections related to the operator $\tilde Q_s$ should then be
resummed. To our knowledge this part of the analysis has not yet been
performed in the literature so far.
