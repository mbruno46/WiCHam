\section{$\epe$ Beyond Leading Logarithms}
        \label{sec:nloepe}
\subsection{Basic Formulae}
           \label{subsec:epeformulae}
The direct CP violation in $K \to \pi\pi$ is described by $\eps'$.
The parameter $\eps'$ is given in terms of the amplitudes $A_0 \equiv
A(K \to (\pi\pi)_{I=0})$ and $A_2 \equiv
A(K \to (\pi\pi)_{I=2})$ as follows
\begin{equation}
\eps' = -\frac{\omega}{\sqrt{2}} \xi (1 - \Omega) \exp(i \Phi) \, ,
\label{eq:epsprim}
\end{equation}
where
\begin{equation}
\xi = \frac{\IM A_0}{\RE A_0} \, , \quad
\omega = \frac{\RE A_2}{\RE A_0} \, , \quad
\Omega = \frac{1}{\omega} \frac{\IM A_2}{\IM A_0}
\label{eq:xiomega}
\end{equation}
and $\Phi = \pi/2 + \delta_2 - \delta_0 \approx \pi/4$.

When using \eqn{eq:epsprim} and \eqn{eq:xiomega} in phenomenological
applications one usually takes $\RE A_0$ and $\omega$ from
experiment, i.e.
\begin{equation}
\RE A_0 = 3.33 \cdot 10^{-7}\gev
\qquad
\RE A_2 = 1.50 \cdot 10^{-8}\gev
\qquad
\omega = 0.045
\label{eq:ReA0data}
\end{equation}
where the last relation reflects the so-called $\Delta I=1/2$ rule. The
main reason for this strategy is the unpleasant fact that until today
nobody succeded in fully explaining this rule which to a large extent is
believed to originate in the long-distance QCD contributions. We will be
more specific about this in the next section. On the other hand the
imaginary parts of the amplitudes in \eqn{eq:xiomega} being related to
CP violation and the top quark physics should be dominated by
short-distance contributions. Therefore $\IM A_0$ and $\IM A_2$ are
usually calculated using the effective hamiltonian given in
\eqn{eq:HeffdF1:1010}. Using this hamiltonian and the experimental
values for $\eps$, $\RE A_0$ and $\omega$ the ratio $\epe$ can be
written as follows
\begin{equation}
\epe = \IM \lambda_t \left[ P^{(1/2)} - P^{(3/2)} \right]
\label{eq:epe}
\end{equation}
where
\begin{eqnarray}
P^{(1/2)} &=& \sum P_i^{(1/2)} = r \sum y_i \langle Q_i\rangle_0
(1-\Omega_{\eta+\eta'})
\label{eq:P12} \\
P^{(3/2)} &=& \sum P_i^{(3/2)} = \frac{r}{\omega}
\sum y_i \langle Q_i\rangle_2
\label{eq:P32}
\end{eqnarray}
with
\begin{equation}
r = \frac{G_F \omega}{2 |\eps| \RE A_0} \, .
\label{eq:repe}
\end{equation}
Here the hadronic matrix element shorthand notation is
\begin{equation}
\langle Q_i\rangle_I \equiv \langle (\pi\pi)_I | Q_i | K \rangle
\label{eq:QiKpp}
\end{equation}
and the sum in \eqn{eq:P12} and \eqn{eq:P32} runs over all contributing
operators. This means for $\mu > \mc$ also contributions from operators
$Q^c_{1,2}$ to $P^{(1/2)}$ and $P^{(3/2)}$ have to be taken into
account. These are necessary for $P^{(1/2)}$ and $P^{(3/2)}$ to be
independent of the renormalization scale $\mu$. Next,
\begin{equation}
\Omega_{\eta+\eta'} = \frac{1}{\omega} \frac{(\IM A_2)_{\rm
I.B.}}{\IM A_0}
\label{eq:Omegaeta}
\end{equation}
represents the contribution stemming from isospin breaking in the quark masses
($m_u \not= m_d$). For $\Omega_{\eta+\eta'}$ we will take
\begin{equation}
\Omega_{\eta+\eta'} = 0.25 \pm 0.05
\label{eq:Omegaetadata}
\end{equation}
which is in the ball park of the values obtained in the $1/N_c$ approach
\cite{burasgerard:87} and in chiral perturbation theory
\cite{donoghueetal:86}, \cite{lusignoli:89}. $\Omega_{\eta+\eta'}$ is
independent of $\mt$.

The numerical values of the Wilson coefficients $y_i$ have been already given
in section \ref{sec:HeffdF1:1010:numres}. We therefore turn now our
attention to the hadronic matrix elements \eqn{eq:QiKpp} which
constitute the main source of uncertainty in the calculation of
$\epe$.

\subsection{Hadronic Matrix Elements for $K \to \pi\pi$}
           \label{subsec:matelKpp}
The hadronic matrix elements $\langle Q_i \rangle_I$ depend generally
on the renormalization scale $\mu$ and on the scheme used to
renormalize the operators $Q_i$. These two dependences are canceled by
those present in the Wilson coefficients $C_i(\mu)$ so that the
resulting physical amplitudes do not depend on $\mu$ and on the
renormalization scheme of the operators.  Unfortunately the accuracy of
the present non-perturbative methods used to evalutate $\langle Q_i
\rangle_I$, like lattice methods or $1/N_c$ expansion, is not
sufficient to obtain the required $\mu$ and scheme dependences of
$\langle Q_i \rangle_I$. A review of the existing methods and their
comparison can be found in \cite{burasetal:92d}, \cite{ciuchini:95}.
In view of this situation it has been suggested \cite{burasetal:92d} to
determine as many matrix elements $\langle Q_i \rangle_I$ as possible
from the leading CP conserving $K \to \pi\pi$ decays, for which the
experimental data are summarized in \eqn{eq:ReA0data}. To this end it
turned out to be very convenient to determine $\langle Q_i \rangle_I$
at a scale $\mu = \mc$.  Using the renormalization group evolution one
can then find $\langle Q_i \rangle_I$ at any other scale $\mu \not=
\mc$. The details of this procedure can be found in
\cite{burasetal:92d}. Here we simply summarize the results of this
work.

We first express the matrix elements
$\langle Q_i \rangle_I$ in terms of the non-perturbative parameters
$B_i^{(1/2)}$ and $B_i^{(3/2)}$ for $\langle Q_i \rangle_0$ and
$\langle Q_i \rangle_2$, respectively. For $\mu \le \mc$ we have
\cite{burasetal:92d}
\begin{eqnarray}
\langle Q_1 \rangle_0 &=& -\,\frac{1}{9} X B_1^{(1/2)} \, ,
\label{eq:Q10} \\
\langle Q_2 \rangle_0 &=&  \frac{5}{9} X B_2^{(1/2)} \, ,
\label{eq:Q20} \\
\langle Q_3 \rangle_0 &=&  \frac{1}{3} X B_3^{(1/2)} \, ,
\label{eq:Q30} \\
\langle Q_4 \rangle_0 &=&  \langle Q_3 \rangle_0 + \langle Q_2 \rangle_0
                          -\langle Q_1 \rangle_0 \, ,
\label{eq:Q40} \\
\langle Q_5 \rangle_0 &=&  \frac{1}{3} B_5^{(1/2)} 
                           \langle \overline{Q_6} \rangle_0 \, ,
\label{eq:Q50} \\
\langle Q_6 \rangle_0 &=&  -\,4 \sqrt{\frac{3}{2}} 
\left[ \frac{m_{\rm K}^2}{\ms(\mu) + \md(\mu)}\right]^2
\frac{F_\pi}{\kappa} \,B_6^{(1/2)} \, ,
\label{eq:Q60} \\
\langle Q_7 \rangle_0 &=& 
- \left[ \frac{1}{6} \langle \overline{Q_6} \rangle_0 (\kappa + 1) 
         - \frac{X}{2} \right] B_7^{(1/2)} \, ,
\label{eq:Q70} \\
\langle Q_8 \rangle_0 &=& 
- \left[ \frac{1}{2} \langle \overline{Q_6} \rangle_0 (\kappa + 1) 
         - \frac{X}{6} \right] B_8^{(1/2)} \, ,
\label{eq:Q80} \\
\langle Q_9 \rangle_0 &=& 
\frac{3}{2} \langle Q_1 \rangle_0 - \frac{1}{2} \langle Q_3 \rangle_0 \, ,
\label{eq:Q90} \\
\langle Q_{10} \rangle_0 &=& 
    \langle Q_2 \rangle_0 + \frac{1}{2} \langle Q_1 \rangle_0
  - \frac{1}{2} \langle Q_3 \rangle_0 \, ,
\label{eq:Q100}
\end{eqnarray}
\begin{eqnarray}
\langle Q_1 \rangle_2 &=& 
\langle Q_2 \rangle_2 = \frac{4 \sqrt{2}}{9} X B_1^{(3/2)} \, ,
\label{eq:Q122} \\
\langle Q_i \rangle_2 &=&  0 \, , \qquad i=3,\ldots,6 \, ,
\label{eq:Q362} \\
\langle Q_7 \rangle_2 &=& 
  -\left[ \frac{\kappa}{6 \sqrt{2}} \langle \overline{Q_6} \rangle_0
          + \frac{X}{\sqrt{2}}
   \right] B_7^{(3/2)} \, ,
\label{eq:Q72} \\
\langle Q_8 \rangle_2 &=& 
  -\left[ \frac{\kappa}{2 \sqrt{2}} \langle \overline{Q_6} \rangle_0
          + \frac{\sqrt{2}}{6} X
   \right] B_8^{(3/2)} \, ,
\label{eq:Q82} \\
\langle Q_9 \rangle_2 &=& 
   \langle Q_{10} \rangle_2 = \frac{3}{2} \langle Q_1 \rangle_2 \, ,
\label{eq:Q9102}
\end{eqnarray}
where
\begin{equation}
\kappa = \frac{\Lambda_\chi^2}{m_{\rm K}^2 - m_\pi^2} =
         \frac{F_\pi}{F_{\rm K} - F_\pi} \, ,
\label{eq:kappaQi}
\end{equation}
\begin{equation}
X = \sqrt{\frac{3}{2}} F_\pi \left( m_{\rm K}^2 - m_\pi^2 \right) \, ,
\label{eq:XQi}
\end{equation}
and
\begin{equation}
\langle \overline{Q_6} \rangle_0 =
   \frac{\langle Q_6 \rangle_0}{B_6^{(1/2)}} \, .
\label{eq:Q60bar}
\end{equation}
The actual numerical values used for $m_{\rm K}$, $m_\pi$, $F_{\rm K}$,
$F_\pi$ are collected in appendix \ref{app:numinput}.

In the vacuum insertion method $B_i=1$ independent of $\mu$. In QCD,
however, the hadronic parameters $B_i$ generally depend on the
renormalizations scale $\mu$ and the renormalization scheme considered.

\subsection{$\langle Q_i(\mu) \rangle_2$ for $(V-A)\otimes (V-A)$ Operators}
           \label{subsec:Qi2VmAVmA}
The matrix elements $\langle Q_1 \rangle_2$, $\langle Q_2 \rangle_2$,
$\langle Q_9 \rangle_2$ and $\langle Q_{10} \rangle_2$ can to a very
good approximation be determined from $\RE A_2$ in
\eqn{eq:ReA0data} as functions of $\Lms$, $\mu$ and the renormalization
scheme considered. To this end it is useful to set $\aem=0$, as the
$\ord(\aem)$ effects in CP conserving amplitudes, such as the
contributions of electroweak penguins, are very small. One then finds
\begin{equation}
\langle Q_1(\mu) \rangle_2 = \langle Q_2(\mu) \rangle_2 =
\frac{10^6\gev^2}{1.77} \frac{\RE A_2}{z_+(\mu)} =
\frac{8.47 \cdot 10^{-3}\gev^3}{z_+(\mu)}
\label{eq:Q122data}
\end{equation}
and comparing with \eqn{eq:Q122}
\begin{equation}
B_1^{(3/2)}(\mu) = \frac{0.363}{z_+(\mu)}
\label{eq:B321}
\end{equation}
with $z_+ = z_1 + z_2$.
Since $z_+(\mu)$ depends on the scale $\mu$ and the renormalization
scheme used, \eqn{eq:B321} gives automatically the scheme and $\mu$
dependence of $B_1^{(3/2)}$ and of the related matrix elements $\langle
Q_1 \rangle_2$, $\langle Q_2 \rangle_2$,
$\langle Q_9 \rangle_2$ and $\langle Q_{10} \rangle_2$. The impact of
$\ord(\aem)$ corrections on this result has been analysed in
\cite{burasetal:92d}. It amounts only to a few percent as expected.
These corrections are of course included in the numerical analysis
presented in this reference and here as well. Using $\mu=\mc=1.3\gev$,
$\Lms^{(4)}=325\mev$ and $z_+(\mc)$ of table \ref{tab:wc10smu13}
we find according to \eqn{eq:B321}
\begin{equation}
B_{1,NDR}^{(3/2)}(\mc) =  0.453
\qquad
B_{1,HV}^{(3/2)}(\mc) =  0.472 \, .
\label{eq:B321mc}
\end{equation}

\noindent
The following comments should be made:
\begin{itemize}
\item
$B_1^{(3/2)}(\mu)$ decreases with increasing $\mu$.
\item
The extracted value for $B_1^{(3/2)}$ is by more than a factor of two
smaller than the vacuum insertion estimate.
\item
It is compatible with the $1/N_c$ value $B_1^{(3/2)}(1\gev) \approx
0.55$ \cite{bardeen:87b} and somewhat smaller than the lattice result
$B_1^{(3/2)}(2\gev) \approx 0.6$ \cite{ciuchini:95}.
\end{itemize}

\subsection{$\langle Q_i(\mu) \rangle_0$ for $(V-A)\otimes (V-A)$ Operators}
           \label{subsec:Qi0VmAVmA}
The determination of $\langle Q_i(\mu) \rangle_0$ matrix elements is
more involved because several operators may contribute to $\RE
A_0$. The main idea of \cite{burasetal:92d} is then to set $\mu=\mc$, as
at this scale only $Q_1$ and $Q_2$ operators contribute to $\RE
A_0$ in the HV scheme. One then finds $\langle Q_1(\mc) \rangle_0$ as a
function of  $\langle Q_2(\mc) \rangle_0$
\begin{equation}
\langle Q_1(\mc) \rangle_0 = \frac{10^6\gev^2}{1.77} \frac{\RE
A_0}{z_1(\mc)} - \frac{z_2(\mc)}{z_1(\mc)} \langle Q_2(\mc) \rangle_0
\label{eq:Q10mc}
\end{equation}
where the reference in $\langle Q_{1,2}(\mc) \rangle_0$ to the HV scheme
has been suppressed for convenience. Using next the relations
\eqn{eq:Q40}, \eqn{eq:Q90} and \eqn{eq:Q100} one is able to obtain
$\langle Q_4(\mc) \rangle_0$, $\langle Q_9(\mc) \rangle_0$ and $\langle
Q_{10}(\mc) \rangle_0$ as functions of $\langle Q_2(\mc) \rangle_0$ and
$\langle Q_3(\mc) \rangle_0$. Because $\langle Q_3(\mc) \rangle_0$ is
colour suppressed it is less essential for this analysis than $\langle
Q_2(\mc) \rangle_0$. Moreover its Wilson coefficient is small and
similarly to $\langle Q_9(\mc) \rangle_0$ and $\langle 
Q_{10}(\mc) \rangle_0$ also $\langle Q_3(\mc) \rangle_0$ has only a
small impact on $\epe$. On the other hand the coefficient $y_4$ is
substantial and consequently $\langle Q_4(\mc) \rangle_0$ plays a
considerable role in the analysis of $\epe$. The matrix element $\langle
Q_3(\mc) \rangle_0$ has then an indirect impact on $\epe$ through
relation \eqn{eq:Q40}. For numerical evaluation, $\langle Q_3(\mc)
\rangle_0$ of \eqn{eq:Q30} with $B_3^{(1/2)} = 1$ can be used keeping in
mind that this may introduce a small uncertainty in the final analysis.
This uncertainty has been investigated in \cite{burasetal:92d}.

Once the matrix elements in question have been determined as functions
of $\langle Q_2(\mc) \rangle_0$ in the HV scheme, they can be found by a
finite renormalization in any other scheme. Details can be found in
\cite{burasetal:92d}.

If one in addition makes the very plausible assumption valid in all
known non-perturbative approaches that $\langle Q_-(\mc) \rangle_0 \ge
\langle Q_+(\mc) \rangle_0 \ge 0$ the experimental value of $\RE
A_0$ in \eqn{eq:ReA0data} together with \eqn{eq:Q10mc} and table
\ref{tab:wc10smu13} implies for $\Lms^{(4)}=325\mev$
\begin{equation}
B_{2,LO}^{(1/2)}(\mc)  =  5.7 \pm 1.1
\qquad
B_{2,NDR}^{(1/2)}(\mc) =  6.6 \pm 1.0
\qquad
B_{2,HV}^{(1/2)}(\mc) =  6.2 \pm 1.0 \, .
\label{eq:B122mc}
\end{equation}
The extraction of $B_1^{(1/2)}(\mc)$ and of an analogous parameter
$B_4^{(1/2)}(\mc)$ are presented in detail in \cite{burasetal:92d}.
$B_1^{(1/2)}(\mc)$ depends very sensitively on $B_2^{(1/2)}(\mc)$ and
its central value is as high as 15. $B_4^{(1/2)}(\mc)$ is less sensitive
and typically by (10--15)\,\% lower than $B_2^{(1/2)}(\mc)$. In any case
this analysis shows very large departures from the results of the
vacuum insertion method.

\subsection{$\langle Q_i(\mu) \rangle_{0,2}$ for $(V-A)\otimes (V+A)$ Operators}
           \label{subsec:Qi0VmAVpA}
The matrix elements of the $(V-A) \otimes (V+A)$ operators $Q_5$--$Q_8$
cannot be constrained by CP conserving data and one has to rely on
existing non-perturbative methods to calculate them. Fortunately, there
are some indications that the existing non-perturbative estimates of
$\langle Q_i(\mu) \rangle_{0,2}$, $i=5,\ldots,8$ are more reliable than
the corresponding calculations for $(V-A) \otimes (V-A)$ operators.

First of all, the parameters $B_{5,6}^{(1/2)}$ \cite{kilcup:91},
\cite{sharpe:91} and $B_{7,8}^{(3/2)}$ \cite{francoetal:89},
\cite{kilcup:91}, \cite{sharpe:91}, \cite{bernardsoni:91} calculated in
the lattice approach
\begin{equation}
B_{5,6}^{(1/2)} = 1.0 \pm 0.2 
\qquad
B_{7,8}^{(3/2)} = 1.0 \pm 0.2 
\label{eq:B1258}
\end{equation}
agree well with the vacuum insertion values ($B_i=1$) and in the case
of $B_6^{(1/2)}$ and $B_8^{(3/2)}$ with the $1/N_c$ approach ($B_6^{(1/2)}
= B_8^{(3/2)} = 1$) \cite{bardeen:87a}, \cite{burasgerard:87}.

We note next that with fixed values for $B_{5,6}^{(1/2)}$ and
$B_{7,8}^{(3/2)}$ the $\mu$-dependence of $\langle Q_{5,6} \rangle_0$
and $\langle Q_{7,8} \rangle_2$ is governed by the $\mu$ dependence of
$\ms(\mu)$. For $\langle Q_6 \rangle_0$ and $\langle Q_8 \rangle_2$
this property has been first found in the $1/N_c$ approach
\cite{burasgerard:87}: in the large-$N_c$ limit the anomalous
dimensions of $Q_6$ and $Q_8$ are simply twice the anomalous dimension
of the mass operator leading to $\sim 1/\ms^2(\mu)$ for the
corresponding matrix elements. Another support comes from a
renormalization study in \cite{burasetal:92d}. In this analysis the
$B_i$-factors in \eqn{eq:B1258} have been set to unity at $\mu=\mc$.
Subsequently the evolution of the matrix elements in the range $1\gev
\le \mu \le 4\gev$ has been calculated showing that for the NDR scheme
$B_{5,6}^{(1/2)}$ and $B_{7,8}^{(3/2)}$ were $\mu$ independent within
an accuracy of (2--3)\,\%. The $\mu$ dependence in the HV scheme has
been found to be stronger but still below 10\,\%.

Concerning $B_{7,8}^{(1/2)}$ one can simply set $B_{7,8}^{(1/2)}=1$ as
the matrix elementes $\langle Q_{7,8} \rangle_0$ play only a minor role
in the $\epe$ analysis.

In summary, our treatment of $\langle Q_i \rangle_{0,2}$, $i=5,\ldots 8$
follows the one used in \cite{burasetal:92d}. We will set
\begin{equation}
B_{7,8}^{(1/2)}(\mc) = 1
\qquad
B_5^{(1/2)}(\mc) = B_6^{(1/2)}(\mc)
\qquad
B_7^{(3/2)}(\mc) = B_8^{(3/2)}(\mc)
\label{eq:B1278mc}
\end{equation}
and we will treat $B_6^{(1/2)}(\mc)$ and $B_8^{(3/2)}(\mc)$ as free
parameters in the neighbourhood of the values given in \eqn{eq:B1258}.
Then the main uncertainty in the values of $\langle Q_i \rangle_{0,2}$,
$i=5,\ldots 8$ results from the value of the strange quark mass
$\ms(\mc)$. The present estimates give
\begin{equation}
\ms(\mc) = (170 \pm 20)\mev
\label{eq:msmc}
\end{equation}
with the lower values coming from recent lattice calculations
\cite{alltonetal:94} and the higher ones from QCD sum rules
\cite{jaminmuenz:95}, \cite{chetyrkinetal:95}.

\subsection{The Four Dominant Contributions to $\epe$}
           \label{subsec:epe4dom}
$P^{(1/2)}$ and $P^{(3/2)}$ in \eqn{eq:epe} can be written as linear
combinations of two independent hadronic parameters $B_6^{(1/2)}$ and
$B_8^{(3/2)}$ \cite{burasetal:92d}. This $B_i$-expansion reads
\begin{eqnarray}
P^{(1/2)} &=& a_0^{(1/2)} +
  \left[ \frac{178\mev}{\ms(\mc)+\md(\mc)} \right]^2 a_6^{(1/2)} B_6^{(1/2)}
\label{eq:P12Bi} \\
P^{(3/2)} &=& a_0^{(3/2)} +
  \left[ \frac{178\mev}{\ms(\mc)+\md(\mc)} \right]^2 a_8^{(3/2)} B_8^{(3/2)}
  \, .
\label{eq:P32Bi}
\end{eqnarray}
Here $a_0^{(1/2)}$ and $a_0^{(3/2)}$ effectively summarize all
dependences other than $B_6^{(1/2)}$ and $B_8^{(3/2)}$, especially
$B_2^{(1/2)}$ in the case of $a_0^{(1/2)}$.
Note that in contrast to \cite{burasetal:92d} we have absorbed the
dependence on $B_2^{(1/2)}$ into $a_0^{(1/2)}$ and we have exhibited the
dependence on $\ms$ which was not shown explicitly there.
The residual $\ms$ dependence present in $a_0^{(1/2)}$ and
$a_0^{(3/2)}$ is negligible.  Setting $\mu=\mc$, and using the strategy
for hadronic matrix elements outlined above one finds the coefficients
$a_i^{(1/2)}$ and $a_i^{(3/2)}$ as functions of $\Lms$, $\mt$ and the
renormalization scheme considered. These dependences are given in
tables \ref{tab:bip12} and \ref{tab:bip32}. We should however stress
that $P^{(1/2)}$ and $P^{(3/2)}$ are independent of $\mu$ and the
renormalization scheme considered.

\begin{table}[htb]
\caption[]{$B_i$-expansion coefficients for $P^{(1/2)}$.
\label{tab:bip12}}
\begin{center}
\begin{tabular}{|c|c||c|c||c|c||c|c|}
& & \multicolumn{2}{c||}{LO} &
  \multicolumn{2}{c||}{NDR} &
  \multicolumn{2}{c|}{HV} \\
\hline
$\Lms^{(4)}\,[\mev]$ & $\mt\,[\gev]$ &
$a_0^{(1/2)}$ & $a_6^{(1/2)}$ &
$a_0^{(1/2)}$ & $a_6^{(1/2)}$ &
$a_0^{(1/2)}$ & $a_6^{(1/2)}$ \\
\hline
    & 155  &  --2.138 & 5.110  &  --2.251 & 4.676  &  --2.215 & 4.159 \\
215 & 170  &  --2.070 & 5.138  &  --2.187 & 4.698  &  --2.150 & 4.181 \\
    & 185  &  --1.996 & 5.162  &  --2.117 & 4.716  &  --2.081 & 4.200 \\
\hline 
    & 155  &  --2.231 & 6.540  &  --2.414 & 6.255  &  --2.362 & 5.389 \\
325 & 170  &  --2.161 & 6.576  &  --2.350 & 6.282  &  --2.298 & 5.416 \\
    & 185  &  --2.085 & 6.606  &  --2.281 & 6.306  &  --2.229 & 5.439 \\
\hline 
    & 155  &  --2.288 & 8.171  &  --2.549 & 8.417  &  --2.473 & 6.972 \\
435 & 170  &  --2.212 & 8.214  &  --2.482 & 8.451  &  --2.406 & 7.005 \\
    & 185  &  --2.130 & 8.251  &  --2.409 & 8.480  &  --2.333 & 7.035
\end{tabular}
\end{center}
\end{table}

\begin{table}[htb]
\caption[]{$B_i$-expansion coefficients for $P^{(3/2)}$.
\label{tab:bip32}}
\begin{center}
\begin{tabular}{|c|c||c|c||c|c||c|c|}
& & \multicolumn{2}{c||}{LO} &
  \multicolumn{2}{c||}{NDR} &
  \multicolumn{2}{c|}{HV} \\
\hline
$\Lms^{(4)}\,[\mev]$ & $\mt\,[\gev]$ &
$a_0^{(3/2)}$ & $a_8^{(3/2)}$ & $a_0^{(3/2)}$ &
$a_8^{(3/2)}$ & $a_0^{(3/2)}$ & $a_8^{(3/2)}$ \\
\hline
    & 155  &  --0.797 & 1.961  &  --0.819 & 1.887  &  --0.838 & 2.114 \\
215 & 170  &  --0.880 & 2.602  &  --0.900 & 2.438  &  --0.919 & 2.666 \\
    & 185  &  --0.965 & 3.296  &  --0.983 & 3.036  &  --1.002 & 3.263 \\
\hline 
    & 155  &  --0.788 & 2.645  &  --0.814 & 2.639  &  --0.837 & 2.894 \\
325 & 170  &  --0.870 & 3.422  &  --0.895 & 3.305  &  --0.917 & 3.560 \\
    & 185  &  --0.956 & 4.264  &  --0.978 & 4.027  &  --1.000 & 4.281 \\
\hline 
    & 155  &  --0.779 & 3.425  &  --0.809 & 3.622  &  --0.835 & 3.899 \\
435 & 170  &  --0.861 & 4.360  &  --0.889 & 4.435  &  --0.915 & 4.712 \\
    & 185  &  --0.947 & 5.372  &  --0.971 & 5.316  &  --0.998 & 5.593
\end{tabular}
\end{center}
\end{table}

Inspecting \eqn{eq:P12Bi}, \eqn{eq:P32Bi} and tables \ref{tab:bip12},
\ref{tab:bip32} we identify the following four contributions which
govern the ratio $\epe$ at scales $\mu=\ord(\mc)$:
\renewcommand{\theenumi}{\roman{enumi}}
\begin{enumerate}
\item
\label{enum:i}
The contribution of $(V-A) \otimes (V-A)$ operators to $P^{(1/2)}$ is
dominantly represented by $a_0^{(1/2)}$. This term is to a large extent
fixed by the experimental value of $A_0$ and consequently is only very
weakly dependent on $\Lms$ and the renormalization scheme considered.
The weak dependence on $\mt$ results from small contributions of
electroweak penguin operators.  Taking $\Lms^{(4)} = 325\mev$,
$\mu=\mc$ and $\mt = 170\gev$ we have $a_0^{(1/2)} \approx -2.3$ for
both schemes considered.  We observe that the contribution of $(V-A)
\otimes (V-A)$ operators, in particular $Q_4$, to $\epe$ is {\em
negative}.
\item
\label{enum:ii}
The contribution of $(V-A) \otimes (V+A)$ QCD penguin operators to
$P^{(1/2)}$ is given by the second term in \eqn{eq:P12Bi}. This
contribution is large and {\em positive}. The coefficient $a_6^{(1/2)}$
depends sensitively on $\Lms$ which results from the strong dependence
of $y_6$ on the QCD scale. The dependence on $\mt$ is very weak on the
other hand. Taking $\Lms^{(4)} = 325\mev$, $\ms(\mc)=170\mev$ and $\mt
= 170\gev$ and setting as an example $B_6^{(1/2)} = 1$ in the NDR and
HV schemes we find a positve contribution to $\epe$ amounting to 6.3
and 5.4 in the NDR and HV scheme, respectively.

\item
\label{enum:iii}
The contribution of the $(V-A) \otimes (V-A)$ electroweak penguin
operators $Q_9$ and $Q_{10}$ to $P^{(3/2)}$ is represented by
$a_0^{(3/2)}$. As in the case of the contribution \ref{enum:i}, the
matrix elements contributing to $a_0^{(3/2)}$ are fixed by the CP
conserving data, this time by the amplitude $A_2$. Consequently, the
scheme and the $\Lms$ dependence of $a_0^{(3/2)}$ is very weak. The
sizeable $\mt$ dependence of $a_0^{(3/2)}$ results from the $\mt$
dependence of $y_9 + y_{10}$. $a_0^{(3/2)}$ contributes {\em positively}
to $\epe$. For $\mt = 170\gev$ this contribution is roughly 0.9 for
both renormalization schemes and the full range of $\Lms$ considered.
\item
\label{enum:iv}
The contribution of the $(V-A) \otimes (V+A)$ electroweak penguin
operators $Q_7$ and $Q_8$ to $P^{(3/2)}$ is represented by the second
term in \eqn{eq:P32Bi}. This contribution depends sensitively on $\mt$
and $\Lms$ as could be expected on the basis of $y_7$ and $y_8$. Taking
again $B_8^{(3/2)}=1$ in both renormalization schemes we find for the
central values of $\Lms^{(4)}$, $\mt$ and $\mc$ a {\em
negative} contribution to $\epe$ equal to $-3.9$ and $-3.6$ for the NDR
and HV scheme, respecetively.
\end{enumerate}
\renewcommand{\theenumi}{\arabic{enumi}}

Before analysing $\epe$ numerically in more detail, let us just set
$\IM \lambda_t = 1.3 \cdot 10^{-4}$ and $B_6^{(1/2)} = B_8^{(3/2)} = 1$
in both schemes. Then for the central values of the remaining
parameters one obtains $\epe = 2.0 \cdot 10^{-4}$ and $\epe = 0.6 \cdot
10^{-4}$ for the NDR and HV scheme, respectively. This strong scheme
dependence can only be compensated for by having $B_6^{(1/2)}$ and
$B_8^{(3/2)}$ different in the two schemes considered. As we will see
below the strong cancellations between various contributions at $\mt
\approx 170\gev$ make the prediction for $\epe$ rather uncertain. One
should also stress that the formulation presented here does not exhibit
analytically the $\mt$ dependence. As the coefficients $a_0^{(3/2)}$
and $a_8^{(3/2)}$ depend very sensitively on $\mt$ it is useful to
display this dependence in an analytic form.

\subsection{An Analytic Formula for $\epe$}
           \label{subsec:epeanalytic}
As shown in \cite{buraslauten:93} it is possible to cast the above
discussion into an analytic formula which exhibits the $\mt$ dependence
together with the dependence on $\ms$, $B_6^{(1/2)}$ and $B_8^{(3/2)}$.
Such an analytic formula should be useful for those phenomenologists
and experimentalists who are not interested in getting involved with
the technicalities discussed in preceding sections.

In order to find an analytic expression for $\epe$ which exactly
reproduces the results discussed above one uses the PBE presented in
section \ref{sec:PBE}. The resulting analytic expression for $\epe$ is
then given as follows
\begin{equation}
\epe =
\IM \lambda_t F(x_t)
\label{eq:epePBE}
\end{equation}
where
\begin{equation}
F(x_t) = P_0 + P_X X_0(x_t) + P_Y Y_0(x_t) + P_Z Z_0(x_t) + P_E E_0(x_t)
\label{eq:Fxt}
\end{equation}
with the $\mt$ dependent functions listed in section \ref{sec:PBE}. The
coefficients $P_i$ are given in terms of $B_6^{(1/2)} \equiv
B_6^{(1/2)}(\mc)$, $B_8^{(3/2)} \equiv B_8^{(3/2)}(\mc)$ and $\ms(\mc)$
as follows
\begin{equation}
P_i = r_i^{(0)} + \left[ \frac{178\mev}{\ms(\mc)+\md(\mc)} \right]^2
\left(r_i^{(6)} B_6^{(1/2)} + r_i^{(8)} B_8^{(3/2)} \right) \, .
\label{eq:pbePi}
\end{equation}
The $P_i$ are $\mu$ and renormalization scheme independent. They depend
however on $\Lms$. In table \ref{tab:pbendr} we give the numerical
values of $r_i^{(0)}$, $r_i^{(6)}$ and $r_i^{(8)}$ for different values
of $\Lms$ at $\mu=\mc$ in the NDR renormalization scheme. Analogous
results in the HV scheme are given in table \ref{tab:pbehv}. The
coefficients $r_i^{(0)}$, $r_i^{(6)}$ and $r_i^{(8)}$do not depend on
$\ms(\mc)$ as this dependence has been factored out. $r_i^{(0)}$ does,
however, depend on the particular choice for the parameter
$B_2^{(1/2)}$ in the parametrization of the matrix element $\langle Q_2
\rangle_0$. The values given in the tables correspond to the central
values in \eqn{eq:B122mc}. Variation of $B_2^{(1/2)}$ in the full
allowed range introduces an uncertainty of at most 18\,\% in the
$r_i^{(0)}$ column of the tables.  Since the parameters $r_i^{(0)}$
give only subdominant contributions to $\epe$ keeping $B_2^{(1/2)}$ and
$r_i^{(0)}$ at their central values is a very good approximation.

For different scales $\mu$ the numerical values in the tables change
without modifying the values of the $P_i$'s as it should be. To this
end also $B_6^{(1/2)}$ and $B_8^{(3/2)}$ have to be modified as they
depend albeit weakly on $\mu$.

Concerning the scheme dependence we note that whereas $r_0$ coefficients
are scheme dependent, the coefficients $r_i$, $i=X, Y, Z, E$ do not show
any scheme dependence. This is related to the fact that the $\mt$
dependence in $\epe$ enters first at the NLO level and consequently all
coefficients $r_i$ in front of the $\mt$ dependent functions must be
scheme independent. That this turns out to be indeed the case is a nice
check of our calculations.

Consequently when changing the renormalization scheme one is only
obliged to change appropriately $B_6^{(1/2)}$ and $B_8^{(3/2)}$ in the
formula for $P_0$ in order to obtain a scheme independence of $\epe$.
In calculating $P_i$ where $i \not= 0$, $B_6^{(1/2)}$ and $B_8^{(3/2)}$
can in fact remain unchanged, because their variation in this part
corresponds to higher order contributions to $\epe$ which would have to
be taken into account in the next order of perturbation theory.

For similar reasons the NLO analysis of $\epe$ is still insensitive to
the precise definition of $\mt$. In view of the fact that the NLO
calculations of $\IM \lambda_t$ have been done with
$\mt=\overline{m}_t(\mt)$ we will also use  this definition in
calculating $F(x_t)$.

\begin{table}[htb]
\caption[]{$\Delta S=1$ PBE coefficients for various $\Lms$ in the NDR scheme.
\label{tab:pbendr}}
\begin{center}
\begin{tabular}{|c||c|c|c||c|c|c||c|c|c|}
& \multicolumn{3}{c||}{$\Lms^{(4)}=215\mev$} &
  \multicolumn{3}{c||}{$\Lms^{(4)}=325\mev$} &
  \multicolumn{3}{c| }{$\Lms^{(4)}=435\mev$} \\
\hline
$i$ & $r_i^{(0)}$ & $r_i^{(6)}$ & $r_i^{(8)}$ &
      $r_i^{(0)}$ & $r_i^{(6)}$ & $r_i^{(8)}$ &
      $r_i^{(0)}$ & $r_i^{(6)}$ & $r_i^{(8)}$ \\
\hline
 0  & --2.644 & 4.784 & 0.876 & --2.749 & 6.376 & 0.689 & --2.845 & 8.547
    & 0.436 \\
$X$ & 0.555 & 0.008 & 0 & 0.521 & 0.012 & 0 & 0.495 & 0.017 & 0 \\
$Y$ & 0.422 & 0.037 & 0 & 0.385 & 0.046 & 0 & 0.356 & 0.057 & 0 \\
$Z$ & 0.074 & --0.007 & --4.798 & 0.149 & --0.009 & --5.789 & 0.237 & --0.011
    & --7.064 \\
$E$ & 0.209 & --0.591 & 0.205 & 0.181 & --0.727 & 0.265 & 0.152 & --0.892
    & 0.342
\end{tabular}
\end{center}
\end{table}

\begin{table}[htb]
\caption[]{$\Delta S=1$ PBE coefficients for various $\Lms$ in the HV scheme.
\label{tab:pbehv}}
\begin{center}
\begin{tabular}{|c||c|c|c||c|c|c||c|c|c|}
& \multicolumn{3}{c||}{$\Lms^{(4)}=215\mev$} &
  \multicolumn{3}{c||}{$\Lms^{(4)}=325\mev$} &
  \multicolumn{3}{c| }{$\Lms^{(4)}=435\mev$} \\
\hline
$i$ & $r_i^{(0)}$ & $r_i^{(6)}$ & $r_i^{(8)}$ &
      $r_i^{(0)}$ & $r_i^{(6)}$ & $r_i^{(8)}$ &
      $r_i^{(0)}$ & $r_i^{(6)}$ & $r_i^{(8)}$ \\
\hline
 0  & --2.631 & 4.291 & 0.668 & --2.735 & 5.548 & 0.457 & --2.830 & 7.163
    & 0.185 \\
$X$ & 0.555 & 0.008 & 0 & 0.521 & 0.012 & 0 & 0.495 & 0.017 & 0 \\
$Y$ & 0.422 & 0.037 & 0 & 0.385 & 0.046 & 0 & 0.356 & 0.057 & 0 \\
$Z$ & 0.074 & --0.007 & --4.798 & 0.149 & --0.009 & --5.789 & 0.237 & --0.011
    & --7.064 \\
$E$ & 0.209 & --0.591 & 0.205 & 0.181 & --0.727 & 0.265 & 0.152 & --0.892
    & 0.342
\end{tabular}
\end{center}
\end{table}

The inspection of tables \ref{tab:pbendr} and \ref{tab:pbehv} shows
that the terms involving $r_0^{(6)}$ and $r_Z^{(8)}$ dominate the ratio
$\epe$. The function $Z_0(x_t)$ representing a gauge invariant
combination of $Z^0$- and $\gamma$-penguins grows rapidly with $\mt$
and due to $r_Z^{(8)} < 0$ these contributions suppress $\epe$ strongly
for large $\mt$ \cite{flynn:89}, \cite{buchallaetal:90}. These two
dominant terms $r_0^{(6)}$ and $r_Z^{(8)}$ correspond essentially to
the second terms in \eqn{eq:P12Bi} and \eqn{eq:P32Bi}, respectively.
The first term in \eqn{eq:P12Bi} corresponds roughly to $r_0^{(0)}$
given here, while the first term in \eqn{eq:P32Bi} is represented to a
large extent by the positve contributions of $X_0(x_t)$ and $Y_0(x_t)$.
The last term in \eqn{eq:Fxt} representing the residual $\mt$
dependence of QCD penguins plays only a minor role in the full analysis
of $\epe$.

\subsection{Numerical Results}
           \label{subsec:epenumres}
Let us define two effective B-factors:
\begin{equation}\label{7e}
(B_i^{(j)}(m_c))_{eff}= 
\left [\frac{178\mev}{\bar m_s(m_c)+\bar m_d(m_c)} \right ]^2 B_i^{(j)}(m_c)
\end{equation}
In fig.\ \ref{fig:lepemt170} we show $\epe$ for $\mt=170\gev$ as a function
of $\Lms$ for different choices of the effective $B_i$ factors.
We show here only the results in the NDR scheme. As discussed above
$\epe$ is generally lower in the HV scheme, if the same values for
$B_6^{(1/2)}$ and $B_8^{(3/2)}$ are used in both schemes. In view of the
fact that the differences between NDR and HV schemes are smaller than
the uncertainties in $B_6^{(1/2)}$ and $B_8^{(3/2)}$ we think it is
sufficient to present only the results in the NDR scheme here. The
results in the HV scheme can be found in \cite{burasetal:92d},
\cite{ciuchini:95}.

\begin{figure}[htb]
\vspace{0.15in}
\centerline{
\epsfysize=6in
\rotate[r]{
\epsffile{ps/lemt170.ps}
}}
\vspace{0.15in}
\caption[]{
The ranges of $\epe$ in the NDR scheme as a function of $\Lms^{(4)}$
for $\mt=170\gev$ and present (light grey) and future (dark grey)
parameter ranges given in appendix \ref{app:numinput}. The three pairs
of $\epe$ plots correspond to hadronic parameter sets
(a) $(B_6^{(1/2)}(\mc))_{\rm eff}=1.5$, $(B_8^{(3/2)}(\mc))_{\rm eff}=1.0$,
(b) $(B_6^{(1/2)}(\mc))_{\rm eff}=1.0$, $(B_8^{(3/2)}(\mc))_{\rm eff}=1.0$,
and
(c) $(B_6^{(1/2)}(\mc))_{\rm eff}=1.0$, $(B_8^{(3/2)}(\mc))_{\rm eff}=1.5$,
respectively.
\label{fig:lepemt170}}
\end{figure}

Fig.\ \ref{fig:lepemt170} shows strong dependence of $\epe$ on $\Lms$.
However the main uncertainty originates in the poor knowledge of
$(B_i)_{eff}$.  In case a) in which the QCD-penguin contributions
dominate, $\epe$ can reach values as high as $1 \cdot 10^{-3}$.
However, in case c) the electroweak penguin contributions are large
enough to cancel essentially the QCD-penguin contributions completely.
Consequently in this case $|\epe|< 2 \cdot 10^{-5}$ and the standard
model prediction of $\epe$ cannot be distinguished from a superweak
theory. As shown in fig.\ \ref{fig:lepemt155} higher values of $\epe$
can be obtained for $\mt = 155\gev$ although still $\epe < 13 \cdot
10^{-4}$.

\begin{figure}[htb]
\vspace{0.15in}
\centerline{
\epsfysize=6in
\rotate[r]{
\epsffile{ps/lemt155.ps}
}}
\vspace{0.15in}
\caption[]{
Same as fig.\ \ref{fig:lepemt170} but for $\mt=155\gev$.
\label{fig:lepemt155}}
\end{figure}

\begin{figure}[htb]
\vspace{0.15in}
\centerline{
\epsfysize=6in
\rotate[r]{
\epsffile{ps/lemt185.ps}
}}
\vspace{0.15in}
\caption[]{
Same as fig.\ \ref{fig:lepemt170} but for $\mt=185\gev$.
\label{fig:lepemt185}}
\end{figure}

For $\mt = 185\gev$ the values of $\epe$ are correspondingly smaller
and in case c) small negative values are found for $\epe$.  In figs.\
\ref{fig:lepemt170}--\ref{fig:lepemt185} the dark grey regions refer to
the future ranges for $\IM\lambda_t$. Of course one should hope that
also the knowledge of $(B_i)_{eff}$ and of $\Lms^{(4)}$ will be
improved in the future so that a firmer prediction for $\epe$ can be
obtained.

\begin{figure}[htb]
\vspace{0.15in}
\centerline{
\epsfysize=6in
\rotate[r]{
\epsffile{ps/mteper1r2f.ps}
}}
\vspace{0.15in}
\caption[]{
The ranges of $\epe$ in the NDR scheme as a function of $\mt$
for $\Lms^{(4)}=325\mev$ and present (light grey) and future (dark grey)
parameter ranges given in appendix \ref{app:numinput}. The three bands
correspond to hadronic parameter sets
(a) $(B_6^{(1/2)}(\mc))_{\rm eff}=1.5$, $(B_8^{(3/2)}(\mc))_{\rm eff}=1.0$,
(b) $(B_6^{(1/2)}(\mc))_{\rm eff}=1.0$, $(B_8^{(3/2)}(\mc))_{\rm eff}=1.0$,
and
(c) $(B_6^{(1/2)}(\mc))_{\rm eff}=1.0$, $(B_8^{(3/2)}(\mc))_{\rm eff}=1.5$,
respectively.
\label{fig:mteper1r2f}}
\end{figure}

Finally, fig.\ \ref{fig:mteper1r2f} shows the interrelated influence of
$\mt$ and the two most important hadronic matrix elements for penguin
operators on the theoretical prediction of $\epe$. For a dominant QCD
penguin matrix element $<Q_6>_0$ $\epe$ stays positive for all $\mt$
values considered. $\epe \approx 0$ becomes possible for equally
weighted matrix elements $<Q_6>_0$ and $<Q_8>_2$ around $\mt=205\gev$.
A dominant electroweak pengiun matrix element $<Q_8>_2$ shifts the
point $\epe \approx 0$ to $\mt \approx 165\gev$ and even allows for a
negative $\epe$ for higher values of $\mt$. The key issue to understand
this behaviour of $\epe$ is the observation that the $Q_6$ contribution
to $\epe$ is positive and only weakly $\mt$ dependent. On the other
hand the contribution coming from $Q_8$ is negative and shows a strong
$\mt$ dependence.

The results in fig.\ \ref{fig:lepemt170}--\ref{fig:mteper1r2f}
use only the $\eps_K$ constraint. In order to complete our analysis we
want to impose also the $x_d$-constraint and vary $m_s(m_c)$,
$B^{(1/2)}_6$ and $B^{(3/2)}_8$ in the full ranges given in
\eqn{eq:B1258} and \eqn{eq:msmc}.

This gives for the ``present'' scenario
\begin{equation}
-2.1 \cdot 10^{-4} \le \epe \le 13.2 \cdot 10^{-4}
\label{eq:eperangepresent}
\end{equation}
to be compared with
\begin{equation}
-1.1 \cdot 10^{-4} \le \epe \le 10.4 \cdot 10^{-4}
\label{eq:eperangefuture}
\end{equation}
in the case of the ``future'' scenario. In both cases the
$x_d$-constraint has essentially no impact on the predicted range for
$\epe$.
\\
Finally, extending the ``future'' scenario to $\ms(\mc)=(170 \pm
10)\mev$, $\Lms^{(4)}=(325 \pm 50)\mev$ and $B^{(1/2)}_6,
B^{(3/2)}_8=1.0 \pm 0.1$ would give
\begin{equation}
0.3 \cdot 10^{-4} \le \epe \le 5.4 \cdot 10^{-4}
\label{eq:eperangefuture2}
\end{equation}
again with no impact from imposing the $x_d$-constraint.

Allowing for the additional variation $B_{2,NDR}^{(1/2)}(\mc) = 6.6 \pm
1.0$ extends ranges \eqn{eq:eperangepresent}--\eqn{eq:eperangefuture2}
to $-2.5 \cdot 10^{-4} \le \epe \le 13.7 \cdot 10^{-4}$, $-1.5 \cdot
10^{-4} \le \epe \le 10.8 \cdot 10^{-4}$ and $0.1 \cdot 10^{-4} \le
\epe \le 5.8 \cdot 10^{-4}$, respectively.

An analysis of the Rome group \cite{ciuchini:95} gives
$\RE(\epe)=(3.1\pm 2.5)\cdot 10^{-4}$ which is compatible with our
results. Similar results are found with hadronic matrix elements
calculated in the chiral quark model \cite{bertolinietal:94},
\cite{bertolinietal:95}.

The difference in the range for $\epe$ presentend here by us and the
Rome group is related to the different treatment of theoretical and
experimental errors. Whereas we simply scan all parameters within one
standard deviation, \cite{ciuchini:95} use Gaussian distributions in
treating the experimental errors. Consequently our procedure is more
conservative. We agree however with these authors that values for
$\epe$ above $1 \cdot 10^{-3}$ although not excluded are very
improbable. This should be contrasted with the work of the Dortmund
group \cite{froehlich:91}, \cite{heinrichetal:92} which finds values
for $\epe$ in the ball park of $(2 - 3) \cdot 10^{-3}$. We do not know
any consistent framework for hadronic matrix elements which would give
such high values within the Standard Model.

The experimental situation on $\RE(\epe)$ is unclear at present.  While
the result of the NA31 collaboration at CERN with $\RE(\epe) = (23 \pm
7)\cdot 10^{-4}$ \cite{barr:93} clearly indicates direct CP violation,
the value of E731 at Fermilab, $\RE(\epe) = (7.4 \pm 5.9)\cdot 10^{-4}$
\cite{gibbons:93}, is compatible with superweak theories
\cite{wolfenstein:64} in which $\epe = 0$.  The E731 result is in the
ball park of the theoretical estimates.  The NA31 value appears a bit
high compared to the range given in \eqn{eq:eperangepresent} above.

Hopefully, in about three years the experimental situation concerning
$\epe$ will be clarified through the improved measurements by the two
collaborations at the $10^{-4}$ level and by experiments at the $\Phi$
factory in Frascati.  One should also hope that the theoretical
situation of $\epe$ will improve by then to confront the new data.
