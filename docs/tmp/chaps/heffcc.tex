\section{The Effective $\Delta F=1$ Hamiltonian: Current-Current Operators}
   \label{sec:HeffdF1:22}

\subsection{Operators}
   \label{sec:HeffdF1:22:op}
We begin our compendium by presenting the parts of effective
hamiltonians involving the current-current operators only.
These operators will be generally denoted by $Q_1$ and $Q_2$,
although their flavour structure depends on the decay considered.
To be specific we will consider
\begin{equation}\label{B1}
Q_1=(\bar b_i c_j)_{V-A} (\bar u_j d_i)_{V-A}
\qquad 
Q_2=(\bar b_i c_i)_{V-A} (\bar u_j d_j)_{V-A}
\end{equation}
\begin{equation}\label{B2}
Q_1=(\bar s_i c_j)_{V-A} (\bar u_j d_i)_{V-A}
\qquad 
Q_2=(\bar s_i c_i)_{V-A} (\bar u_j d_j)_{V-A}
\end{equation}
\begin{equation}\label{B3}
Q_1=(\bar s_i u_j)_{V-A} (\bar u_j d_i)_{V-A}
\qquad 
Q_2=(\bar s_i u_i)_{V-A} (\bar u_j d_j)_{V-A}
\end{equation}
for $\Delta B=1$, $\Delta C=1$ and $\Delta S=1$ decays respectively.
Then the corresponding effective hamiltonians are given by
\begin{equation}\label{B4}
H_{eff}(\Delta B=1)=\frac{G_F}{\sqrt{2}}V_{cb}^{*}V_{ud}
\lbrack C_1(\mu) Q_1+C_2(\mu)Q_2 \rbrack
\qquad
(\mu=O(m_b))
\end{equation}
\begin{equation}\label{B5}
H_{eff}(\Delta C=1)=\frac{G_F}{\sqrt{2}}V_{cs}^{*}V_{ud}
\lbrack C_1(\mu) Q_1+C_2(\mu)Q_2 \rbrack
\qquad
(\mu=O(m_c))
\end{equation}
\begin{equation}\label{B6}
H_{eff}(\Delta S=1)=\frac{G_F}{\sqrt{2}}V_{us}^{*}V_{ud}
\lbrack C_1(\mu) Q_1+C_2(\mu)Q_2 \rbrack
\qquad
(\mu=O(1\gev))
\end{equation}
As we will see in subsequent sections these hamiltonians have to be
generalized to include also penguin operators. This however will not
change the Wilson coefficients $C_1(\mu)$ and $C_2(\mu)$ except for
small $O(\aem)$ corrections in a complete analysis which includes
also electroweak penguin operators. For this reason it is useful to
present the results for $C_{1,2}$ separately as they can be used in a
large class of decays.

When analyzing $Q_1$ and $Q_2$ in isolation, it is useful to work with
the operators $Q_{\pm}$ and their coefficients $z_{\pm}$ defined by
\begin{equation}\label{B7}
Q_{\pm}=\frac{1}{2} (Q_2\pm Q_1)
\qquad
\qquad
z_\pm=C_2\pm C_1 \, .
\end{equation}
$Q_+$ and $Q_-$ do not mix under renormalization and the expression for
$z_\pm(\mu)$ is very simple.

\subsection{Wilson Coefficients and RG Evolution}
   \label{sec:HeffdF1:22:wcrg}
The initial conditions for $z_\pm$ at $\mu=M_W$ are obtained using the
matching procedure between the full (fig.\ \ref{fig:1loopful}\,(a)--(c))
and effective (fig.\ \ref{fig:1loopeff}\,(a)--(c)) theory summarized in
section \ref{sec:basicform:wc:rgf}. Given the initial conditions for
$z_\pm$ at scale $\mu=M_W$
\begin{equation}\label{B8}
z_\pm(M_W)=1+\frac{\as(M_W)}{4\pi}B_\pm
\end{equation}
and using the NLO RG evolution formula \eqn{cjua} for the case without
mixing one finds for the Wilson coefficients of $Q_\pm$ at some scale $\mu$
\begin{equation}\label{B9}
z_\pm(\mu)=\left[1+\frac{\as(\mu)}{4\pi}J_\pm\right]
      \left[\frac{\as(M_W)}{\as(\mu)}\right]^{d_\pm}
\left[1+\frac{\as(M_W)}{4\pi}(B_\pm-J_\pm)\right]
\end{equation}
with
\begin{equation}\label{B10}
J_\pm=\frac{d_\pm}{\beta_0}\beta_1-\frac{\gamma^{(1)}_\pm}{2\beta_0}
\qquad\qquad
d_\pm=\frac{\gamma^{(0)}_\pm}{2\beta_0} \, ,
\end{equation}
where the coefficients $\beta_0$ and $\beta_1$ of the QCD
$\beta$-function are given in \eqn{b0b1}. Furthermore the LO and NLO
expansion coefficients for the anomalous dimensions $\gamma_\pm$ of
$Q_\pm$ in \eqn{B10} and the coefficients $B_\pm$ in \eqn{B8} are given by
\begin{equation}\label{B11}
\gamma^{(0)}_\pm=\pm 12 \frac{N\mp 1}{2N}
\end{equation}
\begin{equation}\label{B12}
\gamma^{(1)}_{\pm}=\frac{N\mp 1}{2N}
\left[-21\pm\frac{57}{N}\mp\frac{19}{3}N \pm
\frac{4}{3}f-2\beta_0\kappa_\pm\right]
\end{equation}
\begin{equation}\label{B13}
B_\pm=\frac{N\mp 1}{2N}\left[\pm 11+\kappa_\pm\right]
\end{equation}
with $N$ being the number of colors.  Here we have introduced the
parameter $\kappa_\pm$ which conveniently distinguishes between various
renormalization schemes
\begin{equation}
\kappa_\pm = \left\{ \begin{array}{rl}
    0 & \qquad {\rm NDR}  \\
\mp 4 & \qquad {\rm HV}
\end{array}\right. \, .
\label{B14}
\end{equation}

Thus, using $N=3$ in the following, $J_\pm$ in (\ref{B10}) can also be
written as
\begin{equation}\label{B15}
J_\pm=(J_\pm)_{\rm NDR}+\frac{3\mp 1}{6}\kappa_\pm
=(J_\pm)_{\rm NDR}\pm\frac{\gamma^{(0)}_\pm}{12}\kappa_\pm
\end{equation}
Setting $\gamma_\pm^{(1)}$, $B_\pm$ and $\beta_1$ to zero gives the
leading logarithmic approximation \cite{altarelli:74}, \cite{gaillard:74}.

The NLO calculations in the NDR scheme and in the HV scheme have been
presented in \cite{burasweisz:90}.  In writing (\ref{B12}) we have
incorporated the $-2 \gamma^{(1)}_J$ correction in the HV scheme
resulting from the non-vanishing two--loop anomalous dimension of the
weak current.
\begin{equation}
\gamma^{(1)}_J = \left\{
\begin{array}{lcl}
                           0 & \qquad & \mbox{NDR} \\
\frac{N^2 -1}{N} \, 2\beta_0 & \qquad & \mbox{HV}
\end{array}
\right.
\label{eq:wcanom}
\end{equation}
The NLO corrections $\gamma^{(1)}_\pm$ in the
dimensional reduction scheme (DRED) have been first considered in
\cite{altarelli:81} and later confirmed in \cite{burasweisz:90}. Here
one has $\kappa_\pm = \mp 6 - N$. This value for $\kappa_\pm$ in DRED
incorporates also a finite renormalization of $\as$ in order to work in
all schemes with the usual $\overline{MS}$ coupling.

As already discussed in section \ref{sec:basicform:wc:rgdep}, the
expression $(B_\pm-J_\pm)$ is scheme independent.  The scheme
dependence of the Wilson coeffcients $z_\pm(\mu)$ originates then
entirely from the scheme dependence of $J_\pm$ at the lower end of the
evolution which can be seen explicitly in (\ref{B15}).

In order to exhibit the $\mu$ dependence on the same footing as the
scheme dependence, it is useful to rewrite (\ref{B9}) in the case of
B--decays as follows:
\begin{equation}\label{B16}
z_\pm(\mu)=\left[1+\frac{\as(m_b)}{4\pi} \tilde J_\pm(\mu)\right]
      \left[\frac{\as(M_W)}{\as(m_b)}\right]^{d_\pm}
\left[1+\frac{\as(M_W)}{4\pi}(B_\pm-J_\pm)\right]
\end{equation}
with 

\begin{equation}\label{B17}
\tilde J_\pm(\mu)=(J_\pm)_{NDR}\pm 
\frac{\gamma^{(0)}_\pm}{12}\kappa_\pm
+\frac{\gamma^{(0)}_\pm}{2}\ln(\frac{\mu^2}{m^2_b})
\end{equation}
summarizing both the renormalization scheme dependence and the
$\mu$--dependence. Note that in the first parenthesis in (\ref{B16}) we
have set $\as(\mu)=\as(m_b)$ as the difference in the scales
in this correction is still of higher order.  We also note that a
change of the renormalization scheme can be compensated by a change in
$\mu$. From (\ref{B17}) we find generally
\begin{equation}\label{B18}
\mu_i^\pm=\mu_{NDR} \, \exp\left(\mp\frac{\kappa_\pm^{(i)}}{12}\right)
\end{equation}
where $i$ denotes a given scheme. From (\ref{B14}) we then have
\begin{equation}\label{B19}
\mu_{HV}=\mu_{NDR} \, \exp\left(\frac{1}{3}\right)
\end{equation}
Evidently the change in $\mu$ relating HV and NDR\footnote{ The
relation $\mu_{\rm DRED}^{\pm} = \mu_{\rm NDR} \, \exp\left(\frac{2\pm
1}{4}\right)$ between NDR and DRED is more involved. In any case
$\mu_{\rm HV}$ and $\mu_{\rm DRED}^\pm$ are larger than $\mu_{\rm
NDR}$.} is the same for $z_+$ and $z_-$ and consequently for
$C_i(\mu)$.

This discussion shows that a meaningful analysis of the $\mu$
dependence of $C_i(\mu)$ can only be made simultaneously with the
analysis of the scheme dependence.

The coefficients $C_i(\mu)$ for B-decays can now be calculated using
\begin{equation}\label{B20}
C_1(\mu)=\frac{z_+(\mu)-z_-(\mu)}{2}
\qquad\qquad
C_2(\mu)=\frac{z_+(\mu)+z_-(\mu)}{2}
\end{equation}
To this end we set $f=5$ in the formulae above and use the two-loop
$\as(\mu)$ of eq.\ \eqn{amu} with $\Lambda \equiv \Lms^{(5)}$. The
actual numerical values used for $\as(M_Z)$ or equivalently
$\Lms^{(5)}$ are collected in appendix \ref{app:numinput} together with
other numerical input parameters.

In the case of D-decays and K-decays the relevant scales are
$\mu=\ord(m_c)$ and $\mu=\ord(1\gev)$, respectively. In order to
calculate $C_i(\mu)$ for these cases one has to evolve these
coefficients first from $\mu=\ord(m_b)$ further down to $\mu=\ord(m_c)$ in
an effective theory with $f=4$. Matching $\as^{(5)}(m_b) = \as^{(4)}(m_b)$
we find to a very good approximation $\Lms^{(4)}=(325\pm110)\mev$.
Unfortunately, the necessity to evolve $C_i(\mu)$ from $\mu=M_W$ down to
$\mu=m_c$ in two different effective theories ($f=5$ and $f=4$) and
eventually in the case of K-decays with $f=3$ for $\mu< m_c$ makes the
formulae for $C_i(\mu)$ in D--decays and K--decays rather complicated.
They can be found in \cite{burasetal:92d}. Fortunately all these
complications can be avoided by a simple trick, which reproduces the
results of \cite{burasetal:92d} to better than $1.5\%$.  In order to
find $C_i(\mu)$ for $1\gev\leq\mu\leq 2\gev$ one can simply use the
master formulae given above with $\Lms^{(5)}$ replaced by $\Lms^{(4)}$
and an ``{effective}'' number of active flavours $f=4.15$. The latter
effective value for $f$ allows to obtain a very good agreement with
\cite{burasetal:92d}.  This can be verified by comparing the results
presented here with those in tables \ref{tab:wc6smu1} and
\ref{tab:wc6smu2} where no ``tricks'' have been used.  The nice feature
of this method is that the $\mu$ and renormalization scheme dependences
of $C_i(\mu)$ can still be studied in simple terms.

The numerical coefficients $C_i(\mu)$ for B--decays are shown in tables
\ref{tab:c1B} and \ref{tab:c2B} for different $\mu$ and  $\Lms^{(5)}$.
In addition to the results for the NDR and HV renormalization schemes
we show the LO values\footnote{The results for the DRED scheme can be
found in \cite{buras:94}.}. The corresponding results for K--decays and
D--decays are given in tables \ref{tab:c1KD} and \ref{tab:c2KD}.

\begin{table}[htb]
\caption[]{The coefficient $C_1(\mu)$ for  B-decays.}
\label{tab:c1B}
\begin{center}
\begin{tabular}{|c|c|c|c||c|c|c||c|c|c|}
& \multicolumn{3}{c||}{$\Lms^{(5)}=140\mev$} &
  \multicolumn{3}{c||}{$\Lms^{(5)}=225\mev$} &
  \multicolumn{3}{c| }{$\Lms^{(5)}=310\mev$} \\
\hline
$\mu [{\rm GeV}]$ &LO & NDR & HV & LO & NDR & HV & LO & NDR & HV  \\
\hline
\hline
4.0 & --0.274 & --0.175 & --0.211 & --0.310 & --0.197 & --0.239 & --0.341 &
--0.216 & --0.264  \\
\hline
5.0 & --0.244 & --0.151 & --0.184 & --0.274 & --0.169 & --0.208 & --0.300 &
--0.185 & --0.228  \\
\hline
6.0 & --0.221 & --0.133 & --0.164 & --0.248 & --0.148 & --0.184 & --0.269 &
--0.161 & --0.201  \\
\hline
7.0 & --0.203 & --0.118 & --0.148 & --0.226 & --0.132 & --0.166 & --0.246 &
--0.143 & --0.181  \\
\hline
8.0 & --0.188 & --0.106 & --0.135 & --0.209 & --0.118 & --0.151 & --0.226 &
--0.128 & --0.164  \\
\end{tabular}
\end{center}
\end{table}

\begin{table}[htb]
\caption[]{The coefficient $C_2(\mu)$ for B-decays.}
\label{tab:c2B}
\begin{center}
\begin{tabular}{|c|c|c|c||c|c|c||c|c|c|}
& \multicolumn{3}{c||}{$\Lms^{(5)}=140\mev$} &
  \multicolumn{3}{c||}{$\Lms^{(5)}=225\mev$} &
  \multicolumn{3}{c| }{$\Lms^{(5)}=310\mev$} \\
\hline
$\mu [{\rm GeV}]$ & LO& NDR & HV & LO & NDR & HV & LO & NDR & HV  \\
\hline
\hline
4.0 & 1.121 & 1.074 & 1.092 & 1.141 & 1.086 & 1.107 & 1.158 &
1.096 & 1.120  \\
\hline
5.0 & 1.105 & 1.062 & 1.078 & 1.121 & 1.072 & 1.090 & 1.135 &
1.080 & 1.101  \\
\hline
6.0 & 1.093 & 1.054 & 1.069 & 1.107 & 1.062 & 1.079 & 1.118 &
1.068 & 1.087  \\
\hline
7.0 & 1.084 & 1.047 & 1.061 & 1.096 & 1.054 & 1.069 & 1.106 &
1.059 & 1.077  \\
\hline
8.0 & 1.077 & 1.042 & 1.055 & 1.087 & 1.047 & 1.062 & 1.096 &
1.052 & 1.069  \\
\end{tabular}
\end{center}
\end{table}

\noindent
From tables \ref{tab:c1B}--\ref{tab:c2Bnlo} we observe:
\begin{itemize}
\item
The scheme dependence of the Wilson coefficients is sizable.  This is
in particular the case of $C_1$ which vanishes in the absence of QCD
corrections.
\item
The differences between LO and NLO results in the case of $C_1$ are
large showing the importance of next--to--leading corrections. In fact
in the NDR scheme the corrections may be as large as $70\%$.  This
comparison of LO and NLO coefficients can however be questioned because
for the chosen values of $\Lms$ one has $\as^{(LO)}(M_Z)=0.135 \pm
0.009$ to be compared with $\as(M_Z)=0.117 \pm 0.007$ \cite{bethke:94},
\cite{webber:94}. Consequently the difference in LO and NLO results for
$C_i$ originates partly in the change in the value of the QCD
coupling.
\item
In view of the latter fact it is instructive to show also the LO
results in which the next-to-leading expression for $\as$ is used.
We give some examples in tables \ref{tab:c1Bnlo} and \ref{tab:c2Bnlo}.
Now the differences between LO and NLO results is considerably smaller
although still as large as $30-40\%$ in the case of $C_1$ and the NDR
scheme.
\item
In any case the inclusion of NLO corrections in NDR and HV schemes
weakens the impact of QCD on the Wilson coefficients of
current--current operators.  It is however important to keep in mind
that such a behavior is specific to the scheme chosen and will in
general be different in other schemes, reflecting the unphysical nature
of the Wilson coefficient functions.
\end{itemize}

\begin{table}[htb]
\caption[]{The coefficient $C_1(\mu)$ for K-decays and D-decays.}
\label{tab:c1KD}
\begin{center}
\begin{tabular}{|c|c|c|c||c|c|c||c|c|c|}
& \multicolumn{3}{c||}{$\Lms^{(4)}=215\mev$} &
  \multicolumn{3}{c||}{$\Lms^{(4)}=325\mev$} &
  \multicolumn{3}{c| }{$\Lms^{(4)}=435\mev$} \\
\hline
$\mu [{\rm GeV}]$ &LO & NDR & HV & LO & NDR & HV & LO & NDR & HV  \\
\hline
\hline
1.00 & --0.602 & --0.410 & --0.491 & --0.742 & --0.510 & --0.631 & --0.899 &
--0.632 & --0.825  \\
\hline
1.25 & --0.529 & --0.356 & --0.424 & --0.636 & --0.430 & --0.523 & --0.747 &
--0.512 & --0.642  \\
\hline
1.50 & --0.478 & --0.319 & --0.379 & --0.565 & --0.378 & --0.457 & --0.653 &
--0.439 & --0.543  \\
\hline
1.75 & --0.439 & --0.291 & --0.346 & --0.514 & --0.340 & --0.410 & --0.587 & 
--0.390 & --0.478  \\
\hline
2.00 & --0.409 & --0.269 & --0.320 & --0.475 & --0.311 & --0.375 & --0.537 &
--0.353 & --0.431  \\
\end{tabular}
\end{center}
\end{table}

\begin{table}[htb]
\caption[]{The coefficient $C_2(\mu)$ for K-decays and D-decays.}
\label{tab:c2KD}
\begin{center}
\begin{tabular}{|c|c|c|c||c|c|c||c|c|c|}
& \multicolumn{3}{c||}{$\Lms^{(4)}=215\mev$} &
  \multicolumn{3}{c||}{$\Lms^{(4)}=325\mev$} &
  \multicolumn{3}{c| }{$\Lms^{(4)}=435\mev$} \\
\hline
$\mu [{\rm GeV}]$ &LO & NDR & HV & LO & NDR & HV & LO & NDR & HV  \\
\hline
\hline
1.00 & 1.323 & 1.208 & 1.259 & 1.422 & 1.275 & 1.358 & 1.539 &  
1.363 & 1.506  \\
\hline
1.25 & 1.274 & 1.174 & 1.216 & 1.346 & 1.221 & 1.282 & 1.426 & 
1.277 & 1.367  \\
\hline
1.50 & 1.241 & 1.152 & 1.187 & 1.298 & 1.188 & 1.237 & 1.358 & 
1.228 & 1.296  \\
\hline
1.75 & 1.216 & 1.136 & 1.167 & 1.264 & 1.165 & 1.207 & 1.313 & 
1.196 & 1.252  \\
\hline
2.00 & 1.198 & 1.123 & 1.152 & 1.239 & 1.148 & 1.185 & 1.279 &
1.174 & 1.221  \\
\end{tabular}
\end{center}
\end{table}

\begin{table}[htb]
\caption[]{$C_1^{\rm LO}$ and $C_2^{\rm LO}$ for B-decays with $\as$
in NLO.}
\label{tab:c1Bnlo}
\begin{center}
\begin{tabular}{|c|c|c||c|c||c|c|}
& \multicolumn{2}{c||}{$\Lms^{(5)}=140\mev$} &
  \multicolumn{2}{c||}{$\Lms^{(5)}=225\mev$} &
  \multicolumn{2}{c| }{$\Lms^{(5)}=310\mev$} \\
\hline
$\mu [{\rm GeV}]$ & $C_1$ & $C_2$ & $C_1$ & 

$C_2$ & $C_1$ & $C_2$  \\
\hline
\hline
4.0 & --0.244 & 1.105 & --0.274 & 1.121 & --0.301 & 1.135
\\
\hline
5.0 & --0.217 & 1.091 & --0.243 & 1.105 & --0.265 & 1.116
\\
\hline
6.0 & --0.197 & 1.082 & --0.220 & 1.093 & --0.239 & 1.102
\\
\end{tabular}
\end{center}
\end{table}

\begin{table}[htb]
\caption[]{$C_1^{\rm LO}$ and $C_2^{\rm LO}$ for K and D-decays with $\as$
in NLO.}
\label{tab:c2Bnlo}
\begin{center}
\begin{tabular}{|c|c|c||c|c||c|c|}
& \multicolumn{2}{c||}{$\Lms^{(4)}=215\mev$} &
  \multicolumn{2}{c||}{$\Lms^{(4)}=325\mev$} &
  \multicolumn{2}{c| }{$\Lms^{(4)}=435\mev$} \\
\hline
$\mu [{\rm GeV}]$ & $C_1$ & $C_2$ & $C_1$ & $C_2$ & $C_1$ & $C_2$  \\
\hline
\hline
1.0 & --0.524 & 1.271 & --0.664 & 1.366 & --0.851 & 1.502
\\
\hline
1.5 & --0.413 & 1.201 & --0.493 & 1.250 & --0.579 & 1.307
\\
\hline
2.0 & --0.354 & 1.165 & --0.412 & 1.200 & --0.469 & 1.235
\\
\end{tabular}
\end{center}
\end{table}

We have made the whole discussion without invoking HQET (cf.\ section
\ref{sec:HQET}). It is sometimes stated in the literature that at
$\mu=m_b$ in the case of B-decays one {\it has to } switch to HQET.  In
this case for $\mu<m_b$ the anomalous dimensions $\gamma_\pm$ differ
from those given above. We should however stress that switching to HQET
can be done at any $\mu<m_b$ provided the logarithms $\ln(m_b/\mu)$ in
$\langle Q_i \rangle$ do not become too large.  Similar comments apply
to D-decays with respect to $\mu=m_c$. Of course the coefficients $C_i$
calculated in HQET for $\mu<m_b$ are different from the coefficients
presented here. However the corresponding matrix elements $\langle Q_i
\rangle$ in HQET are also different so that the physical amplitudes
remain unchanged.
