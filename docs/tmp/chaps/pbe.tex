\section{Penguin Box Expansion for FCNC Processes}
         \label{sec:PBE}
An important virtue of OPE and RG is that with $m_t > \mw$ the
dependence of weak decays on the top quark mass is very elegantly
isolated. It resides only in the initial conditions for the Wilson
coefficients at scale $\mu \approx \mw$ i.e.~in $C_i(\mw)$. A quick look
at the initial conditions in the previous sections reveals the important
fact that the leading $m_t$-dependence in all decays considered is
represented universally by the $m_t$-dependent functions which result
from exact calculations of the relevant penguin and box diagrams with
internal top quark exchanges. These are the functions
\begin{equation}
S_0(x_t), \quad
B_0(x_t), \quad
C_0(x_t), \quad
D_0(x_t), \quad
E_0(x_t), \quad
D'_0(x_t), \quad
E'_0(x_t)
\label{eq:SBCDE}
\end{equation}
for which explicit expressions are given in \eqn{s0t},
eqs.~\eqn{eq:Bxt}--\eqn{eq:Dxt}, \eqn{eq:Ext}, \eqn{c7} and \eqn{c8},
respectively. In certain decays some of these functions do not appear
because the corresponding penguin or box diagram does not contribute to
the initial conditions. However, the function $C_0(x_t)$ resulting from
the $Z^0$-penguin diagram enters all $\Delta F=1$ decays but $B \to X_s
\gamma$. Having a quadratic dependence on $m_t$, this function
is responsible for the dominant $m_t$-dependence of these
decays. Since the non-leading $m_t$-dependence of $C_0(x_t)$ is gauge
dependent, $C_0(x_t)$ is always accompanied by $B_0(x_t)$ or $D_0(x_t)$
in such a way that this dependence cancels. For this reason it is
useful to replace the gauge dependent functions $B_0(x_t)$, $C_0(x_t)$
and $D_0(x_t)$ by the gauge independent set \cite{buchallaetal:91}
\begin{eqnarray}
X_0(x_t) &=& C_0(x_t) - 4 \, B_0(x_t)           \nn \\
Y_0(x_t) &=& C_0(x_t) - B_0(x_t)                \label{eq:XYZ} \\
Z_0(x_t) &=& C_0(x_t) + \frac{1}{4} \, D_0(x_t) \nn
\end{eqnarray}
as we have already done at various places in this review.  The
inclusion of NLO QCD corrections to \BB-, \KK-mixing and the rare $K$-
and $B$-decays of section~\ref{sec:HeffRareKB} requires the calculation
of QCD corrections to penguin and box diagrams in the full theory. This
results in the functions $\tilde{S}(x_t)=\eta_2 S_0(x_t)$, $X(x_t)$ and
$Y(x_t)$, with the latter two given in \eqn{xx} and \eqn{yy}, respectively.
\\
It turns out however that if the top quark mass is definded as $m_t
\equiv \bar{m}_t(m_t)$ one has
\begin{equation}
\tilde{S}(x_t) = \eta_2 \, S_0(x_t), \quad
X(x_t) = \eta_X \, X_0(x_t), \quad
Y(x_t) = \eta_Y \, Y_0(x_t)
\label{eq:SXY}
\end{equation}
with $\eta_2$, $\eta_X$ and $\eta_Y$ almost independent of $m_t$.
Numerical values of $\eta_X$ and $\eta_Y$ are given in part three.

Consequently with this definition of $m_t$ the basic $m_t$-dependent
functions listed in \eqn{eq:SBCDE} and \eqn{eq:XYZ} represent the
$m_t$-dependence of weak decays at the NLO level to a good
approximation.  It should be remarked that the QCD corrections to
$D_0$, $E_0$, $D'_0$ and $E'_0$ have not been calculated yet. They
would however be only required for still higher order corrections
(NNLO) in the renormalization group improved perturbation theory as
far as $D_0$ and $E_0$ are concerned. On the other hand, in the case
of $D_0'$ and $E_0'$, which are relevant for the $b \to s \gamma$ decay,
these corrections are necessary.

An inspection of the effective hamiltonians derived in the previous
sections shows that for \BB-mixing, \KK-mixing and the rare decays of
section~\ref{sec:HeffRareKB} the $m_t$ dependence of the effective
hamiltonian is explicitly given in terms of the basic functions listed
above. Due to the one step evolution from $\mu_t$ to $\mu_b$ we have
also presented the explicit $\mt$-dependence for $B \to X_s \gamma$ and
$B \to X_s e^+ e^-$ decays. On the other hand in the case of $\Kpipi$
and $K_L \to \pi^0 e^+ e^-$ where the renormalization group evolution
is very complicated the $m_t$ dependence of a given box or penguin
diagram is distributed among various Wilson coefficient functions. In
other words the $m_t$-dependence acquired at scale $\mu \approx
\ord(\mw)$ is hidden in a complicated numerical evaluation of
$\hU(\mu,\mw)$.

For phenomenological applications it is more elegant and more
convenient to have a formalism in which the final formulae for all
amplitudes are given explicitly in terms of the basic $m_t$-dependent
functions discussed above.

In \cite{buchallaetal:91} an approach has been presented which
accomplishes this task. It gives the decay amplitudes as linear
combinations of the basic, universal, process independent but
$m_t$-dependent functions $F_r(x_t)$ of eq.~\eqn{eq:SBCDE} with
corresponding coefficients $P_r$ characteristic for the decay under
consideration. This approach termed ``Penguin Box Expansion'' (PBE) has
the following general form
\begin{equation}
A({\rm decay}) = P_0({\rm decay}) + \sum_r P_r({\rm decay}) \, F_r(x_t)
\label{eq:generalPBE}
\end{equation}
where the sum runs over all possible functions contributing to a given
amplitude. In \eqn{eq:generalPBE} we have separated a $m_t$-independent
term $P_0$ which summarizes contributions stemming from internal quarks
other than the top, in particular the charm quark.

Many examples of PBE appear in this review. Several decays or
transitions depend only on a single function out of the complete set
\eqn{eq:SBCDE}. For completeness we give here the correspondence
between various processes and the basic functions

\begin{center}
\begin{tabular}{lcl}
\BB-mixing &\qquad\qquad& $S_0(x_t)$ \\
$K \to \pi \nu \bar\nu$, $B \to K \nu \bar\nu$, 
$B \to \pi \nu \bar\nu$ &\qquad\qquad& $X_0(x_t)$ \\
$K \to \mu \bar\mu$, $B \to l\bar l$ &\qquad\qquad& $Y_0(x_t)$ \\
$K_L \to \pi^0 e^+ e^-$ &\qquad\qquad& $Y_0(x_t)$, $Z_0(x_t)$, $E_0(x_t)$ \\
$\varepsilon'$ &\qquad\qquad& $X_0(x_t)$, $Y_0(x_t)$, $Z_0(x_t)$,
$E_0(x_t)$ \\
$B \to X_s \gamma$ &\qquad\qquad& $D'_0(x_t)$, $E'_0(x_t)$ \\
$B \to X_s e^+ e^-$ &\qquad\qquad&
$Y_0(x_t)$, $Z_0(x_t)$, $E_0(x_t)$, $D'_0(x_t)$, $E'_0(x_t)$
\end{tabular}
\end{center}

In \cite{buchallaetal:91} an explicit transformation from OPE to
PBE has been made. This transformation and the relation between these
two expansions can be very clearly seen on the basis of 
\begin{equation}
A(P \to F) = \sum_{i,k} \langle F | O_k(\mu) | P \rangle \,
U_{kj}(\mu,\mw) \, C_j(\mw)
\label{eq:RGTransf}
\end{equation}
where $U_{kj}(\mu,\mw)$ represents the renormalization group
transformation from $\mw$ down to $\mu$. As we have seen, OPE puts the
last two factors in this formula together, mixing this way the physics
around $\mw$ with all physical contributions down to very low energy
scales. The PBE is realized on the other hand by putting the first two
factors together and rewriting $C_j(\mw)$ in terms of the basic
functions \eqn{eq:SBCDE}. This results in the expansion of
eq.~\eqn{eq:generalPBE}. Further technical details and the methods for
the evaluation of the coefficients $P_r$ can be found in
\cite{buchallaetal:91}, where further virtues of PBE are discussed.

Finally, we give approximate formulae having power-like
dependence on $x_t$ for the basic, gauge independent functions of PBE
\begin{equation}
\begin{array}{lclclcl}
S_0(x_t)  &=& 0.784 \cdot x_t^{ 0.76}  &\qquad\qquad&
X_0(x_t)  &=& 0.660 \cdot x_t^{0.575} \\
Y_0(x_t)  &=& 0.315 \cdot x_t^{ 0.78}  &\qquad\qquad&
Z_0(x_t)  &=& 0.175 \cdot x_t^{0.93} \\
E_0(x_t)  &=& 0.564 \cdot x_t^{-0.51}  &\qquad\qquad&
D'_0(x_t) &=& 0.244 \cdot x_t^{0.30} \\
E'_0(x_t) &=& 0.145 \cdot x_t^{ 0.19} \, . &\qquad\qquad&  
\end{array}
\label{eq:approxSXYZE}
\end{equation}
In the range $150\gev \le m_t \le 200\gev$ these approximations
reproduce the exact expressions to an accuracy better than 1\%.
