\section{$K_L-K_S$ Mass Difference and $\Delta I=1/2$ Rule}
         \label{sec:mki12}
It is probably a good moment to make a few comments on the $K_L-K_S$
mass difference given by \begin{equation}\label{delmk} \Delta
M=M(K_L)-M(K_S)=3.51\cdot 10^{-15}\gev \end{equation} and the
approximate $\Delta I=1/2$ rule in $K\to\pi\pi$ decays.  As we have
already briefly mentioned in the beginning of section
\ref{subsec:epeformulae}, this empirical rule manifests itself in the
dominance of $\Delta I=1/2$ over $\Delta I=3/2$ decay amplitudes.
It can be expressed as
\begin{equation}\label{deli12}
\frac{{\rm Re}A_0}{{\rm Re}A_2}=22.2
\end{equation}
using the notation of section \ref{subsec:epeformulae}.

\subsection{$\Delta M(K_L-K_S)$}
            \label{sec:mki12:mk}
The $K_L-K_S$ mass difference can be written as
\begin{equation}\label{2rem12}
\Delta M=2 {\rm Re} M_{12}+(\Delta M)_{LD}
\end{equation}
with $M_{12}$ given in \eqn{eq:M12K} and $(\Delta M)_{LD}$
representing long distance contributions, corresponding for instance to
the exchange of intermediate light pseudoscalar mesons ($\pi^0$,
$\eta$). The first term in \eqn{2rem12}, the so-called short distance
contribution, is dominated by the first term in \eqn{eq:M12K} so that
\begin{equation}\label{dmsd}
(\Delta M)_{SD}=\frac{G^2_F}{6\pi^2}F^2_K B_K m_K M^2_W
\left[ \lambda^2_c\eta_1\frac{m^2_c}{M^2_W}+\Delta_{top}\right]
\end{equation}
where $\Delta_{top}$ represents the two top dependent terms in
\eqn{eq:M12K}. In writing \eqn{dmsd} we are neglecting the tiny
imaginary part in $\lambda_c=V^*_{cs}V_{cd}$. A very extensive
numerical analysis of \eqn{dmsd} has been presented by
\cite{herrlichnierste:93}, who calculated the NLO corrections to
$\eta_1$ and also to $\eta_3$ \cite{herrlichnierste:95} which enters
$\Delta_{top}$.  The NLO calculation of the short distance
contributions improves the matching to the non-perturbative matrix
element parametrized by $B_K$ and clarifies the proper definition of
$B_K$ to be used along with the QCD factors $\eta_i$. In addition the
NLO study reveals an enhancement of $\eta_1$ over its LO estimate by
about 20\%. Although sizable, this enhancement can still be considered
being perturbative, as required by the consistency of the calculation.
This increase in $\eta_1$, reinforced by updates in input parameters
($\Lms$), brings $(\Delta M)_{SD}$ closer to the experimental value in
\eqn{delmk}.  With $\Lms^{(4)}=325\mev$ and $\mc=1.3\gev$, giving
$\eta^{NLO}_1=1.38$, one finds that typically $70\%$ of $\Delta M$ can
be described by the short distance component. The exact value is still
somewhat uncertain because $\eta_1$ is rather sensitive to $\Lms$.
Further uncertainties are introduced by the error in $B_K$ and due to
the renormalization scale ambiguity, which is still quite pronounced
even at NLO. Yet the result is certainly more reliable than previous LO
estimates.  Using the old value $\eta^{LO}_1=0.85$, corresponding to
$\mc=1.4\gev$ and $\Lambda_{QCD}=200\mev$, $(\Delta M)_{SD}/\Delta M$
would be below $50\%$, suggesting a dominance of long distance
contributions in $\Delta M$. As discussed in \cite{herrlichnierste:93},
such a situation would be "unnatural" since the long distance component
is formally suppressed by $\Lambda^2_{QCD}/m^2_c$. Hence the short
distance dominance indicated by the NLO analysis is also gratifying in
this respect.
\\
The long distance contributions, to which one can attribute the 
remaining $\sim 30\%$ in $\Delta M$ not explained by the short distance
part, are nicely discussed in \cite{bijnensetal:91}.
\\
In summary, the observed $K_L-K_S$ mass difference can be roughly
described within the standard model after the NLO corrections
have been taken into account. The remaining theoretical uncertainties
in the dominant part in \eqn{dmsd} and the uncertainties in
$(\Delta M)_{LD}$ do not allow however to use $\Delta M$ as a 
constraint on the CKM parameters.

\subsection{The $\Delta I=1/2$ Rule}
            \label{sec:mki12:i12}
Using the effective hamiltonian in \eqn{eq:HeffdF1:1010} and 
keeping only the dominant terms one has
\begin{equation}\label{rea0a2}
\frac{{\rm Re}A_0}{{\rm Re}A_2}\approx
\frac{z_1(\mu)\langle Q_1(\mu)\rangle_0+
      z_2(\mu)\langle Q_2(\mu)\rangle_0+z_6(\mu)\langle Q_6(\mu)\rangle_0} 
{z_1(\mu)\langle Q_1(\mu)\rangle_2+z_2(\mu)\langle Q_2(\mu)\rangle_2}
\end{equation}
where $\langle Q_i\rangle_{0,2}$ are defined in \eqn{eq:QiKpp}.  The
coefficients $z_i(\mu)$ can be found in table \ref{tab:wc10smu1}.  For
the hadronic matrix elements we use the formulae \eqn{eq:Q10},
\eqn{eq:Q20}, \eqn{eq:Q60} and \eqn{eq:Q122}, which have been
discussed in section \ref{subsec:matelKpp}.  We find then, separating
current-current and penguin contributions
\begin{equation}\label{rcrp}
\frac{{\rm Re}A_0}{{\rm Re}A_2}=R_c+R_p
\end{equation}
\begin{equation}\label{rczb}
R_c=\frac{5z_2(\mu) B^{(1/2)}_2-z_1(\mu) B^{(1/2)}_1}{4\sqrt{2}z_+(\mu)
  B^{(3/2)}_1} \qquad\qquad   z_+=z_1+z_2
\end{equation}
\begin{equation}\label{rpzb}
R_p=-11.9\frac{z_6(\mu)}{z_+(\mu)}\frac{B^{(1/2)}_6}{B^{(3/2)}_1}
\left[\frac{178\mev}{m_s(\mu)+m_d(\mu)}\right]^2
\end{equation}
The factor $11.9$ expresses the enhancement of the matrix elements of
the penguin operator $Q_6$ over $\langle Q_{1,2}\rangle$ first pointed
out in \cite{vainshtein:77}. It is instructive to calculate $R_c$ and
$R_p$ using the vacuum insertion estimate for which $B^{(1/2)}_1=$
$B^{(1/2)}_2=$ $B^{(3/2)}_1=$ $B^{(1/2)}_6=1$.  Without QCD effects one
finds then $R_c=0.9$ and $R_p=0$ in complete disagreement with the
data. In table \ref{tab:rcrp} we show the values of $R_c$ and $R_p$ at
$\mu=1\gev$ using the results of table \ref{tab:wc10smu1}. We have set
$m_s+m_d=178\mev$.

\begin{table}[htb]
\caption[]{The quantities $R_c$ and $R_p$ contributing to ${\rm
Re}A_0/{\rm Re}A_2$ as described in the text, calculated using the
vacuum  insertion estimate for the hadronic matrix elements. The Wilson
coefficient functions are evaluated for various $\Lms^{(4)}$in leading
logarithmic approximation as well as in next-to-leading order in two
different schemes (NDR and HV).
\label{tab:rcrp}}
\begin{center}
\begin{tabular}{|c|c|c|c|c|c|c|c|c|c|}
&\multicolumn{3}{c|}{$\Lms^{(4)}=215\mev$}
&\multicolumn{3}{c|}{$\Lms^{(4)}=325\mev$}
&\multicolumn{3}{c|}{$\Lms^{(4)}=435\mev$}\\
 \hline
Scheme&LO&NDR&HV&LO&NDR&HV&LO&NDR&HV\\ \hline
$R_c$&1.8&1.4&1.6&2.0&1.6&1.8&2.4&1.8&2.2\\
 \hline
$R_p$&0.1&0.3&0.1&0.2&0.5&0.2&0.3&1.0&0.4
\end{tabular}
\end{center}
\end{table}

The inclusion of QCD effects enhances both $R_c$ and $R_p$
\cite{gaillard:74}, \cite{altarelli:74}, however even for the highest
values of $\Lms^{(4)}$ the ratio ${\rm Re}A_0/{\rm Re}A_2$ is by at
least a factor of 8 smaller than the experimental value in
\eqn{deli12}. Moreover a considerable scheme dependence is observed.
Lowering $\mu$ would improve the situation, but for $\mu< 1\gev$ the
perturbative calculations of $z_i(\mu)$ can no longer be trusted.
Similarly lowering $m_s$ down to $100\mev$ would increase the penguin
contribution. In view of the most recent estimates in \eqn{eq:msmc}
such a low value of $m_s$ seems to be excluded however. We conclude
therefore, as already known since many years, that the vacuum insertion
estimate fails completely in explaining the $\Delta I=1/2$ rule. As we
have discussed in section \ref{sec:nloepe} the vacuum insertion
estimate $B^{(1/2)}_6=1$ is supported by the $1/N$ expansion approach
and by lattice calculations. Consequently the only solution to the
$\Delta I=1/2$ rule problem appears to be a change in the values of the
remaining $B_i$ factors. For instance repeating the above calculation
with $B^{(3/2)}_1=0.48$, $B^{(1/2)}_2=5$ and  $B^{(1/2)}_1=10$ would
give in the NDR scheme $R_c\approx 20$, $R_p\approx 2$ and ${\rm
Re}A_0/{\rm Re}A_2\approx 22$ in accordance with the experimental
value.

There have been several attempts to explain the $\Delta I=1/2$ rule,
which basically use the effective hamiltonian in
\eqn{eq:HeffdF1:1010} but employ different methods for the hadronic
matrix elements.  In particular we would like to mention the $1/N$
approach \cite{bardeen:87b}, the work of \cite{pichderafael:91} based
on an effective action for four-quark operators, the diquark
approach in \cite{neubertstech:91}, QCD sum rules \cite{jaminpich:94},
the chiral perturbation calculations in \cite{kamboretal:90},
\cite{kamboretal:91} and very recently an analysis \cite{antonellietal:95}
 in the framework of the chiral quark model \cite{cohenmanohar:84}.

With these methods values for ${\rm
Re}A_0/{\rm Re}A_2$ in the range 15--20 can be obtained. It is beyond
the scope of this review to discuss the weak and strong points of each
method, although at least one of us believes that the "meson evolution"
picture advocated in \cite{bardeen:87b} represents the main bulk of the
physics behind the number 22.  In view of the uncertainties present in
these approaches, we have not used them in our analysis of
$\varepsilon'/\varepsilon$, but have constrained the hadronic matrix
elements so that they satisfy the $\Delta I=1/2$ rule exactly.
