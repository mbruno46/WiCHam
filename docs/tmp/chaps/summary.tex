\section{Summary}
        \label{sec:summary}
In this review we have described in detail the present status of higher
order QCD corrections to weak decays of hadrons. We have emphasized
that during the last years considerable progress has been made in this
field through the calculation of the next--to--leading QCD corrections
to essentially all of the most interesting and important processes.
This effort reduced considerably the theoretical uncertainties in the
relevant formulae and thereby improves the determination of the CKM
parameters to be achieved in future experiments. We have illustrated
this with several examples.

In this review we have concentrated on weak decays in the Standard
Model.  The structure of weak decays in extensions of the Standard
Model will generally be modified. Although we do not expect substantial
effects due to "new physics" in tree level decays, the picture of loop
induced processes, such as rare and CP violating decays, may turn out
to be different from the one presented here. The basic structure of QCD
calculations will remain valid, however. In certain extensions of the
Standard Model, in which no new local operators occur, only the initial
conditions to the renormalization group evolution will have to be
modified. In more complicated extensions additional operators can be
present and in addition to the change of the initial conditions, also
the evolution matrix will have to be generalized.

Yet in order to be able to decide whether modifications of the standard
theory are required by the data, it is essential that the theoretical
calculations within the Standard Model itself reach the necessary
precision.  As far as the {\it short distance}
contributions are concerned, we think that in most cases such a
precision has been already achieved.

Important exceptions are the $b \to s \gamma$ and $b \to s g$
transitions for which the complete NLO corrections are not yet
available.  On the other hand the status of {\it long distance}
contributions represented by the hadronic matrix elements of local
operators or equivalently by various $B_i$ parameters, is much less
satisfactory. This is in particular the case of non--leptonic decays,
where the progress is very slow. Yet without these difficult
non--perturbative calculations it is impossible to give reliable
theoretical predictions for non-leptonic decays even if the Wilson
coefficients of the relevant operators have been calculated with high
precision. Moreover these coefficients have unphysical renormalization
scale and renormalization scheme dependences which can only be canceled
by the corresponding dependences in the hadronic matrix elements. All
efforts should be made to improve the status of non-perturbative
calculations.

The next ten years should be very exciting for the field of weak
decays.  The experimental efforts in several laboratories will provide
many new results for the rare and CP violating decays which will offer
new tests of the Standard Model and possibly signal some "new
physics".  As we have stressed in this review the NLO calculations
presented here will play undoubtedly an important role in these
investigations. Let us just imagine that $B_s^0-\bar B_s^0$ mixing and
the branching ratios for $K^+ \to \pi^+\nu\bar\nu$, $K_L\to \pi^0
\nu\bar\nu$, $B \to X_s \nu \bar\nu$ and $B_s \to \mu^+\mu^-$ have been
measured to an acceptable accuracy.  Having in addition at our disposal
accurate values of $|V_{ub}/V_{cb}|$, $|V_{cb}|$, $m_t$, $F_B$, $B_B$
and $B_K$ as well as respectable results for the angles
$(\alpha,\beta,\gamma)$ from the CP asymmetries in B--decays, we could
really get a great insight into the physics of quark mixing and CP
violation. One should hope that this progress on the experimental side
will be paralleled by the progress in the calculations of hadronic
matrix elements as well as by the calculations of QCD corrections in
potential extensions of the Standard Model.

We would like to end our review with a summary of theoretical
predictions and present experimental results for the rare and CP
violating decays discussed by us. This summary is given in table
\ref{tab:sumtab}.

\begin{table}[htb]
\caption[]{Summary of theoretical predictions and experimental results
for the rare and CP violating processes discussed in this review. The
entry ``input'' indicates that the corresponding measurement is used to
determine or to constrain CKM parameters needed for the calculation of
other decays. For $B(K_L\to\mu^+\mu^-)$ the theoretical value refers
only to the short-distance contribution.  In the case of $B(K_L \to
\pi^0 e^+e^-)$ the SM prediction corresponds to the contribution from
direct CP violation.
The SM predictions for $K^+\to\pi^+\nu\bar\nu$ and
$K_L\to\pi^0\nu\bar\nu$ include the isospin breaking corrections
considered in \cite{marcianoparsa:95}.
\label{tab:sumtab}}
\begin{center}
\begin{tabular}{|c|c|c|l|}
{\bf Quantity}  &  {\bf SM Prediction}  &  {\bf Experiment}  &
\phantom{XXX} {\bf Exp. Reference} \\
\hline \hline
\multicolumn{4}{|c|}{\bf K--Decays}\\
\hline
$|\varepsilon_K|$ & input & $(2.266\pm 0.023)\cdot 10^{-3}$
 & \cite{particledata:94} \\
\hline
$\varepsilon'/\varepsilon$ & $(5.6\pm 7.7)\cdot 10^{-4}$
 & $(15\pm 8)\cdot 10^{-4}$ & \cite{particledata:94} \\
\hline
$B(K_L\to\pi^0e^+e^-)$ & $(4.5 \pm 2.8) \cdot 10^{-12}$
[$\hbox{CP}_{\rm dir}$] & $<4.3\cdot 10^{-9}$
 & \cite{harrisetal:93} )\\
\hline
$B(K^+\to\pi^+\nu\bar\nu)$ & $(1.0\pm 0.4)\cdot 10^{-10}$
 & $<2.4\cdot 10^{-9}$ & \cite{adleretal:95} \\
\hline
$B(K_L\to\pi^0\nu\bar\nu)$ & $(2.9\pm 1.9)\cdot 10^{-11}$
 & $<5.8\cdot 10^{-5}$ & \cite{weaveretal:94} \\
\hline
$B(K_L\to\mu^+\mu^-)$ & $(1.3\pm 0.7)\cdot 10^{-9}$ [SD]
 & $(7.4\pm 0.4)\cdot 10^{-9}$ & \cite{particledata:94} \\
\hline
$|\Delta_{LR}(K^+\to\pi^+\mu^+\mu^-)|$ & $(6\pm 3)\cdot 10^{-3}$
 & --- & ---\\
\hline
\multicolumn{4}{|c|}{\bf B--Decays}\\
\hline
$x_d$ & input & $0.75\pm 0.06$ & \cite{browderhonscheid:95} \\
\hline
$B(B\to X_s\gamma)$ & $(2.8\pm
0.8)\cdot 10^{-4}$
 & $(2.32\pm 0.67)\cdot 10^{-4}$ & \cite{CLEO:94} \\
\hline
$B(B\to X_s\nu\bar\nu)$ & $(4.0\pm 0.9)\cdot 10^{-5}$
 & $< 3.9 \cdot 10^{-4}$ & \cite{grossmanetal:95} \\
\hline
$B(B_s\to\tau^+\tau^-)$ & $(1.1\pm 0.7)\cdot 10^{-6}$
 & --- & ---\\
\hline
$B(B_s\to\mu^+\mu^-)$ & $(5.1\pm 3.3)\cdot 10^{-9}$
 & $<8.4\cdot 10^{-6}$ & \cite{krolletal:95} \\
\hline
$B(B_s\to e^+e^-)$ & $(1.2\pm 0.8)\cdot 10^{-13}$
 & --- & ---\\
\hline
$B(B_d\to\mu^+\mu^-)$ & $\sim 10^{-10}$ & $<1.6\cdot 10^{-6}$ &
\cite{krolletal:95} \\
\hline
$B(B_d\to e^+e^-)$ & $\sim 10^{-14}$ & $<5.9\cdot 10^{-6}$ &
\cite{ammaretal:94} \\
\end{tabular}
\end{center}
\end{table}

Let us hope that the next ten years will bring a further reduction of
uncertainties in the theoretical predictions and will provide us with
accurate measurements of various branching ratios for which, as seen in
table \ref{tab:sumtab}, only  upper bounds are available at present.
