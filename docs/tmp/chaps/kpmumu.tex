\section{The Decays \klmm and $K^+\to\pi^+\mu^+\mu^-$}
\label{sec:KLmm}
\subsection{General Remarks on \klmm}
\label{sec:KLmm:GeneralKL}
The rare decay \klmm is CP conserving and in addition to its
short-distance part receives important contributions from the
two-photon intermediate state, which are difficult to calculate
reliably \cite{gengng:90}, \cite{belangergeng:91}, \cite{ko:92}.

This latter fact is rather unfortunate because the
short-distance part is, similarly to $\kpn$, free of hadronic
uncertainties and if extracted from the data would give a useful
determination of the Wolfenstein parameter $\varrho$. The separation
of the short-distance from the long-distance piece in the measured
rate is very difficult however.
\\
In spite of all this we will present here the analysis of the short-distance
contribution because on one hand it may turn out to be useful
one day for \klmm and on the other hand it also plays an important
role in a parity violating asymmetry, which can be measured in
$K^+\to\pi^+\mu^+\mu^-$. We will discuss this latter topic later on in
this section.
\\
The analysis of $(\klm)_{SD}$ proceeds in essentially the same
manner as for $\kpn$. The only difference enters through the lepton
line in the box contribution. This change introduces two new
functions $Y_{NL}$ and $Y(x_t)$ for the charm and top
contributions respectively (section \ref{sec:HeffRareKB:klmm}),
which will be discussed in detail below.

\subsection{Master Formulae for $(K_L \to \mu^+ \mu^-)_{\rm SD}$}
\label{sec:KLmm:MasterKL}
Using the effective hamiltonian (\ref{hklm}) and relating
$\langle 0|(\bar sd)_{V-A}|K_L\rangle$ to $B(K^+\to\mu^+\nu)$
we find
\begin{equation}\label{bklm}
B(\klm)_{SD}=\kappa_\mu\left[\frac{\relc}{\lambda}P_0(Y)+
\frac{\relt}{\lambda^5} Y(x_t)\right]^2
\end{equation}
\begin{equation}\label{kapm}
\kappa_\mu=\frac{\alpha^2 B(K^+\to\mu^+\nu)}{\pi^2\sin^4\Theta_W}
\frac{\tau(K_L)}{\tau(K^+)}\lambda^8=1.68\cdot 10^{-9}
\end{equation}
where we have used
\begin{equation}\label{klmpar}
\alpha=\frac{1}{129}\qquad \sin^2\Theta_W=0.23\qquad
B(K^+\to\mu^+\nu)=0.635
\end{equation}
The function $Y(x)$ of (\ref{yy}) can also be written as
\begin{equation}\label{yeta}
Y(x)=\eta_Y\cdot Y_0(x) \qquad\quad \eta_Y=1.026\pm 0.006
\end{equation}
where $\eta_Y$ summarizes the NLO corrections discussed in section
\ref{sec:HeffRareKB:klmm}. With $m_t\equiv m_t(m_t)$ this QCD factor
depends only very weakly on $m_t$. The range in (\ref{yeta})
corresponds to $150\gev\leq m_t\leq 190\gev$. The dependence on
$\Lambda_{\overline{MS}}$ can be neglected. Next
\begin{equation}\label{p0kldef}
P_0(Y)=\frac{Y_{NL}}{\lambda^4}
\end{equation}
with $Y_{NL}$ calculated in section \ref{sec:HeffRareKB:klmm}.
Values for $P_0(Y)$ as a function of $\Lambda_{\overline{MS}}$
and $m_c\equiv m_c(m_c)$ are collected in table \ref{tab:P0KL}.

\begin{table}[htb]
\caption[]{The function $P_0(Y)$ for various $\Lms^{(4)}$ and $m_c$.
\label{tab:P0KL}}
\begin{center}
\begin{tabular}{|c|c|c|c|}
&\multicolumn{3}{c|}{$P_0(Y) $}\\
\hline
$\Lms^{(4)}$ / $m_c$  & $1.25\gev$ & $1.30\gev$ & $1.35\gev$\\
\hline
$215\mev$ & 0.132 & 0.141 & 0.151 \\
$325\mev$ & 0.140 & 0.149 & 0.159 \\
$435\mev$ & 0.145 & 0.156 & 0.166
\end{tabular}
\end{center}
\end{table}

Using the improved Wolfenstein parametrization and the approximate
formulae (\ref{2.51}) -- (\ref{2.53}) we can next write
\begin{equation}\label{bklmnum}
B(\klm)_{SD}=1.68\cdot 10^{-9} A^4 Y^2(x_t)\frac{1}{\sigma}
\left(\bar\varrho_0-\bar\varrho\right)^2
\end{equation}
with
\begin{equation}\label{rhosig}
\bar\varrho_0=1+\frac{P_0(Y)}{A^2 Y(x_t)}\qquad
\sigma=\left(\frac{1}{1-\frac{\lambda^2}{2}}\right)^2
\end{equation}
The "experimental" value of $B(\klm)_{SD}$ determines the value of
$\bar\varrho$ given by
\begin{equation}\label{rhor0}
\bar\varrho=\bar\varrho_0-\bar r_0
\qquad\qquad
\bar r_0^2=\frac{1}{A^4 Y^2(x_t)}\left[
\frac{\sigma B(\klm)_{SD}}{1.68\cdot 10^{-9}}\right]
\end{equation}
Similarly to $r_0$ in the case of $\kpn$, the value of $\bar r_0$
is fully determined by the top contribution which has only a very
weak renormalization scale ambiguity after the inclusion of
$\ord(\as)$ corrections. The main scale ambiguity resides in
$\bar\varrho_0$ whose departure from unity measures the relative
importance of the charm contribution.

\subsection{Numerical Analysis of $(\klm)_{SD}$}
\label{sec:KLmm:NumericalKL}
\subsubsection{Renormalization Scale Uncertainties}
\label{sec:KLmm:NumericalKL:RSU}
We will now investigate the uncertainties in $Y(x_t)$, $Y_{NL}$,
$B(\klm)_{SD}$ and $\bar\varrho$ related to the dependence of
these quantities on the choice of the renormalization scales $\mu_t$
and $\mu_c$. To this end we proceed as in section
\ref{sec:Kpnn:NumericalKp:RSU}. We fix all the remaining parameters
as given in (\ref{mcmtnum}) and (\ref{vcbubnum}) and we vary
$\mu_c$ and $\mu_t$ within the ranges stated in (\ref{muctnum}).

\begin{figure}[hbt]
\vspace{0.10in}
\centerline{
\epsfysize=5in
\rotate[r]{
\epsffile{ps/mucY.ps}
} }
\vspace{0.08in}
\caption[]{
Charm quark function $Y_{NL}$ compared to the leading-log
result $Y_L$ and the case without QCD as functions of $\mu_c$.
\label{fig:kpmumu:YNL}}
\end{figure}

Fig.\ \ref{fig:kpmumu:YNL} shows the charm function $Y_{NL}$ compared
to the leading-log result $Y_L$ and the case without QCD as a function
of $\mu_c$.  We note the following points:
\begin{itemize}
\item
The residual slope of $Y_{NL}$ is considerably smaller than
in $Y_L$ although still sizable. The variation of $Y$ with $\mu$
defined as $(Y(1\gev)-Y(3\gev))/Y(m_c)$ is 53\% in NLLA compared
to 92\% in LLA.
\item
There is a strong enhancement of $Y_0$ through QCD corrections
in contrast to the suppression found in the case of $X_0$.
\end{itemize}

\begin{figure}[hbt]
\vspace{0.10in}
\centerline{
\epsfysize=5in
\rotate[r]{
\epsffile{ps/mutY.ps}
} }
\vspace{0.08in}
\caption[]{
Top quark function $Y(x_t)$ as a function of $\mu_t$ for fixed
$\mt(\mt)=170\gev$ with (solid curve) and without (dashed curve) $\ord(\as)$
corrections.
\label{fig:kpmumu:Y}}
\end{figure}

In fig.\ \ref{fig:kpmumu:Y} we show the analogous results for $Y(x_t)$
as a function of $\mu_t$. The observed features are similar to the ones
found in the case of $X(x_t)$:
\begin{itemize}
\item
Considerable reduction of the scale uncertainties in NLLA
relative to the LLA with a tiny residual uncertainty after the
inclusion of NLO corrections.
\item
Small NLO correction for the choice $\mu_t=m_t$ as summarized
by $\eta_Y$ in (\ref{yeta}).
\end{itemize}
Using (\ref{bklm}) and varying $\mu_{c,t}$ in the ranges (\ref{muctnum})
we find that for our choice of input parameters the uncertainty in
$B(\klm)_{SD}$
\begin{equation}\label{varbklmLO}
0.816\cdot 10^{-9}\leq B(\klm)_{SD}\leq 1.33\cdot 10^{-9}
\end{equation}
present in the leading order is reduced to
\begin{equation}\label{varbklmNLO}
1.02\cdot 10^{-9}\leq B(\klm)_{SD}\leq 1.25\cdot 10^{-9}
\end{equation}
after including NLO corrections. Here we have assumed $\bar\varrho=0$.
\\
Similarly we find
\begin{equation}\label{varrhbLO}
-0.117\leq \bar\varrho\leq 0.165  \qquad {\rm LLA}
\end{equation}
\begin{equation}\label{varrhbNLO}
0.011\leq \bar\varrho\leq 0.134  \qquad {\rm NLLA}
\end{equation}
where we have set $B(\klm)_{SD}=1\cdot 10^{-9}$.
We observe again a considerable reduction of the theoretical
error when the NLO effects are included in the analyses. Also in this
case the remaining ambiguity is largely dominated by the uncertainty
in the charm sector.

\subsubsection{Expectations for $B(\klm)_{SD}$}
\label{sec:KLmm:NumericalKL:EfB}
We finally quote the standard model expectation for the 
short-distance contribution to the \klmm branching ratio. Using the
analysis of $\varepsilon_K$ and the constraint implied by
$B_d-\bar B_d$ mixing in analogy to the case of \kpnn described in
section \ref{sec:Kpnn:NumericalKp:EfB}, we find
\begin{equation}\label{klmm1}
0.6\cdot 10^{-9}\leq B(\klm)_{SD}\leq 2.0\cdot 10^{-9}
\end{equation}
and
\begin{equation}\label{klmm2}
0.9\cdot 10^{-9}\leq B(\klm)_{SD}\leq 1.2\cdot 10^{-9}
\end{equation}
for present parameter uncertainties and our "future" scenario,
respectively. The relevant sets of input parameters and their errors
are collected in appendix \ref{app:numinput}.  Removing the $x_d$
constraint would increase the upper bounds in \eqn{klmm1} and
\eqn{klmm2} to $3.5 \cdot 10^{-9}$ and $2.2 \cdot 10^{-9}$,
respectively.

\subsection{General Remarks on $K^+\to\pi^+\mu^+\mu^-$}
\label{sec:KLmm:GeneralKp}
Obviously, the short distance effective hamiltonian in \eqn{hklm} also
gives rise to an amplitude for the transition $K^+\to\pi^+\mu^+\mu^-$.
This amplitude, however, is by three orders of magnitude smaller than
the dominant contribution to $K^+\to\pi^+\mu^+\mu^-$ given by the
one-photon exchange diagram \cite{eckeretal:87} and is therefore
negligible in the total decay rate. On the other hand the coupling to
the muon pair is purely vector-like for the one-photon amplitude,
whereas it contains an axial vector part in the case of the SD
contribution mediated by $Z^0$-penguin and W-box diagrams.  Thus, as
was pointed out by \cite{savagewise:90} and discussed in detail in
\cite{luetal:92}, the {\em interference\/} of the one-photon and the SD
contribution, which is odd under parity, generates a parity violating
longitudinal muon polarization asymmetry

\begin{equation}\label{delr}
\Delta_{LR}=\frac{\Gamma_R-\Gamma_L}{\Gamma_R+\Gamma_L}
\end{equation}
in the decay $K^+\to\pi^+\mu^+\mu^-$. Here $\Gamma_R$ ($\Gamma_L$)
denotes the rate of producing a right- (left-) handed $\mu^+$, that is
a $\mu^+$ with spin along (opposite to) its three-momentum direction.
In this way a measurement of the asymmetry $\Delta_{LR}$ could probe
the phenomenologically interesting short distance physics, which is not
visible in the total rate.

The $K^+\to\pi^+\gamma^\ast$ vertex is described by a form factor
$f(s)$ ($s$ being the invariant mass squared of the muon pair), that
determines the one-photon amplitude and hence the total rate of
$K^+\to\pi^+\mu^+\mu^-$, but also enters the asymmetry $\Delta_{LR}$.
This formfactor has been analyzed in detail in \cite{eckeretal:87}
within the framework of chiral perturbation theory. The imaginary part
$\IM f(s)$ turns out to be much smaller than ${\RE} f(s)$ and can
safely be neglected in the calculation of $\Delta_{LR}$. For this
reason $f(s)\approx {\RE} f(s)$, which depends on a constant not fixed
by chiral perturbation theory, may also be directly extracted from
experimental data on $K^+\to\pi^+e^+e^-$ \cite{alliegro:92}, sensitive
to $|f(s)|$. We follow \cite{luetal:92} in adopting this procedure.

The dominance of ${\RE} f(s)$ further implies that $\Delta_{LR}$
actually measures the real part of the short distance amplitude.  As
emphasized in \cite{belangeretal:93}, $\Delta_{LR}$ is therefore
closely related to the short distance part of $K_L\to\mu^+\mu^-$ and
could possibly yield useful information on this contribution, which is
difficult to extract from experimental results on $K_L\to\mu^+\mu^-$.
Like $(K_L\to\mu^+\mu^-)_{SD}$, $\Delta_{LR}$ is in particular a
measure of the Wolfenstein parameter $\varrho$.

The authors of \cite{luetal:92} have also considered potential long
distance contributions to $\Delta_{LR}$ originating from two-photon
exchange amplitudes. Unfortunately these are very difficult to
calculate in a reliable manner. The discussion in \cite{luetal:92}
indicates however, that they are likely to be much smaller than the
short distance contributions considered above. We will focus here on
the short distance part, keeping in mind the uncertainty due to
possible non-negligible long distance corrections.

One should stress that the short distance part by itself, although
calculable in a well defined perturbative framework, is not completely
free from theoretical uncertainty. The natural context to discuss this
issue is a next-to-leading order analysis, which for $\Delta_{LR}$ has
been presented in \cite{buchallaburas:94b}, generalizing the previous
leading log calculations \cite{savagewise:90}, \cite{luetal:92},
\cite{belangeretal:93}. We will summarize the results of
\cite{buchallaburas:94b} below.

We finally mention that other asymmetries in $K^+\to\pi^+\mu^+\mu^-$,
which are odd under time reversal and are also sensitive to short
distance contributions, have been discussed in the literature
\cite{savagewise:90}, \cite{luetal:92}, \cite{agrawaletal:91},
\cite{agrawaletal:92}. They involve both the $\mu^+$ and $\mu^-$
polarizations and are considerably more difficult to measure than
$\Delta_{LR}$. Possibilities for measuring the polarization of muons
from $K^+\to\pi^+\mu^+\mu^-$ in future experiments, based on studying
the angular distribution of $e^\pm$ from muon decay, are described in
\cite{kuno:92}.

\subsection{Master Formulae for $\Delta_{LR}$}
\label{sec:KLmm:MasterDeLR}
The absolute value of the asymmetry $\Delta_{LR}$ can be written as
\begin{equation}\label{drxi}
|\Delta_{LR}|=r\cdot |{\RE}\xi|
\end{equation}
The factor $r$ arises from phase space integrations. It depends only on
the particle masses $m_K$, $m_\pi$ and $m_\mu$, on the form factors of
the matrix element $\langle\pi^+\mid(\bar sd)_{V-A}\mid K^+\rangle$, as
well as on the form factor of the $K^+\to\pi^+\gamma^\ast$ transition,
relevant for the one-photon amplitude. In addition $r$ depends on a
possible cut which may be imposed on $\theta$, the angle between the
three-momenta of the $\mu^-$ and the pion in the rest frame of the
$\mu^+\mu^-$ pair.  Without any cuts one has $r=2.3$ \cite{luetal:92}. If
$\cos\theta$ is restricted to lie in the region $-0.5\leq\cos\theta\leq
1.0 $, this factor is increased to $r=4.1$.  As discussed in
\cite{luetal:92}, such a cut in $\cos\theta$ could be useful since it
enhances $\Delta_{LR}$ by 80\% with only a 22\% decrease in the total
number of events.

${\RE}\xi$ is a function containing the information on the
short distance physics. It depends on CKM parameters, the QCD scale
$\Lambda_{\overline{MS}}$, the quark masses $m_t$ and $m_c$ and is given by
\begin{equation}
\label{rexi}
{\RE}\xi=
\kappa\cdot\left[\frac{\relc}{\lambda}P_0(Y)+\frac{\relt}{\lambda^5}Y(x_t)
\right]            
\end{equation}
\begin{equation}\label{kap}
\kappa=\frac{\lambda^4}{2\pi\sin^2\Theta_W(1-\frac{\lambda^2}{2})}
 =1.66\cdot 10^{-3}   
\end{equation}
Here $\lambda=|V_{us}|=0.22$, $\sin^2\Theta_W=0.23$,
$x_t=m^2_t/M^2_W$, $\lambda_i=V^\ast_{is}V_{id}$ and
\begin{equation}\label{p0ynl}
P_0(Y)=\frac{Y_{NL}}{\lambda^4}
\end{equation}
The functions $Y_{NL}$ and $Y(x_t)$ represent the charm and the top
contribution, respectively. They are to next-to-leading logarithmic
accuracy given in \eqn{ynl} and \eqn{yy} and have already been
discussed in chapter \ref{sec:HeffRareKB:klmm} and in the previous
sections on the phenomenology of $(K_L\to\mu^+\mu^-)_{SD}$. Numerical
values for $P_0(Y)$ can be found in table \ref{tab:P0KL}.
From (\ref{drxi}) and (\ref{rexi}) we can obtain $\relt$ expressed as a
function of $|\Delta_{LR}|$
\begin{equation}\label{rltdlr}
\relt=-\lambda^5\frac{{|\Delta_{LR}|}/{r \kappa}-
  \left(1-\frac{\lambda^2}{2}\right) P_0(Y)}{Y(x_t)}  
\end{equation}
Since $\relt$ is related to the Wolfenstein parameter $\bar\varrho$
(see section \ref{sec:sewm}), one may use (\ref{rltdlr}) to extract
$\bar\varrho$ from a given value of $|\Delta_{LR}|$.

\subsection{Numerical Analysis of $\Delta_{LR}$}
\label{sec:KLmm:NumericalDeLR}
To illustrate the phenomenological implications of the next-to-leading
order calculation, let us consider the following scenario. We assume a
typical value for $\Delta_{LR}$, allowing for an uncertainty of $\pm
10\%$
\begin{equation}\label{delrnum}
\Delta_{LR}=(6.0\pm 0.6)\cdot 10^{-3}
\end{equation}
Here a cut on $\cos\theta$, $-0.5\leq\cos\theta\leq 1.0$,
is understood.
Next we take ($m_i\equiv\bar m_i(m_i)$)
\begin{equation}\label{mtcV}
m_t=(170\pm 5)\gev\quad m_c=(1.30\pm 0.05)\gev\quad
V_{cb}=0.040\pm 0.001  
\end{equation}
\begin{equation}\label{lams}
\Lms^{(4)}=(300 \pm 50)\mev   
\end{equation}
Table \ref{tab:rhoDeLR} shows the central value of $\bar\varrho$ that
is extracted from $\Delta_{LR}$ in our example together with the
uncertainties associated to the relevant input. Combined errors due to
a simultaneous variation of several parameters can be obtained to a
good approximation by simply adding the errors in table \ref{tab:rhoDeLR}.

\begin{table}[htb]
\caption[]{$\bar\varrho$ determined from $\Delta_{LR}$ for the scenario
described in the text together with the uncertainties related to
various input parameters.
\label{tab:rhoDeLR}}
\begin{center}
\begin{tabular}{|c||c||c|c|c|c|c|}
&&$\Delta(\Delta_{LR})$&$\Delta(m_t)$
&$\Delta(V_{cb})$&$\Delta(m_c)$&$\Delta(\Lambda_{\overline{MS}})$
\\ \hline
$\bar\varrho$&$-0.06$&$\pm 0.13$&$\pm 0.05$&$\pm 0.06$&$\pm 0.01$&
$\pm 0.00$
\end{tabular}
\end{center}
\end{table}

These errors should be compared with the purely theoretical uncertainty
of the short distance calculation, estimated by a variation of the
renormalization scales $\mu_c$ and $\mu_t$. Varying these scales as given
in \eqn{muctnum} and keeping all other parameters at their central
values we find
\begin{equation}\label{rhor}
-0.15\leq\bar\varrho\leq -0.03 \qquad {\rm (NLLA)}   
\end{equation}
\begin{equation}\label{rhor0num}
-0.31\leq\bar\varrho\leq 0.02 \qquad {\rm (LLA)}   
\end{equation}
We observe that at NLO the scale ambiguity is reduced by almost a
factor of 3 compared to the leading log approximation. However, even in
the NLLA the remaining uncertainty is still sizable, though moderate in
comparison with the errors in table \ref{tab:rhoDeLR}.  Note that the
remaining error in (\ref{rhor}) is almost completely due to the charm
sector, since the scale uncertainty in the top contribution is
practically eliminated at NLO.
\\
We remark that for definiteness we have incorporated the numerically 
important piece $x_c/2$ in the leading log expression for the
charm function $Y$, although this is strictly speaking a next-to-leading
order term. This procedure corresponds to a central value of
$\bar\varrho=-0.12$ in LLA. Omitting the $x_c/2$ term and employing the
strict leading log result shifts this value to $\bar\varrho=-0.20$.
Within NLLA this ambiguity is avoided in a natural way.
\\
Finally we give the Standard Model expectation for $\Delta_{LR}$,
based on the short distance contribution in (\ref{drxi}), for the
Wolfenstein parameter $\varrho$ in the range
$-0.25\leq\varrho\leq 0.25$, $V_{cb}=0.040\pm 0.004$ and
$m_t=(170\pm 20)\gev$. Including the uncertainties due to
$m_c$, $\Lambda_{\overline{MS}}$, $\mu_c$ and $\mu_t$ and
imposing the cut $-0.5\leq\cos\theta\leq 1$, we find
\begin{equation}\label{dlr1}
3.0\cdot 10^{-3}\leq |\Delta_{LR}|\leq 9.6\cdot 10^{-3}   
\end{equation}
employing next-to-leading order formulae.
Anticipating improvements in $V_{cb}$, $m_t$ and $\varrho$ we also
consider a future scenario in which
$\varrho=0.00\pm 0.02$, $V_{cb}=0.040\pm 0.001$ and
$m_t=(170\pm 5)\gev$. The very precise determination of $\varrho$
used here should be achieved through measuring CP asymmetries
in B decays in the LHC era \cite{buras:94b}. Then (\ref{dlr1}) reduces to
\begin{equation}\label{dlr2}
4.8\cdot 10^{-3}\leq |\Delta_{LR}|\leq 6.6\cdot 10^{-3}   
\end{equation}
\\
One should mention that although the top contribution dominates the
short distance prediction for $|\Delta_{LR}|$, the charm part is still
important and should not be neglected, as done in
\cite{belangeretal:93}.  It is easy to convince oneself that the charm
sector contributes to $\bar\varrho$ the sizable amount
$\Delta\bar\varrho_{charm}\approx 0.2$.  Furthermore, as we have shown
above, the charm part is the dominant source of theoretical uncertainty
in the short distance calculation of $\Delta_{LR}$.

To summarize, we have seen that the scale ambiguity in the
perturbative short distance contribution to $\Delta_{LR}$ can be
considerably reduced by incorporating next-to-leading order QCD
corrections. The corresponding theoretical error in the
determination of $\bar\varrho$ from an anticipated measurement
of $|\Delta_{LR}|$ is then decreased by a factor of 3, in a typical
example. Unfortunately the remaining scale uncertainty is quite
visible even at NLO. In addition there are further uncertainties due to
various input parameters and due to possible long distance effects.
Together this implies that the accuracy to which $\bar\varrho$ can be
extracted from $\Delta_{LR}$ appears to be limited and $\Delta_{LR}$
can not fully compete with the "gold-plated" $K\to\pi\nu\bar\nu$
decay modes. Still, a measurement of $\Delta_{LR}$ might give
interesting constraints on SM parameters, $\bar\varrho$ in particular,
and we feel it is worthwhile to further pursue this interesting
additional possibility.
