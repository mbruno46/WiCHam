\section{Inclusive B Decays}
\label{sec:InclB}
\subsection{General Remarks}
\label{sec:InclB:General}
Inclusive decays of $B$ mesons constitute an important testing ground
for our understanding of strong interaction dynamics in its interplay
with the weak forces. At the same time inclusive semileptonic modes
provide useful information on $|V_{cb}|$.
\\
Due to quark-hadron duality inclusive decays of heavy mesons can, in
general, be calculated more reliably than corresponding exclusive
modes. During recent years a systematic formulation for the treatment
of inclusive heavy meson decays has been developed. It is based on
operator product and heavy quark expansion, which are applied to the
$B$ meson inclusive width, expressed as the absorptive part of the $B$
forward scattering amplitude
\begin{equation}\label{gabx}
\Gamma(B\to X)=\frac{1}{2m_B}{\IM}\left(i \int d^4x
 \langle B | T \, {\cal H}^{(X)}_{eff}(x){\cal H}^{(X)}_{eff}(0) | B
\rangle\right)
\end{equation}
Here ${\cal H}^{(X)}_{eff}$ is the part of the complete $\Delta B=1$
effective hamiltonian that contributes to the particular inclusive
final state $X$ under consideration. E.g.\ for inclusive semileptonic decays
\begin{equation}\label{hbsl}
{\cal H}^{(SL)}_{eff,\Delta B=1}=\frac{G_F}{\sqrt{2}} V_{cb}
(\bar cb)_{V-A} \sum_{l=e,\mu,\tau}(\bar l\nu_l)_{V-A}+ h.c.
\end{equation}
For nonleptonic modes the relevant expression is the $\Delta B=1$ short
distance effective hamiltonian given in \eqn{eq:HeffdB1:66}.  It has
been shown in \cite{Chay}, \cite{Bj}, \cite{bigietal:92},
\cite{bigietal:93}, \cite{manoharwise:94}, \cite{bloketal:94},
\cite{falketal:94}, \cite{mannel:94}, \cite{Bigi}, that the leading
term in a systematic expansion of (\ref{gabx}) in $1/m_b$ is determined
by the decay width of a free b-quark calculated in the parton picture.
Furthermore, the nonperturbative corrections to this perturbative
result start at order $(\Lambda/m_b)^2$, where $\Lambda$ is a hadronic
scale $\sim 1\gev$, and are quite small in the case of B decays.  In
the light of this formulation it becomes apparent that the
perturbative, partonic description of heavy hadron decay is thus
promoted from the status of a model calculation to the leading
contribution in a systematic expansion based on QCD.  We will still
comment on the $(\Lambda/m_b)^2$ corrections below.  In the following
we will however concentrate on the leading quark level analysis of
inclusive $B$ decays. As we shall see, the treatment of short-distance
QCD effects at the next-to-leading order level -- at least for the
dominant modes -- is of crucial importance for a proper understanding
of these processes.
\\
The calculation of b-quark decay starts from the effective 
$\Delta B=1$ hamiltonian containing the relevant four-fermion
operators multiplied by Wilson coefficients. To obtain the decay rate,
the matrix elements (squared) of these operators have to be calculated
perturbatively to the required order in $\as$. While in LLA a
zeroth order evaluation is sufficient, ${\cal O}(\as)$ virtual
gluon effects (along with real gluon bremsstrahlung contributions for
the proper cancellation of infrared divergences in the inclusive rate)
have to be taken into account at NLO. In this way the renormalization
scale and scheme dependence present in the coefficient functions is
canceled to the considered order (${\cal  O}(\as)$) in the
decay rate. Thus, by contrast to low energy decays, in the case of
inclusive heavy quark decay, a physical final result can be obtained
within perturbation theory alone.
\\
Our goal will be in particular to review the present status of the
theoretical prediction for the $B$ meson semileptonic branching ratio
$B_{SL}$. This quantity has received some attention in recent years
since theoretical calculations \cite{altarellipetrarca:91},
\cite{tanimoto:92}, \cite{palmerstech:93}, \cite{bigietal:94},
\cite{falketal:94b} tended to yield values around $12.5-13.5\%$, above
the experimental figure $B_{SL}=(10.4\pm 0.4)\%$
\cite{particledata:94}.  However, these earlier analyses have not been
complete in regard to the inclusion of final state mass effects and NLO
QCD corrections in the nonleptonic widths. More precisely, these
calculations took into account mass effects appropriate for the leading
order in QCD along with NLO QCD corrections obtained for massless final
state quarks.  Recently the most important of these -- so far lacking
-- mass effects have been properly included in the NLO QCD calculation
through the work of \cite{baganetal:94a}, \cite{baganetal:94b},
\cite{baganetal:95}. These ${\cal O}(\as)$ mass effects tend to
decrease $B_{SL}$ and, according to the analysis of these authors
essentially bring it, within theoretical uncertainties, into agreement
with the experimental number.  Before further discussing these issues,
it is appropriate to start with a short overview summarizing the
possible b-quark decay modes and classifying their relative
importance.

\subsection{b-Quark Decay Modes}
\label{sec:InclB:bdecay}
First of all, a b-quark can decay {\it semileptonically\/} to the
final states $cl\bar\nu_l$ and $ul\bar\nu_l$ with $l=e$, $\mu$, $\tau$.
\\
In the case of nonleptonic final states we may distinguish three classes:
Decays induced through current-current operators alone (class I),
decays induced by both current-current and penguin operators (class II)
and pure penguin transitions (class III). We have

\begin{center}
\begin{tabular}{|c|l|}
\hline
Class & \multicolumn{1}{c|}{Final State} \\
\hline
I    & $c\bar ud$, \quad $c\bar us$; \qquad  $u\bar cs$, \quad $u\bar cd$ \\
II   & $c\bar cs$, \quad $c\bar cd$; \qquad  $u\bar ud$, \quad $u\bar us$ \\
III  & $d\bar dd$, \quad $d\bar ds$; \qquad  $s\bar sd$, \quad $s\bar ss$ \\
\hline
\end{tabular}
\end{center}

Clearly there is a rich structure of possible decay modes even at the
quark level and a complete treatment would be quite complicated.
However, not all of these final states are equally important. In order
to perform the analysis of b-quark decay, in particular in view of the
calculation of $B_{SL}$, it is useful to identify the most important
channels and to introduce appropriate approximations in dealing
with less prominent decays. To organize the procedure, we make the
following observations:
\begin{itemize}
\item
The dominant, i.e. CKM allowed and tree-level induced, decays are
$b\to cl\nu$, $b\to c\bar ud$ and $b\to c\bar cs$. For these a complete
NLO calculation including final state mass effects is necessary.
\item
The channels $c\bar us$, $c\bar cd$, $u\bar cd$,
$u\bar us$ may be incorporated with excellent accuracy into the modes
$c\bar ud$, $c\bar cs$, $u\bar cs$, $u\bar ud$, respectively, using
the approximate CKM unitarity in the first two generations.
The error introduced thereby through the $s$-$d$ mass difference is
entirely negligible.
\item
Penguin transitions are generally suppressed by the smallness of their
Wilson coefficient functions, which are typically of the order of
a few percent. For this reason, one may neglect the pure penguin
decays of class III altogether as their decay rates involve
penguin coefficients squared.
\item 
Furthermore we may neglect the penguin contributions to the CKM
suppressed $b\to u$ transitions of class II.
\item
In addition one may treat the remaining smaller effects, namely $b\to u$
transitions and the interference of penguins with the leading
current-current contribution in $b\to c\bar cs$ within the
leading log approximation.
\item
Finally, rare, flavor-changing neutral current b-decay modes are
negligible in the present context as well.
\end{itemize}
Next we will write down expressions for the relevant decay rate
contributions we have discussed.
\\
For the dominant modes $b\to cl\nu$, $b\to c\bar ud$ and
$b\to c\bar cs$ (without penguin effects) one has at
next-to-leading order:
\begin{equation}\label{bcln}
\Gamma(b\to cl\nu)=\Gamma_0 P(x_c,x_l,0)\left[
 1+\frac{2\as(\mu)}{3\pi} g(x_c,x_l,0)\right]
\end{equation}
\begin{eqnarray}\label{bcud}
\Gamma(b\to c\bar ud)&=&\Gamma_0 P(x_c,0,0)\left[2L^2_++L^2_-+
 \frac{\as(M_W)-\as(\mu)}{2\pi}(2L^2_+ R_++L^2_- R_-)\right.
\nonumber\\
 &+&\frac{2\as(\mu)}{3\pi}\left(\frac{3}{4}(L_+-L_-)^2 
g_{11}(x_c)+\frac{3}{4}(L_++L_-)^2g_{22}(x_c)\right.
\nonumber\\
&&+\left.\left.\frac{1}{2}
(L^2_+-L^2_-)(g_{12}(x_c)-12 \ln\frac{\mu}{m_b})\right)\right]
\end{eqnarray}
\begin{eqnarray}\label{bccs}
\Gamma(b\to c\bar cs)&=&\Gamma_0 P(x_c,x_c,x_s)\left[2L^2_++L^2_-+
 \frac{\as(M_W)-\as(\mu)}{2\pi}(2L^2_+ R_++L^2_- R_-)\right.
\nonumber\\
 &+&\frac{2\as(\mu)}{3\pi}\left(\frac{3}{4}(L_+-L_-)^2 
h_{11}(x_c)+\frac{3}{4}(L_++L_-)^2h_{22}(x_c)\right.
\nonumber\\
&&+\left.\left.\frac{1}{2}
(L^2_+-L^2_-)(h_{12}(x_c)-12 \ln\frac{\mu}{m_b})\right)\right]
\end{eqnarray}
Eq.\ \eqn{bccs} neglects small strange quark mass effects in the NLO
terms, which have however been included in the numerical analysis in
\cite{baganetal:95}.
In the equations above $\Gamma_0=G^2_Fm^5_b|V_{cb}|^2/(192\pi^3)$ and
$P(x_1,x_2,x_3)$ is the leading order phase space factor given for
arbitrary masses $x_i=m_i/m_b$ by
\begin{equation}\label{px123}
P(x_1,x_2,x_3)=12\int\limits_{(x_2+x_3)^2}^{(1-x_1)^2} \frac{ds}{s}
(s-x^2_2-x^2_3)(1+x^2_1-s) w(s,x^2_2,x^2_3) w(s,x^2_1,1)
\end{equation}
\begin{equation}\label{wabc}
w(a,b,c)=(a^2+b^2+c^2-2ab-2ac-2bc)^{1/2}
\end{equation}
$P$ is a completely symmetric function of its arguments.
\\
Furthermore
\begin{equation}\label{lpmmu}
L_\pm=L_\pm(\mu)=\left[\frac{\as(M_W)}{\as(\mu)}
\right]^{d_\pm}
\end{equation}
with $d_+=6/23$, $d_-=-12/23$ (see (\ref{B10})) and $\mu={\cal
O}(m_b)$.  The scheme independent $R_\pm$ come from the NLO
renormalization group evolution and are given by $R_\pm=B_\pm-J_\pm$
(see (\ref{B9})).  For $f=5$ flavors $R_+=6473/3174$,
$R_-=-9371/1587$.  Note that the leading dependence of $L_\pm$ on the
renormalization scale $\mu$ is canceled to ${\cal O}(\as)$ by the
explicit $\mu$-dependence in the $\as$-correction terms.  Virtual
gluon and bremsstrahlung corrections to the matrix elements of four
fermion operators are contained in the mass dependent functions $g$,
$g_{ij}$ and $h_{ij}$.
\\
The function $g(x_1,x_2,x_3)$ is available for arbitrary $x_1$, $x_2$,
$x_3$ from \cite{hokimpham:83}, \cite{hokimpham:84}. The special case
$g(x_1,0,0)$ has been analysed also in \cite{CM:78}. Analytical
expressions have been given in \cite{nir:89} for $g(x_1,0,0)$ and in
\cite{baganetal:94a} for $g(0,x_2,0)$.  The functions $g_{11}(x)$,
$g_{12}(x)$ and $g_{22}(x)$ are calculated analytically in
\cite{baganetal:94a}. Furthermore, as discussed in
\cite{baganetal:94a}, $h_{11}(x)$ and $h_{22}(x)$ can be obtained from
the work of \cite{hokimpham:83}, \cite{hokimpham:84}. Finally,
$h_{12}(x)$ has been determined in \cite{baganetal:95}.
For the full mass dependence of these functions we refer the
reader to the cited literature. Here we quote the results obtained in
the massless limit.  These have been computed in \cite{altarelli:81},
\cite{buchalla:93} for $g_{ij}$, $h_{ij}$ ($g_{ij}(0)=h_{ij}(0)$)
\begin{equation}\label{ghij0}
g_{11}(0)=g_{22}(0)=\frac{31}{4}-\pi^2
\qquad g_{12}(0)=g_{11}(0)-\frac{19}{2}
\end{equation}
Furthermore
\begin{equation}\label{g000}
g(0,0,0)=\frac{25}{4}-\pi^2
\end{equation}
In table \ref{tab:bNLOnum} we have listed some typical numbers
extracted from \cite{baganetal:94b}, \cite{baganetal:95} illustrating
the impact of charm mass effects (for $x_c=0.3$) in the NLO correction
terms by giving the enhancment factor of the NLO over the LO results.
There are of course various ambiguities involved in this comparison.
The numbers in table \ref{tab:bNLOnum} are therefore merely intended to
show the general trend.  Note the sizable enhancement through NLO mass
effects in the nonleptonic channels, in particular $b\to c\bar cs$. A
large QCD enhancement in the latter case has also been reported in
\cite{voloshin:94}.

\begin{table}[htb]
\caption[]{Typical values for the ratio of NLO to LO results for dominant
b-decay channels with (I) and without (II) including finite charm mass
effects in the NLO correction terms. The leading order final state mass
effects (through the function $P$) are taken into account in all
cases.
\label{tab:bNLOnum}}
\begin{center}
\begin{tabular}{|c||c|c|c|c|}
 & $b\to ce\nu$ & $b\to c\tau\nu$ & $b\to c\bar ud$ & $b\to c\bar cs$\\
\hline
I & 0.85 & 0.88 & 1.06 & 1.32 \\
\hline
II & 0.79 & 0.80 & 1.01 & 1.02
\end{tabular}
\end{center}
\end{table}

To complete the presentation of b decay modes we next write down
expressions for the CKM suppressed channels $b \to u l \nu$, $b\to
u\bar cs$ and $b\to u\bar ud$ (without penguins) as well as the
contribution to the $b\to c\bar cs$ rate due to interference of the
leading, current-current type transitions with penguin operators.
Restricting ourselves to the LLA for these small contributions we
obtain
\begin{equation}\label{buln}
\Gamma(b\to u\sum_l l\nu)=\Gamma_0 |\frac{V_{ub}}{V_{cb}}|^2
\sum_l P(0,x_l,0)
\end{equation} 
\begin{equation}\label{bucs}
\Gamma(b\to u\bar cs)=\Gamma_0 |\frac{V_{ub}}{V_{cb}}|^2
 P(0,x_c,x_s) \left[2L^2_++L^2_-\right]
\end{equation} 
\begin{equation}\label{buud}
\Gamma(b\to u\bar ud)=\Gamma_0 |\frac{V_{ub}}{V_{cb}}|^2
  \left[2L^2_++L^2_-\right]
\end{equation} 
\begin{eqnarray}\label{pbccs}
\Delta\Gamma_{penguin}(b\to c\bar cs) &=& 6\Gamma_0P(x_c,x_c,x_s)
\left[c_1\left(c_3+\frac{1}{3}c_4+F(c_5+\frac{1}{3}c_6)\right)\right.
\nonumber \\
&&+\left.c_2\left(\frac{1}{3}c_3+c_4+F(\frac{1}{3}c_5+c_6)\right)\right]
\end{eqnarray}
where $c_1, \ldots, c_6$ are the leading order Wilson coefficients and
\begin{equation}\label{fpeng}
F=\frac{6x^2_c}{P(x_c,x_c,x_s)}\int\limits_{(x_c+x_s)^2}^{(1-x_c)^2}
\frac{ds}{s^2}(s+x^2_s-x^2_c)(1+s-x^2_c) w(s,x^2_c,x^2_s) w(1,s,x^2_c)
\end{equation}
Numerically we have for $|V_{ub}/V_{cb}|=0.1$
\begin{equation}\label{bupnum1}
\Gamma(b\to u\sum_l l\nu)\approx 0.024 \Gamma_0 \qquad
\Gamma(b\to u\bar cs)\approx 0.017 \Gamma_0
\end{equation}
\begin{equation}\label{bupnum2}
\Gamma(b\to u\bar ud)\approx 0.034 \Gamma_0 \qquad
\Delta\Gamma_{penguin}(b\to c\bar cs)\approx -0.041 \Gamma_0
\end{equation}
Note that the contribution due to the interference with penguin
transitions in $b\to c\bar cs$ is negative. Hence, in addition to 
being small the effects in (\ref{bupnum1}) and (\ref{bupnum2}) tend
to cancel each other in the total nonleptonic width.

Finally one may also incorporate nonperturbative corrections. These
have been derived in \cite{bigietal:92} and are also discussed in
\cite{baganetal:94a}. As mentioned above, nonperturbative effects are
suppressed by two powers of the heavy b-quark mass and amount typically
to a few percent. For details we refer the reader to the cited
articles.

\subsection{The B Meson Semileptonic Branching Ratio}
\label{sec:InclB:BSL}
An important application of the results described in the
previous section is the theoretical prediction for the inclusive
semileptonic branching ratio of $B$ mesons
\begin{equation}\label{bsldef}
B_{SL}=\frac{\Gamma(B\to Xe\nu)}{\Gamma_{tot}(B)}
\end{equation}
On the parton level $\Gamma(B\to Xe\nu)\simeq\Gamma(b\to ce\nu)$
and
\begin{equation}\label{gtotb}
\Gamma_{tot}(B)\simeq \sum_{l=e,\mu,\tau}\Gamma(b\to cl\nu)+
\Gamma(b\to c\bar ud)+\Gamma(b\to c\bar cs)+
\Delta\Gamma_{penguin}(b\to c\bar cs)+\Gamma(b\to u)
\end{equation}
Here we have applied the approximations discussed above. $\Gamma(b\to u)$
summarizes the $b\to u$ transitions.
\\
Based on a similar treatment of the partonic rates, including in
particular next-to-leading QCD corrections for the dominant channels
and also incorporating nonperturbative corrections, the authors of
\cite{baganetal:94b}, \cite{baganetal:95} have carried out an analysis
of $B_{SL}$ and estimated the theoretical uncertainties. They obtain
\cite{baganetal:95}
\begin{equation}\label{bslnum}
B_{SL} = (12.0 \pm 1.4)\%
\qquad \mbox{and} \qquad
B_{SL} = (11.2 \pm 1.7)\%
\end{equation}
using pole and $\overline{MS}$ masses, respectively. The error is
dominated in both cases by the renormalization scale uncertainty ($\mb/2
< \mu < 2 \mb$). Note also the sizable scheme ambiguity.
\\
Within existing uncertainties, the theoretical prediction does not disagree
significantly with the experimental value $B_{SL,exp}=(10.4\pm 0.4)\%$
\cite{particledata:94}, although it seems to lie still somewhat on the
high side.
\\
It is amusing to note, that the naive mode counting estimate for
$B_{SL}$, neglecting QCD and final state mass effects completely,
yields $B_{SL}=1/9=11.1\%$ in (almost) "perfect agreement" with
experiment. Including the final state masses, still neglecting QCD,
enhances this number to $B_{SL}=15.8\%$. Incorporating in addition QCD
effects at the leading log level {\it increases\/} the hadronic modes,
thus leading to a {\it decrease\/} in $B_{SL}$, resulting typically in
$B_{SL}=14.7\%$. A substantial further decrease is finally brought
about through the NLO QCD corrections, which both further enhance
hadronic channels, in particular $b\to c\bar cs$, and simultaneously
reduce $b\to ce\nu$. As pointed out in \cite{baganetal:94b},
\cite{baganetal:95} and illustrated in table \ref{tab:bNLOnum} final
state mass effects in the NLO correction terms play a nonnegligile role
for this enhancement of hadronic decays. The nonperturbative effects
also lead to a slight decrease of $B_{SL}$.
\\
In short, leading final state mass effects and QCD corrections, acting
in opposite directions on $B_{SL}$, tend to cancel each other, 
resulting in a number for $B_{SL}$ not too different from the
simple modecounting guess.

We finally mention that, besides a calculation of $B_{SL}$, the
partonic treatment of heavy meson decay has further important
applications, such as the determination of $|V_{cb}|$ from inclusive
semileptonic $B$ decay, $B\to X_ce\nu$. Analyses of this type have been
presented in \cite{lukesavage:94}, \cite{bigiuraltsev:94},
\cite{ballnierste:94}, \cite{shifmanetal:95}.

Exact results beyond the presently known NLO accuracy seem extremely
difficult to obtain, even for relatively simple quantities like the
semileptonic b-quark decay rate.  There exist however calculations in
the literature devoted to the investigation of these higher order
perturbative effects. Due to the severe technical difficulties, those
calculations require additional assumptions. For instance, in an
interesting study \cite{braunetal:95} have investigated the effects of
the running of $\as$ on the semileptonic b-quark decay rate to all
orders in perturbation theory. This calculation is equivalent to a
resummation of all terms of the form $\as(\beta_0\as)^n$, which are
related to one-gluon exchange diagrams containing an arbitrary number
$n$ of fermion bubbles. The work of \cite{braunetal:95} applies the
renormalon techniques developped in \cite{benekebraun:95},
\cite{balletal:95} and generalizes the ${\cal O}(\beta_0\alpha^2_s)$
results computed in \cite{lukeetal:95}. The underlying idea is similar
in spirit to the BLM approach \cite{brodskyetal:83}.  An important
application of the result is the extraction of $|V_{cb}|$
\cite{braunetal:95}.  The formalism has also been used to study higher
order QCD corrections to the $\tau$ lepton hadronic width
\cite{balletal:95}.  Irrespective of the ultimate reliability of the
approximation, these investigations are useful from a conceptual point
of view as they help to illustrate important features of the higher
order behavior of the perturbative expansion.

In principle the discussion we have given for
b-decays may of course, with appropriate modifications, be applied to
the case of charm as well.  However here the nonperturbative
corrections to the parton picture, which scale like $1/m^2_Q$ with the
heavy quark mass $m_Q$, are by an order of magnitude larger than for B
mesons and accurate theoretical predictions are much more difficult to
obtain \cite{blokshifman:93}.
