\section{Guide to Effective Hamiltonians}
         \label{sec:Heffguide}
In order to facilitate the presentation of effective
hamiltonians in weak decays we give a complete compilation of the
relevant operators below. Divided into six classes, these 
operators play a dominant role in the phenomenology of weak decays.
The six classes are given as follows

\medskip

\leftline{\bf Current-Current Operators (fig.\ \ref{fig:oporig}\,(a)):}

\begin{eqnarray}
Q_{1} = \left( \bar s_{i} u_{j}  \right)_{\rm V-A}
            \left( \bar u_{j}  d_{i} \right)_{\rm V-A}
&\qquad&
Q_{2} = \left( \bar s u \right)_{\rm V-A}
            \left( \bar u d \right)_{\rm V-A}
\label{eq:Q12}
\end{eqnarray}

\leftline{\bf QCD-Penguins Operators (fig.\ \ref{fig:oporig}\,(b)):}

\begin{eqnarray}
Q_{3} = \left( \bar s d \right)_{\rm V-A}
   \sum_{q} \left( \bar q q \right)_{\rm V-A}
&\qquad&
Q_{4} = \left( \bar s_{i} d_{j}  \right)_{\rm V-A}
   \sum_{q} \left( \bar q_{j}  q_{i} \right)_{\rm V-A}
\label{eq:Q34} \\
Q_{5} = \left( \bar s d \right)_{\rm V-A}
   \sum_{q} \left( \bar q q \right)_{\rm V+A}
&\qquad&
Q_{6} = \left( \bar s_{i} d_{j}  \right)_{\rm V-A}
   \sum_{q} \left( \bar q_{j}  q_{i} \right)_{\rm V+A}
\label{eq:Q56}
\end{eqnarray}

\leftline{\bf Electroweak-Penguins Operators (fig.\ \ref{fig:oporig}\,(c)):}

\begin{eqnarray}
Q_{7} = \frac{3}{2} \left( \bar s d \right)_{\rm V-A}
         \sum_{q} e_{q} \left( \bar q q \right)_{\rm V+A}
&\qquad&
Q_{8} = \frac{3}{2} \left( \bar s_{i} d_{j} \right)_{\rm V-A}
         \sum_{q} e_{q} \left( \bar q_{j}  q_{i}\right)_{\rm V+A}
\label{eq:Q78} \\
Q_{9} = \frac{3}{2} \left( \bar s d \right)_{\rm V-A}
         \sum_{q} e_{q} \left( \bar q q \right)_{\rm V-A}
&\qquad&
Q_{10} = \frac{3}{2} \left( \bar s_{i} d_{j} \right)_{\rm V-A}
         \sum_{q} e_{q} \left( \bar q_{j}  q_{i}\right)_{\rm V-A}
\label{eq:Q910}
\end{eqnarray}

\leftline{\bf Magnetic-Penguins Operators (fig.\ \ref{fig:oporig}\,(d)):}

\begin{eqnarray}
Q_{7\gamma} = \frac{e}{8\pi^2} \mb \bar{s}_i \sigma^{\mu\nu}
              (1+\gamma_5) b_i F_{\mu\nu}
&\qquad&
Q_{8G} = \frac{g}{8\pi^2} \mb \bar{s}_i \sigma^{\mu\nu}
        (1+\gamma_5)T^a_{ij} b_j G^a_{\mu\nu}
\label{eq:Q78mag}
\end{eqnarray}

\leftline{\bf $\Delta S=2$ and $\Delta B=2$ Operators
(fig.\ \ref{fig:oporig}\,(e)):}
\begin{eqnarray}
Q(\Delta S=2) = (\bar s d)_{V-A} (\bar s d)_{V-A} 
&\qquad&
Q(\Delta B=2) = (\bar b d)_{V-A} (\bar b d)_{V-A} 
\label{eq:QdSB2}
\end{eqnarray}

\leftline{\bf Semi-Leptonic Operators (fig.\ \ref{fig:oporig}\,(f)):}

\begin{eqnarray}
Q_{7V}  = (\bar s d)_{V-A} (\bar e e)_{V} 
&\qquad&
Q_{7A} = (\bar s d)_{V-A} (\bar e e)_{A}
\label{eq:Q7V7A} \\
Q_{9V}  = (\bar b s)_{V-A} (\bar e e)_{V} 
&\qquad&
Q_{10A} = (\bar b s)_{V-A} (\bar e e)_{A}
\label{eq:Q9V10A} \\
Q(\bar\nu \nu) = (\bar s d)_{V-A} (\bar \nu \nu)_{V-A} 
&\qquad&
Q(\bar\mu \mu) = (\bar s d)_{V-A} (\bar \mu \mu)_{V-A} 
\label{eq:Qnnmm}
\end{eqnarray}
where indices in color singlet currents have been suppressd for
simplicity.

For illustrative purposes, typical diagrams in the full theory from
which the operators \eqn{eq:Q12}--\eqn{eq:Qnnmm} originate are shown in
fig.\ \ref{fig:oporig}.

\begin{figure}[htb]
\vspace{0.10in}
\centerline{
\epsfysize=5in
\epsffile{ps/oporig.ps}
}
\vspace{0.08in}
\caption[]{Typical diagrams in the full theory from which the operators
\eqn{eq:Q12}--\eqn{eq:Qnnmm} originate.  The cross in diagram (d) means
a mass-insertion. It indicates that magnetic penguins originate from
the mass-term on the external line in the usual QCD or QED penguin
diagrams.
\label{fig:oporig}}
\end{figure}

The operators listed above will enter our review in a systematic
fashion. We begin in section \ref{sec:HeffdF1:22} with the presentation
of the effective hamiltonians involving the current-current operators
$Q_1$ and $Q_2$ only.  These effective hamiltonians are given in
(\ref{B4}), (\ref{B5}) and (\ref{B6}) for $\Delta B=1$, $\Delta C=1$
and $\Delta S=1$ non-leptonic decays, respectively.

In section \ref{sec:HeffdF1:66} we will generalize the hamiltonians
(\ref{B4}) and (\ref{B6}) to include the QCD-penguin operators
$Q_3-Q_6$.
The corresponding expressions are given in (\ref{eq:HeffdB1:66}) and
(\ref{eq:HeffKpp}), respectively.  This generalization does not affect
the Wilson coefficients of $Q_1$ and $Q_2$.

Next in section \ref{sec:HeffdF1:1010} the $\Delta S=1$ and $\Delta
B=1$ hamiltonians of section \ref{sec:HeffdF1:66} will be generalized
to include the electroweak penguin operators $Q_7-Q_{10}$. These
generalized hamiltonians are given in (\ref{eq:HeffdF1:1010}) and
(\ref{eq:HeffdB1:1010}) for $\Delta S=1$ and $\Delta B=1$ non-leptonic
decays, respectively.  The inclusion of the electroweak penguin
operators implies the inclusion of $QED$ effects. Consequently the
coefficients of the operators $Q_1-Q_6$ given in this section will
differ slightly from the ones presented in the previous sections.

In section \ref{sec:HeffKpe} the effective hamiltonian for $K_L\to
\pi^0 e^+e^- $ will be presented.  It is given in (\ref{eq:HeffKpe}).
This hamiltonian can be considered as a generalization of the $\Delta
S=1$ hamiltonian (\ref{eq:HeffKpp}) presented in section
\ref{sec:HeffdF1:66} to include the semi-leptonic operators $Q_{7V}$
and $Q_{7A}$. This generalization does not modify the numerical values
of the $\Delta S=1$ coefficients $C_i$ ($i=1,\ldots,6$) given in
section \ref{sec:HeffdF1:66}.

In section \ref{sec:Heff:BXsgamma} we will discuss the effective
hamiltonian for $B\to X_s\gamma$. It is written down in
(\ref{eq:HeffBXsgamma}). This hamiltonian can be considered as a
generalization of the $\Delta B=1$ hamiltonian (\ref{eq:HeffdB1:66})
to include the magnetic penguin operators $Q_{7\gamma}$ and $Q_{8G}$.
This generalization does not modify the numerical values of the $\Delta
B=1$ coefficients $C_i$ ($i=1,\ldots,6$) from section \ref{sec:HeffdF1:66}.

In section \ref{sec:Heff:BXsee} we present the effective hamiltonian
for $B\to X_s e^+ e^- $. It is to be found in (\ref{eq:Heff2atmu}) and
can be considered as the generalization of the $B\to X_s\gamma$
hamiltonian to include the semi-leptonic operators $Q_{9V}$ and
$Q_{10A}$.  The coefficients $C_i$ ($i=1,\ldots,6,7\gamma,8G$) given in
section \ref{sec:Heff:BXsgamma} are not affected by this
generalization.

In section \ref{sec:HeffRareKB} the effective hamiltonians for
$K^+\to\pi^+\nu\bar\nu$, $K_L\to\mu^+\mu^-$, $K_L\to \pi^0\nu\bar\nu$
($B\to X_{s,d}\nu\bar\nu$) and $B\to l^+l^-$ will be discussed.  They
are given in (\ref{hkpn}), (\ref{hklm}), (\ref{hxnu}) and (\ref{hyll})
respectively. Each of these hamiltonians involves only a single
operator:  $Q(\nu\bar\nu)$ or $Q(\mu\bar\mu)$ for
$K^+\to\pi^+\nu\bar\nu$ $(K_L\to \pi^0\nu\bar\nu)$ and
$K_L\to\mu^+\mu^-$ with analogous operators for $B\to
X_{s,d}\nu\bar\nu$ and $B\to l^+l^-$.

Finally, sections \ref{sec:HeffKKbar} and \ref{sec:HeffBBbar} present
the effective hamiltonians for $\Delta S=2$ and $\Delta B=2$
transitions, respectively. These hamiltonians involve the operators
$Q(\Delta S=2)$ and $Q(\Delta B=2)$ and can be found in (\ref{hds2})
and (\ref{hdb2}).

In table \ref{tab:heffguide} we give the list of effective hamiltonians
to be presented below, the equations in which they can be found and the
list of operators entering different hamiltonians.

\begin{table}[htb]
\caption[]{Compilation of various processes, equation no.\ of the
corresponding effective hamiltonians and contributing operators.}
\label{tab:heffguide}
\begin{center}
\begin{tabular}{|l|l|l|}
\multicolumn{1}{|c|}{\bf Process} &
\multicolumn{1}{c|}{\bf Cf.\ Equation} &
\multicolumn{1}{c|}{\bf Contributing Operators} \\
\hline
$\Delta F=1$, $F=B,C,S$ current-current &
   \eqn{B4}--\eqn{B6} & $Q_1, Q_2$       \\
$\Delta F=1$ pure QCD &
   \eqn{eq:HeffKpp}, \eqn{eq:HeffdB1:66} & $Q_1,\ldots,Q_6$ \\
$\Delta F=1$ QCD and electroweak &
   \eqn{eq:HeffdF1:1010}, \eqn{eq:HeffdB1:1010} & $Q_1,\ldots,Q_{10}$ \\
\hline
$\Kpiee$ & \eqn{eq:HeffKpe} & $Q_1,\ldots,Q_6, Q_{7V}, Q_{7A}$ \\
\hline
$B\to X_s\gamma$ & \eqn{eq:HeffBXsgamma} &
   $Q_1,\ldots,Q_6, Q_{7\gamma}, Q_{8G}$ \\
$B\to X_s e^+e^-$ & \eqn{eq:Heff2atmu} & 
   $Q_1,\ldots,Q_6, Q_{7\gamma}, Q_{8G}, Q_{9V}, Q_{10A}$ \\
\hline
$\kpn$, $(\klm)_{SD}$, $K_L\to\pi^0\nu\bar\nu$, &
   \eqn{hkpn}, \eqn{hklm}, \eqn{hxnu} & $Q(\bar\nu \nu), Q(\bar\mu \mu)$ \\
$B\to X_{s, d}\nu\bar\nu$, $B\to l^+l^-$ & \eqn{hyll} & \\
\hline
$K^0-\bar K^0$ mixing & \eqn{hds2} & $Q(\Delta S=2)$ \\
$B^0-\bar B^0$ mixing & \eqn{hdb2} & $Q(\Delta B=2)$ \\
\end{tabular}
\end{center}
\end{table}
