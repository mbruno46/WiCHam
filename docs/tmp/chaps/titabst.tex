\renewcommand{\thefootnote}{\fnsymbol{footnote}}
% \setcounter{footnote}{1}

\preprint{\small
$
\begin{array}{r}
MPI-Ph/95-104 \\
TUM-T31-100/95 \\
FERMILAB-PUB-95/305-T \\
SLAC-PUB\ 7009
% \\
% hep-ph/9512323
\end{array}
$
}

\title{WEAK DECAYS BEYOND LEADING LOGARITHMS}

\author{\bf
        Gerhard~{Buchalla}${}^{3}$,\,
        Andrzej~J.~{Buras}${}^{1,2}$,\,
        Markus~E.~{Lautenbacher}${}^{1,4}$
        \footnote{email:~{\tt
buchalla@fnth20.fnal.gov,\,buras@feynman.t30.physik.tu-muenchen.de, \\
\phantom{XXXXXX}lauten@feynman.t30.physik.tu-muenchen.de}}
}

\address{
\ \\
${}^{1}$ Physik Department, Technische Universit\"at M\"unchen, \\
D-85748 Garching, Germany.
\\
% \svs
${}^{2}$ Max-Planck-Institut f\"ur Physik -- Werner-Heisenberg-Institut, \\
F\"ohringer Ring 6, D-80805 M\"unchen, Germany.
\\
% \svs
${}^{3}$ Theoretical Physics Department,  \\
Fermi National Accelerator Laboratory, \\
P.O.\ Box 500, Batavia, IL 60510, USA.
\\
% \svs
${}^{4}$ SLAC Theory Group, Stanford University, \\
P.O.\ Box 4349, Stanford, CA 94309, USA.
\\
\ 
}

\date{November 1995}

\maketitle
\thispagestyle{empty}

\centerline{to appear in Reviews of Modern Physics}

\begin{abstract}
\noindent We review the present status of QCD corrections to weak
decays beyond the leading logarithmic approximation including
particle-antiparticle mixing and rare and CP violating decays. After
presenting the basic formalism for these calculations we discuss in
detail the effective hamiltonians for all decays for which the
next-to-leading corrections are known. Subsequently, we present the
phenomenological implications of these calculations. In particular we
update the values of various parameters and we incorporate new
information on $m_t$ in view of the recent top quark discovery.  One of
the central issues in our review are the theoretical uncertainties
related to renormalization scale ambiguities which are substantially
reduced by including next-to-leading order corrections.  The impact of
this theoretical improvement on the determination of the
Cabibbo-Kobayashi-Maskawa matrix is then illustrated in various cases.
\end{abstract}

%%%%%%%%%%%%%%%%%%%%%%%%%%%%%%%%%%%%%%%%%%%%%%%%%%%%%%%%%%%%%%%%%%%%%%%%%
% Table of Contents
%%%%%%%%%%%%%%%%%%%%%%%%%%%%%%%%%%%%%%%%%%%%%%%%%%%%%%%%%%%%%%%%%%%%%%%%%
\setcounter{footnote}{0}
\renewcommand{\thefootnote}{\arabic{footnote}}

\newpage
\setcounter{page}{1}
\renewcommand{\thepage}{\roman{page}}
\tableofcontents
\newpage

\renewcommand{\thepage}{\arabic{page}}
\setcounter{page}{1}
