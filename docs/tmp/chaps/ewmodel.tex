\section{Standard Electroweak Model}
         \label{sec:sewm}
\subsection{Particles and Interactions}
            \label{sec:sewm:particles}
Throughout this review we will work in the context of the three
generation model of quarks and leptons based on the gauge group
$SU(3)\otimes SU(2)_L\otimes U(1)_Y $ spontaneously broken down
 to $SU(3)\otimes U(1)_Q$.
Here $Y$ and $Q$ denote the weak hypercharge and the electric charge
generators, respectively. $SU(3)$ stands for $QCD$ which will be discussed
in more detail in the following section. Here we would like to recall certain
features of the electroweak part of the Standard Model which will be
important for our considerations.

The left-handed leptons and quarks are put in $ SU(2)_L $ doublets
\begin{equation}\label{2.31}
\left(\begin{array}{c}
\nu_e \\
e^-
\end{array}\right)_L\qquad
\left(\begin{array}{c}
\nu_\mu \\
\mu^-
\end{array}\right)_L\qquad
\left(\begin{array}{c}
\nu_\tau \\
\tau^-
\end{array}\right)_L
\end{equation}
\begin{equation}\label{2.66}
\left(\begin{array}{c}
u \\
d^\prime
\end{array}\right)_L\qquad
\left(\begin{array}{c}
c \\
s^\prime
\end{array}\right)_L\qquad
\left(\begin{array}{c}
t \\
b^\prime
\end{array}\right)_L       
\end{equation}
with the corresponding right-handed fields transforming as singlets
under $ SU(2)_L $. The primes are discussed below.
The relevant electroweak charges $Q$, $Y$ and the third component of
the weak isospin $T_3$ are collected in table~\ref{tab:ewcharges}.

\begin{table}[htb]
\caption[]{Electroweak charges $Q$, $Y$ and the third component of
the weak isospin $T_3$ for quarks and leptons in the Standard Model.
\label{tab:ewcharges}}
\begin{center}
\begin{tabular}{|c||c|c|c|c|c|c|c|}
 & $\nu^e_L$ & $ e^-_L$ & $ e^-_R$ & $u_L$ & $d_L$ & $u_R$ & $d_R$ \\
\hline
$Q$ & 0 & $-1$ & $-1$ & 2/3 & $-1/3$ & 2/3 & $-1/3$ \\
\hline
$T_3$ &1/2 & $-1/2$ & 0 & 1/2 & $-1/2$ & 0 & 0 \\
\hline
$Y$ & $-1$ & $-1$ & $-2$ & 1/3 & 1/3 & 4/3 & $-2/3$
\end{tabular}
\end{center}
\end{table}

The electroweak interactions of quarks and leptons are mediated by
the massive weak gauge bosons $W^\pm$ and $Z^o$ and by the photon $A$.
These interactions are summarized by the Lagrangian
\begin{equation}\label{3}
{\cal L}_{int}={\cal L}_{\rm CC}+{\cal L}_{\rm NC}
\end{equation}
where
\begin{equation}\label{4}
{\cal L}_{\rm CC}=\frac{g_2}{2 \sqrt{2}}(J^+_\mu W^{+\mu}+J^-_\mu W^{-\mu})
\end{equation}
describes the {\it charged current} interactions and
\begin{equation}\label{5}
{\cal L}_{\rm NC}=
 e J^{em}_\mu A^{\mu}+ \frac{g_2}{2 \cos \Theta_W} J^o_\mu Z^\mu
\end{equation}
the {\it neutral current} interactions. Here $e$ is the QED coupling constant,
$g_2$ is the $SU(2)_L$ coupling constant and $\Theta_W$ is the Weinberg
angle. The currents are given as follows
\begin{equation}\label{6}
J^+_\mu=
(\bar{u} d')_{V-A} +
(\bar{c} s')_{V-A} +
(\bar{t} b')_{V-A} +
(\bar{\nu}_e e)_{V-A} +
(\bar{\nu}_\mu \mu)_{V-A} +
(\bar{\nu}_\tau \tau)_{V-A}
\end{equation}

\begin{equation}\label{7}
J^{em}_\mu=\sum_f {Q_f \bar f \gamma_\mu f}
\end{equation}
\begin{equation}\label{8}
J^o_\mu=\sum_f \bar f \gamma_\mu (v_f-a_f\gamma_5) f
\end{equation}
\begin{equation}\label{9}
v_f=T^f_3-2 Q_f \sin^2\Theta_W
\qquad
a_f=T^f_3
\end{equation}
where $Q_f$ and $T^f_3$ denote the charge and the third component of the
weak isospin of the left-handed fermion $f_L$.


In our discussion of weak decays an important role is played by
the Fermi constant:
\begin{equation}\label{2.100}
\frac{G_F}{\sqrt{2}}=\frac{g^2_2}{8 M^2_W}
\end{equation}
which has the value
\begin{equation}\label{2.98}
G_F=1.16639\cdot 10^{-5}\gev^{-2}
\end{equation}
Other values of the relevant parameters will be collected in appendix
\ref{app:numinput}. 

The interactions between the gauge bosons are standard and can be
found in any textbook on gauge theories.

The primes in (\ref{2.66}) indicate that the weak eigenstates 
$(d^\prime,s^\prime,b^\prime)$ are not equal to the corresponding
mass eigenstates $(d,s,b)$, but are rather linear combinations of
the latter. This is expressed through the relation
\begin{equation}\label{2.67}
\left(\begin{array}{c}
d^\prime \\ s^\prime \\ b^\prime
\end{array}\right)=
\left(\begin{array}{ccc}
V_{ud}&V_{us}&V_{ub}\\
V_{cd}&V_{cs}&V_{cb}\\
V_{td}&V_{ts}&V_{tb}
\end{array}\right)
\left(\begin{array}{c}
d \\ s \\ b
\end{array}\right)
\end{equation}
where the unitary matrix connecting theses two sets of states is the
Cabibbo-Kobayashi-Maskawa (CKM) matrix.  Many parametrizations of this
matrix have been proposed in the literature.  We will use in this
review two parametrizations:  the standard parametrization
recommended by the particle data group and the Wolfenstein
parametrization.

\subsection{Standard Parametrization}
            \label{sec:sewm:stdparam}
Let us introduce the notation
$c_{ij}=cos\theta_{ij}$ and $s_{ij}=sin\theta_{ij}$ with $i$ and $j$
being generation labels ($i,j=1,2,3$). The standard parametrization is
then given as follows \cite{particledata:94}
\begin{equation}\label{2.72}
V=
\left(\begin{array}{ccc}
c_{12}c_{13}&s_{12}c_{13}&s_{13}e^{-i\delta}\\ -s_{12}c_{23}
-c_{12}s_{23}s_{13}e^{i\delta}&c_{12}c_{23}-s_{12}s_{23}s_{13}e^{i\delta}&
s_{23}c_{13}\\ s_{12}s_{23}-c_{12}c_{23}s_{13}e^{i\delta}&-s_{23}c_{12}
-s_{12}c_{23}s_{13}e^{i\delta}&c_{23}c_{13}
\end{array}\right)
\end{equation}
where $\delta$ is the phase necessary for CP violation. $c_{ij}$ and
$s_{ij}$ can all be chosen to be positive and $\delta$ may vary in the
range $0\le\delta\le 2\pi$. However the measurements
of CP violation in K decays force $\delta$ to be in the range
 $0<\delta<\pi$. 

The extensive phenomenology of the last years 
has shown that
$s_{13}$ and $s_{23}$ are small numbers: $\ord(10^{-3})$ and ${\cal
O}(10^{-2})$,
respectively. Consequently to an excellent accuracy $c_{13}=c_{23}=1$
and the four independent parameters are given as follows
\begin{equation}\label{2.73}
s_{12}=| V_{us}|, \quad s_{13}=| V_{ub}|, \quad s_{23}=|
V_{cb}|, \quad \delta
\end{equation}
with the phase $\delta$ extracted from CP violating transitions or 
loop processes sensitive to $| V_{td}|$. The latter fact is based
on the observation that
 for $0\le\delta\le\pi$, as required by the analysis of CP violation,
there is a one--to--one correspondence between $\delta$ and $|V_{td}|$
given by
\begin{equation}\label{10}
| V_{td}|=\sqrt{a^2+b^2-2 a b \cos\delta},
\qquad
a=| V_{cd} V_{cb}|,
\qquad
b=| V_{ud} V_{ub}|
\end{equation} 

\subsection{Wolfenstein Parameterization Beyond Leading Order}
            \label{sec:sewm:wolfparam}
We will also use the Wolfenstein parametrization
\cite{wolfenstein:83}.
It is an approximate parametrization of the CKM matrix in which
each element is expanded as a power series in the small parameter
$\lambda=| V_{us}|=0.22$
\begin{equation}\label{2.75} 
V=
\left(\begin{array}{ccc}
1-{\lambda^2\over 2}&\lambda&A\lambda^3(\varrho-i\eta)\\ -\lambda&
1-{\lambda^2\over 2}&A\lambda^2\\ A\lambda^3(1-\varrho-i\eta)&-A\lambda^2&
1\end{array}\right)
+\ord(\lambda^4)
\end{equation}
and the set (\ref{2.73}) is replaced by
\begin{equation}\label{2.76}
\lambda, \qquad A, \qquad \varrho, \qquad \eta \, .
\end{equation}
The Wolfenstein parameterization
has several nice features. In particular it offers in conjunction with the
unitarity triangle a very transparent geometrical
representation of the structure of the CKM matrix and allows to derive
several analytic results to be discussed below. This turns out to be very
useful in the phenomenology of rare decays and of CP violation.

When using the Wolfenstein parametrization one should remember that it
is an approximation and that in certain situations neglecting
$\ord(\lambda^4)$ terms may give wrong results. The question then
arises how to find $\ord(\lambda^4)$ and higher order terms ?  The
point is that since \eqn{2.75} is only an approximation the {\em exact}
definiton of $\lambda$ is not unique by terms of the neglected order
$\ord(\lambda^4)$. This is the reason why in different papers in the
literature different $\ord(\lambda^4)$ terms can be found. They simply
correspond to different definitons of the expansion parameter
$\lambda$.  Obviously the physics does not depend on this choice.  Here
it suffices to find an expansion in $\lambda$ which allows for simple
relations between the parameters (\ref{2.73}) and (\ref{2.76}).  This
will also restore the unitarity of the CKM matrix which in the
Wolfenstein parametrization as given in (\ref{2.75}) is not satisfied
exactly.

To this end
we go back to (\ref{2.72}) and we impose the relations \cite{burasetal:94b}
\begin{equation}\label{2.77} 
s_{12}=\lambda
\qquad
s_{23}=A \lambda^2
\qquad
s_{13} e^{-i\delta}=A \lambda^3 (\varrho-i \eta)
\end{equation}
to {\it  all orders} in $\lambda$. In view of the comments made above
this can certainly be done. It follows that
\begin{equation}\label{2.84} 
\varrho=\frac{s_{13}}{s_{12}s_{23}}\cos\delta
\qquad
\eta=\frac{s_{13}}{s_{12}s_{23}}\sin\delta
\end{equation}
We observe that (\ref{2.77}) and (\ref{2.84}) represent simply
the change of variables from (\ref{2.73}) to (\ref{2.76}).
Making this change of variables in the standard parametrization 
(\ref{2.72}) we find the CKM matrix as a function of 
$(\lambda,A,\varrho,\eta)$ which satisfies unitarity exactly!
We also note that in view of $c_{13}=1-\ord(\lambda^6)$ the relations
between $s_{ij}$ and $| V_{ij}|$ in (\ref{2.73}) are 
satisfied to high accuracy. The relations in (\ref{2.84}) have
been first used in \cite{schmidtlerschubert:92}.
However, the improved treatment of the unitarity
triangle presented below goes beyond the analysis of these authors.

The procedure outlined above gives automatically the corrections to the
Wolfenstein parametrization in (\ref{2.75}).  Indeed expressing
(\ref{2.72}) in terms of Wolfenstein parameters using (\ref{2.77})
and then expanding in powers of $\lambda$ we recover the
matrix in (\ref{2.75}) and in addition find explicit corrections of
$\ord(\lambda^4)$ and higher order terms. $V_{ub}$ remains unchanged. The
corrections to $V_{us}$ and $V_{cb}$ appear only at $\ord(\lambda^7)$ and
$\ord(\lambda^8)$, respectively.  For many practical purposes the
corrections to the real parts can also be neglected.
The essential corrections to the imaginary parts are:
\begin{equation}\label{2.83g}
\Delta V_{cd}=-iA^2 \lambda^5\eta
\qquad
\Delta V_{ts}=-iA\lambda^4\eta 
\end{equation}
These two corrections have to be
taken into account in the discussion of CP violation.
On the other hand the imaginary part of $V_{cs}$ which in our expansion
in $\lambda$ appears only at $\ord(\lambda^6)$ can be fully neglected. 

In order to improve the accuracy of the unitarity triangle discussed
below we will also include the $\ord(\lambda^5)$ correction to $V_{td}$
which gives
\begin{equation}\label{2.83d}
 V_{td}= A\lambda^3 (1-\bar\varrho-i\bar\eta) 
\end{equation}
with
\begin{equation}\label{2.88d}
\bar\varrho=\varrho (1-\frac{\lambda^2}{2})
\qquad
\bar\eta=\eta (1-\frac{\lambda^2}{2}).
\end{equation}
%
In order to derive analytic results we need accurate explicit expressions
for $\lambda_i=V_{id}^{}V_{is}^*$ where $i=c,t$. We have
\begin{equation}\label{2.51}
 Im\lambda_t= -Im\lambda_c=\eta A^2\lambda^5 
\end{equation}
\begin{equation}\label{2.52}
 Re\lambda_c=-\lambda (1-\frac{\lambda^2}{2})
\end{equation}
\begin{equation}\label{2.53}
 Re\lambda_t= -(1-\frac{\lambda^2}{2}) A^2\lambda^5 (1-\bar\varrho) 
\end{equation}
Expressions (\ref{2.51}) and (\ref{2.52}) represent to an accuracy of
0.2\% the exact formulae obtained using (\ref{2.72}). The expression
(\ref{2.53}) deviates by at most 2\% from the exact formula in the
full range of parameters considered. 
In order to keep the analytic
expressions in the phenomenological applications in a transparent form
we have dropped a small $\ord(\lambda^7)$ term in deriving (\ref{2.53}).
After inserting the expressions (\ref{2.51})--(\ref{2.53}) in exact
formulae for quantities of interest, further expansion in $\lambda$
should not be made. 

\subsection{Unitarity Triangle Beyond Leading Order}
            \label{sec:sewm:utriag}
The unitarity of the CKM matrix provides us with several relations
of which
\begin{equation}\label{2.87h}
V_{ud}^{}V_{ub}^* + V_{cd}^{}V_{cb}^* + V_{td}^{}V_{tb}^* =0
\end{equation}
is the most useful one.
In the complex plane the relation (\ref{2.87h}) can be represented as
a triangle, the so-called ``unitarity--triangle'' (UT).
Phenomenologically this triangle is very interesting as it involves
simultaneously the elements $V_{ub}$, $V_{cb}$ and $V_{td}$ which are
under extensive discussion at present.

In the usual analyses of the unitarity triangle only terms ${\cal
O}(\lambda^3)$ are kept in (\ref{2.87h}) \cite{burasharlander:92},
\cite{nir:74}, \cite{harrisrosner:92}, \cite{schmidtlerschubert:92},
\cite{dibdunietzgilman:90}, \cite{alilondon:95}. It is however
straightforward to include the next-to-leading $\ord(\lambda^5)$
terms \cite{burasetal:94b}. We note first that
\begin{equation}\label{2.88a}
V_{cd}^{}V_{cb}^*=-A\lambda^3+\ord(\lambda^7).
\end{equation}
%
Thus to an excellent accuracy $V_{cd}^{}V_{cb}^*$ is real with
$| V_{cd}^{}V_{cb}^*|=A\lambda^3$.
Keeping $\ord(\lambda^5)$ corrections and rescaling all terms in
(\ref{2.87h})
by $A \lambda^3$ 
we find
\begin{equation}\label{2.88b}
 \frac{1}{A\lambda^3}V_{ud}^{}V_{ub}^*
=\bar\varrho+i\bar\eta
\qquad,
\qquad
 \frac{1}{A\lambda^3}V_{td}^{}V_{tb}^*
=1-(\bar\varrho+i\bar\eta)
\end{equation}
with $\bar\varrho$ and $\bar\eta$ defined in (\ref{2.88d}). 
Thus we can represent (\ref{2.87h}) as the unitarity triangle 
in the complex $(\bar\varrho,\bar\eta)$
plane. This is  shown in fig.~\ref{fig:triangle}.
 The length of the side $CB$ which lies
on the real axis equals unity when eq.~(\ref{2.87h}) is rescaled by
$V_{cd}^{}V_{cb}^*$. We observe that beyond the leading order
in $\lambda$ the point A {\it does not} correspond to  $(\varrho,\eta)$ but to
 $(\bar\varrho,\bar\eta)$.
Clearly within 3\% accuracy $\bar\varrho=\varrho$ and $\bar\eta=\eta$.
Yet in the distant future the accuracy of experimental results and
theoretical calculations may improve considerably so that the more
accurate formulation given here will be appropriate.

\begin{figure}[htb]
\vspace{0.05in}
\centerline{
% adjust actual size of fig. in text here
\epsfysize=1.5in
\epsffile{ps/triangle.ps}
% use this if rotation of fig. is needed
% \rotate[r]{
% \epsffile{ps/triangle.ps}
% }
}
\vspace{0.05in}
\caption[]{\small\sl
Unitarity triangle in the complex $(\bar\varrho,\bar\eta)$ plane.
\label{fig:triangle}}
\end{figure}
 
Using simple trigonometry one can calculate $\sin(2\phi_i$), $\phi_i=
\alpha, \beta, \gamma$, in terms of $(\bar\varrho,\bar\eta)$ with the result:
\begin{equation}\label{2.89}
\sin(2\alpha)=\frac{2\bar\eta(\bar\eta^2+\bar\varrho^2-\bar\varrho)}
  {(\bar\varrho^2+\bar\eta^2)((1-\bar\varrho)^2
  +\bar\eta^2)}  
\end{equation}
\begin{equation}\label{2.90}
\sin(2\beta)=\frac{2\bar\eta(1-\bar\varrho)}{(1-\bar\varrho)^2 + \bar\eta^2}
\end{equation}
 \begin{equation}\label{2.91}
\sin(2\gamma)=\frac{2\bar\varrho\bar\eta}{\bar\varrho^2+\bar\eta^2}=
\frac{2\varrho\eta}{\varrho^2+\eta^2}
\end{equation}
%
The lengths $CA$ and $BA$ in the
rescaled triangle of fig.~\ref{fig:triangle} to be denoted by $R_b$ and $R_t$,
respectively, are given by
%
\begin{equation}\label{2.94}
R_b \equiv \frac{| V_{ud}^{}V^*_{ub}|}{| V_{cd}^{}V^*_{cb}|}
= \sqrt{\bar\varrho^2 +\bar\eta^2}
= (1-\frac{\lambda^2}{2})\frac{1}{\lambda}
\left| \frac{V_{ub}}{V_{cb}} \right|
\end{equation}
\begin{equation}\label{2.95}
R_t \equiv \frac{| V_{td}^{}V^*_{tb}|}{| V_{cd}^{}V^*_{cb}|} =
 \sqrt{(1-\bar\varrho)^2 +\bar\eta^2}
=\frac{1}{\lambda} \left| \frac{V_{td}}{V_{cb}} \right|
\end{equation}
The expressions for $R_b$ and $R_t$ given here in terms of
$(\bar\varrho, \bar\eta)$ 
are excellent approximations. Clearly $R_b$ and $R_t$
can also be determined by measuring two of the angles $\phi_i$:
\begin{equation}\label{2.96}
R_b=\frac{\sin(\beta)}{\sin(\alpha)}=
\frac{\sin(\alpha+\gamma)}{\sin(\alpha)}=
\frac{\sin(\beta)}{\sin(\gamma+\beta)}
\end{equation}
\begin{equation}\label{2.97}
R_t=\frac{\sin(\gamma)}{\sin(\alpha)}=
\frac{\sin(\alpha+\beta)}{\sin(\alpha)}=
\frac{\sin(\gamma)}{\sin(\gamma+\beta)}
\end{equation}

