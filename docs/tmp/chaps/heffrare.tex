\section{Effective Hamiltonians for Rare $K$- and $B$-Decays}
         \label{sec:HeffRareKB}
\subsection{Overview}
            \label{sec:HeffRareKB:overview}
In the present section we will summarize the effective hamiltonians
valid at next-to-leading logarithmic accuracy in QCD, which describe
the semileptonic rare Flavour Changing Neutral Current (FCNC)
transitions $\kpn$, $(\klm)_{SD}$, $K_L\to\pi^0\nu\bar\nu$, $B\to X_{s,
d}\nu\bar\nu$ and $B\to l^+l^-$.  These decay modes all are very
similar in their structure and it is natural to discuss them together.
On the other hand they differ from the decays $\Kpipi$, $K \to \pi e^+
e^-$, $B \to X_s \gamma$ and $B \to X_s e^+ e^-$ discussed in previous
sections. Before giving the detailed formulae, it will be useful to
recall the most important general features of this class of processes
first. In addition, characteristic differences between the specific
modes will also become apparent from our presentation.
\begin{itemize}
\item
Within the Standard Model all the decays listed above are loop-induced
semileptonic FCNC processes determined by $Z^0$-penguin and box diagrams
(fig.~\ref{fig:1loopful}\,(d) and (e)).\\
In particular, a distinguishing feature of the present class of decays
is the absence of a photon penguin contribution. For the decay modes
with neutrinos in the final state this is obvious, since the photon
does not couple to neutrinos. For the mesons decaying into a charged
lepton pair the photon penguin amplitude vanishes due to vector current
conservation.\\
An important consequence is, that the decays considered here exhibit a
hard GIM suppression, quadratic in (small) internal quark masses, which
is a property of the $Z^0$-penguin  and box graphs. By contrast, the
GIM suppression resulting from photon penguin contributions is
logarithmic. Decays where the photon penguin contributes are for
example $K_L\to\pi^0e^+e^-$ and $B \to X_s e^+ e^-$. The differences in
the basic structure of these processes, resulting from the different
pattern of GIM suppression, are the reason why we have discussed
$K_L\to\pi^0e^+e^-$ and $B \to X_s e^+ e^-$ in a separate context.
\item
The investigation of low energy rare decay processes allows to probe,
albeit indirectly, high energy scales of the theory. Of particular
interest is the sensitivity to properties of the top quark,
its mass $m_t$ and its CKM couplings $V_{ts}$ and $V_{td}$.
\item
A particular and very important advantage of the processes under
discussion is, that theoretically clean predictions can be obtained.
The reasons for this are:
\begin{itemize}
\item The low energy hadronic
matrix elements required are just the matrix elements of quark currents
between hadron states, which can be extracted from the leading
(non-rare) semileptonic decays. Other long-distance contributions
are negligibly small.\\
An exception is the decay $\klm$ receiving important contributions from
the two-photon intermediate state, which are difficult to calculate
reliably. However, the short-distance part $(\klm)_{SD}$ alone, which we
shall discuss here, is on the same footing as the other modes. The
essential difficulty for phenomenological applications then is to
separate the short-distance from the long-distance piece in the
measured rate.
\item According to the comments just made, the processes at hand are
short-distance processes, calculable within a perturbative framework,
possibly including renormalization group improvement. The necessary
separation of the short-distance dynamics from the low energy matrix
elements  is achieved by means of an operator product expansion.
The scale ambiguities, inherent to perturbative QCD, essentially
constitute the only theoretical uncertainties present in the analysis.
These uncertainties are well under control as they may be
systematically reduced through calculations beyond leading order.
\end{itemize}
\item
The points made above emphasize, that the short-distance dominated
loop-induced FCNC decays provide highly promising possibilities to
investigate flavourdynamics at the quantum level. However, the very fact
that these processes are based on higher order electroweak effects,
which makes them interesting theoretically, at the same time implies,
that the branching ratios will be very small and not easy to
access experimentally.
\end{itemize}
The effective hamiltonians governing the decays
$\kpn$, $(\klm)_{SD}$, $K_L\to\pi^0\nu\bar\nu$,
$B\to X_{s, d}\nu\bar\nu$, $B\to l^+l^-$,
resulting from the $Z^0$-penguin and box-type contributions, can all be
written down in the following general form
\begin{equation}\label{hnr} {\cal H}_{eff}={G_F \over{\sqrt 2}}{\alpha\over 2\pi \sin^2\Theta_W}
 \left( \lambda_c F(x_c) + \lambda_t F(x_t)\right)
 (\bar nn^\prime)_{V-A}(\bar rr)_{V-A}  \end{equation}
where $n$, $n^\prime$ denote down-type quarks
($n, n^\prime=d, s, b$ but $n\not= n^\prime$) and $r$ leptons,
$r=l, \nu_l$ ($l=e, \mu, \tau$). The $\lambda_i$ are products of CKM elements,
in the general case $\lambda_i=V^*_{in}V_{in^\prime}^{}$. Furthermore
$x_i=m^2_i/M^2_W$.\\
The functions $F(x_i)$ describe the dependence on the internal
up-type quark masses $m_i$ (and on lepton masses if necessary)
and are understood to include QCD corrections.
They are increasing functions of the quark masses, a property that is
particularly important for the top contribution.\\
Crucial features of the structure of the hamiltonian
are furthermore determined
by the hard GIM suppression characteristic for this class of decays.
First we note that the dependence of the hamiltonian on the internal
quarks comes in the form
\begin{equation}\label{lifx}
\sum_{i=u, c, t}\lambda_i F(x_i)=
\lambda_c(F(x_c)-F(x_u))+\lambda_t(F(x_t)-F(x_u))  \end{equation}
where we have used the unitarity of the CKM matrix. Now, hard GIM
suppression means that for $x\ll 1$ $F$ behaves quadratically in
the quark masses. In the present case we have
\begin{equation}\label{fxln}
F(x)\sim x\ln x\qquad\qquad  {\rm for}\quad x\ll 1   \end{equation}
The first important consequence is, that $F(x_u)\approx 0$ can be
neglected. The hamiltonian acquires the form anticipated in \eqn{hnr}.
It effectively consists of a charm and a top contribution. Therefore
the relevant energy scales are $M_W$ or $m_t$ and, at least, $m_c$,
which are large compared to $\Lambda_{QCD}$. This fact indicates the
short-distance nature of these processes.\\
A second consequence of \eqn{fxln} is that $F(x_c)/F(x_t)\approx
\ord(10^{-3})\ll 1$. Together with the weighting introduced by the
CKM factors this relation determines the relative importance of
the charm versus the top contribution in \eqn{hnr}. As seen in
table~\ref{tab:lambdaexp} a simple pattern emerges if one writes down
the order of magnitude of $\lambda_c$, $\lambda_t$ in terms of powers
of the Wolfenstein expansion parameter $\lambda$.

\begin{table}[htb]
\caption[]{
Order of magnitude of CKM parameters relevant for the various decays,
expressed in powers of the Wolfenstein parameter $\lambda=0.22$. In the
case of $K_L\to\pi^0\nu\bar\nu$, which is CP-violating, only the
imaginary parts of $\lambda_{c, t}$ contribute.
\label{tab:lambdaexp}}
\begin{center}
\begin{tabular}{|r|c|c|c|c|}
&$\kpn$&$K_L\to\pi^0\nu\bar\nu$&$B\to X_s\nu\bar\nu$&
$B\to X_d\nu\bar\nu$\\
&$(\klm)_{SD}$&&$B_s\to l^+l^-$&$B_d\to l^+l^-$\\  \hline\hline
$\lambda_c$&$\sim\lambda$&(${\rm Im}\lambda_c\sim\lambda^5$)&
$\sim\lambda^2$&$\sim\lambda^3$\\  \hline
$\lambda_t$&$\sim\lambda^5$&(${\rm Im}\lambda_t\sim\lambda^5$)&
$\sim\lambda^2$&$\sim\lambda^3$
\end{tabular}
\end{center}
\end{table}

For the CP-violating decay $K_L\to\pi^0\nu\bar\nu$ and the B-decays
the CKM factors $\lambda_c$ and $\lambda_t$ have the same order of
magnitude. In view of $F(x_c)\ll F(x_t)$ the charm contribution is
therefore negligible and these decays are entirely determined by the
top sector.\\
For \kpnn and $(\klm)_{SD}$ on the other hand $\lambda_t$ is
suppressed compared to $\lambda_c$ by a factor of order
$\ord(\lambda^4)\approx\ord(10^{-3})$, which roughly compensates
for the $\ord(10^3)$ enhancement of $F(x_t)$ over $F(x_c)$. Hence
the top and charm contributions have the same order of magnitude and
must both be taken into account.

In principle, as far as flavordynamics is concerned, the top and the
charm sector have the same structure. The only difference comes from
the quark masses. However, this difference has striking implications
for the detailed formalism necessary to treat the strong interaction
corrections. We have $m_t/M_W=\ord(1)$ and $m_c/M_W\ll 1$.
Correspondingly the QCD coupling $\as$ is also somewhat smaller at
$m_t$ than at $m_c$. For the charm contribution this implies that one
can work to lowest order in the mass ratio $m_c/M_W$. On the other
hand, for the same reason, logarithmic QCD corrections
$\sim\as\ln M_W/m_c$ are large and have to be resummed to all
orders in perturbation theory by renormalization group methods.
On the contrary, no large logarithms are present in the top sector, so that
ordinary perturbation theory is applicable, but all orders in $m_t/M_W$ have
to be taken into account. In fact we see that from the
point of view of QCD corrections the charm and top contributions are
quite ``complementary'' to each other, representing in a sense
opposite limiting cases.\\
We are now ready to list the explicit expressions for the effective
hamiltonians.

\subsection{The Decay \kpnn}
            \label{sec:HeffRareKB:kpnn}
\subsubsection{The Next-to-Leading Order Effective Hamiltonian}
               \label{sec:HeffRareKB:kpnn:heff}
The final result for the effective hamiltonian inducing \kpnn can
be written as
\begin{equation}\label{hkpn} {\cal H}_{eff}={G_F \over{\sqrt 2}}{\alpha\over 2\pi \sin^2\Theta_W}
 \sum_{l=e,\mu,\tau}\left( V^{\ast}_{cs}V_{cd} X^l_{NL}+
V^{\ast}_{ts}V_{td} X(x_t)\right)
 (\bar sd)_{V-A}(\bar\nu_l\nu_l)_{V-A} \, .
\end{equation}
The index $l$=$e$, $\mu$, $\tau$ denotes the lepton flavor.
The dependence on the charged lepton mass, resulting from the box-graph,
is negligible for the top contribution. In the charm sector this is the
case only for the electron and the muon, but not for the $\tau$-lepton.
\\
The function $X(x)$, relevant for the top part, reads
to $\ord(\as)$ and to all orders in $x=m^2/M^2_W$
\begin{equation}\label{xx} X(x)=X_0(x)+\aspi X_1(x) \end{equation}
with \cite{inamilim:81}
\begin{equation}\label{xx0} X_0(x)={x\over 8}\left[ -{2+x\over 1-x}+{3x-6\over (1-x)^2}\ln x\right] \end{equation}
and the QCD correction \cite{buchallaburas:93b}
\begin{eqnarray}\label{xx1}
X_1(x)=&-&{23x+5x^2-4x^3\over 3(1-x)^2}+{x-11x^2+x^3+x^4\over (1-x)^3}\ln x
\nonumber\\
&+&{8x+4x^2+x^3-x^4\over 2(1-x)^3}\ln^2 x-{4x-x^3\over (1-x)^2}L_2(1-x)
\nonumber\\
&+&8x{\partial X_0(x)\over\partial x}\ln x_\mu
\end{eqnarray}
where $x_\mu=\mu^2/M^2_W$ with $\mu=\ord(m_t)$ and
\begin{equation}\label{l2} L_2(1-x)=\int^x_1 dt {\ln t\over 1-t}   \end{equation}
The $\mu$-dependence in the last term in (\ref{xx1}) cancels to the
order considered the $\mu$-dependence of the leading term $X_0(x(\mu))$.
\\
The expression corresponding to $X(x_t)$ in the charm sector is the function
$X^l_{NL}$. It results from the RG calculation in NLLA and is given
as follows:
\begin{equation}\label{xlnl}X^l_{NL}=C_{NL}-4 B^{(1/2)}_{NL}  \end{equation}
$C_{NL}$ and $B^{(1/2)}_{NL}$ correspond to the $Z^0$-penguin and the
box-type contribution, respectively. One has \cite{buchallaburas:94}
\begin{eqnarray}\label{cnln}
\lefteqn{C_{NL}={x(m)\over 32}K^{{24\over 25}}_c\left[\left({48\over 7}K_++
 {24\over 11}K_--{696\over 77}K_{33}\right)\left({4\pi\over\as(\mu)}+
 {15212\over 1875} (1-K^{-1}_c)\right)\right.}\nonumber\\
&&+\left(1-\ln{\mu^2\over m^2}\right)(16K_+-8K_-)-{1176244\over 13125}K_+-
 {2302\over 6875}K_-+{3529184\over 48125}K_{33} \nonumber\\
&&+\left. K\left({56248\over 4375}K_+-{81448\over 6875}K_-+{4563698\over 144375}K_{33}
  \right)\right]
\end{eqnarray}
where
\begin{equation}\label{kkc} K={\as(M_W)\over\as(\mu)}\qquad
  K_c={\as(\mu)\over\as(m)}  \end{equation}
\begin{equation}\label{kkn} K_+=K^{{6\over 25}}\qquad K_-=K^{{-12\over 25}}\qquad
          K_{33}=K^{{-1\over 25}}  \end{equation}
\begin{eqnarray}\label{bnln}
\lefteqn{B^{(1/2)}_{NL}={x(m)\over 4}K^{24\over 25}_c\left[ 3(1-K_2)\left(
 {4\pi\over\as(\mu)}+{15212\over 1875}(1-K^{-1}_c)\right)\right.}\nonumber\\
&&-\left.\ln{\mu^2\over m^2}-
  {r\ln r\over 1-r}-{305\over 12}+{15212\over 625}K_2+{15581\over 7500}K K_2
  \right]
\end{eqnarray}
Here $K_2=K^{-1/25}$, $m=m_c$, $r=m^2_l/m^2_c(\mu)$ and $m_l$ is the
lepton mass.  We will at times omit the index $l$ of $X^l_{NL}$.  In
(\ref{cnln}) -- (\ref{bnln}) the scale is $\mu=\ord(m_c)$.  The
two-loop expression for $\as(\mu)$ is given in \eqn{amu}.  Again -- to
the considered order -- the explicit $\ln(\mu^2/m^2)$ terms in
(\ref{cnln}) and (\ref{bnln}) cancel the $\mu$-dependence of the
leading terms.
\\
These formulae give the complete next-to-leading effective hamiltonian
for $\kpn$. The leading order expressions \cite{novikovetal:77},
\cite{ellishagelin:83}, \cite{dibetal:91}, \cite{buchallaetal:91} are
obtained by replacing $X(x_t)\to X_0(x_t)$ and $X^l_{NL}\to X_L$ with
$X_L$ found from (\ref{cnln}) and (\ref{bnln}) by retaining there only
the $1/\as(\mu)$ terms. In LLA the one-loop expression should be used
for $\as$.  This amounts to setting $\beta_1=0$ in \eqn{amu}.  The
numerical values for $X_{NL}$ for $\mu = \mc$ and several values of
$\Lms^{(4)}$ and $\mc(\mc)$ are given in table \ref{tab:xnlnum}. The
$\mu$ dependence will be discussed in part three.

\begin{table}[htb]
\caption[]{The functions $X^e_{NL}$ and $X^\tau_{NL}$
for various $\Lms^{(4)}$ and $\mc$.
\label{tab:xnlnum}}.
\begin{center}
\begin{tabular}{|c|c|c|c|c|c|c|}
& \multicolumn{3}{c|}{$X^e_{NL}/10^{-4}$} &
  \multicolumn{3}{c|}{$X^\tau_{NL}/10^{-4}$} \\
\hline
$\Lms^{(4)}\ [\mev]\;\backslash\;\mc\ [\gev]$ &
1.25 & 1.30 & 1.35 & 1.25 & 1.30 & 1.35 \\
\hline
215 & 10.55  & 11.40  & 12.28 & 7.16 & 7.86 & 8.59 \\
325 &  9.71  & 10.55  & 11.41 & 6.32 & 7.01 & 7.72 \\
435 &  8.75  &  9.59  & 10.45 & 5.37 & 6.05 & 6.76
\end{tabular}
\end{center}
\end{table}

\subsubsection{Z-Penguin and Box Contribution in the Top Sector}
               \label{sec:HeffRareKB:kpnn:Zptop}
For completeness we give here in addition the expressions for the
$Z^0$-penguin function $C(x)$ and the box function $B(x,1/2)$ separately,
which contribute to $X(x)$ in \eqn{xx} according to
\begin{equation}\label{xc4b} X(x)=C(x) - 4 \, B(x,1/2)  \end{equation}
The functions $C$ and $B$ depend on the gauge of the $W$-boson.
In 't Hooft--Feynman-gauge ($\xi=1$) they read
\begin{equation}\label{cx01} C(x)=C_0(x)+{\as\over 4\pi} C_1(x)   \end{equation}
where \cite{inamilim:81}
\begin{equation}\label{cx0} C_0(x) = {x \over 8} \left[{6-x\over 1-x} +
   {3x+2\over (1-x)^2} \ln x\right]   \end{equation}
and \cite{buchallaburas:93a}
\begin{eqnarray}\label{cx1}
C_1(x) &=& {29x + 7x^2+4x^3 \over 3(1-x)^2}
           - {x - 35x^2 -3 x^3 -3 x^4\over 3(1-x)^3} \ln x\nonumber\\
       & &- {20x^2 - x^3+x^4\over 2(1-x)^3} \ln^2 x
           + {4x+x^3\over (1-x)^2} L_2 (1-x)\nonumber\\
       & &+ 8x {\partial C_0(x)\over \partial x} \ln x_\mu
\end{eqnarray}
Similarly
\begin{equation}\label{bx01p}
B(x, 1/2) = B_0(x) + {\as\over 4\pi} B_1 (x, 1/2)   \end{equation}
with the one-loop function \cite{inamilim:81}
\begin{equation}\label{bx0}
B_0(x) = {1\over 4}\left[{x\over 1-x} + {x\over (1-x)^2} \ln x\right] \end{equation}
and \cite{buchallaburas:93b}
\begin{eqnarray}\label{bx1p}
B_1(x, 1/2)&=&{13x + 3x^2\over 3(1-x)^2} -
{x-17x^2\over 3(1-x)^3} \ln x
-  {x+3x^2\over (1-x)^3}\ln^2x + {2x\over(1-x)^2} L_2(1-x)\nonumber\\
& &+ 8x {\partial B_0(x)\over \partial x} \ln x_\mu
\end{eqnarray}
The gauge dependence of $C$ and $B$ is canceled in the
combination $X$ \eqn{xc4b}. The second argument in $B(x,1/2)$ indicates the
weak isospin of the external leptons (the neutrinos in this case).

\subsubsection{The $Z$-Penguin Contribution in the Charm Sector}
               \label{sec:HeffRareKB:kpnn:Zpcharm}
In the next two paragraphs we would like to summarize the essential
ingredients of the RG calculation for the charm sector leading to
\eqn{cnln} and \eqn{bnln}. In particular we present the operators involved,
the initial values for the RG evolution of the Wilson coefficients and
the required two-loop anomalous dimensions. We will first treat
the $Z^0$-penguin contribution \eqn{cnln} and discuss the box part \eqn{bnln}
subsequently. Further details can be found in \cite{buchallaburas:94}.

At renormalization scales of the order $\ord(M_W)$ and after
integrating out the $W$- and $Z$-bosons the effective hamiltonian
responsible for the $Z^0$-penguin contribution of the charm sector is given
by
\begin{equation}\label{hzop}{\cal H}^{(Z)}_{eff,c}={G_F \over{\sqrt 2}}
{\alpha\over 2\pi \sin^2\Theta_W}\lambda_c{\pi^2\over 2 M^2_W}
\left( v_+ O_+ +v_- O_- +v_3 Q\right) \end{equation}
where the operator basis is
\begin{equation}\label{o1} O_1=
   -i\int d^4x\ T\left((\bar s_ic_j)_{V-A}(\bar c_jd_i)_{V-A}\right)(x)\
       \left((\bar c_kc_k)_{V-A}(\bar\nu\nu)_{V-A}\right)(0)\ -
       \{c\to u\}    \end{equation}
\begin{equation}\label{o2} O_2=
   -i\int d^4x\ T\left((\bar s_ic_i)_{V-A}(\bar c_jd_j)_{V-A}\right)(x)\
       \left((\bar c_kc_k)_{V-A}(\bar\nu\nu)_{V-A}\right)(0)\ -
       \{c\to u\}    \end{equation}
\begin{equation}\label{opm} O_\pm ={1\over 2}(O_2 \pm O_1) \end{equation}
\begin{equation}\label{qnu} Q={m^2\over g^2} (\bar sd)_{V-A}(\bar\nu\nu)_{V-A}   \end{equation}
The Wilson coefficients at $\mu=M_W$ are
($\vec v^T\equiv(v_+,v_-,v_3)$)
\begin{equation}\label{vmw} \vec v(M_W)=\vec v^{(0)}+\frac{\as(M_W)}{4 \pi}
\vec v^{(1)}
\end{equation}
\begin{equation}\label{v0} {\vec v^{(0)T}}=(1,1,0)  \end{equation}
\begin{equation}\label{v1} {\vec v^{(1)T}}=(B_+,B_-,B_3)  \end{equation}
where in the NDR scheme ($\overline{MS}$, anticommuting $\gamma_5$
in $D\not=4$ dimensions)
\begin{equation}\label{bpm3} B_\pm=\pm 11{N\mp 1\over 2N}\qquad B_3=16  \end{equation}
with $N$ denoting the number of colors.\\
In the basis of operators $\{ O_+,O_-,Q\}$ the matrix of anomalous
dimensions has the form
\begin{equation}\label{gz} \gamma =
 \left(\begin{array}{ccc} \gamma_+ & 0 & \gamma_{+3} \\
                           0 & \gamma_- & \gamma_{-3} \\
                           0 & 0 & \gamma_{33}
    \end{array}\right)   \end{equation}
with the perturbative expansion
\begin{equation}\label{ga2}
\gamma(\as)=\aspi \gamma^{(0)}+\left(\aspi\right)^2\gamma^{(1)}
\end{equation}
The nonvanishing entries of the anomalous dimension matrix read
\begin{equation}\label{gzij}
\begin{tabular}{lcl}
$\gamma^{(0)}_{33}=2(\gamma_{m0}-\beta_0)$ &\qquad\qquad&
   $\gamma^{(1)}_{33}=2(\gamma_{m1}-\beta_1)$ \\  & & \\
$\gamma^{(0)}_\pm=\pm 6{N\mp 1\over N}$ &  &
   $\gamma^{(1)}_\pm={N\mp 1\over 2N}\left(-21\pm{57\over N}\mp{19\over 3}N\pm{4\over 3}f
     \right)$ \\                                 & & \\
$\gamma^{(0)}_{\pm 3}=\pm 8(N\pm 1)$ &  &
   $\gamma^{(1)}_{\pm 3}=C_F(\pm 88 N-48)$ \\
\end{tabular}  \end{equation}
where $\gamma_{m0}$, $\gamma_{m1}$, $\beta_0$, $\beta_1$ can be found
in \eqn{gm01} and \eqn{b0b1}, respectively. The expressions
$\gamma^{(1)}$ refer to the NDR scheme, consistent with the scheme
chosen for $\vec v(M_W)$. Following the general method for the solution
of the RG equations explained in section~\ref{sec:basicform:wc:rgf}, we
can compute the Wilson coefficients $\vec v(\mu)$ at a scale
$\mu=\ord(m_c)$. It is convenient to work in an effective four-flavor
theory ($f=4$) in the full range of the RG evolution from $M_W$ down to
$\mu$. The possible inclusion of a $b$-quark threshold would change the
result for $X_{NL}$ by not more than 0.1\% and can therefore be safely
neglected.\\ After integrating out the charm quark at the scale
$\mu=\ord(m_c)$, the $Z^0$-penguin part of the charm contribution to
the effective hamiltonian becomes
\begin{equation}\label{hzc} {\cal H}^{(Z)}_{eff,c}={G_F \over{\sqrt 2}}
  {\alpha\over 2\pi \sin^2\Theta_W} \lambda_c\ C_{NL}\
   (\bar sd)_{V-A}(\bar\nu\nu)_{V-A}  \end{equation}
\begin{equation}\label{cnl}
C_{NL}={x(\mu)\over 32}\left[{1\over 2}\left(1-\ln{\mu^2\over m^2}\right)
  \left(\gamma^{(0)}_{+3} K_++\gamma^{(0)}_{-3} K_-\right)+
  {4\pi\over\as(\mu)}v_3(\mu)\right] \, .
\end{equation}
The explicit expression for $v_3(\mu)$ as obtained from solving the RG
equation is given in \cite{buchallaburas:94}. Inserting this
expression in \eqn{cnl}, expressing the charm quark mass $m(\mu)$ in
terms of $m(m)$ and setting $N=3$, $f=4$, we finally end up with
\eqn{cnln}.

\subsubsection{The Box Contribution in the Charm Sector}
               \label{sec:HeffRareKB:kpnn:boxcharm}
The RG analysis for the box contribution proceeds in analogy to the
$Z^0$-penguin case. The box part is even somewhat simpler. When the
$W$ boson is integrated out, the hamiltonian based on the box diagram
reads
\begin{equation}\label{hbop}{\cal H}^{(B)}_{eff,c}=-{G_F \over{\sqrt 2}}
{\alpha\over 2\pi \sin^2\Theta_W}\lambda_c\left(-{\pi^2\over M^2_W}\right)
\left( c_1 O +c_2 Q\right) \end{equation}
\begin{equation}\label{ob} O=
   -i\int d^4x\ T\left((\bar sc)_{V-A}(\bar \nu l)_{V-A}\right)(x)\
       \left((\bar l\nu)_{V-A}(\bar cd)_{V-A}\right)(0)\ -
       \{c\to u\}    \end{equation}
with $Q$ alread given in \eqn{qnu}.
The Wilson coefficients at $M_W$ in the NDR scheme are given by
\begin{equation}\label{cmw}
\vec c^T(M_W)\equiv (c_1(M_W),c_2(M_W))=(1,0)+\frac{\as(M_W)}{4 \pi} (0,B_2)
  \qquad B_2=-36  \end{equation}
In the operator basis $\{O,Q\}$ the anomalous dimension matrix  has the form
\begin{equation}\label{gb} \gamma =
 \left(\begin{array}{cc}   0 & \gamma_{12} \\
                           0 & \gamma_{22}
    \end{array}\right)   \end{equation}
When expanded as
\begin{equation}\label{gb2}
\gamma=\aspi \gamma^{(0)}+\left(\aspi\right)^2\gamma^{(1)}  \end{equation}
the non-zero elements read (NDR scheme for $\gamma^{(1)}$)
\begin{equation}\label{gbij}
\begin{tabular}{lcl}
$\gamma^{(0)}_{22}=2(\gamma_{m0}-\beta_0)$ &\qquad\qquad&
   $\gamma^{(1)}_{22}=2(\gamma_{m1}-\beta_1)$ \\  & & \\
$\gamma^{(0)}_{12}=-32$ &  &
   $\gamma^{(1)}_{12}=80C_F$ \\
\end{tabular}  \end{equation}
Finally, after integrating out charm at $\mu=\ord(m_c)$
\begin{equation}\label{hbcn} {\cal H}^{(B)}_{eff,c}=-{G_F \over{\sqrt 2}}
  {\alpha\over 2\pi \sin^2\Theta_W} \lambda_c\ 4 B^{(1/2)}_{NL}\
   (\bar sd)_{V-A}(\bar\nu_l\nu_l)_{V-A}  \end{equation}
\begin{equation}\label{bnl} B^{(1/2)}_{NL}=-{x(\mu)\over 64}\left[
16\left(\ln{\mu^2\over m^2}+
{5\over 4}+{r\ln r\over 1-r}\right) + \frac{4 \pi}{\as(\mu)} c_2(\mu) \right]
\end{equation}
(\ref{hbcn}) is written here for one neutrino flavor. The index $(1/2)$
refers to the weak isospin of the final state leptons.
From this result \eqn{bnln} can be derived ($N=3$, $f=4$).
The explicit expression for $c_2(\mu)$ can be found in
\cite{buchallaburas:94}.

Although Wilson coefficients and anomalous dimensions depend on the
renormalization scheme, the final results in \eqn{cnln} and \eqn{bnln}
are free from this dependence. The argument proceeds as in the
general case presented in section~\ref{sec:basicform:wc:rgdep}.

\subsubsection{Discussion}
               \label{sec:HeffRareKB:kpnn:disc}
It is instructive to consider furthermore the function $X(x)$ in the
limiting case of small masses ($x\ll 1$), keeping only terms linear
in $x$ and including $\ord(\as)$ corrections:
\begin{equation}\label{xxc} X(x)\doteq -{3\over 4}x\ln x-{1\over 4}x+
     \aspi\left(-2x\ln^2 x-7x\ln x-{23+2\pi^2\over 3}x\right)  \end{equation}
This simple and transparent expression can be regarded as a common
limiting case of the top- and the charm contribution: On the one hand
it follows from keeping only terms linear in $x$ in the top function
\eqn{xx}. On the other hand it can be obtained
(up to the last term in \eqn{xxc} which is $\ord(\as x)$ and
goes beyond the NLLA) from expanding $X_{NL}$ \eqn{xlnl} (for $m_l=0$)
to first order in $\as$.\\
This exercise provides one with a nice cross-check between the
rather different looking functions $X_{NL}$ and $X(x_t)$ of the
charm- and the top sector. Viewed the other way around, \eqn{xxc} may
serve to further illustrate the complementary character of the
calculations necessary in each of the two sectors. $X(x_t)$ is the
generalization of \eqn{xxc} that includes all the higher order mass terms.
$X_{NL}$ on the other hand generalizes \eqn{xxc} to include all the
leading logarithmic, $\ord(x \alpha^n_s \ln^{n+1}x)$, as well as
the next-to-leading logarithmic $\ord(x \alpha^n_s \ln^{n}x)$
corrections, to all orders $n$ in $\as$. Of these only the terms
with $n=0$ and $n=1$ are contained in \eqn{xxc}.\\
Applying the approximation \eqn{xxc} to the charm part directly, one can
furthermore convince oneself, that the $\ord(\as)$
correction term would amount to more than 50\% of the lowest order
result. This observation illustrates very clearly the necessity to
go beyond straightforward perturbation theory and to employ the
RG summation technique. The importance of going still to
next-to-leading order accuracy in the RG calculation is suggested by the
relatively large size of the $\ord(x \as \ln x)$ term.
Note also, that formally the non-logarithmic mass term $(-x/4)$ in
\eqn{xxc} is a next-to-leading order effect in the framework of
RG improved perturbation theory. The same is true for the dependence
on the charged lepton mass, which can be taken into account
consistently only in NLLA.

A crucial issue is the residual dependence of the functions $X_{NL}$
and $X(x_t)$ on the corresponding renormalization scales $\mu_c$ and
$\mu_t$. Since the quark current operator in \eqn{hnr} has no
anomalous dimension, its matrix elements do not depend on the
renormalization scale. The same must then hold for the coefficient
functions $X_{NL}$ and $X(x_t)$. However, in practice this is only
true up to terms of the neglected order in perturbation theory.
The resulting scale ambiguities represent the theoretical uncertainties
present in the calculation of the short-distance dominated processes
under discussion. They can be systematically reduced by going to
higher orders in the analysis. In table~\ref{tab:scaledep} we compare
the order of the residual scale dependence in LLA and in NLLA for the
top- and the charm contribution.

\begin{table}[htb]
\caption[]{Residual scale ambiguity in the top and charm sector
in LLA and NLLA.
\label{tab:scaledep}}
\begin{center}
\begin{tabular}{|r|c|c|}
&Top Sector ($\mu_t=\ord(m_t)$) & Charm Sector ($\mu_c=\ord(m_c)$)
\\  \hline
LLA & $\ord(\as)$&$\ord(x_c)$ \\  \hline
NLLA & $\ord(\alpha^2_s)$&$\ord(\as x_c)$
\end{tabular}
\end{center}
\end{table}

For numerical investigations we shall use $1\gev\leq\mu_c\leq 3\gev$
for the renormalization scale $\mu_c=\ord(m_c)$ in the charm sector.
Similarly, in the case of the top contribution we choose
$\mu_t=\ord(m_t)$ in the range $100\gev\leq\mu_t\leq 300\gev$ for
$m_t=170\gev$. Then, comparing LLA and NLLA, the theoretical
uncertainty due to scale ambiguity is typically reduced from
$\ord(10\%)$ to $\ord(1\%)$ in the top sector and from more than 50\%
to less than 20\% in the charm sector. Here the quoted percentages refer
to the total variation $(X_{max}-X_{min})/X_{central}$ of the functions
$X(x_t)$ or $X_{NL}$ within the range of scales considered.
Phenomenological implications of this gain in accuracy will be
discussed in section~\ref{sec:Kpnn}.

\subsection{The Decay $(\klm)_{SD}$}
            \label{sec:HeffRareKB:klmm}
\subsubsection{The Next-to-Leading Order Effective Hamiltonian}
               \label{sec:HeffRareKB:klmm:heff}
The analysis of $(\klm)_{SD}$ proceeds in essentially the same
manner as for $\kpn$. The only difference is introduced through the
reversed lepton line in the box contribution. In particular there is
no lepton mass dependence, since only massless neutrinos appear as
virtual leptons in the box diagram.\\
The effective hamiltonian in next-to-leading order can be written as
follows:
\begin{equation}\label{hklm}{\cal H}_{eff}=-{G_F \over{\sqrt 2}}{\alpha\over 2\pi \sin^2\Theta_W}
 \left( V^{\ast}_{cs}V_{cd} Y_{NL}+
V^{\ast}_{ts}V_{td} Y(x_t)\right)
 (\bar sd)_{V-A}(\bar\mu\mu)_{V-A} + h.c. \end{equation}
The function $Y(x)$ is given by
\begin{equation}\label{yy}
Y(x) = Y_0(x) + \aspi Y_1(x)\end{equation}
where \cite{inamilim:81}
\begin{equation}\label{yy0}
Y_0(x) = {x\over 8}\left[{4-x\over 1-x}+{3x\over (1-x)^2}\ln x\right]
\end{equation}
and \cite{buchallaburas:93b}
\begin{eqnarray}\label{yy1}
Y_1(x) = &&{4x + 16 x^2 + 4x^3 \over 3(1-x)^2} -
           {4x - 10x^2-x^3-x^4\over (1-x)^3} \ln x\nonumber\\
         &+&{2x - 14x^2 + x^3 - x^4\over 2(1-x)^3} \ln^2 x
           + {2x + x^3\over (1-x)^2} L_2(1-x)\nonumber\\
         &+&8x {\partial Y_0(x) \over \partial x} \ln x_\mu
\end{eqnarray}
The RG expression $Y_{NL}$ representing the charm contribution reads
\begin{equation}\label{ynl} Y_{NL}=C_{NL}-B^{(-1/2)}_{NL}  \end{equation}
where $C_{NL}$ is the $Z^0$-penguin part given in (\ref{cnln}) and
$B^{(-1/2)}_{NL}$ is the box contribution in the charm sector, relevant
for the case of final state leptons with weak isospin $T_3=-1/2$.
One has \cite{buchallaburas:94}
\begin{eqnarray}\label{bmnln}
\lefteqn{B^{(-1/2)}_{NL}={x(m)\over 4}K^{24\over 25}_c\left[ 3(1-K_2)\left(
 {4\pi\over\as(\mu)}+{15212\over 1875}(1-K^{-1}_c)\right)\right.}\nonumber\\
&&-\left.\ln{\mu^2\over m^2}-
  {329\over 12}+{15212\over 625}K_2+{30581\over 7500}K K_2
  \right]
\end{eqnarray}
Note the simple relation to $B^{(1/2)}_{NL}$ in (\ref{bnln}) (for
$r=0$)
\begin{equation}\label{dbnl}
B^{(-1/2)}_{NL}-B^{(1/2)}_{NL}={x(m)\over 2}K^{24\over 25}_c (K K_2-1)
\end{equation}
More details on the RG analysis in this case may be found in
\cite{buchallaburas:94}.

\begin{table}[htb]
\caption[]{The function $Y_{NL}$ for various $\Lms^{(4)}$ and $\mc$.
\label{tab:ynlnum}}.
\begin{center}
\begin{tabular}{|c|c|c|c|}
&\multicolumn{3}{c|}{$Y_{NL}/10^{-4}$}\\
\hline
$\Lms^{(4)}\ [\mev]\;\backslash\;\mc\ [\gev]$ & 1.25 & 1.30 & 1.35 \\
\hline
215 & 3.09 & 3.31 & 3.53 \\
325 & 3.27 & 3.50 & 3.73 \\
435 & 3.40 & 3.64 & 3.89
\end{tabular}
\end{center}
\end{table}

\subsubsection{Discussion}
               \label{sec:HeffRareKB:klmm:disc}
The gauge independent function $Y$ can be decomposed into the
$Z^0$-penguin- and the box contribution
\begin{equation}\label{yxcb}  Y(x)=C(x)-B(x, -1/2)    \end{equation}
In Feynman-gauge for the $W$ boson $C(x)$ is given in \eqn{cx01}.
In the same gauge the box contribution reads
\begin{equation}\label{bx01m}
B(x,-1/2)=B_0(x)+{\as\over 4\pi} B_1(x,-1/2)
\end{equation}
with $B_0(x)$ from \eqn{bx0} and
\begin{eqnarray}\label{bx1m}
B_1(x, -1/2)&=&{25x-9x^2\over 3(1-x)^2}
+ {11x + 5x^2\over 3(1-x)^3}\ln x
- {x+3x^2\over(1-x)^3} \ln^2x + {2x\over (1-x)^2}L_2(1-x)\nonumber \\
& &+ 8x {\partial B_0(x)\over \partial x} \ln x_\mu
\end{eqnarray}
The equality $B(x,1/2)=B(x,-1/2)$ at the one-loop level is a
particular property of the Feynman-gauge. It is violated by
$\ord(\as)$ corrections. There is however a very simple
relation between $B_1(x,1/2)$ and $B_1(x,-1/2)$
\begin{equation}\label{b1pm}
B_1(x, -1/2)- B_1(x, 1/2) = 16 B_0(x)   \end{equation}

We add a few comments on the most important differences between
$Y_{NL}$ and $X_{NL}$.\\
Expanding $Y_{NL}$ to first order in $\as$ we find
\begin{equation}\label{ylin}
Y_{NL}\doteq {1\over 2}x +{\as\over 4\pi} x \ln^2x +\ord(\as x)  \end{equation}
In contrast to $X_{NL}$ both the terms of $\ord(x\ln x)$ and of
$\ord(\as x\ln x)$ are absent in $Y_{NL}$. The cancellation
of the leading $\ord(x\ln x)$ terms between $Z^0$-penguin and box
contribution implies that the non-leading $\ord(x)$ term plays a
much bigger role for $Y_{NL}$. A second consequence are the increased
importance of QCD effects and the related larger sensitivity to $\mu_c$,
resulting in a bigger theoretical uncertainty for $Y_{NL}$ than it
happened to be the case for $X_{NL}$. In addition, whereas $X(x_c)$
is suppressed by $\sim 30\%$ through QCD effects, the zeroth order
expression for $Y$ is enhanced by as much as a factor of about 2.5.
Nevertheless, QCD corrections included, $X_{NL}$ still exceeds $Y_{NL}$
by a factor of four, so that $Y_{NL}$ is less important for
$(\klm)_{SD}$ than $X_{NL}$ is for $\kpn$. Although the impact of the
bigger uncertainties in $Y_{NL}$ is thus somewhat reduced in the
complete result for $(\klm)_{SD}$, the remaining theoretical uncertainty
due to scale ambiguity is still larger than for $\kpn$. It will be
investigated numerically in section \ref{sec:KLmm}. The numerical
values for $Y_{NL}$ for $\mu=\mc$ and several values of $\Lms^{(4)}$
and $\mc(\mc)$ are given in table \ref{tab:ynlnum}.

\subsection{The Decays $K_L\to\pi^0\nu\bar\nu$, $B\to X_{s,d}\nu\bar\nu$ and
            $B_{s,d}\to l^+l^-$}
            \label{sec:HeffRareKB:klpinn}
After the above discussion it is easy to write down also the effective
hamiltonians for $K_L\to\pi^0\nu\bar\nu$, $B\to X_{s,d}\nu\bar\nu$
and $B_{s,d}\to l^+l^-$. As we have seen, only the top contribution is
important in these cases and we can write
\begin{equation}\label{hxnu}
{\cal H}_{eff} = {G_F\over \sqrt 2} {\alpha \over
2\pi \sin^2 \Theta_W} V^\ast_{tn} V_{tn^\prime}
X (x_t) (\bar nn^\prime)_{V-A} (\bar\nu\nu)_{V-A} + h.c.   \end{equation}
for the decays $K_L\to\pi^0\nu\bar\nu$, $B\to X_s\nu\bar\nu$
and $B\to X_d\nu\bar\nu$, with $(\bar nn^\prime)=(\bar sd)$, $(\bar bs)$,
$(\bar bd)$ respectively. Similarly
\begin{equation}\label{hyll}
{\cal H}_{eff} = -{G_F\over \sqrt 2} {\alpha \over
2\pi \sin^2 \Theta_W} V^\ast_{tn} V_{tn^\prime}
Y (x_t) (\bar nn^\prime)_{V-A} (\bar ll)_{V-A} + h.c.   \end{equation}
for $B_s\to l^+l^-$ and $B_d\to l^+l^-$, with
$(\bar nn^\prime)=(\bar bs)$, $(\bar bd)$.
The functions $X$, $Y$ are given in \eqn{xx} and \eqn{yy}.

