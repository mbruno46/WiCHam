\section{Comments on Input Parameters}
         \label{sec:inputparams}
The phenomenology of weak decays depends sensitively on a number of
input parameters. We have collected the numerical values of these
parameters in appendix \ref{app:numinput}. To this end we have
frequently used the values quoted by \cite{particledata:94}. The basis
for our choice of the numerical values for various non-perturbative
parameters, such as $B_K$ or $F_B$, will be given in the course of our
presentation. In certain cases, like the B-meson life-times and the
size of the $B^0_d-\bar B^0_d$ mixing, for which the experimental world
averages change constantly we have chosen values, which are in the ball
park of those presented at various conferences and workshops during the
summer of 1995. Here we would like to comment briefly on three
important parameters:  $\left| V_{cb} \right|$, $\left| V_{ub}/V_{cb}
\right|$ and $\mt$.  The importance of these parameters lies in the
fact that many branching ratios and also the CP violation in the
Standard Model depend sensitively on them.

\subsection{CKM Element $\left| V_{cb} \right|$}
         \label{subsec:inputparams:Vcb}
During the last two years there has been a considerable progress made
by experimentalists \cite{patterson:95} and theorists in the extraction of
$\left| V_{cb} \right|$ from the exclusive and inclusive B-decays.  In
these investigations the HQET in the case of exclusive decays and the
Heavy Quark Expansions for inclusive decays played a considerable
role.  In particular we would like to mention the important papers
\cite{neubert:94b}, \cite{shifmanetal:95} and \cite{braunetal:95} on
the basis of which one is entitled to use:
\begin{equation}
\label{eq:Vcberr}
\left| V_{cb} \right|=0.040\pm0.003\quad =>\quad A=0.82\pm 0.06
\end{equation}
This should be compared with an error of $\pm 0.006$  for 
$\left| V_{cb} \right|$ quoted still in 1993. The corresponding
reduction of the error in $A$ by a factor of 2 has important
consequences for the phenomenology of weak decays.

\subsection{CKM Element Ratio $\left| V_{ub}/V_{cb} \right|$}
         \label{subsec:inputparams:Vubcb}
Here the situation is much worse and the value
\begin{equation}
\label{eq:Vubcberr}
\left| \frac{V_{ub}}{V_{cb}} \right|=0.08\pm0.02
\end{equation}
quoted by \cite{particledata:94} appears to be still valid. There is a
hope that the error could be reduced by a factor of 2 to 4 in the
coming years both due to the theory \cite{braunetal:95} and the recent
CLEO measurements of the exclusive semileptonic decays $B \to
(\pi,\varrho)l\nu_l$ \cite{thorndike:95}.

\subsection{Top Quark Mass $\mt$}
         \label{subsec:inputparams:mt}
Next it is important to stress that the discovery of the top quark
\cite{abeetal:94a}, \cite{abeetal:94b}, \cite{abeetal:95}, \cite{D0:95}
and its mass measurement had an important impact on the field of rare
decays and CP violation reducing considerably one potential
uncertainty. It is however important to keep in mind that the parameter
$\mt$, the top quark mass, used in weak decays is not equal to the one
used in the electroweak precision studies at LEP or SLD. In the latter
investigations the so-called pole mass is used, whereas in all the NLO
calculations listed in table \ref{tab:processes} and used in this
review, $\mt$ refers to the running current top quark mass normalized
at $\mu=\mt$:  $\bar \mt(\mt)$. One has
\begin{equation}
\mt^{\rm (pole)}=\bar \mt(\mt)
\left[ 1+\frac{4}{3}\frac{\as(\mt)}{\pi}\right]
\label{eq:mtpolebar}
\end{equation}
so that for $\mt={\cal O}(170\gev)$, $\bar \mt(\mt)$ is typically by
$8\gev$ smaller than $\mt^{\rm (Pole)}$. This difference will matter in
a few years.

In principle any definition $\bar \mt(\mu_t)$ with $\mu_t=\ord(\mt)$
could be used. In the leading order this arbitrariness in the choice of
$\mu_t$ introduces a potential theoretical uncertainty in those
branching ratios which depend sensitively on the top quark mass. The
inclusion of NLO corrections reduces this uncertainty considerably so
that the resulting branching ratios remain essentially independent of
the choice of $\mu_t$. We have discussed this point already in previous
sections. Numerical examples will be given in this part below.  The
choice $\mu_t=\mt$ turns out to be convenient and will be adopted in
what follows.

Using the $\mt^{\rm (pole)}$ quoted by CDF \cite{abeetal:94a},
\cite{abeetal:94b}, \cite{abeetal:95} together with the relation
\eqn{eq:mtpolebar} we find roughly
\begin{equation}
\mt\equiv\bar \mt(\mt)=(170\pm15)\gev
\end{equation}
which we will use in our phenomenological applications. In principle an
error of $\pm 11\gev$ could be used but we prefer to be
conservative.
