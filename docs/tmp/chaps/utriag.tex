\section{$\eps_K$, $B^0$-$\bar B^0$ Mixing and the Unitarity Triangle}
        \label{sec:epsBBUT}
\subsection{Basic Formula for $\eps_K$}
            \label{subsec:epsformula}
The indirect CP violation in $K \to \pi\pi$ is described by the well
known parameter $\eps_K$. The general formula for $\eps_K$ is given as
follows
\begin{equation}
\eps_K = \frac{\exp(i \pi/4)}{\sqrt{2} \Delta M_K} \,
\left( \IM M_{12} + 2 \xi \RE M_{12} \right)
\label{eq:epsdef}
\end{equation}
where
\begin{equation}
\xi = \frac{\IM A_0}{\RE A_0}
\label{eq:xi}
\end{equation}
with $A_0 \equiv A(K \to (\pi\pi)_{I=0})$ and $\Delta M_K$ being
the $K_L$-$K_S$ mass difference. The off-diagonal element $M_{12}$ in
the neutral $K$-meson mass matrix represents the $K^0$-$\bar K^0$
mixing. It is given by
\begin{equation}
2 m_K M^*_{12} = \langle \bar K^0| \Heff(\Delta S=2) |K^0\rangle
\label{eq:M12Kdef}
\end{equation}
where $\Heff(\Delta S=2)$ is the effective hamiltonian of eq.\
\eqn{hds2}. Defining the renormalization group invariant parameter
$B_K$ by
\begin{equation}
B_K = B_K(\mu) \left[ \as^{(3)}(\mu) \right]^{-2/9} \,
\left[ 1 + \frac{\as^{(3)}(\mu)}{4\pi} J_3 \right]
\label{eq:BKrenorm}
\end{equation}
\begin{equation}
\langle \bar K^0| (\bar s d)_{V-A} (\bar s d)_{V-A} |K^0\rangle
\equiv \frac{8}{3} B_K(\mu) F_K^2 m_K^2
\label{eq:KbarK}
\end{equation}
and using \eqn{hds2} we find
\begin{equation}
M_{12} = \frac{G_F^2}{12 \pi^2} F_K^2 B_K m_K \mw^2
\left[ {\lambda_c^*}^2 \eta_1 S_0(x_c) + {\lambda_t^*}^2 \eta_2 S_0(x_t) +
2 {\lambda_c^*} {\lambda_t^*} \eta_3 S_0(x_c, x_t) \right]
\label{eq:M12K}
\end{equation}
where the functions $S_0(x_i)$ and $S_0(x_i, x_j)$ are those of eq.\
\eqn{s0c}--\eqn{s0ct}. $F_K$ is the $K$-meson decay constant and $m_K$
the $K$-meson mass. The coefficient $J_3$ is given in \eqn{zd0} and the
QCD factors $\eta_i$ have been discussed in section
\ref{sec:HeffKKbar}. Their numerical values are
\begin{equation}
\eta_1 = 1.38 
\qquad
\eta_2 = 0.57
\qquad
\eta_3 = 0.47 \, .
\label{eq:etaknum}
\end{equation}
The last term in \eqn{eq:epsdef} constitutes at
most a 2\,\% correction to $\eps_K$ and consequently can be neglected
in view of other uncertainties, in particular those connected with
$B_K$.  Inserting \eqn{eq:M12K} into \eqn{eq:epsdef} we find
\begin{equation}
\eps_K = C_{\eps} B_K \IM\lambda_t \left\{
\RE\lambda_c \left[ \eta_1 S_0(x_c) - \eta_3 S_0(x_c, x_t) \right] -
\RE\lambda_t \eta_2 S_0(x_t) \right\} \exp(i \pi/4)
\label{eq:epsformula}
\end{equation}
where we have used the unitarity relation $\IM\lambda_c^* = {\rm
Im}\lambda_t$ and we have neglected $\RE\lambda_t/\RE\lambda_c
= \ord(\lambda^4)$ in evaluating $\IM(\lambda_c^* \lambda_t^*)$.
The numerical constant $C_\eps$ is given by
\begin{equation}
C_\eps = \frac{G_F^2 F_K^2 m_K \mw^2}{6 \sqrt{2} \pi^2 \Delta M_K}
       = 3.78 \cdot 10^4 \, .
\label{eq:Ceps}
\end{equation}
Using the standard parametrization of \eqn{2.72} to evaluate ${\rm
Im}\lambda_i$ and $\RE\lambda_i$, setting the values for $s_{12}$,
$s_{13}$, $s_{23}$ and $\mt$ in accordance with appendix
\ref{app:numinput} and taking a value for $B_K$ (see below) one can
determine the phase $\delta$ by comparing \eqn{eq:epsformula} with the
experimental value for $\eps_K$.

Once $\delta$ has been determined in this manner one can find the
corresponding point $(\bar\varrho,\bar\eta)$ by using \eqn{2.84} and
\eqn{2.88d}. Actually for a given set ($s_{12}$, $s_{13}$, $s_{23}$,
$\mt$, $B_K$) there are two solutions for $\delta$ and consequently two
solutions for $(\bar\varrho,\bar\eta)$. In order to see this clearly it is
useful to use the Wolfenstein parametrization in which ${\rm
Im}\lambda_t$, $\RE\lambda_c$ and $\RE\lambda_t$ are given to
a very good approximation by \eqn{2.51}--\eqn{2.53}. We then find that
\eqn{eq:epsformula} and the experimental value for $\eps_K$ specify a
hyperbola in the $(\bar\varrho,\bar\eta)$ plane given by
\begin{equation}
\bar\eta \left\{ (1 - \bar\varrho) A^2 \eta_2 S_0(x_t) + P_0(\eps)
\right\} A^2 B_K = 0.226 \, .
\label{eq:hyperbola}
\end{equation}
where
\begin{equation}
P_0(\eps) =
\left[ \eta_3 S_0(x_c, x_t) - \eta_1 x_c \right] \frac{1}{\lambda^4} \, .
\label{eq:p0eps}
\end{equation}
The hyperbola \eqn{eq:hyperbola} intersects the circle given by
\eqn{2.94} in two points which correspond to the two solutions for
$\delta$ mentioned earlier.

The position of the hyperbola \eqn{eq:hyperbola} in the
$(\bar\varrho,\bar\eta)$ plane depends on $\mt$, $|V_{cb}|=A \lambda^2$
and $B_K$. With decreasing $\mt$, $|V_{cb}|$ and $B_K$ the
$\eps_K$-hyperbola moves away from the origin of the
$(\bar\varrho,\bar\eta)$ plane. When the hyperbola and the circle
\eqn{2.94} touch each other lower bounds consistent with $\eps_K^{\rm
exp}$ for $\mt$, $|V_{cb}|$, $|V_{ub}/V_{cb}|$ and $B_K$ can be found.
The lower bound on $\mt$ is discussed in \cite{buras:93}.
Corresponding results for $|V_{ub}/V_{cb}|$ and $B_K$ are shown in
fig.\ \ref{fig:ut:vubcbmin} and \ref{fig:ut:bkmin}, respectively.
They will be discussed below.
\\
Moreover approximate analytic expressions for these bounds can be
derived. One has
\begin{eqnarray}
(\mt)_{\rm min} &=& \mw \left[ \frac{1}{2 A^2} \left( \frac{1}{A^2 B_K
R_b} - 1.4 \right) \right]^{0.658}
\label{eq:mtmin} \\
|V_{ub}/V_{cb}|_{\rm min} &=&
\frac{\lambda}{1-\lambda^2/2} \,
\left[ A^2 B_K \left( 2 x_t^{0.76} A^2 + 1.4 \right) \right]^{-1}
\label{eq:Vubcbmin} \\
(B_K)_{\rm min} &=& \left[ A^2 R_b \left( 2 x_t^{0.76} A^2 + 1.4 \right)
                    \right]^{-1}
\label{eq:BKmin}
\end{eqnarray}

Concerning the parameter $B_K$, the analyses of \cite{sharpe:94},
\cite{ishizuka:93} ($B_K=0.83\pm 0.03$) using the lattice method and of
\cite{bijnenspardes:95} using a somewhat modified form of the $1/N$
approach of \cite{bardeenetal:88}, \cite{gerard:90} give results in the
ball park of the $1/N$ result $B_K=0.70\pm 0.10$ obtained some time ago
in \cite{bardeenetal:88}, \cite{gerard:90}. In particular the analysis
of \cite{bijnenspardes:95} seems to have explained the difference
between these values for $B_K$ and the lower values obtained using the
QCD Hadronic Duality approach \cite{pichderaf:85}, \cite{pradesetal:91}
($B_K=0.39\pm 0.10$) or using SU(3) symmetry and PCAC ($B_K=1/3$)
\cite{donoghueetal:82}. These higher values of $B_K$ are also found in
the most recent lattice analysis \cite{crisafullietal:95}
($B_K=0.86 \pm 0.15$) and in the lattice calculations of Bernard and
Soni ($B_K=0.78 \pm 0.11$) and the JLQCD group ($B_K=0.67 \pm 0.07$)
with the quoted values obtained on the basis on the review by
\cite{soni:95}. In our numerical analysis we will use
\begin{equation}
B_K = 0.75 \pm 0.15 \, .
\label{eq:BKnum}
\end{equation}

\subsection{Basic Formula for $B^0$-$\bar B^0$ Mixing}
            \label{subsec:BBformula}
The $B^0$-$\bar B^0$ mixing is usually described by
\begin{equation}
x_{d,s} \equiv \frac{(\Delta M)_{B_{d,s}}}{\Gamma_{B_{d,s}}} =
\frac{2 |M_{12}|_{B_{d,s}}}{\Gamma_{B_{d,s}}}
\label{eq:xdsdef}
\end{equation}
where $(\Delta M)_{B_{d,s}}$ is the mass difference between the mass
eigenstates in the $B_d^0-\bar B_d^0$ system and the $B_s^0-\bar B_s^0$
system, respectively, and $\Gamma_{B_{d,s}} = 1/\tau_{B_{d,s}}$ with
$\tau_{B_{d,s}}$ being the corresponding lifetimes. The off-diagonal
term $M_{12}$ in \eqn{eq:xdsdef} is given by
\begin{equation}
2 m_B |M_{12}| = |\langle \bar B^0| \Heff(\Delta B=2) |B^0\rangle|
\label{eq:M12Bdef}
\end{equation}
where $\Heff(\Delta B=2)$ is the effective hamiltonian of
\eqn{hdb2}. Defining the renormalization group invariant parameter $B_B$ by
\begin{equation}
B_B = B_B(\mu) \left[ \as^{(5)}(\mu) \right]^{-6/23} \,
\left[ 1 + \frac{\as^{(5)}(\mu)}{4\pi} J_5 \right]
\label{eq:BBrenorm}
\end{equation}
\begin{equation}
\langle \bar B^0| (\bar b d)_{V-A} (\bar b d)_{V-A} |B^0\rangle
\equiv \frac{8}{3} B_B(\mu) F_B^2 m_B^2
\label{eq:BbarB}
\end{equation}
and using \eqn{hdb2} we find
\begin{equation}
x_{d,s} = \tau_{B_{d,s}} \frac{G_F^2}{6 \pi^2} \eta_B m_{B_{d,s}} (
B_{B_{d,s}} F_{B_{d,s}}^2 ) \mw^2 S_0(x_t) |V_{t(d,s)}|^2
\label{eq:xds}
\end{equation}
with the QCD factor $\eta_B$ discussed in section \ref{sec:HeffBBbar}
and given by $\eta_B=0.55$.

The measurement of $B_d^0$-$\bar B_d^0$ mixing allows then to determine
$|V_{td}|$ or $R_t$ of \eqn{2.95}
\begin{equation}
|V_{td}| = A \lambda^3 R_t
\qquad
\qquad
R_t = 1.52 \frac{R_0}{\sqrt{S_0(x_t)}}
\label{eq:VtdRt}
\end{equation}
where
\begin{equation}
R_0 = 
\left[ \frac{0.040}{|V_{cb}|} \right]
\left[ \frac{200\mev}{ \sqrt{B_{B_d}} F_{B_d} } \right]
\left[ \frac{x_d}{0.75}       \right]^{0.5} 
\left[ \frac{1.6\,ps}{\tau_B} \right]^{0.5}
\left[ \frac{0.55}{\eta_B}    \right]^{0.5} \, .
\label{eq:R0}
\end{equation}
which gives setting $\eta_B = 0.55$
\begin{equation}
|V_{td}| = 8.56 \cdot 10^{-3}
\left[ \frac{170\gev}{\bar m_t(m_t)} \right]^{0.76} 
\left[ \frac{200\mev}{ \sqrt{B_{B_d}} F_{B_d} } \right]
\left[ \frac{x_d}{0.75}             \right]^{0.5} 
\left[ \frac{1.6\,ps}{\tau_B}       \right]^{0.5} \, .
\label{eq:Vtdnum}
\end{equation}

There is a vast literature on the lattice calculations of $F_B$. The
most recent results are somewhat lower than quoted a few years ago.
Based on a review by \cite{sachrajda:94}, the recent extensive study by
\cite{duncanetal:94} and the analyses in \cite{bernardetal:94},
\cite{drapermcneile:94} we conclude that $F_{B_d}=(180\pm40)\mev$. This
together with the earlier result of the European Collaboration
\cite{abadaetal:92} for $B_B$, gives $F_{B_d}\sqrt{B_{B_d}}=194\pm
45\mev$. A reduction of the error in this important quantity is
desirable. These results for $F_B$ are compatible with the results
obtained using QCD sum rules (e.g. \cite{baganetal:92},
\cite{neubert:92}). An interesting upper bound $F_{B_d}<195\mev$ using
QCD dispersion relations has also recently been obtained
\cite{boydetal:94}. In our numerical analysis we will use
\begin{equation}
\sqrt{B_{B_d}} F_{B_d} = (200 \pm 40)\mev \, .
\label{eq:Fbnum}
\end{equation}

The accuracy of the determination of $R_t$ can be considerably improved
by measuring simultaneously the $B_s^0$-$\bar B_s^0$ mixing described by
$x_s$. We have
\begin{equation}
R_t = \frac{1}{\sqrt{R_{ds}}} \sqrt{\frac{x_d}{x_s}} \frac{1}{\lambda}
\sqrt{1 - \lambda^2 (1 - 2 \varrho)}
\qquad
R_{ds} = \frac{\tau_{B_d}}{\tau_{B_s}} \frac{m_{B_d}}{m_{B_s}}
\left[ \frac{F_{B_d} \sqrt{B_{B_d}}}{F_{B_s} \sqrt{B_{B_s}}} \right]^2 \, .
\label{eq:Rt}
\end{equation}
Note that $\mt$ and $|V_{cb}|$ have been eliminated in this way and that
$R_{ds}$ depends only on $SU(3)$-flavour breaking effects which contain
much smaller theoretical uncertainties than the hadronic matrix elements
in $x_d$ and $x_s$ separately.
Provided $x_d/x_s$ has been accurately measured a determination of
$R_t$ within $\pm 10\%$ should be possible. Indeed the most recent
lattice results \cite{duncanetal:94}, \cite{baxteretal:94} give
$F_{B_s}/F_{B_d} = 1.22\pm0.04$. A similar result $F_{B_s}/F_{B_d} =
1.16\pm0.05$ has been obtained using QCD sum rules \cite{narison:94}.
It would be useful to know $B_{B_s}/B_{B_d}$ with a similar precision.
For $B_{B_s}=B_{B_d}$ we find using the lattice result $R_{ds} =
0.66 \pm 0.07$.

\subsection{$\sin(2\beta)$ from $\eps_K$ and $B^0$-$\bar B^0$ Mixing}
           \label{subsec:sin2bepskBB}
Combining \eqn{eq:hyperbola} and \eqn{eq:xds} one can derive an analytic
formula for $\sin(2\beta)$ \cite{burasetal:94b}
\begin{equation}
\sin(2\beta) = \frac{1}{1.16 A^2 \eta_2 R_0^2}
\left[ \frac{0.226}{A^2 B_K} - \bar\eta P_0(\eps) \right] \, .
\label{eq:sin2b}
\end{equation}
$P_0(\eps)$ is weakly dependent on $\mt$ and for $155\gev \le \mt \le
185\gev$ one has $P_0(\eps) \approx 0.31 \pm 0.02$. As $\bar\eta \le
0.45$ for $|V_{ub}/V_{cb}| \le 0.1$ the first term in parenthesis is
generally by a factor of 2--3 larger than the second term. Since this
dominant term is independent of $\mt$, the values for $\sin(2\beta)$
extracted from $\eps_K$ and $B^0$-$\bar B^0$ mixing show only a weak
dependence on $\mt$ as stressed in particular in \cite{rosner:00}.

Since in addition $A^2 R_0^2$ is independent of $|V_{cb}|$, the dominant
uncertainty in this determination of $\sin(2\beta)$ resides in $A^2 B_K$
in the first term in the parenthesis and in $F_{B_d} \sqrt{B_{B_d}}$
contained in $R_0^2$.

\subsection{Phenomenological Analysis}
           \label{subsec:phenoUT}
We will now combine the analyses of $\eps_K$ and of $B^0_d-\bar B^0_d$
mixing to obtain allowed ranges for several quantities of interest. We
consider two sets of input parameters, which are collected in the
appendix. The first set represents the present situation. The second
set can be considered as a ``future vision'' in which the errors on
various input parameters have been decreased. It is plausible that such
errors will be achieved at the end of this decade, although one cannot
guarantee that the central values will remain. In table
\ref{tab:predictions} we show the results for $\delta$, $\IM\lambda_t$,
$\sin 2\alpha$, $\sin 2\beta$, $\sin \gamma$, $|V_{td}|$ and $x_s$.
They correspond to the two sets of parameters in question, with and
without the constraint from $B^0_d-\bar B^0_d$ mixing. The results for
$\IM\lambda_t$ and $|V_{td}|$ will play an important role in the
phenomenology of rare decays and CP violation.  For completeness we
also show the expectations for $\sin 2\alpha$, $\sin 2\beta$ and
$\sin\gamma$ which enter various CP asymmetries in B-decays. As already
discussed in detail in \cite{burasetal:94b}, $\sin 2\alpha$ cannot be
predicted accurately this way.  On the other hand $\sin 2\beta$ and
$\sin\gamma$ are more constrained and the resulting ranges for these
quantities indicate that large CP asymmetries should be observed in a
variety of B-decays.

\begin{table}[htb]
\caption[]{
Predictions for various quantities using present and future input
parameter ranges given in appendix \ref{app:numinput}.
${\rm Im}\lambda_t$ and $|V_{td}|$ are given
in units of $10^{-4}$ and $10^{-3}$, respectively. $\delta$ is in degrees.
\label{tab:predictions}}
\begin{center}
\begin{tabular}{|c|r|r|r|r|}
& \multicolumn{2}{c|}{no   $x_d$ constraint} &
  \multicolumn{2}{c|}{with $x_d$ constraint} \\
\hline
& \multicolumn{1}{c|}{Present} & \multicolumn{1}{c|}{Future} &
  \multicolumn{1}{c|}{Present} & \multicolumn{1}{c|}{Future} \\
\hline
$\delta$                     & 37.7 -- 160.0 & 57.4 -- 144.9
                             & 37.7 -- 140.2 & 58.5 -- 93.3 \\
${\rm Im}\lambda_t$              & 0.64 -- 1.75 & 0.82 -- 1.50
                             & 0.87 -- 1.75 & 1.12 -- 1.50 \\
$|V_{td}| $                  & 6.7 -- 13.5 & 7.7 -- 12.1
                             & 6.7 -- 11.9 & 7.8 -- 9.3 \\
$x_s$                        &  --  &  -- 
                             & 11.1 -- 47.0 & 19.6 -- 29.6 \\
$\sin 2\alpha$               & --0.86 -- 1.00 & --0.323 -- 1.00
                             & --0.86 -- 1.00 & --0.30 -- 0.73 \\
$\sin 2\beta$                & 0.21 -- 0.80 & 0.34 -- 0.73
                             & 0.34 -- 0.80 & 0.57 -- 0.73 \\
$\sin\gamma$                 & 0.34 -- 1.00 & 0.58 -- 1.00
                             & 0.61 -- 1.00 & 0.85 -- 1.00 \\
\end{tabular}
\end{center}
\end{table}

\begin{figure}[hbt]
\vspace{0.10in}
\centerline{
\epsfysize=5in
\rotate[r]{
\epsffile{ps/mtImLtAll_xd.ps}
} }
\vspace{0.08in}
\caption[]{
Present (left) and future (right) allowed ranges for $\IM(\lambda_t)$.
The ranges have been obtained by fitting $\eps_K$ in \eqn{eq:epsformula} to 
the experimental value. Input parameter ranges are given in appendix
\ref{app:numinput}. The impact of the additional constraint coming from
$x_d$ is illustrated by the dashed lines. With the $x_d$ constraint
imposed the solution $\pi/2 < \delta < \pi$ is completely eliminated
for the future scenario.
\label{fig:ut:imtimltxd}}
\end{figure}

In fig.\ \ref{fig:ut:imtimltxd} we show $\IM\lambda_t$ as a function of
$\mt$. In fig.\ \ref{fig:ut:vubcbmin} the lower bound on
$|V_{ub}/V_{cb}|$ resulting from the $\eps_K$-constraint is shown as a
function of $|V_{cb}|$ for various values of $B_K$. To this end we have
set $\mt=185\gev$.  For lower values of $\mt$ the lower bound on
$|V_{ub}/V_{cb}|$ is stronger.  A similar analysis has been made by
\cite{herrlichnierste:95}. The latter work and the plot in
fig.\ \ref{fig:ut:vubcbmin} demonstrate clearly the impact of the
$\eps_K$ constraint on the allowed values of $|V_{ub}/V_{cb} |$ and
$|V_{cb}|$.  Simultaneously small values of $|V_{ub}/V_{cb} |$ and $|
V_{cb}|$, although still consistent with tree-level decays, are not
allowed by the size of the indirect CP violation observed in $K \to
\pi\pi$.  Another representation of this behaviour is shown in
fig.\ \ref{fig:ut:bkmin} where we plot the minimal value of $B_{K}$
consistent with the experimental value of $\eps_K$ as a function of
$V_{cb}$ for different $|V_{ub}/V_{cb} |$ and $\mt < 185\gev$.

\begin{figure}[hbt]
\vspace{0.10in}
\centerline{
\epsfysize=5in
\rotate[r]{
\epsffile{ps/vubcbmin.ps}
} }
\vspace{0.08in}
\caption[]{
$|V_{ub}/V_{cb}|_{\rm min}$ for $\mt \le 185\gev$ and various choices
of $B_K$.
\label{fig:ut:vubcbmin}}
\end{figure}

\begin{figure}[hbt]
\vspace{0.10in}
\centerline{
\epsfysize=5in
\rotate[r]{
\epsffile{ps/bkmin.ps}
} }
\vspace{0.08in}
\caption[]{
$(B_K)_{\rm min}$ of eq.\ \eqn{eq:BKmin} for $\mt \le 185\gev$ and
various choices of $|V_{ub}/V_{cb}|$.
\label{fig:ut:bkmin}}
\end{figure}

Finally in fig.\ \ref{fig:ut:rhoeta} we show the allowed ranges in the
$(\bar\rho,\bar\eta)$ plane obtained using the information from
$V_{cb}$, $|V_{ub}/V_{cb}|$, $\eps_K$ and $B^0_d-\bar B^0_d$ mixing.
In this plot we also show the impact of a future measurement of
$B^0_s-\bar B^0_s$ mixing with $x_s=10$, $15$, $25$, $40$, which by
means of the formula (\ref{eq:Rt}) gives an important measurement of
the side $R_t$ of the unitarity triangle. Whereas at present a broad
range in the $(\bar\rho,\bar\eta)$ plane is allowed, the situation
might change in the future allowing only the values $0 \le \bar \rho
\le 0.2$ and $0.30 \le \bar \eta \le 0.40$.  This results in smaller
ranges for various quantities of interest as explicitly seen in table
\ref{tab:predictions}.

\begin{figure}[hbt]
\vspace{0.10in}
\centerline{
\epsfysize=5in
\rotate[r]{
\epsffile{ps/rhoeta.ps}
} }
\vspace{0.08in}
\caption[]{
Present (a) and future (b) allowed ranges for the upper corner A
of the UT using data from $K^0-\bar K^0$-, $B^0-\bar B^0$-mixing
and tree-level $B$-decays. Input parameter ranges are given in appendix
\ref{app:numinput}.
The solid lines correspond to $(R_t)_{\rm max}$ from eq.\eqn{eq:Rt}
using $R_{ds}=0.66$ and $x_s \ge 10, 15, 25$ and $40$, respectively.
\label{fig:ut:rhoeta}}
\end{figure}

Other analyses of the unitarity triangle can be found in
\cite{pecceiwang:95}, \cite{ciuchini:95}, \cite{herrlichnierste:95},
\cite{alilondon:95}.
