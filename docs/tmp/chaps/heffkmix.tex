\section{The Effective Hamiltonian for $K^0-\bar K^0$ Mixing}
         \label{sec:HeffKKbar}

\subsection{General Structure}
\label{sec:HeffKKbar:General}
The following chapter is devoted to the presentation of the
effective hamiltonian for $\Delta S=2$ transitions. This hamiltonian
incorporates the short-distance physics contributing to
$K^0-\bar K^0$ mixing and is essential for the description of
CP violation in the neutral K-meson system.
\\
Being a FCNC process, $K^0-\bar K^0$ mixing can only occur at the
loop level within the Standard Model. To lowest order it is induced
through the box diagrams in fig.\ \ref{fig:oporig}\,(e).
Including QCD corrections the
effective low energy hamiltonian, to be derived from these diagrams,
can be written as follows $(\lambda_i = V_{is}^* V_{id}^{})$
\begin{eqnarray}\label{hds2}
{\cal H}^{\Delta S=2}_{eff}&=&\frac{G^2_F}{16\pi^2}M^2_W
 \left[\lambda^2_c\eta_1 S_0(x_c)+\lambda^2_t \eta_2 S_0(x_t)+
 2\lambda_c\lambda_t \eta_3 S_0(x_c, x_t)\right] \times
\nonumber\\
& & \times \left[\as(\mu)\right]^{-2/9}\left[
  1 + \frac{\as(\mu)}{4\pi} J_3\right]  Q + h. c.
\end{eqnarray}
This equation, together with \eqn{eta1}, \eqn{eta2}, \eqn{eta3}
for $\eta_1$, $\eta_2$ and $\eta_3$ respectively, represents the
complete next-to-leading order short-distance hamiltonian for
$\Delta S=2$ transitions. (\ref{hds2}) is valid for scales $\mu$
below the charm threshold $\mu_c=\ord(m_c)$. In this case
${\cal H}^{\Delta S=2}_{eff}$ consists of a single four-quark operator
\begin{equation}\label{qsdsd}
Q=(\bar sd)_{V-A}(\bar sd)_{V-A}
\end{equation}
which is multiplied by the corresponding coefficient function.
It is useful and customary to decompose this function into a
charm-, a top- and a mixed charm-top contribution, as displayed
in (\ref{hds2}). This form is obtained upon eliminating $\lambda_u$
by means of CKM matrix unitarity and setting $x_u=0$. The basic
electroweak loop contributions without QCD correction are then
expressed through the functions $S_0$, which read \cite{inamilim:81}
\begin{equation}\label{s0c}
S_0(x_c)\doteq x_c
\end{equation}
\begin{equation}\label{s0t}
S_0(x_t)=\frac{4x_t-11x^2_t+x^3_t}{4(1-x_t)^2}-
 \frac{3x^3_t \ln x_t}{2(1-x_t)^3}
\end{equation}
\begin{equation}\label{s0ct}
S_0(x_c, x_t)=x_c\left[\ln\frac{x_t}{x_c}-\frac{3x_t}{4(1-x_t)}-
 \frac{3 x^2_t\ln x_t}{4(1-x_t)^2}\right]
\end{equation}
Here again we keep only linear terms in $x_c\ll 1$, but of course
all orders in $x_t$.
\\
Short-distance QCD effects are described through the correction
factors $\eta_1$, $\eta_2$, $\eta_3$ and the explicitly
$\as$-dependent terms in (\ref{hds2}). The discussion of
these corrections will be the subject of the following sections.
\\
Without QCD, i.e. in the limit $\as\to 0$, one has
$\eta_i [\as]^{-2/9}\to 1$. In general, the complete
coefficient function multiplying $Q$ in (\ref{hds2}) contains the
QCD effects at high energies $\mu_W=\ord(M_W)$,
$\mu_t=\ord(m_t)$ together with their RG evolution down to the
scale $\mu=\ord(1\gev)$. A common ingredient for the three
different contributions in (\ref{hds2}) is the anomalous dimension
of the operator $Q$ and the corresponding evolution of its coefficient.
The Fierz symmetric flavor structure of $Q$ implies that it acquires
the same anomalous dimension as the Fierz symmetric operator
$Q_+=(Q_2+Q_1)/2$ (see section \ref{sec:HeffdF1:22}), explicitly
\begin{equation}\label{gamq1}
\gamma=\frac{\as}{4\pi}\gamma^{(0)}+
       \left(\frac{\as}{4\pi}\right)^2\gamma^{(1)}
\end{equation}
\begin{equation}\label{gamq2}
\gamma^{(0)}=6\frac{N-1}{N}\qquad
\gamma^{(1)}=\frac{N-1}{2N}\left[-21+\frac{57}{N}-\frac{19}{3}N+
    \frac{4}{3} f\right]\quad ({\rm NDR})
\end{equation}
The resulting evolution of the coefficient of $Q$ between general
scales $\mu_1$ and $\mu$ then reads
\begin{equation}\label{cqmu}
C_Q(\mu)=\left[1+\frac{\as(\mu)-\as(\mu_1)}{4\pi}J_f\right]
  \left[\frac{\as(\mu_1)}{\as(\mu)}\right]^{d_f}
  C_Q(\mu_1)
\end{equation}
where
\begin{equation}\label{zd0}
d_f=\frac{\gamma^{(0)}}{2\beta_0} \qquad
J_f= \frac{d_f}{\beta_0}\beta_1 - \frac{\gamma^{(1)}}{2\beta_0}
\end{equation}
depend on the number of active flavors $f$. At the lower end of the
evolution $f=3$. The terms in (\ref{cqmu}) depending on
$\as(\mu)$ are factored out explicitly in (\ref{hds2}) to
exhibit the $\mu$-dependence of the coefficient function in the
$f=3$ regime, which has to cancel the corresponding $\mu$-dependence
of the hadronic matrix element of $Q$ between meson states in
physical applications. A similar comment applies to the scheme
dependence entering $J_f$ in (\ref{zd0}) through the scheme
dependence of $\gamma^{(1)}$. Splitting off the $\mu$-dependence in
(\ref{hds2}) is of course not unique. The way it is done belongs to the
{\em definition} of the $\eta_i$-factors.
\\
Let us finally compare the structure of (\ref{hds2}) with the
effective hamiltonians for rare decays discussed in chapter
\ref{sec:HeffRareKB}. Common features of both types of processes
include:
\begin{itemize}
\item
Both are generated to lowest order via electroweak FCNC loop
transitions involving heavy quarks.
\item
They contain a charm and a top contribution.
\item
The hamiltonian consists of a single dimension-6 operator.
\end{itemize}
Besides these similarities, however, there are also a few important
differences, which have their root in the fact that the $\Delta S=2$
box diagrams involve two distinct quark lines as compared to the single
quark line in semileptonic rare decays:
\begin{itemize}
\item
The CKM parameter combinations $\lambda_i$ appear quadratically in
(\ref{hds2}) instead of only linearly.
\item
(\ref{hds2}) in addition receives contributions from a mixed
top-charm sector. This part in fact turns out to have the most
involved structure of the three contributions.
\item
The operator $Q$ has a non-vanishing QCD anomalous dimension, resulting
in a non-trivial scale and scheme dependence of the Wilson
coefficient.
\item
The hadronic matrix element of the four-quark operator $Q$ is
a considerably more complicated object than the quark current
matrix elements in semileptonic rare decays.
\end{itemize}
We will now present the complete next-to-leading order results for
$\eta_2$, $\eta_1$ and $\eta_3$ in turn and discuss their most
important theoretical features. The first leading log calculations of
$\eta_1$ have been presented in \cite{vainshteinetal:76},
\cite{novikovetal:77} and of $\eta_2$ in \cite{vysotskij:80}. The
complete leading log calculation inlcuding also $\eta_3$ has been first
performed in \cite{gilmanwise:83}.  Leading order calculations in the
presence of a heavy top can be found in \cite{kaufmanetal:89},
\cite{flynn:90}, \cite{dattaetal:90} and \cite{dattaetal:95}.

\subsection{The Top Contribution -- $\eta_2$}
\label{sec:HeffKKbar:eta2}

The basic structure of the top quark sector in
${\cal H}^{\Delta S=2}_{eff}$ is easy to understand. First the
top quark is integrated out, along with the $W$, at a matching
scale $\mu_t=\ord(m_t)$, leaving a $m_t$-dependent coefficient
normalized at $\mu_t$, multiplying the single operator $Q$.
Subsequently the coefficient is simply renormalized down to
scales $\mu=\ord(1\gev)$ by means of (\ref{cqmu}). Including
NLO corrections the resulting QCD factor $\eta_2$ from (\ref{hds2})
may be written (in $\overline{MS}$) as follows \cite{burasjaminweisz:90}
\begin{eqnarray}\label{eta2}
\eta_2&=&\left[\as(m_c)\right]^{6/27}
 \left[\frac{\as(m_b)}{\as(m_c)}\right]^{6/25}
 \left[\frac{\as(\mu_t)}{\as(m_b)}\right]^{6/23}
\\
& & \cdot\left[1+\frac{\as(m_c)}{4\pi}(J_4-J_3)+
         \frac{\as(m_b)}{4\pi}(J_5-J_4) \right.
\nonumber\\
& & + \left.\frac{\as(\mu_t)}{4\pi} \left(
 \frac{S_1(x_t)}{S_0(x_t)}+B_t-J_5+\frac{\gamma^{(0)}}{2}
 \ln\frac{\mu^2_t}{M^2_W}+\gamma_{m0}
 \frac{\partial\ln S_0(x_t)}{\partial\ln x_t}\ln\frac{\mu^2_t}{M^2_W}
\right)\right] \nonumber
\end{eqnarray}
where $\gamma_{m0} = 6 C_F$,
\begin{equation}\label{btndr}
B_t=5\frac{N-1}{2N}+3\frac{N^2-1}{2N}\qquad ({\rm NDR})
\end{equation}
and
\begin{equation}\label{s1x}
S_1(x)=\frac{N-1}{2N}S^{(8)}_1(x)+\frac{N^2-1}{2N}S^{(1)}_1(x)
\end{equation}
\begin{eqnarray}\label{s1x8}
S^{(8)}_1(x)=&-&\frac{64-68x-17x^2+11x^3}{4(1-x)^2}+
                \frac{32-68x+32x^2-28x^3+3x^4}{2(1-x)^3}\ln x
\nonumber\\
&+&\frac{x^2(4-7x+7x^2-2x^3)}{2(1-x)^4}\ln^2 x+
   \frac{2x(4-7x-7x^2+x^3)}{(1-x)^3}L_2(1-x)
\nonumber\\
&+&\frac{16}{x}\left(\frac{\pi^2}{6}-L_2(1-x)\right)
\end{eqnarray}
\begin{eqnarray}\label{s1x1}
S^{(1)}_1(x)=&-&\frac{x(4-39x+168x^2+11x^3)}{4(1-x)^3}-
                \frac{3x(4-24x+36x^2+7x^3+x^4)}{2(1-x)^4}\ln x
\nonumber\\
&+&\frac{3x^3(13+4x+x^2)}{2(1-x)^4}\ln^2 x-
   \frac{3x^3(5+x)}{(1-x)^3}L_2(1-x)
\end{eqnarray}
where the dilogarithm $L_2$ is defined in \eqn{l2}.

In the expression (\ref{eta2}) we have taken into account the heavy
quark thresholds at $m_b$ and $m_c$ in the RG evolution. As it must
be, the dependence on the threshold scales is of the neglected
order $\ord(\alpha^2_s)$. In fact the threshold ambiguity is here of
$\ord(\as^2)$ also in LLA since $\gamma^{(0)}$ is flavor independent.
It turns out that this dependence is also very weak numerically and
we therefore set $\mu_c=m_c$ and $\mu_b=m_b$. Furthermore it is
a good approximation to neglect the b-threshold completely
using an effective 4-flavor theory from $\mu_t$ down to $m_c$. This
can be achieved by simply substituting $m_b\to\mu_t$ in (\ref{eta2}).
\\
The leading order expression for $\eta_2$ is given by the first three
factors on the r.h.s. of (\ref{eta2}). The fourth factor represents
the next-to-leading order generalization. Let us discuss now the
most interesting and important features of the NLO result for $\eta_2$
exhibited in (\ref{eta2}).
\begin{itemize}
\item
$\eta_2$ is proportional to the initial value of the Wilson
coefficient function at $\mu_t=\mw$
\begin{equation}\label{s01x}
S(x)=S_0(x)+\frac{\as}{4\pi}\left(S_1(x)+B_t S_0(x)\right)
\end{equation}
which has to be extracted from the box graphs in fig.\
\ref{fig:oporig}\,(e) and the corresponding gluon correction diagrams
after a proper factorization of long- and short-distance
contributions.
\item
$S(x)$ in (\ref{s01x}) is similar to the functions $X(x)$ and
$Y(x)$ in sections \ref{sec:HeffRareKB:kpnn:heff} and
\ref{sec:HeffRareKB:klmm:heff} except that $S(x)$ is scheme
dependent due to the renormalization that is required for the
operator $Q$. This scheme dependence enters (\ref{s01x}) through the
scheme dependent constant $B_t$, given in the NDR scheme in
(\ref{btndr}). This scheme dependence is canceled in the combination
$B_t-J_5$ by the two-loop anomalous dimension contained in $J_5$.
Likewise the scheme dependence of $J_f$ cancels in the differences
$(J_{f}-J_{f-1})$ as is evident from the discussion of section
\ref{sec:basicform:wc:rgdep}.
\item
A very important point is the dependence on the high energy matching
scale $\mu_t$. This dependence enters the NLO
$\as(\mu_t)$-correction in (\ref{eta2}) in two distinct ways:
First as a term proportional to $\gamma^{(0)}$ and, secondly, in
conjunction with $\gamma_{m0}$. The first of these terms cancels to
$\ord(\as)$ the $\mu_t$-dependence present in the leading
term $[\as(\mu_t)]^{6/23}$. The second, on the other hand, leads
to an $\ord(\as)$ $\mu_t$-dependence of $\eta_2$ which is
just the one needed to cancel the $\mu_t$-ambiguity of the leading
function $S_0(x_t(\mu_t))$ in the product $\eta_2 S_0(x_t)$, such
that in total physical results become independent of $\mu_t$ to
$\ord(\as)$. From these observations it is obvious that one
may interpret $\mu_t$ in the first case as the initial scale of the
RG evolution and in the second case as the scale at which the top
quark mass is defined. These two scales need not necessarily have the
same value.
\\
The important point is, that to leading logarithmic accuracy the
$\mu_t$-de\-pen\-dence of both $\eta^{LO}_2(\mu_t)$ and
$S_0(x_t(\mu_t))$ remains uncompensated, leaving a dis\-tur\-bing\-ly
large uncertainty in the short-distance calculation.
\item
It is interesting to note that in the limit $\mt\gg \mw$ the dependence
on $\mu_t$ enters $\eta_2$ as $\ln\mu_t/\mt$, rather than
$\ln\mu_t/\mw$. This feature provides a formal justification for
choosing $\mu_t=\ord(\mt)$ instead of $\mu_t=\ord(\mw)$. An explicit
expression for the large $\mt$ limit in the similar case of $\eta_{2B}$
may be found in section \ref{sec:HeffBBbar}.
\item
Although at NLO the product $\eta_2 S_0(x_t)$ depends only very weakly
on the precise value of $\mu_t$ as long as it is of $\ord(m_t)$, the
choice $\mu_t=m_t$ is again convenient:  With this choice $\eta_2$
becomes almost independent of the top quark mass $m_t(m_t)$. By
contrast, for $\mu_t=M_W$, say, $\eta_2$ would decrease with rising
$m_t(m_t)$ in order to compensate for the increase of $S_0(x_t(M_W))$
due to the use of a -- for high $m_t$ -- ``unnaturally'' low scale
$M_W$.
\item
As mentioned above the dependence of the Wilson coefficient on the
low energy scale $\mu$ and the remaining scheme dependence ($J_3$)
has been factored out explicitly in (\ref{hds2}). Therefore the
QCD correction factor $\eta_2$ is {\it by definition\/}
scale and scheme independent on the lower end of the RG evolution.
\end{itemize}

\subsection{The Charm Contribution -- $\eta_1$}
\label{sec:HeffKKbar:eta1}

The calculation of $\eta_1$ beyond leading logs has been presented in
great detail in \cite{herrlichnierste:93}, \cite{herrlich:94}. Our task
here will be to briefly describe the basic procedure and to summarize
the main results.
\\
In principle the charm contribution is similar in structure to the
top contribution. However, since the quark mass $m_c\ll M_W$,
the charm degrees of freedom can no longer be integrated out
simultaneously with the $W$ boson, which would introduce large
logarithmic corrections $\sim\as\ln M_W/m_c$. To resum these logarithms
one first constructs an effective theory at a scale $\ord(M_W)$, where
the $W$ boson is removed. The relevant operators are subsequently
renormalized down to scales $\mu_c=\ord(m_c)$, where the charm quark is
then integrated out. After this step only the operator $Q$
(\ref{qsdsd}) remains and $\eta_1$ is finally obtained as discussed in
section \ref{sec:HeffKKbar:General}.
\\
Let us briefly outline this procedure for the case at hand.
After integrating out $W$ the effective hamiltonian to first order
in weak interactions, which is needed for the charm contribution,
can be written as
\begin{equation}\label{hc1}
{\cal H}^{(1)}_c=\frac{G_F}{\sqrt 2}\sum_{q, q^\prime=u, c}
 V^\ast_{q^\prime s}V_{qd}\left(C_+Q^{q^\prime q}_+ +
 C_-Q^{q^\prime q}_- \right)
\end{equation}
where we have introduced the familiar $\Delta S=1$ four-quark
operators in the multiplicatively renormalizable basis
\begin{equation}\label{qqpm}
Q^{q^\prime q}_\pm=\frac{1}{2}\left[
(\bar s_iq^\prime_i)_{V-A}(\bar q_jd_j)_{V-A}\pm
(\bar s_iq^\prime_j)_{V-A}(\bar q_jd_i)_{V-A} \right]
\end{equation}
We remark that no penguin operators appear in the present case
due to GIM cancellation between charm quark and up quark
contributions.
\\
$\Delta S=2$ transitions occur to second order in the effective
interaction (\ref{hc1}). The $\Delta S=2$ effective hamiltonian
is therefore given by
\begin{equation}\label{heff2c}
{\cal H}^{\Delta S=2}_{eff,c}=-\frac{i}{2}\int d^4x\
T \left( {\cal H}^{(1)}_c(x) {\cal H}^{(1)}_c(0) \right)
\end{equation}
Inserting (\ref{hc1}) into (\ref{heff2c}), keeping only pieces that can
contribute to the charm box diagrams and taking the GIM constraints
into account, one obtains
\begin{equation}\label{hds2c}
{\cal H}^{\Delta S=2}_{eff,c}=\frac{G^2_F}{2}\lambda^2_c
 \sum_{i,j=+,-}C_i C_j O_{ij}
\end{equation}
where
\begin{equation}\label{oijdef}
O_{ij}=-\frac{i}{2}\int d^4x\ T\left[
Q^{cc}_i(x)Q^{cc}_j(0)-Q^{uc}_i(x)Q^{cu}_j(0)-
Q^{cu}_i(x)Q^{uc}_j(0)+Q^{uu}_i(x)Q^{uu}_j(0) \right]
\end{equation}
From the derivation of (\ref{hds2c}) it is evident, that the
Wilson coefficients of the bilocal operators $O_{ij}$ are simply given
by the product $C_i C_j$ of the coefficients pertaining to the
local operators $Q_i$, $Q_j$. The evolution of the $C_i$ from $M_W$
down to $\mu_c$ proceeds in the standard fashion and is described
by equations of the type shown in (\ref{cqmu}) with the
appropriate anomalous dimensions inserted. In the following we list
the required ingredients.
\\
The Wilson coefficients at scale $\mu=M_W$ read
\begin{equation}\label{cpmw}
C_\pm(M_W)=1+\frac{\as(M_W)}{4\pi}B_\pm
\end{equation}
\begin{equation}\label{bpmn}
B_\pm=\pm 11\frac{N\mp 1}{2N}   \qquad {\rm (NDR)}
\end{equation}
The two-loop anomalous dimensions are
\begin{equation}\label{gapm1}
\gamma_\pm=\frac{\as}{4\pi}\gamma^{(0)}_\pm+
       \left(\frac{\as}{4\pi}\right)^2\gamma^{(1)}_\pm
\end{equation}
\begin{equation}\label{gapm2}
\gamma^{(0)}_\pm=\pm 6\frac{N\mp 1}{N}\quad
\gamma^{(1)}_\pm=\frac{N\mp 1}{2N}\left[-21\pm\frac{57}{N}\mp\frac{19}{3}N
 \pm\frac{4}{3} f\right]\quad ({\rm NDR})
\end{equation}
For $i,j=+,-$ we introduce
\begin{equation}\label{dzi}
d^{(f)}_i=\frac{\gamma^{(0)}_i}{2\beta_0} \qquad
J^{(f)}_i= \frac{d^{(f)}_i}{\beta_0}\beta_1 - \frac{\gamma^{(1)}_i}{2\beta_0}
\end{equation}
and
\begin{equation}\label{dzij}
d^{(f)}_{ij}=d^{(f)}_i+d^{(f)}_j \qquad
J^{(f)}_{ij}=J^{(f)}_i+J^{(f)}_j
\end{equation}
The essential step consists in matching (\ref{hds2c}) onto an
effective theory without charm, which will contain the single
operator $Q=(\bar sd)_{V-A}(\bar sd)_{V-A}$. In NLO this matching
has to be performed to $\ord(\as)$. At a normalization
scale $\mu_c$ it reads explicitly, expressed in terms of operator
matrix elements $(i, j=+, -)$
\begin{equation}\label{oijq}
\langle O_{ij}\rangle=\frac{m^2_c(\mu_c)}{8\pi^2}\left[ \tau_{ij}+
 \frac{\as(\mu_c)}{4\pi}\left( \kappa_{ij}
 \ln\frac{\mu^2_c}{m^2_c}+\beta_{ij}\right)\right]
 \langle Q\rangle
\end{equation}
\begin{equation}\label{tauij}
\tau_{++}=\frac{N+3}{4}
\qquad
\tau_{+-}=\tau_{-+}=-\frac{N-1}{4}
\qquad
\tau_{--}=\frac{N-1}{4}
\end{equation}
\begin{equation}\label{kapij}
\kappa_{++}=3(N-1)\tau_{++}
\qquad
\kappa_{+-}=\kappa_{-+}=3(N+1)\tau_{+-}
\qquad
\kappa_{--}=3(N+3)\tau_{--}
\end{equation}
The $\beta_{ij}$ are scheme dependent. In the NDR scheme they are
given by \cite{herrlichnierste:93}
\begin{eqnarray}\label{betij}
\beta_{++} &=& (1-N)\left(\frac{N^2-6}{12N}\pi^2+
 3\frac{-N^2+2N+13}{4N} \right)  \nonumber \\
\beta_{+-}=\beta_{-+} &=& (1-N)\left(\frac{-N^2+2N-2}{12N}\pi^2+
 \frac{3N^2+13}{4N} \right)    \\
\beta_{--} &=& (1-N)\left(\frac{N^2-4N+2}{12N}\pi^2-
 \frac{3N^2+10N+13}{4N} \right)  \nonumber
\end{eqnarray}
Now, starting from (\ref{hds2c}), evolving $C_i$ from $M_W$ down to
$\mu_c$, integrating out charm at $\mu_c$ with the help of
(\ref{oijq}), evolving the resulting coefficient according to
(\ref{cqmu}) and recalling the definition of $\eta_1$ in
(\ref{hds2}), one finally obtains
\begin{eqnarray}\label{eta1}
\eta_1=&&[\as(\mu_c)]^{d_3}\sum_{i,j=+,-}
\left(\frac{\as(m_b)}{\as(\mu_c)}\right)^{d^{(4)}_{ij}}
\left(\frac{\as(M_W)}{\as(m_b)}\right)^{d^{(5 )}_{ij}}\times
\nonumber \\
&&\times\left[\tau_{ij}+\frac{\as(\mu_c)}{4\pi}\left(\kappa_{ij}
\ln\frac{\mu^2_c}{m^2_c}+\tau_{ij}(J^{(4)}_{ij}-J_3)+\beta_{ij}
\right)+\right. \nonumber \\
&&\left. +\tau_{ij}\left(\frac{\as(m_b)}{4\pi}
(J^{(5)}_{ij}-J^{(4)}_{ij})+\frac{\as(M_W)}{4\pi}
(B_i+B_j-J^{(5)}_{ij})\right)\right]
\end{eqnarray}
We conclude this section with a discussion of a few important
issues concerning the structure of this formula.
\begin{itemize}
\item
(\ref{eta1}), as first obtained in \cite{herrlichnierste:93}, represents the
next-to-leading order generalization of the leading log expression
for $\eta_1$ given in \cite{gilmanwise:83}. The latter follows as a
special case of (\ref{eta1}) when the $\ord(\as)$ correction
terms are put to zero.
\item
The expression (\ref{eta1}) is independent of the renormalization
scheme up to terms of the neglected order $\ord(\alpha^2_s)$.
We have written $\eta_1$ in a form, in which this scheme
independence becomes manifest: While the various $J$-terms, $B_i$ and
$\beta_{ij}$ in (\ref{eta1}) all depend on the renormalization
scheme when considered separately, the combinations
$\tau_{ij}(J^{(4)}_{ij}-J_3)+\beta_{ij}$,
$J^{(5)}_{ij}-J^{(4)}_{ij}$ and $B_i+B_j-J^{(5)}_{ij}$
are scheme invariant.
\item
The product $\eta_1(\mu_c) x_c(\mu_c)$ is independent of $\mu_c$
to the considered order,
\begin{equation}
\frac{d}{d\ln\mu_c}\eta_1(\mu_c) x_c(\mu_c)=\ord(\alpha^2_s)
\end{equation}
in accordance with the requirements of renormalization group
invariance. The cancellation of the $\mu_c$-dependence to
$\ord(\as)$ is related to the presence of an explicitly
$\mu_c$-dependent term at NLO in (\ref{eta1}) and is
guaranteed through the identity
\begin{equation}\label{kapid}
\kappa_{ij}=\tau_{ij}\left(\gamma_{m0}+\frac{\gamma^{(0)}}{2}-
\frac{\gamma^{(0)}_i+\gamma^{(0)}_j}{2}\right)
\end{equation}
which is easily verified using (\ref{gm01}), (\ref{gamq2}),
(\ref{gapm2}), (\ref{tauij}) and (\ref{kapij}).
\item
Also the ambiguity in the scale $\mu_W$, at which $W$ is integrated out,
is reduced from $\ord(\as)$ to $\ord(\alpha^2_s)$
when going from leading to next-to-leading order. As mentioned
above the dependence on the $b$-threshold scale $\mu_b$ is
$\ord(\alpha^2_s)$ in NLLA as well as in LLA.
Numerically the dependence on $\mu_b$ is very small. Also the
variation of the result with the high energy matching scale $\mu_W$
is considerably weaker than the residual dependence on $\mu_c$.
Therefore we have set $\mu_b=m_b$ and $\mu_W=M_W$ in (\ref{eta1}).
In numerical analyses we will take the dominant $\mu_c$-dependence
as representative for the short-distance scale ambiguity of $\eta_1$.
The generalization to the case $\mu_W\not= M_W$ is discussed in
\cite{herrlichnierste:93}. The more general case $\mu_b\not= m_b$ is trivially
obtained by substituting $m_b\to\mu_b$ in (\ref{eta1}).
\item
Note that due to the GIM structure of $O_{ij}$ no mixing under infinite
renormalization occurs between $O_{ij}$ and the local operator $Q$.
This is related to the absence of the logarithm in the function
$S_0(x_c)$ in \eqn{s0c}.
\end{itemize}
It is instructive to compare the results obtained for $\eta_1$
and $\eta_2$. Expanding (\ref{eta1}) to first order in $\as$,
in this way ``switching off'' the RG summations, we find
\begin{eqnarray}\label{e1lim}
&&[\as(\mu)]^{-2/9}\left(1+\frac{\as(\mu)}{4\pi}J_3\right)
\eta_1\doteq  \\
&& 1+\frac{\as}{4\pi}\left[\frac{\gamma^{(0)}}{2}\left(
\ln\frac{\mu^2}{M^2_W}+\ln\frac{m^2}{M^2_W}-1+\frac{2}{9}\pi^2\right)
+\gamma_{m0}\left(\ln\frac{\mu^2}{m^2}+\frac{1}{3}\right)\right]
\nonumber
\end{eqnarray}
where we have replaced $\mu_c\to\mu$ and $m_c\to m$. In deriving
(\ref{e1lim}) besides (\ref{kapid}) the following identities are
useful
\begin{equation}\label{tauid}
\sum_{i,j=+,-}\tau_{ij}=1\qquad
\sum_{i,j=+,-}\tau_{ij}\frac{\gamma^{(0)}_i+\gamma^{(0)}_j}{2}=
\gamma^{(0)}
\end{equation}
\begin{equation}\label{bid}
\sum_{i,j=+,-}\tau_{ij}(B_i+B_j)=2 B_+
\end{equation}
The same result (\ref{e1lim}) is obtained from $\eta_2$ as well,
if we set $m_c=m_b=\mu_t=\mu$, $m_t=m$ in (\ref{eta2}) and expand
for $m\ll M_W$. This exercise yields a useful cross-check between
the calculations for $\eta_1$ and $\eta_2$. In addition it gives
some further insight into the structure of the QCD corrections to
$\Delta S=2$ box diagrams, establishing $\eta_1$ and $\eta_2$ as two
different generalizations of the same asymptotic limit (\ref{e1lim}).

\subsection{The Top-Charm Contribution -- $\eta_3$}
\label{sec:HeffKKbar:eta3}
To complete the description of the $K^0-\bar K^0$ effective hamiltonian we
now turn to the mixed top-charm component, defined by the contribution
$\sim\lambda_c\lambda_t$ in (\ref{hds2}), and the associated QCD
correction factor $\eta_3$. The short distance QCD effects have been
first obtained within the leading log approximation by
\cite{gilmanwise:83}. The calculation of $\eta_3$ at next-to-leading
order is due to the work of \cite{herrlichnierste:95},
\cite{nierste:95}. As already mentioned, the renormalization group
analysis necessary for $\eta_3$ is more involved than in the cases of
$\eta_1$ and $\eta_2$. The characteristic differences will become clear
from the following presentation.
\\
We begin by writing down the relevant $\Delta S=1$ hamiltonian,
obtained after integrating out $W$ and top, which provides the basis
for the construction of the $\Delta S=2$ effective hamiltonian we want
to derive. It reads
\begin{equation}\label{hct1}
{\cal H}^{(1)}_{ct}=\frac{G_F}{\sqrt 2}\left(\sum_{q, q^\prime=u, c}
 V^\ast_{q^\prime s}V_{qd}\sum_{i=1,2}C_iQ^{q^\prime q}_i -
 \lambda_t\sum_{i=3}^6 C_i Q_i \right)
\end{equation}
with
\begin{equation}\label{qpq12}
Q^{q^\prime q}_1=(\bar s_iq^\prime_j)_{V-A}(\bar q_jd_i)_{V-A} \qquad
Q^{q^\prime q}_2=(\bar s_iq^\prime_i)_{V-A}(\bar q_jd_j)_{V-A}
\end{equation}
and corresponds to the hamiltonian (\ref{eq:HeffKppc}), discussed in
chapter \ref{sec:HeffdF1:66}, except that we have included the
$\Delta C=1$ components $Q^{uc}_i$, $Q^{cu}_i$, which contribute in the
analysis of $\eta_3$.
By contrast to the simpler case of $\eta_1$ presented in the previous
section, now also the penguin operators $Q_i$, $i=3,\ldots, 6$
(\ref{eq:Kppbasis}) have to be considered. Being proportional to
$\lambda_t=V^\ast_{ts}V_{td}$ they will contribute to the
$\lambda_c\lambda_t$-part of (\ref{hds2}). We remark in this context
that, on the other hand, the penguin contribution to the 
$\lambda^2_t$-sector is entirely negligible. Since only light quarks
are involved in $Q_3,\ldots ,Q_6$, the double penguin diagrams, which 
would contribute to the $\lambda^2_t$-piece of the $\Delta S=2$
amplitude, are suppressed by at least a factor of $m^2_b/m^2_t$
compared with the dominant top-exchange effects discussed in section
\ref{sec:HeffKKbar:eta2}.
\\
At second order in (\ref{hct1}) $\Delta S=2$ transitions are generated.
Inserting (\ref{hct1}) in an expression similar to (\ref{heff2c}),
eliminating $\lambda_u$ by means of $\lambda_u=-\lambda_c-\lambda_t$
and collecting the terms proportional to $\lambda_c\lambda_t$, we
obtain the top-charm component of the effective $\Delta S=2$
hamiltonian in the form
\begin{equation}\label{hds2ct}
{\cal H}^{\Delta S=2}_{eff,ct}=\frac{G^2_F}{2}\lambda_c\lambda_t
 \sum_{i=\pm}\left[\sum_{j=1}^6 C_i C_j Q_{ij}
 +C_{7i} Q_7\right]
\end{equation}
where   
\begin{equation}\label{qij12}
Q_{ij}=-\frac{i}{2}\int d^4x\ T\left[
2Q^{uu}_i(x)Q^{uu}_j(0)-Q^{uc}_i(x)Q^{cu}_j(0)-
Q^{cu}_i(x)Q^{uc}_j(0) \right]
\end{equation}
for $j=1, 2$ and
\begin{equation}\label{qij36}
Q_{ij}=-\frac{i}{2}\int d^4x\ T\left[
\left(Q^{uu}_i(x)-Q^{cc}_i(x)\right)Q_j(0)+
Q_j(x)\left(Q^{uu}_i(0)-Q^{cc}_i(0)\right)\right]
\end{equation}
for $j=3, \ldots,6$.
\\
In defining these operators we have already omitted bilocal products
with flavor structure like $(\bar su\bar ud)\cdot(\bar sc\bar cd)$,
which cannot contribute to $\Delta S=2$ box diagrams. Furthermore,
for the factor entering the bilocal operators with index
$i$ we have changed the basis from $Q^{q^\prime q}_{1,2}$ to
$Q^{q^\prime q}_\pm$ given in (\ref{qqpm}).
In addition local counterterms proportional to the $\Delta S=2$
operator
\begin{equation}\label{q7def}
Q_7=\frac{m^2_c}{g^2}(\bar sd)_{V-A}(\bar sd)_{V-A}
\end{equation}
have been added to (\ref{hds2ct}). These are necessary here because
the bilocal operators can in general mix into $Q_7$ under infinite
renormalization, a fact related to the logarithm present in the
leading term $-x_c\ln x_c$ entering $S_0(x_c, x_t)$ in (\ref{s0ct}).
This behaviour is in contrast to the charm contribution, where the
corresponding function $S_0(x_c)=x_c$ does not contain a logarithmic term
and consequently no local $\Delta S=2$ counterterm is necessary in
(\ref{hds2c}). On the other hand the situation here is analogous to the
case of the charm contribution to the effective hamiltonian for
$K^+\to\pi^+\nu\bar\nu$ in section \ref{sec:HeffRareKB:kpnn} which
similarly behaves as $x_c\ln x_c$ in lowest order and correspondingly
requires a counterterm, as displayed in (\ref{hzop}) and (\ref{hbop}).

After integrating out top and $W$ at the high energy matching scale
$\mu_W={\cal O}(M_W)$, the Wilson coefficients $C_j$, $j=1, \ldots 6$
of (\ref{hct1}) and (\ref{hds2ct}) are given in the NDR-scheme by
(see section \ref{sec:HeffdF1:66})
\begin{eqnarray}\label{ctmuw}
\vec C^T(\mu_W) & = & (0, 1, 0, 0, 0, 0)+ 
  \frac{\alpha_s(\mu_W)}{4\pi}
  \left(6, -2, -\frac{2}{9}, \frac{2}{3}, -\frac{2}{9}, \frac{2}{3}
  \right)\ln\frac{\mu_W}{M_W} \nonumber\\ 
  & + & \frac{\alpha_s(\mu_W)}{4\pi}
  \left(\frac{11}{2}, -\frac{11}{6}, -\frac{1}{6}\tilde E_0(x_t),
  \frac{1}{2}\tilde E_0(x_t), -\frac{1}{6}\tilde E_0(x_t),
  \frac{1}{2}\tilde E_0(x_t)\right)
\end{eqnarray}
and $C_\pm=C_2\pm C_1$. $\tilde E_0(x_t)$ can be found in
(\ref{eq:Exttilde}).
The coefficient of $Q_7$ is obtained through matching the $\Delta S=2$
matrix element of the effective theory (\ref{hds2ct}) to the
corresponding full theory matrix element, which is in the required
approximation ($x_c\ll 1$) given by (compare (\ref{hds2}))
\begin{equation}\label{afullct}
A_{full, ct}=\frac{G^2_F}{16\pi^2}M^2_W 2\lambda_c\lambda_t
S_0(x_c,x_t)\langle Q\rangle
\end{equation}
At next-to-leading order this matching has to be done to one loop,
including finite parts. Note that here the loop effect is due to
electroweak interactions and QCD does not contribute explicitly in this
step.  The matching condition determines the sum $C_7\equiv
C_{7+}+C_{7-}$, which in the NDR scheme and with the conventional
definition of evanescent operators, \cite{burasweisz:90}, see also
\cite{herrlichnierste:95}, \cite{nierste:95}, reads
\begin{equation}\label{c7muw}
C_7(\mu_W)=\frac{\alpha_s(\mu_W)}{4\pi}\left[-8\ln\frac{\mu_W}{M_W}+
4\ln x_t-\frac{3x_t}{1-x_t}-\frac{3x^2_t\ln x_t}{(1-x_t)^2}+2\right]
\end{equation}
at next-to-leading order.
In leading log approximation one simply would have $C_7(\mu_W)=0$.
\\
The distribution of $C_7$ among $C_{7+}$ and $C_{7-}$ is arbitrary
and has no impact on the physics. For example we may choose
\begin{equation}\label{c7pm}
C_{7+}=C_7\qquad\qquad  C_{7-}=0
\end{equation}
Having determined the initial values of the Wilson coefficients
\begin{equation}\label{cvpmt}
\vec C^{(\pm)T}\equiv (C_\pm C_1, \ldots, C_\pm C_6, C_{7\pm})
\end{equation}
at a scale $\mu_W$, $\vec C^{(\pm)}(\mu_W)$, the next step consists
in solving the RG equations to determine $\vec C^{(\pm)}(\mu_c)$
at the charm mass scale $\mu_c={\cal O}(m_c)$. The renormalization
group evolution of $\vec C^{(\pm)}$ is given by
\begin{equation}\label{rgct}
\frac{d}{d\ln\mu}\vec C^{(\pm)}(\mu)=\gamma^{(\pm)T}_{ct}
  \vec C^{(\pm)}(\mu)
\end{equation}
\begin{equation}\label{gpmct}
\gamma^{(\pm)}_{ct}=
\left( \begin{array}{cc}
\gamma_s+\gamma_\pm\cdot 1 & \vec\gamma_{\pm 7}\\
\vec 0^T  & \gamma_{77}  \end{array}  \right)
\end{equation}
Here $\gamma_s$ is the standard $6\times 6$ anomalous dimension matrix
for the $\Delta S=1$ effective hamiltonian including QCD penguins
from (\ref{eq:gsexpKpp}), (\ref{eq:gs0Kpp}) and (\ref{eq:gs1ndrN3Kpp})
(NDR-scheme). Similarly $\gamma_\pm$ are the anomalous dimensions of the
current-current operators. They can be obtained as
$\gamma_\pm=\gamma_{s,11}\pm\gamma_{s,12}$ and are also given in
section \ref{sec:HeffdF1:22}.
\\
$\gamma_{77}$ represents the anomalous dimension of $Q_7$ (\ref{q7def})
and reads
\begin{equation}\label{g77}
\gamma_{77}=\gamma_++2\gamma_m+2\beta(g)/g=
\frac{\alpha_s}{4\pi}\gamma^{(0)}_{77}+
\left(\frac{\alpha_s}{4\pi}\right)^2\gamma^{(1)}_{77}
\end{equation}
For $N=3$ and in NDR
\begin{equation}\label{g7701}
\gamma^{(0)}_{77}=-2+\frac{4}{3}f\qquad
\gamma^{(1)}_{77}=\frac{175}{3}+\frac{152}{9}f
\end{equation}
Finally $\vec\gamma_{\pm 7}$, the vector of anomalous dimensions expressing
the mixing of the bilocal operators $Q_{\pm i}$ ($i=1,\ldots,6$)
into $Q_7$, is given by
\begin{equation}\label{gpm7}
\vec\gamma_{\pm 7}=
\frac{\alpha_s}{4\pi}\vec\gamma^{(0)}_{\pm 7}+
\left(\frac{\alpha_s}{4\pi}\right)^2\vec\gamma^{(1)}_{\pm 7}
\end{equation}
where
\begin{equation}\label{gp70}
\vec\gamma^{(0)T}_{+7}=
\left(-16,-8,-32,-16,32,16\right)
\end{equation}
\begin{equation}\label{gm70}
\vec\gamma^{(0)T}_{-7}=
\left(8,0,16,0,-16,0\right)
\end{equation}
\begin{equation}\label{gp71}
\vec\gamma^{(1)T}_{+7}=
\left(-212,-28,-424,-56,\frac{1064}{3},\frac{832}{3}\right)
\end{equation}
\begin{equation}\label{gm71}
\vec\gamma^{(1)T}_{-7}=
\left(276,-92,552,-184,-\frac{1288}{3},0 \right)
\end{equation}
The scheme-dependent numbers in $\vec\gamma^{(1)}_{\pm 7}$ are given
here in the NDR-scheme with the conventional treatment of evanescent
operators as described in \cite{burasweisz:90}, \cite{herrlichnierste:95},
\cite{nierste:95}.
\\
In order to solve the RG equation (\ref{rgct}) it is useful
\cite{herrlichnierste:95}, \cite{nierste:95} to define the
eight-dimensional vector ($\vec C^T=(C_1,\ldots,C_6)$)
\begin{equation}\label{d8def}
\vec D^T=(\vec C^T,C_{7+}/C_+,C_{7-}/C_-)
\end{equation}
which obeys
\begin{equation}\label{rgd8}
\frac{d}{d\ln\mu}\vec D=\gamma^T_{ct}\vec D
\end{equation}
where
\begin{equation}\label{gct8}
\gamma_{ct}=
\left(\begin{array}{ccc}
\gamma_s & \vec\gamma_{+7} & \vec\gamma_{-7} \\
\vec 0^T & \gamma_{77}-\gamma_+ & 0 \\
\vec 0^T & 0 & \gamma_{77}-\gamma_-
\end{array}\right)
\end{equation}
The solution of (\ref{rgd8}) proceeds in the standard fashion as
described in section \ref{sec:basicform:wc:rgf} and has the form
\begin{equation}\label{dcsol}
\vec D(\mu_c)=U_4(\mu_c,\mu_b)M(\mu_b)U_5(\mu_b,\mu_W)\vec D(\mu_W)
\end{equation}
similarly to (\ref{cthr}).
The b-quark-threshold matching matrix $M(\mu_b)$ is an $8\times 8$ 
matrix whose $6\times 6$ submatrix $M_{ij}$, $i,j=1,\ldots,6$
is identical to the matrix $M$ described in section
\ref{sec:HeffdF1:66:Mm}. The remaining elements are $M_{77}=M_{88}=1$
and zero otherwise.
From (\ref{dcsol}) the Wilson coefficients $C_i(\mu_c)$ are
obtained as
\begin{equation}\label{cdrel}
C_i(\mu_c)=D_i(\mu_c)\quad i=1,\ldots,6\qquad
C_7(\mu_c)=C_+(\mu_c)D_7(\mu_c)+C_-(\mu_c)D_8(\mu_c)
\end{equation}

The final step in the calculation of $\eta_3$ consists in removing
the charm degrees of freedom from the effective theory.
Without charm the effective short-distance hamiltonian corresponding
to (\ref{hds2ct}) can be written as 
\begin{equation}\label{hctq}
{\cal H}^{\Delta S=2}_{eff,ct}=\frac{G^2_F}{2}\lambda_c\lambda_t
C_{ct}  Q
\end{equation}
The matching condition is obtained by equating the matrix elements of
(\ref{hds2ct}) and (\ref{hctq}), evaluated at a scale 
$\mu_c={\cal O}(m_c)$. At next-to-leading order one needs the finite
parts of the matrix elements of $Q_{ij}$, which can be written in the
form
\begin{equation}\label{qijq}
\langle Q_{ij}(\mu_c)\rangle=\frac{m^2_c(\mu_c)}{8\pi^2}
r_{ij}(\mu_c)\langle Q\rangle
\end{equation}
where in the renormalization scheme described above after eq. (\ref{gm71})
the $r_{ij}$ are given by
\begin{equation}\label{rijt}
r_{ij}(\mu_c)=
\left\{ \begin{array}{ll}
(4\ln(\mu_c/m_c)-1)\tau_{ij} & j=1,2 \\
(8\ln(\mu_c/m_c)-4)\tau_{ij} & j=3,4 \\
(-8\ln(\mu_c/m_c)+4)\tau_{ij} & j=5,6 \end{array}\right.
\end{equation}
\begin{equation}\label{tjodd}
\tau_{\pm 1}=\tau_{\pm 3}=\tau_{\pm 5}=(1\pm 3)/2
\end{equation}
\begin{equation}\label{tjevn}
\tau_{+j}=1\qquad \tau_{-j}=0\qquad  \mbox{$j$ even}
\end{equation}
Using (\ref{qijq}), the matching condition at $\mu_c$ between
(\ref{hds2ct}) and (\ref{hctq}) implies
\begin{equation}\label{cctmuc}
C_{ct}(\mu_c)=\sum_{i=\pm}\sum_{j=1}^6 C_i(\mu_c) C_j(\mu_c)
\frac{m^2_c(\mu_c)}{8\pi^2}r_{ij}(\mu_c)+C_7(\mu_c)
\frac{m^2_c(\mu_c)}{4\pi\alpha_s(\mu_c)}
\end{equation}
Evolving $C_{ct}$ from $\mu_c$ to $\mu<\mu_c$ in a three-flavor theory
using (\ref{cqmu}) and comparing (\ref{hctq}) with (\ref{hds2}), 
we obtain the final result
\begin{displaymath}
\eta_3=\frac{x_c(\mu_c)}{S_0(x_c(\mu_c),x_t(\mu_W))}\alpha_s(\mu_c)^{2/9}
\Bigl[ \frac{\pi}{\alpha_s(\mu_c)}C_7(\mu_c)
\left(1-\frac{\alpha_s(\mu_c)}{4\pi}J_3\right)+ 
\end{displaymath}
\begin{equation}\label{eta3}
 +\frac{1}{2}\sum_{i=\pm}\sum_{j=1}^6 C_i(\mu_c) C_j(\mu_c)
r_{ij}(\mu_c)\Bigr] 
\end{equation}
One may convince oneself, that $\eta_3 S_0(x_c,x_t)$ is independent of the 
renormalization scales, in particular of $\mu_c$, up to terms of 
${\cal O}(x_c \alpha^{2/9}_s \alpha_s)$. 

Furthermore, using the formulae given in this
section, it is easy to see from the explicit expression (\ref{eta3}),
that $\eta_3 \alpha^{-2/9}_s\to 1$ in the limit $\alpha_s\to 0$,
as it should indeed be the case.
\\
The next-to-leading order formula (\ref{eta3}) for $\eta_3$, first
calculated in \cite{herrlichnierste:95}, \cite{nierste:95}, provides
the generalization of the leading log result obtained by
\cite{gilmanwise:83}. It is instructive to compare (\ref{eta3}) with
the leading order approximation, which can be written as
\begin{equation}\label{eta3lo}
\eta^{LO}_3=\alpha_s(\mu_c)^{2/9}
\frac{-\pi C^{LO}_7(\mu_c)}{\alpha_s(\mu_c)\ln x_c}
\end{equation}
using the notation of (\ref{eta3}). $C^{LO}_7$ denotes the coefficient
$C_7$, restricted to the leading logarithmic approximation.
Formula (\ref{eta3lo}), derived here as a special case of (\ref{eta3}),
is equivalent to the result obtained in \cite{gilmanwise:83}.
\\
If penguin operators and the b-quark threshold in the RG
evolution are neglected, it is possible to write down in closed form
a relatively simple, explicit expression for $\eta_3$. Using a 4-flavor
effective theory for the evolution from the $W$-scale down to the
charm scale, we find in this approximation
\begin{eqnarray}\label{eta3apx}
\eta_3 &=& \frac{x_c(\mu_c)}{S_0(x_c(\mu_c),x_t)}\alpha_s(\mu_c)^{2/9}\cdot
\nonumber \\
&\cdot& \Biggl[ \frac{\pi}{\alpha_s(\mu_c)}\left(-\frac{18}{7}K_{++}-
\frac{12}{11}K_{+-}+\frac{6}{29}K_{--}+\frac{7716}{2233}K_7\right)
\left(1-\frac{\alpha_s(\mu_c)}{4\pi}\frac{307}{162}\right)+ \nonumber \\
& & + \left(\ln\frac{\mu_c}{m_c}-\frac{1}{4}\right)
\left(3 K_{++}-2 K_{+-}+ K_{--}\right) + \nonumber \\
& & +\frac{262497}{35000}K_{++}-\frac{123}{625}K_{+-}+
\frac{1108657}{1305000}K_{--}-\frac{277133}{50750}K_7+ \nonumber \\
& & + K\left(-\frac{21093}{8750}K_{++}+\frac{13331}{13750}K_{+-}-
\frac{10181}{18125}K_{--}-\frac{1731104}{2512125}K_7\right)+ \nonumber \\
& & +\left(\ln x_t-\frac{3 x_t}{4(1-x_t)}-\frac{3x^2_t\ln x_t}{4(1-x_t)^2}
+\frac{1}{2}\right) K K_7 \Biggr]
\end{eqnarray}
where
\begin{equation}\label{kpm}
K_{++}=K^{12/25}\qquad K_{+-}=K^{-6/25}\qquad K_{--}=K^{-24/25}
\end{equation}
\begin{equation}\label{k7k}
K_7=K^{1/5}\qquad  K=\frac{\alpha_s(M_W)}{\alpha_s(\mu_c)}
\end{equation}
Here we have set $\mu_W=M_W$. (\ref{eta3apx}) represents the
next-to-leading order generalization of an approximate formula for the
leading log $\eta_3$, also omitting gluon penguins, that has been first
given in \cite{gilmanwise:83}.
The analytical expression for $\eta_3$ in \eqn{eta3apx}
provides an excellent approximation, deviating generally by less
than $1\%$ from the full result.

\subsection{Numerical Results}
\label{sec:HeffKKbar:Num}

\subsubsection{General Remarks}
\label{sec:HeffKKbar:Num:Rem}
After presenting the theoretical aspects of the short-distance
QCD factors $\eta_1$, $\eta_2$ and $\eta_3$ in the previous
sections, we shall now turn to a discussion of their
numerical values. However, before considering explicit numbers,
we would like to make a few general remarks.
\\
First of all, it is important to recall
that in the matrix element
$\langle\bar K^0|{\cal H}^{\Delta S=2}_{eff}|K^0\rangle$
(see (\ref{hds2})), only the complete products
\begin{equation}\label{setaiq}
S_{0i}\cdot \eta_i[\as(\mu)]^{-2/9}\left[
1+\frac{\as(\mu)}{4\pi}J_3\right]
\langle\bar K^0|Q(\mu)|K^0\rangle\equiv
C_i(\mu) \langle\bar K^0|Q(\mu)|K^0\rangle
\end{equation}
are physically relevant.  Here $S_{0i}$ denote the appropriate quark
mass dependent functions $S_0$ for the three contributions ($i=1$, $2$,
$3$) in (\ref{hds2}).  None of the factors in (\ref{setaiq}) is
physically meaningful by itself. In particular, there is some
arbitrariness in splitting the product (\ref{setaiq}) into the
short-distance part and the matrix element of $Q$ (\ref{qsdsd})
containing long distance contributions. This arbitrariness has of
course no impact on the physical result. However, it is essential to
employ a definition for the operator matrix element that is consistent
with the short-distance QCD factor used.
\\
Conventionally, the matrix element $\langle\bar K^0|Q|K^0\rangle$
is expressed in terms of the so-called bag parameter $B_K(\mu)$
defined through
\begin{equation}\label{bkdef}
\langle\bar K^0|Q(\mu)|K^0\rangle\equiv\frac{8}{3}F^2_K m^2_K B_K(\mu)
\end{equation}
where $m_K$ is the kaon mass and $F_K=160 MeV$ is the kaon decay
constant. In principle, one could just use the scale- and scheme
dependent bag factor $B_K(\mu)$ along with the coefficient
functions $C_i(\mu)$ as defined by (\ref{setaiq}), evaluated
at the same scale and in the same renormalization scheme.
However, it has become customary to define the short-distance
QCD correction factors $\eta_i$ by splitting off from the
Wilson coefficient $C_i(\mu)$ the factor
$[\as(\mu)]^{-2/9}[1+\as(\mu)/(4\pi)\ J_3]$, which
carries the dependence on the renormalization scheme and the scale $\mu$.
This factor is then
attributed to the matrix element of $Q$, formally cancelling its
scale and scheme dependence. Accordingly one defines a
renormalization scale and scheme invariant bag parameter $B_K$
(compare (\ref{setaiq}), (\ref{bkdef}))
\begin{equation}\label{bkbkmu}
B_K\equiv [\as(\mu)]^{-2/9}\left[
1+\frac{\as(\mu)}{4\pi}J_3\right] B_K(\mu)
\end{equation}
If the $\eta_i$ as described in this report are employed to
describe the short-distance QCD corrections, eq. (\ref{bkbkmu})
is the consistent definition to be used for the kaon bag
parameter.
\\
Eventually the quantity $B_K(\mu)$ should be calculated within
lattice QCD. At present, the analysis of \cite{sharpe:94}, for
example, gives a central value of $B_K(2 GeV)_{NDR}=0.616$, with
some still sizable uncertainty. For a recent review see also
\cite{soni:95}.
This result already incorporates the lattice-continuum
theory matching and refers to the usual NDR scheme.
It is clear that the NLO calculation of short-distance QCD effects
is essential for consistency with this matching and for a proper
treatment of the scheme dependence. Both require ${\cal O}(\as)$
corrections, which go beyond the leading log approximation.
\\
To convert to the scheme invariant parameter $B_K$ one uses
(\ref{bkbkmu}) with the NDR-scheme value for $J_3=307/162$ to
obtain $B_K=0.84$. Note that the factor involving $J_3$ in (\ref{bkbkmu}),
which appears at NLO, increases the r.h.s. of (\ref{bkbkmu}) by
$\approx 4.5\%$. The leading factor $\as^{-2/9}$ is about $1.31$.
Of course, the fact that there is presently still a rather large
uncertainty in the calculation of the hadronic matrix element is
somewhat forgiving, regarding the precise definition of $B_K$.
However, as the lattice calculations improve further and the errors
decrease, the issue of a consistent definition of the $\eta_i$
and $B_K$ will become crucial and it is important to keep
relation (\ref{bkbkmu}) in mind.

Let us next add a side remark concerning the separation of the
full amplitude into the formally RG invariant factors $\eta_i$
and $B_K$.
This separation is essentially unique, up to trivial constant factors,
if the evolution from the charm scale $\mu_c$ down to a "hadronic"
scale $\mu<\mu_c$ is written in the resummed form as shown in
(\ref{cqmu}) and one requires that all factors depending on the scale
$\mu$ are absorbed into the matrix element. On the other hand the
hadronic scale $\mu={\cal O}(1 GeV)$ is not really much different
from the charm scale $\mu_c={\cal O}(m_c)$, so that the logarithms
$\ln\mu/\mu_c$ are not very large. Therefore one could argue that
it is not necessary to resum those logarithms. In this case the
first two factors on the r.h.s. of (\ref{cqmu}) could be expanded
to first order in $\as$ and the amplitude (\ref{setaiq})
would read
\begin{equation}\label{ciqmumuc}
C_i(\mu_c)\left(1+\frac{\as}{\pi}\ln\frac{\mu}{\mu_c}\right)
\langle\bar K^0|Q(\mu)|K^0\rangle
\end{equation}
From this expression it is obvious, that the separation of the
physical amplitude into scheme invariant short-distance factors
and a scheme invariant matrix element is in general not unique.
This illustrates once more the ambiguity existing for theoretical
concepts such as operator matrix elements or QCD correction factors,
which only cancels in physical quantities.
\\
For definiteness, we will stick to the RG improved form also for the
evolution between $\mu_c$ and $\mu$ and the definitions for
$\eta_i$ and $B_K$ that we have discussed in detail above.


\subsubsection{Results for $\eta_1$, $\eta_2$ and $\eta_3$}
\label{sec:HeffKKbar:Num:Res}
We are now ready to quote numerical results for the
short-distance QCD corrections $\eta_i$ at next-to-leading order
and to compare them with the leading order approximation.
\\
The factors $\eta_1$ and $\eta_3$ have been analyzed in detail in
\cite{herrlichnierste:93} and \cite{nierste:95}.
Here we summarize briefly their main results.
Using our central parameter values $m_c(m_c)=1.3 GeV$,
$\Lms^{(4)}=0.325 GeV$, $m_t(m_t)=170 GeV$ and
fixing the scales as $\mu_c=m_c$, $\mu_W=M_W$ for $\eta_1$,
$\mu_W=130 GeV$ for $\eta_3$, one obtains at NLO
\begin{equation}\label{eta13nlo}
\eta_1=1.38\qquad\qquad \eta_3=0.47
\end{equation}
This is to be compared with the LO values corresponding to the same
input $\eta^{LO}_1=1.12$, $\eta^{LO}_3=0.35$.  We note that the
next-to-leading order corrections are sizable, typically $20\%-30\%$,
but still perturbative.  The numbers above may be compared with the
leading log values $\eta^{LO}_1=0.85$ and $\eta^{LO}_3=0.36$ that have
been previously used in the literature, based on the choice $m_c=1.4
GeV$, $\Lambda_{QCD}=0.2 GeV$ and $\mu_W=M_W$.  The considerable
difference between the two LO values for $\eta_1$ mainly reflects the
large dependence of $\eta_1$ on $\Lambda_{QCD}$.
\\
In fact, when the QCD scale is allowed to vary within
$\Lms^{(4)}=(0.325\pm 0.110)GeV$, the
value for $\eta_1$ (NLO) changes by $\sim\pm 35\%$. The leading order
result $\eta^{LO}_1$ appears to be slightly less sensitive to
$\Lambda_{QCD}$. However, in this approximation the relation of
$\Lambda_{QCD}$ to $\Lms^{(4)}$ is not well
defined, which introduces an additional source of uncertainty
when working to leading logarithmic accuracy.
\\
The situation is much more favorable in the case of $\eta_3$, where
the sensitivity to $\Lms^{(4)}$ is quite small,
$\sim\pm 3\%$.
Likewise the dependence on the charm quark mass is very small for both
$\eta_1$ and $\eta_3$. Using $m_c(m_c)=(1.3\pm 0.05)GeV$ and the
central value for $\Lms^{(4)}$ it is about $\pm 4\%$ for $\eta_1$ and
entirely negligible for $\eta_3$.
\\
Finally, there are the purely theoretical uncertainties due to the
renormalization scales. They are dominated by the ambiguity related
to $\mu_c$. The products $S_0(x_c(\mu_c))\cdot\eta_1(\mu_c)$ and
$S_0(x_c(\mu_c),x_t)\cdot\eta_3(\mu_c)$ are independent of $\mu_c$
up to terms of the neglected order in RG improved perturbation theory.
In the case of $S_0(x_c(\mu_c))\cdot\eta_1(\mu_c)$
($S_0(x_c(\mu_c),x_t)\cdot\eta_3(\mu_c)$) the remaining sensitivity to
$\mu_c$ amounts to typically $\pm 15\%$ ($\pm 7\%$) at NLO. These
scale dependences are somewhat reduced compared to the leading order
calculation, where the corresponding uncertainty is around $\pm 30\%$
($\pm 10\%$).
\\
To summarize, sizable uncertainties are still associated with the
number for the QCD factor $\eta_1$, whose central value is found to
be $\eta_1=1.38$ \cite{herrlichnierste:93}.
On the other hand, the prediction for $\eta_3$ appears to be quite
stable and can be reliably determined as $\eta_3=0.47\pm 0.03$
\cite{herrlichnierste:95}, \cite{nierste:95}.  One should emphasize
however, that these conclusions have their firm basis only within the
framework of a complete NLO analysis, as the one performed in
\cite{herrlichnierste:93}, \cite{nierste:95}.  Fortunately the quantity
$\eta_1$, for which a high precision seems difficult to achieve, plays
a less important role in the phenomenology of indirect CP violation.

Finally, we turn to a brief discussion of $\eta_2$ \cite{burasjaminweisz:90},
representing the short-distance QCD effects of the top-quark
contribution. For central parameter values, in particular
$\Lms^{(4)}=0.325 GeV$ and $m_t(m_t)=170 GeV$, and for $\mu_t=m_t(m_t)$
the numerical value is
\begin{equation}\label{eta2num}
\eta_2=0.574
\end{equation}
Varying the QCD scale within
$\Lms^{(4)}=(0.325\pm 0.110) GeV$ results in a
$\pm 0.5\%$ change in $\eta_2$. The dependence on $m_t(m_t)$ is even
smaller, only $\pm 0.3\%$ for $m_t(m_t)=(170\pm 15)GeV$.
It is worthwhile to compare the NLO results with the leading log
approximation. Using the same input as before yields a central value of
$\eta^{LO}_2=0.612$, about $7\%$ larger as the NLO result (\ref{eta2num}).
However, what is even more important than the difference in central
values is the quite striking reduction of scale uncertainty when going
from the leading log approximation to the full NLO treatment.
Recall that the $\mu_t$-dependence in $\eta_2$ has to cancel the
scale dependence of the function $S_0(x_t(\mu_t))$. Allowing for a
typical variation of the renormalization scale $\mu_t={\cal O}(m_t)$
from $100 GeV$ to $300 GeV$ results in a sizable change in
$S_0(x_t(\mu_t)) \eta^{LO}_2$ of $\pm 9\%$. In fact, in leading order
the $\mu_t$-dependence of $\eta_2$ has even the wrong sign, re-inforcing
the scale dependence present in $S_0(x_t(\mu_t))$ instead of
reducing it. The large sensitivity to the unphysical parameter $\mu_t$
is essentially eliminated (to $\pm 0.4\%$) for $\eta_2 S_0(x_t)$ at
NLO, a quite remarkable improvement of the theoretical accuracy. The
situation here is similar to the case of the top-quark dominated
rare K and B decays discussed in sections
\ref{sec:HeffRareKB}, \ref{sec:Kpnn} and \ref{sec:BXnnBmm}.
For a further illustration of the reduction in scale uncertainty see
the discussion of the analogous case of $\eta_{2B}$ in section
\ref{sec:HeffBBbar:Num}.
\\
The dependence of $\eta_2$ on the charm and bottom threshold scales
$\mu_c={\cal O}(m_c)$ and  $\mu_b={\cal O}(m_b)$ is also extremely
weak. Taking $1GeV\leq\mu_c\leq 3GeV$ and $3GeV\leq\mu_b\leq 9GeV$
results in a variation of $\eta_2$ by merely $\pm 0.26\%$ and
$\pm 0.06\%$, respectively.
\\
In summary, the NLO result for $\eta_2 S_0(x_t)$ is, by contrast to the
leading logarithmic approximation, essentially free from theoretical
uncertainties. Furthermore, $\eta_2$ is also rather insensitive to the
input parameters $\Lms$ and $m_t$. The top
contribution plays the dominant role for indirect CP violation in the
neutral kaon system. The considerable improvement in the theoretical
analysis of the short-distance QCD factor $\eta_2$ brought about by
the next-to-leading order calculation is therefore particularly
satisfying.
