\section{The Effective $\Delta F=1$ Hamiltonian: Inclusion of QCD Penguin
         Operators}
         \label{sec:HeffdF1:66}
In section \ref{sec:HeffdF1:22} we have restricted ourselves 
to current-current operators when considering
QCD corrections to the effective $\Delta F=1$ ($F=B, C, S$)
hamiltonian for weak decays.

As already mentioned in section \ref{sec:basicform:rg:pop} e.g.\ for
the $\dS$ case the special flavour structure of $Q_{2} = \left( \bar s
u \right)_{\rm V-A} \left( \bar u d \right)_{\rm V-A}$ allows not only
for QCD corrections of the current-current type as in
fig.\ \ref{fig:1loopeff}\,(a)--(c) from which the by now well known
second current-current operator $Q_1$ is created.  For a complete
treatment of QCD corrections all possible ways of attaching a gluon to
the initial weak $\Delta F=1$ transition operator $Q_2$ have to be
taken into account. Therefore attaching gluons to $Q_2$ in the form of
diagrams (d.1) and (d.2) in fig.\ \ref{fig:1loopeff}, generates a
completely new set of four-quark operators, the so-called QCD penguin
operators, usually denoted as $Q_3, \ldots, Q_6$\footnote{Obviously,
whether or not it is possible to form a closed fermion loop as in a
type-1 insertion or to connect the two currents to yield a continuous
fermion line as required for a type-2 insertion strongly depends on the
flavour structure of the operator considered. E.g.\ for $Q_2$ only the
type-2 penguin diagram contributes. This feature can be exploited
to obtain NLO anomalous dimension matrices in the NDR scheme 
without the necessity of calculating closed fermion loops with $\gamma_5$
\cite{burasetal:92b}, \cite{burasetal:92c}.}. This procedure is often
referred to as inserting $Q_2$ into type--1 and type--2 penguin
diagrams.


The $\dS$ effective hamiltonian for $\Kpipi$ at scales $\mu < \mc$ then
reads
\begin{equation}
\Heff(\dS) = \frac{G_F}{\sqrt{2}} \V{us}^* \V{ud}^{} \sum_{i=1}^{6}
\left( z_i(\mu) + \tau \; y_i(\mu) \right) Q_i \, ,
\label{eq:HeffKpp}
\end{equation}
with
\begin{equation}
\tau = -\frac{\V{ts}^*\V{td}^{}}{\V{us}^*\V{ud}^{}} \, .
\label{eq:tauKpp}
\end{equation}
The set of four-quark operators $\vec{Q}(\mu)$ and Wilson coefficients
$\vec{z}(\mu)$ and $\vec{y}(\mu)$ will be discussed one by one in
the subsections below.

\subsection{Operators}
            \label{sec:HeffdF1:66:op}
The basis of four-quark operators for the $\dS$ effective hamiltonian
in \eqn{eq:HeffKpp} is given in explicit form by
\begin{eqnarray}
Q_{1} & = & \left( \bar s_{i} u_{j}  \right)_{\rm V-A}
            \left( \bar u_{j}  d_{i} \right)_{\rm V-A}
\, , \nn \\
Q_{2} & = & \left( \bar s u \right)_{\rm V-A}
            \left( \bar u d \right)_{\rm V-A}
\, , \nn \\
Q_{3} & = & \left( \bar s d \right)_{\rm V-A}
   \sum_{q} \left( \bar q q \right)_{\rm V-A}
\, , \label{eq:Kppbasis} \\
Q_{4} & = & \left( \bar s_{i} d_{j}  \right)_{\rm V-A}
   \sum_{q} \left( \bar q_{j}  q_{i} \right)_{\rm V-A}
\, , \nn \\
Q_{5} & = & \left( \bar s d \right)_{\rm V-A}
   \sum_{q} \left( \bar q q \right)_{\rm V+A}
\, , \nn \\
Q_{6} & = & \left( \bar s_{i} d_{j}  \right)_{\rm V-A}
   \sum_{q} \left( \bar q_{j}  q_{i} \right)_{\rm V+A}
\, . \nn
\end{eqnarray}
As already mentioned, this basis closes under QCD renormalization.  

For $\mu < \mc$ the sums over active quark flavours in \eqn{eq:Kppbasis}
run over $u$, $d$ and $s$.  However, when  $\mb > \mu > \mc$ is
considered also $q=c$ has to be included. Moreover, in this case two
additional current--current operators have to be taken into account
\begin{equation}
Q_1^c = \left(\bar s_i c_j  \right)_{\rm V-A}
        \left(\bar c_j  d_i \right)_{\rm V-A}
\, , \qquad
Q_2^c = \left(\bar s c \right)_{\rm V-A}
        \left(\bar c d \right)_{\rm V-A} \, .
\label{eq:KppQ12c}
\end{equation}
and the effective hamiltonian takes the form
\begin{equation}
\Heff(\dS) = \frac{G_F}{\sqrt{2}} \V{us}^* \V{ud}^{} 
\left[(1-\tau) \sum_{i=1}^{2} z_i(\mu) (Q_i-Q^c_i) +
\tau \; \sum_{i=1}^{6} v_i(\mu)  Q_i \right]
\label{eq:HeffKppc}
\end{equation}

\subsection{Wilson Coefficients}
            \label{sec:HeffdF1:66:wc}
For the Wilson coefficients $y_i(\mu)$ and $z_i(\mu)$ in
eq.~\eqn{eq:HeffKpp} one has
\begin{equation}
y_i(\mu) = v_i(\mu) - z_i(\mu) \, .
\label{eq:WCy}
\end{equation}
The coefficients $z_i$ and $v_i$ are the components of the six
dimensional column vectors $\vec{v}(\mu)$ and $\vec{z}(\mu)$. Their
RG evolution is given by
\begin{equation}
\vec{v}(\mu) =
\hU_3(\mu,\mc) \hM(\mc) \hU_4(\mc,\mb) \hM(\mb) \hU_5(\mb,\mw)
\vec{C}(\mw) \, ,
\label{eq:WCv}
\end{equation}
\begin{equation}
\vec{z}(\mu) = \hU_3(\mu,\mc) \vec{z}(\mc) \, .
\label{eq:WCz}
\end{equation}
Here $\hU_f(m_1,m_2)$ denotes the full NLO evolution matrix for
$f$ active flavours. $\hM(m_i)$ is the matching matrix at quark
threshold $m_i$ given in eq.\ \eqn{mdrt}.  These two matrices will be
discussed in more detail in subsections~\ref{sec:HeffdF1:66:rge} and
\ref{sec:HeffdF1:66:Mm}, respectively.

The initial values $\vC(\mw)$ necessary for the RG evolution of
$\vv(\mu)$ in eq.~\eqn{eq:WCv} can be found according to the procedure
of matching the effective (fig.\ \ref{fig:1loopeff}) onto the full
theory (fig.\ \ref{fig:1loopful}) as summarized in
section~\ref{sec:basicform:wc}.  For the NDR scheme one obtains
\cite{burasetal:92a}
\begin{eqnarray}
C_1(\mw) &=&     \frac{11}{2} \; \frac{\as(\mw)}{4\pi} \, ,
\label{eq:CMw1QCD} \\
C_2(\mw) &=& 1 - \frac{11}{6} \; \frac{\as(\mw)}{4\pi} \, ,
\label{eq:CMw2QCD} \\
C_3(\mw) &=& -\frac{\as(\mw)}{24\pi} \widetilde{E}_0(x_t) \, ,
\label{eq:CMw3QCD} \\
C_4(\mw) &=& \frac{\as(\mw)}{8\pi} \widetilde{E}_0(x_t) \, ,
\label{eq:CMw4QCD} \\
C_5(\mw) &=& -\frac{\as(\mw)}{24\pi} \widetilde{E}_0(x_t) \, ,
\label{eq:CMw5QCD} \\
C_6(\mw) &=& \frac{\as(\mw)}{8\pi} \widetilde{E}_0(x_t) \, ,
\label{eq:CMw6QCD}
\end{eqnarray}
where
\begin{eqnarray}
E_0(x) &=& -\frac{2}{3} \ln x + \frac{x (18 -11 x - x^2)}{12 (1-x)^3} +
          \frac{x^2 (15 - 16 x  + 4 x^2)}{6 (1-x)^4} \ln x \, ,
\label{eq:Ext} \\
\widetilde{E}_0(x_t) &=& E_0(x_t) - \frac{2}{3}
\label{eq:Exttilde}
\end{eqnarray}
with
\begin{equation}
x_t = \frac{m_t^2}{\mw^2} \, .
\label{eq:xt}
\end{equation}
Here $E_0(x)$ results from the evaluation of the gluon penguin diagrams.

The initial values $\vec{C}(\mw)$ in the HV scheme can be found in
\cite{burasetal:92a}.

In order to calculate the initial conditions $\vec{z}(\mc)$ for
$z_i(\mu)$ in eq.~\eqn{eq:WCz} one has to consider the difference
$Q_2^u - Q_2^c$ of $Q_2$-type current-current operators
as can be seen explicitly in \eqn{eq:HeffKppc}. Due to the GIM
mechanism the coefficients $z_i(\mu)$ of penguin operators $Q_i$,
$i\not=1,2$ are zero in 5- and 4-flavour theories. The evolution for
scales $\mu > \mc$ involves then only the current-current operators
$Q^u_i-Q^c_i$, $i=1, 2$,  with initial conditions at
scale $\mu = \mw$
\begin{equation}
z_1(\mw) = C_1(\mw) \, ,
\qquad
z_2(\mw) = C_2(\mw) \, .
\label{eq:zMw12}
\end{equation}
$Q_{1,2}^u\equiv Q_{1,2}$ and $Q_{1,2}^c$ do not mix with each other under
renormalization. We then find
\begin{equation}
\left( \begin{array}{ll} z_1(\mc) \\ z_2(\mc) \end{array} \right) =
\hU_4(\mc,\mb) \; \hM(\mb) \; \hU_5(\mb,\mw) \;
\left( \begin{array}{ll} z_1(\mw) \\ z_2(\mw) \end{array} \right) \, ,
\label{eq:zmc12}
\end{equation}
where this time the evolution matrices $\hU_{4,5}$ contain only the $2
\times 2$ anomalous dimension submatrices describing the mixing between
current-current operators. The matching matrix $\hM(\mb)$ is then also
only the corresponding $2 \times 2$ submatrix of the full $6 \times 6$
matrix in \eqn{eq:MmKpp}. For the particular case of \eqn{eq:zmc12} it
simplifies to a unit matrix. When the charm quark is integrated out the
operators $Q_{1,2}^c$ disappear from the effective hamiltonian and the
coefficients $z_i(\mu)$, $i\not=1,2$ for penguin operators become
non-zero. In order to calculate $z_i(\mc)$ for penguin operators a
proper matching between effective 4- and 3-quark theories,
that is between \eqn{eq:HeffKppc} and \eqn{eq:HeffKpp}, has to be
made. For the 3-quark theory one obtains in the NDR scheme
\cite{burasetal:92d}
\begin{equation}
\vec{z}^{\rm }(\mc) =
\left( \begin{array}{c}
z_1^{\rm }(\mc) \\ z_2^{\rm }(\mc) \\
-\as/(24\pi) F_{\rm s}^{\rm }(\mc) \\ \as/(8\pi) F_{\rm s}^{\rm }(\mc) \\ 
-\as/(24\pi) F_{\rm s}^{\rm }(\mc) \\ \as/(8\pi) F_{\rm s}^{\rm }(\mc)
\end{array} \right) \, ,
\label{eq:zmc}
\end{equation}
where 
\begin{equation}
F_{\rm s}^{\rm }(\mc) =
-\frac{2}{3} \; z_2(\mc)
\label{eq:Fsmc}
\end{equation}
In the HV scheme $z_{1,2}$ are modified and one has $F_{\rm s}^{\rm }(\mc)
= 0$ or $z_i(\mc) = 0$ for $i\not=1,2$.

\subsection{Renormalization Group Evolution and Anomalous Dimension Matrices}
            \label{sec:HeffdF1:66:rge}
The general RG evolution matrix $\hU(m_1,m_2)$ from scale $m_2$ down to
$m_1 < m_2$ reads in pure QCD
\begin{equation}
\hU(m_1,m_2) \equiv T_g \exp
\int_{g(m_2)}^{g(m_1)} \!\! dg' \; \frac{\hg_{\rm s}^T(g'^2)}{\beta(g')} \, ,
\label{eq:UgeneralQCD}
\end{equation}
with $\hg_{\rm s}(g^2)$ being the full $6\times 6$ QCD anomalous dimension
matrix for $Q_1, \ldots, Q_6$.

For the case at hand it can be expanded in terms of $\as$ as follows
\begin{equation}
\hg_{\rm s}(g^2) = \frac{\as}{4\pi} \gs + \frac{\as^2}{(4\pi)^2} \gss
                   + \cdots \, .
\label{eq:gsexpKpp}
\end{equation}
Explicit expressions for $\gs$ and $\gss$ will be given below.

Using eq.\ \eqn{eq:gsexpKpp} the general QCD evolution matrix
$\hU(m_1,m_2)$ of eq.~\eqn{eq:UgeneralQCD} can be written as in \eqn{u0jj}
\cite{burasetal:92a}.
\begin{equation}
\hU(m_1,m_2) = 
\left( 1 + \frac{\as(m_1)}{4\pi} \hJ \right)
\hU^{(0)}(m_1,m_2)
\left( 1 - \frac{\as(m_2)}{4\pi} \hJ \right) \, ,
\label{eq:UQCDKpp}
\end{equation}
where $\hU^{(0)}(m_1,m_2)$ denotes the evolution matrix in the leading
logarithmic approximation and $\hJ$ summarizes the next-to-leading
correction to this evolution. Therefore, the full matrix $\hU(m_1,m_2)$
sums logarithms $(\as t)^n$ and $\as (\as t)^n$ with $t=\ln(m_2^2/m_1^2)$.
Explicit expressions for $\hU^{(0)}(m_1,m_2)$ and $\hJ$ are given in
eqs.~\eqn{u0vd}--\eqn{jvs}.

The LO anomalous dimension matrix $\gs$ of eq.\ \eqn{eq:gsexpKpp} has
the explicit form \cite{gaillard:74}, \cite{altarelli:74},
\cite{vainshtein:77}, \cite{gilman:79}, \cite{guberina:80}
\begin{equation}
\gs = 
\left(
\begin{array}{cccccc}
{{-6}\over N} & 6 & 0 & 0 & 0 & 0 \\ \svs
6 & {{-6}\over N} & {{-2}\over {3 N}} & {2\over 3} & {{-2}\over {3 N}} &\
  {2\over 3} \\ \svs
0 & 0 & {{-22}\over {3 N}} & {{22}\over 3} & {{-4}\over {3 N}} & {4\over 3}
\\ \svs
0 & 0 & 6 - {{2 f}\over {3 N}} & {{-6}\over N} + {{2 f}\over 3} & {{-2\
  f}\over {3 N}} & {{2 f}\over 3} \\ \svs
0 & 0 & 0 & 0 & {6\over N} & -6  \\ \svs
0 & 0 & {{-2 f}\over {3 N}} & {{2 f}\over 3} & {{-2 f}\over {3 N}} & {{-6\
  \left( -1 + {N^2} \right) }\over N} + {{2 f}\over 3}
\end{array}
\right)
\label{eq:gs0Kpp}
\end{equation}
The NLO anomalous dimension matrix $\gss$ of eq.\ \eqn{eq:gsexpKpp}
reads in the NDR scheme \cite{burasetal:92a}, \cite{ciuchini:93}
\begin{equation}
\gssndr\bigl|_{N=3} =
\left(
\begin{array}{cccccc}
-{{21}\over 2} - {{2\,f}\over 9} & {7\over 2} + {{2\,f}\over 3} & {{79}\over\
  9} & -{7\over 3} & -{{65}\over 9} & -{{7}\over{3}} \\ \mvs
{7\over 2} + {{2\,f}\over 3} & -{{21}\over 2} - {{2\,f}\over 9} &\
  -{{202}\over {243}} & {{1354}\over {81}} & -{{1192}\over {243}} &
{904 \over 81} \\ \mvs
0 & 0 & -{{5911}\over {486}} + {{71\,f}\over 9} & {{5983}\over {162}} +\
  {f\over 3} & -{{2384}\over {243}} - {{71\,f}\over 9} &
{1808 \over 81} - {f \over 3} \\ \mvs
0 & 0 & {{379}\over {18}} + {{56\,f}\over {243}} & -{{91}\over 6} +\
  {{808\,f}\over {81}} & -{{130}\over 9} - {{502\,f}\over {243}} &
-{14 \over 3} + {{646\,f} \over 81} \\ \mvs
0 & 0 & {{-61\,f}\over 9} & {{-11\,f}\over 3} & {{71}\over 3} + {{61\,f}\over\
  9} & -99 + {{11\,f} \over 3} \\ \mvs
0 & 0 & {{-682\,f}\over {243}} & {{106\,f}\over {81}} & -{{225}\over 2} +\
  {{1676\,f}\over {243}} & -{1343 \over 6} + {{1348\,f} \over 81}
\end{array}
\right)
\label{eq:gs1ndrN3Kpp}
\end{equation}
In \eqn{eq:gs0Kpp} and \eqn{eq:gs1ndrN3Kpp} $f$ denotes the number of
active quark flavours at a certain scale $\mu$. The corresponding
results for $\gss$ in the HV scheme can either be obtained by direct
calculation or by using the relation \eqn{gpgs}. They can be found in
\cite{burasetal:92a}, \cite{ciuchini:93} where also the $N$
dependence of $\gss$ is given.

\subsection{Quark Threshold Matching Matrix}
            \label{sec:HeffdF1:66:Mm}
As discussed in section~\ref{sec:basicform:wc:rgf} in general a 
matching matrix $\hM(m)$ has to be included in the RG evolution at NLO when
going from a $f$-flavour effective theory to a $(f-1)$-flavour
effective theory at quark threshold $\mu = m$ \cite{burasetal:92a},
\cite{burasetal:92d}.

For the $\dS$ decay $\Kpipi$ in pure QCD one has \cite{burasetal:92a}
\begin{equation}
\hM(m) = 1 + \frac{\as(m)}{4\pi} \; \vardrs^T \, .
\label{eq:MmKpp}
\end{equation}
At the quark thresholds $m=m_b$ and $m=m_c$ the matrix $\vardrs$ reads
\begin{equation}
\vardrs^T = -\, \frac{5}{9} P \, (0, 0, 0, 1, 0, 1)
\label{eq:drsmbcKpp}
\end{equation}
with
\begin{equation}
P^T = (0,0,-\frac{1}{3},1,-\frac{1}{3},1) \, .
\label{eq:PdrsKpp}
\end{equation}

\subsection{Numerical Results for the $\Kpipi$ Wilson Coefficients
            in Pure QCD}
            \label{sec:HeffdF1:66:numres}
\begin{table}[htb]
\caption[]{$\dS$ Wilson coefficients at $\mu=1\gev$ for $\mt=170\gev$.
$y_1 = y_2 \equiv 0$.
\label{tab:wc6smu1}}
\begin{center}
\begin{tabular}{|c|c|c|c||c|c|c||c|c|c|}
& \multicolumn{3}{c||}{$\Lms^{(4)}=215\mev$} &
  \multicolumn{3}{c||}{$\Lms^{(4)}=325\mev$} &
  \multicolumn{3}{c| }{$\Lms^{(4)}=435\mev$} \\
\hline
Scheme & LO & NDR & HV & LO & 
NDR & HV & LO & NDR & HV \\
\hline
$z_1$ & --0.602 & --0.407 & --0.491 & --0.743 & 
--0.506 & --0.636 & --0.901 & --0.622 & --0.836 \\
$z_2$ & 1.323 & 1.204 & 1.260 & 1.423 & 
1.270 & 1.362 & 1.541 & 1.352 & 1.515 \\
\hline
$z_3$ & 0.003 & 0.007 & 0.004 & 0.004 & 
0.013 & 0.007 & 0.006 & 0.022 & 0.015 \\
$z_4$ & --0.008 & --0.022 & --0.010 & --0.012 & 
--0.034 & --0.016 & --0.016 & --0.058 & --0.029 \\
$z_5$ & 0.003 & 0.006 & 0.003 & 0.004 & 
0.007 & 0.004 & 0.005 & 0.009 & 0.005 \\
$z_6$ & --0.009 & --0.021 & --0.009 & --0.013 & 
--0.034 & --0.014 & --0.018 & --0.058 & --0.025 \\
\hline
$y_3$ & 0.029 & 0.023 & 0.026 & 0.036 & 
0.031 & 0.036 & 0.045 & 0.040 & 0.048 \\
$y_4$ & --0.051 & --0.046 & --0.048 & --0.060 & 
--0.056 & --0.059 & --0.069 & --0.066 & --0.072 \\
$y_5$ & 0.012 & 0.004 & 0.013 & 0.013 & 
--0.001 & 0.016 & 0.014 & --0.013 & 0.020 \\
$y_6$ & --0.084 & --0.076 & --0.070 & --0.111 & 
--0.109 & --0.096 & --0.145 & --0.166 & --0.136 \\
\end{tabular}
\end{center}
\end{table}

\begin{table}[htb]
\caption[]{$\dS$ Wilson coefficients at $\mu=\mc=1.3\gev$ for
$\mt=170\gev$ and $f=3$ effective flavours.
$|z_3|,\ldots,|z_6|$ are numerically irrelevant relative to
$|z_{1,2}|$. $y_1 = y_2 \equiv 0$.
\label{tab:wc6smu13}}
\begin{center}
\begin{tabular}{|c|c|c|c||c|c|c||c|c|c|}
& \multicolumn{3}{c||}{$\Lms^{(4)}=215\mev$} &
  \multicolumn{3}{c||}{$\Lms^{(4)}=325\mev$} &
  \multicolumn{3}{c| }{$\Lms^{(4)}=435\mev$} \\
\hline
Scheme & LO & NDR & HV & LO & 
NDR & HV & LO & NDR & HV \\
\hline
$z_1$ & --0.518 & --0.344 & --0.411 & --0.621 & 
--0.412 & --0.504 & --0.727 & --0.487 & --0.614 \\
$z_2$ & 1.266 & 1.166 & 1.207 & 1.336 & 
1.208 & 1.269 & 1.411 & 1.258 & 1.346 \\
\hline
$y_3$ & 0.026 & 0.021 & 0.024 & 0.032 & 
0.027 & 0.031 & 0.039 & 0.035 & 0.040 \\
$y_4$ & --0.050 & --0.046 & --0.048 & --0.059 & 
--0.056 & --0.058 & --0.068 & --0.067 & --0.070 \\
$y_5$ & 0.013 & 0.007 & 0.013 & 0.015 & 
0.005 & 0.016 & 0.016 & 0.001 & 0.018 \\
$y_6$ & --0.075 & --0.067 & --0.062 & --0.095 & 
--0.088 & --0.079 & --0.118 & --0.116 & --0.102 \\
\end{tabular}
\end{center}
\end{table}

\begin{table}[htb]
\caption[]{$\dS$ Wilson coefficients at $\mu=2\gev$ for
$\mt=170\gev$. For $\mu > \mc$ the GIM mechanism gives $z_i \equiv 0$,
$i=3,\ldots,6$. $y_1 = y_2 \equiv 0$.
\label{tab:wc6smu2}}
\begin{center}
\begin{tabular}{|c|c|c|c||c|c|c||c|c|c|}
& \multicolumn{3}{c||}{$\Lms^{(4)}=215\mev$} &
  \multicolumn{3}{c||}{$\Lms^{(4)}=325\mev$} &
  \multicolumn{3}{c| }{$\Lms^{(4)}=435\mev$} \\
\hline
Scheme & LO & NDR & HV & LO & 
NDR & HV & LO & NDR & HV \\
\hline
$z_1$ & --0.411 & --0.266 & --0.318 & --0.477 & 
--0.309 & --0.374 & --0.541 & --0.350 & --0.430 \\
$z_2$ & 1.199 & 1.121 & 1.151 & 1.240 & 
1.145 & 1.185 & 1.282 & 1.170 & 1.220 \\
\hline
$y_3$ & 0.019 & 0.019 & 0.018 & 0.023 & 
0.023 & 0.022 & 0.027 & 0.027 & 0.026 \\
$y_4$ & --0.040 & --0.046 & --0.039 & --0.046 & 
--0.054 & --0.045 & --0.052 & --0.062 & --0.052 \\
$y_5$ & 0.011 & 0.010 & 0.011 & 0.012 & 
0.010 & 0.013 & 0.013 & 0.010 & 0.015 \\
$y_6$ & --0.055 & --0.057 & --0.047 & --0.067 & 
--0.070 & --0.056 & --0.078 & --0.085 & --0.067 \\
\end{tabular}
\end{center}
\end{table}

Tables \ref{tab:wc6smu1}--\ref{tab:wc6smu2} give the $\dS$ Wilson
coefficients for $Q_1,\ldots,Q_6$ in pure QCD.
\\
We observe a visible scheme dependence for all NLO Wilson coefficients.
Notably we find $|y_6|$ to be smaller in the HV than in the NDR scheme.
\\
In addition all coefficients, especially $z_1$ and $y_3,\ldots,y_6$,
show a strong dependence on $\Lms$.
\\
Next, at NLO the absolute values for $z_{1,2}$ and $y_i$ are suppressed
relative to their LO results, except for $y_5$ in HV and $y_{4,6}$ in
NDR for $\mu > \mc$. The latter behaviour is related to the effect of
the matching matrix $M(\mc)$ absent for $\mu > \mc$.
\\
For $y_3,\ldots,y_5$ there is no visible $\mt$ dependence in the range
$\mt = (170 \pm 15)\gev$. For $|y_6|$ there is a relative variation of
$\ord(\pm 1.5\%)$ for in/decreasing $\mt$.
\\
Finally, a comment on the Wilson coefficients in the HV Scheme as
presented here is appropriate. As we have mentioned in section
\ref{sec:HeffdF1:22:wcrg}, the two-loop anomalous dimensions of the
weak current in the HV scheme does not vanish. This peculiar feature of
the HV scheme is also felt in $\gss$.  The diagonal terms in $\gss$
aquire additional universal large $\ord(N^2)$ terms $(44/3)N^2$
which are absent in the NDR scheme. These artificial terms can be
removed by working with $\gss -2 \gamma^{(1)}_J$ instead of $\gss$.
This procedure, adopted in this review and in \cite{burasetal:92d},
corresponds effectively to a finite renormalization of operators which
changes the coefficient of $\as/4\pi$ in $C^{HV}_2(\mw)$ from $- 13/2$
to $- 7/6$. The Rome group \cite{ciuchini:93} has chosen not to make
this additional finite renormalization and consequently their
coefficients in the HV scheme differ from the HV coefficients presented
here by a universal factor. They can be found by using
\begin{equation}
C^{\rm HV}_{\rm Rome}(\mu) =
\left[ 1 -\frac{\as(\mu)}{4\pi} 4 C_F \right]
C^{\rm HV}(\mu)
\label{eq:Crome}
\end{equation}
Clearly this difference is compensated by the corresponding difference in the
hadronic matrix elements of the operators $Q_i$.  

\subsection{The $\dB$ Effective Hamiltonian in Pure QCD}
            \label{sec:HeffdF1:66:dB1}
An important application of the formalism developed in the previous
subsections is for the case of $B$-meson decays.  The LO calculation
can be found e.g.\ in \cite{ponce:81}, \cite{grinstein:89} where the
importance of NLO calculations has already been pointed out.  This
section can be viewed as the generalization of Grinstein's analysis
beyond the LO approximation. We will focus on the $\Delta B=1$, $\Delta
C = 0$ part of the effective hamiltonian which is of particular
interest for the study of CP violation in decays to CP self-conjugate
final states.  The part of the hamiltonian inducing $\Delta B=1$,
$\Delta C = \pm 1$ transitions involves no penguin operators and has
already been discussed in \ref{sec:HeffdF1:22}.

At tree-level the effective hamiltonian of interest here is simply given by
\begin{equation} 
\Heff(\dB) = \frac{G_F}{\sqrt{2}} \sum_{q=u,c} \sum_{q'=d,s}
              V^*_{qb} V^{}_{qq'}
             \left(\bar b q  \right)_{\rm V-A}
             \left(\bar q q' \right)_{\rm V-A} \, .
\label{eq:HeffdB1:tree}
\end{equation} 

The cases $q'=d$ and $q'=s$ can be treated separately and have the same
Wilson coefficients $C_i(\mu)$. Therefore we will restrict the
discussion to $q'=d$ in the following.

Using unitarity of the CKM matrix, $\xi_u + \xi_c + \xi_t = 0$ with
$\xi_i = V^*_{ib} V^{}_{id}$, and the fact that $Q^u_{1,2}$ and
$Q^c_{1,2}$ have the same initial conditions at
$\mu=\mw$ one obtains for the effective $\dB$ hamiltonian at scales
$\mu = \ord(\mb)$
\begin{eqnarray} 
\Heff(\dB) &=& \frac{G_F}{\sqrt{2}} \bigl\{
   \xi_c \, \left[ C_1(\mu) Q_1^c(\mu) + C_2(\mu) Q_2^c(\mu) \right] +
   \xi_u \, \left[ C_1(\mu) Q_1^u(\mu) + C_2(\mu) Q_2^u(\mu) \right] 
\nn \\
 & & - \xi_t \, \sum_{i=3}^{6} C_i(\mu) Q_i(\mu)
\bigr\} \, .
\label{eq:HeffdB1:66}
\end{eqnarray} 
Here
\begin{eqnarray}
Q^q_{1} & = & \left( \bar b_{i} q_{j}  \right)_{\rm V-A}
            \left( \bar q_{j}  d_{i} \right)_{\rm V-A}
\, , \nn \\
Q^q_{2} & = & \left( \bar b q \right)_{\rm V-A}
            \left( \bar q d \right)_{\rm V-A}
\, , \nn \\
Q_{3} & = & \left( \bar b d \right)_{\rm V-A}
   \sum_{q} \left( \bar q q \right)_{\rm V-A}
\, , \label{eq:dB1basis} \\
Q_{4} & = & \left( \bar b_{i} d_{j}  \right)_{\rm V-A}
   \sum_{q} \left( \bar q_{j}  q_{i} \right)_{\rm V-A}
\, , \nn \\
Q_{5} & = & \left( \bar b d \right)_{\rm V-A}
   \sum_{q} \left( \bar q q \right)_{\rm V+A}
\, , \nn \\
Q_{6} & = & \left( \bar b_{i} d_{j}  \right)_{\rm V-A}
   \sum_{q} \left( \bar q_{j}  q_{i} \right)_{\rm V+A}
\, , \nn
\end{eqnarray}
where the summation runs over $q=u,d,s,c,b$. 

The corresponding $\dB$ Wilson coefficients at scale $\mu = \ord(\mb)$
are simply given by a truncated version of eq.\ \eqn{eq:WCv}
\begin{equation}
\vC(\mb) = \hU_5(\mb,\mw) \, \vC(\mw) \, .
\label{eq:HeffdB1:66:wc}
\end{equation}
Here $\hU_5$ is the $6 \times 6$ RG evolution matrix of eq.\
\eqn{eq:UQCDKpp} for $f=5$ active flavours. The initial conditions
$\vC(\mw)$ are identical to those of \eqn{eq:CMw1QCD}--\eqn{eq:CMw6QCD}
for the $\dS$ case.

\subsection{Numerical Results for the $\dB$ Wilson Coefficients in Pure QCD}
            \label{sec:HeffdF1:66:dB1num}
\begin{table}[htb]
\caption[]{$\dB$ Wilson coefficients at $\mu=\overline{m}_{\rm b}(\mb)=
4.40\gev$ for $\mt=170\gev$.
\label{tab:wc6b}}
\begin{center}
\begin{tabular}{|c|c|c|c||c|c|c||c|c|c|}
& \multicolumn{3}{c||}{$\Lms^{(5)}=140\mev$} &
  \multicolumn{3}{c||}{$\Lms^{(5)}=225\mev$} &
  \multicolumn{3}{c| }{$\Lms^{(5)}=310\mev$} \\
\hline
Scheme & LO & NDR & HV & LO & 
NDR & HV & LO & NDR & HV \\
\hline
$C_1$ & --0.272 & --0.164 & --0.201 & --0.307 & 
--0.184 & --0.227 & --0.337 & --0.202 & --0.250 \\
$C_2$ & 1.120 & 1.068 & 1.087 & 1.139 & 
1.078 & 1.101 & 1.155 & 1.087 & 1.113 \\
\hline
$C_3$ & 0.012 & 0.012 & 0.011 & 0.013 & 
0.013 & 0.012 & 0.015 & 0.015 & 0.014 \\
$C_4$ & --0.026 & --0.031 & --0.026 & --0.030 & 
--0.035 & --0.029 & --0.032 & --0.038 & --0.032 \\
$C_5$ & 0.008 & 0.008 & 0.008 & 0.009 & 
0.009 & 0.009 & 0.009 & 0.009 & 0.010 \\
$C_6$ & --0.033 & --0.035 & --0.029 & --0.038 & 
--0.041 & --0.033 & --0.042 & --0.046 & --0.036 \\
\end{tabular}
\end{center}
\end{table}

Table \ref{tab:wc6b} lists the $\dB$ Wilson coefficients for
$Q_1^{u,c},Q_2^{u,c},Q_3,\ldots,Q_6$ in pure QCD.
\\
$C_1$, $C_4$ and $C_6$ show a $\ord(20\%)$ scheme dependence while
this dependence is much weaker for the rest of the coefficients.
\\
Similarly to the $\dS$ case the numerical values for $\dB$ Wilson
coefficients are sensitive to the value of $\Lms$ used to determine
$\as$ for the RG evolution. The sensitivity is however less pronounced
than in the $\dS$ case due to the higher value $\mu=\overline{m}_{\rm
b}(\mb)$ of the renormalization scale.
\\
Finally, one finds no visible $\mt$ dependence in the range $\mt = (170
\pm 15)\gev$.
