\section{The Decay $B \to X_{\lowercase{s}} \gamma$}
         \label{sec:Heff:Bsgamma}
\subsection{General Remarks}
         \label{sec:Heff:Bsgamma:rem}
The $B \to X_s \gamma$ decay is known to be extremely sensitive to the
structure of fundamental interactions at the electroweak scale. As any
FCNC process, it does not arise at the tree level in the Standard
Model. The one-loop W-exchange diagrams that generate this decay at the
lowest order in the Standard Model are small enough to be comparable to
possible nonstandard contributions (charged scalar exchanges, SUSY one
loop diagrams, $W_R$ exchanges in the L--R symmetric models, etc.).\\

The $B \to X_s \gamma$ decay is particularly interesting because its
rate is of order $G_F^2 \aem$, while most of the other FCNC processes
involving leptons or photons are of order $G_F^2 \aem^2$.  The
long-range strong interactions are expected to play a minor role in the
inclusive $\Bsg$ decay.  This is because the mass of the b-quark is
much larger than the QCD scale $\Lambda$. Moreover, the only relevant
intermediate hadronic states $\psi X_s$ are expected to give very small
contributions, as long as we assume no interference between short- and
long-distance terms in the inclusive rate.  Therefore, it has become
quite common to use the following approximate equality to estimate the
$\Bsg$ rate:

\begin{equation}\label{ratios}
\frac{\Gamma(B \to X_s \gamma)}
     {\Gamma(B \to X_c e \bar{\nu}_e)}
 \simeq                                                     
\frac{\Gamma(b \to s \gamma)}
     {\Gamma(b \to c e \bar{\nu}_e)} \equiv R(\mt,\as,\xi)
\end{equation}
where the quantities on the r.h.s are calculated in the spectator model
corrected for short-distance QCD effects. The normalization to the
semileptonic rate is usually introduced in order to cancel the
uncertainties due to the Cabibbo-Kobayashi-Maskawa (CKM) matrix
elements and factors of $\mb^5$ in the r.h.s. of eq. (\ref{ratios}).
Additional support for the approximation given above comes from the
heavy quark expansions.  Indeed the spectator model has been shown to
correspond to the leading order approximation of an expansion in
$1/\mb$.  The first corrections appear at the ${\cal O}(1/\mb^2)$
level. The latter terms have been studied by several authors
\cite{Chay}, \cite{Bj}, \cite{bigietal:92}, \cite{bigietal:93},
\cite{manoharwise:94}, \cite{bloketal:94}, \cite{falketal:94},
\cite{mannel:94}, \cite{Bigi} with the result that they affect
$B(\Bsg)$ and $B(B \to X_c e \bar{\nu}_e)$ by only a few percent.

As indicated above, the ratio $R$ depends only on $\mt$ and $\as$ in
the Standard Model. In extensions of the Standard Model, additional
parameters are present. They have been commonly denoted by $\xi$. The
main point to be stressed here is that $R$ is a calculable function of
its parameters in the framework of a renormalization group improved
perturbation theory. Consequently, the decay in question is
particularly suited for tests of the Standard Model and its extensions.

One of the main difficulties in analyzing the inclusive $\Bsg$ decay is
calculating the short-distance QCD effects due to hard gluon exchanges
between the quark lines of the leading one-loop electroweak diagrams.
These effects are known \cite{Bert}, \cite{Desh}, \cite{Grin},
\cite{grigjanis:88}, \cite{grigjanis:92}, \cite{misiak:91} to enhance
the $B \to X_s \gamma$ rate in the Standard Model by a factor of 2--3,
depending on the top quark mass. So the $B \to X_s \gamma$ decay
appears to be the only known short distance process in the Standard
Model that is dominated by two-loop contributions.

The $B \to X_s \gamma$ decay has already been measured.  In 1993
CLEO reported \cite{CLEO:93} the following branching ratio for the
exclusive $B \to K^* \gamma$ decay
\begin{equation}
B(B \to K^* \gamma) = (4.5 \pm 1.5 \pm 0.9) \times 10^{-5}.
\label{excl}
\end{equation}

In 1994 a first measurement of the inclusive rate has been
presented \cite{CLEO:94}
\begin{equation}
B(B \to X_s\gamma) = (2.32 \pm 0.57 \pm 0.35) \times 10^{-4}
\label{incl}
\end{equation}
where the first error is statistical and the second is systematic.

As we will see below
these experimental findings are in the ball park of the Standard Model
expectations based on the leading logarithmic approximation. 

In fact a complete leading order analysis of $B(B \to
X_s\gamma)$ in the Standard Model has been presented almost a year
before the CLEO result giving \cite{BMMP:94}
\begin{equation}
B(B \to X_s\gamma)_{TH} = (2.8 \pm 0.8) \times 10^{-4}.
\label{theo}
\end{equation}
where the error is dominated by the uncertainty in the choice of the
renormalization scale $\mb/2<\mu<2 \mb$ as first stressed by Ali and
Greub \cite{aligreub:93} and confirmed in \cite{BMMP:94}. Since $\Bsg$ is
dominated by QCD effects, it is not surprising that this
scale-uncertainty in the leading order is particularly large.  Such an
uncertainty, inherent in any finite order of perturbation theory can be
reduced by including next-to-leading order corrections.  Unfortunately,
it will take some time before the $\mu$-dependences present in $\Bsg$
can be reduced in the same manner as it was done for the other decays
\cite{burasjaminweisz:90}, \cite{buchallaburas:93b},
\cite{buchallaburas:94}, \cite{herrlichnierste:93}.  As we already
stated in section \ref{sec:Heff:BXsgamma:wc}, a full next-to-leading
order computation of $\Bsg$ would require calculation of three-loop
mixings between the operators $Q_1,\ldots,Q_6$ and the magnetic penguin
operators $Q_{7\gamma},Q_{8G}$.
Moreover, certain two-loop matrix elements of the relevant operators
should be calculated in the spectator model.  A formal analysis at the
next-to-leading level \cite{BMMP:94} is however very encouraging and
shows that the $\mu$-dependence can be considerably reduced once all
the necessary calculations have been performed.  We will return to this
issue below.

\subsection{The Decay $B\to X_s\gamma$ in the Leading Log Approximation}
         \label{sec:Heff:Bsgamma:lo}
The leading logarithmic calculations \cite{Grin}, \cite{misiak:93},
\cite{aligreub:93}, \cite{CFRS:94}, \cite{CCRV:94a}, \cite{misiak:94},
\cite{BMMP:94} can be summarized in a compact form, as follows:
\begin{equation}\label{main}
R = 
\frac{\Gamma(b \to s \gamma)}{\Gamma(b \to c e
\bar{\nu}_e)}
 =  \frac{|V_{ts}^* V_{tb}^{}|^2}{|V_{cb}|^2} 
\frac{6 \alpha}{\pi f(z)} |C^{(0)eff}_{7\gamma}(\mu)|^2
\end{equation}
where $C^{(0)eff}_{7\gamma}(\mu)$ is the effective coefficient 
given in (\ref{C7eff}) and table \ref{tab:c78effnum}, $z =
\frac{\mc}{\mb}$, and
\begin{equation}\label{g}
f(z) = 1 - 8z^2 + 8z^6 - z^8 - 24z^4 ln \; z           
\end{equation}
is the phase space factor in the semileptonic b-decay. Note, that at
this stage one should  not include the ${\cal O}(\as)$ corrections to
$\Gamma(b \to c e \bar{\nu})$ since they are part of the
next-to-leading effects. For the same reason we do not include the
${\cal O}(\as)$ QCD corrections to the matrix element of the operator
$Q_{7\gamma}$ (the QCD bremsstrahlung $b\to s\gamma+g$ and the virtual
corrections to $b\to s\gamma$) which are known \cite{aligreub:91a},
\cite{aligreub:91b}, \cite{pott:95} and will be a part of a future NLO
analysis.

Formula (\ref{main}) and the expression (\ref{C7eff}) for
$C^{(0)eff}_{7\gamma}(\mu)$ summarize the complete leading logarithmic
(LO) approximation for the $\Bsg$ rate in the Standard Model.  Their
important property is that they are exactly the same in many
interesting extensions of the Standard Model, such as the
Two-Higgs-Doublet Model (2HDM) \cite{Grin}, \cite{Hewett},
\cite{Barger}, \cite{Hayashi}, \cite{BMMP:94} or the Minimal
Supersymmetric Standard Model (MSSM) \cite{Borzum}, \cite{Barbieri},
\cite{Borzum2}.  The only quantities that change are the coefficients
$C^{(0)}_2(\mw)$, $C^{(0)}_{7\gamma}(\mw)$ and $C^{(0)}_{8G}(\mw)$ .
On the other hand in a general $SU(2)_L \times SU(2)_R \times U(1)$
model additional modifications are necessary, because new operators
enter \cite{LR}.

A critical analysis of theoretical and experimental
uncertainties present in the prediction for $B(\Bsg)$ based on the
above formulae has been made \cite{BMMP:94}. Here we just briefly 
list the main findings:

\begin{itemize}
\item
First of all, eq.\ \eqn{main} is based on the spectator model. As we
have mentioned above the heavy quark expansion gives a strong support
for this model in inclusive B-decays.  On a conservative side one can
assume the error due to the use of the spectator model in $\Bsg$ to
amount to at most $\pm 10\%$.
\item
The uncertainty coming from the ratio $z = \frac{\mc}{\mb}$ in the
phase-space factor $f(z)$ for the semileptonic decay is estimated to be
around 6\%.
\item
The error due to the ratio of the CKM parameters in
eq. (\ref{main}) is small. Assuming unitarity of the $3 \times 3$ CKM
matrix and imposing the constraints from the CP-violating parameter
$\epsilon_{\rm K}$ and $B^0-\bar B^0$ mixing one finds
\begin{equation}\label{KM}
\frac{|V_{ts}^* V_{tb}^{}|^2}{|V_{cb}|^2} = 0.95 \pm 0.03     
\end{equation}
\item
There exists an uncertainty due to the determination of $\as$. This
uncertainty is not small because of the importance of QCD corrections
in the considered decay.  For instance the difference between the
ratios $R$ of eq.  (\ref{main}) obtained with help of
$\alpha_{\overline{MS}}(\mz)=0.11$ and $0.13$, respectively, is roughly
20\%.
\item
The dominant uncertainty in eq. (\ref{main}) comes from the
unknown next-to-leading order contributions. This uncertainty is best
signaled by the strong $\mu$-dependence of the leading order
expression (\ref{main}), which is shown by the solid line in
fig.\ \ref{fig:bsg:rmu}, for the case $\mt=170\gev$. 

\begin{figure}[hbt]
\vspace{0.10in}
\centerline{
\epsfysize=5in
\rotate[r]{
\epsffile{ps/bsgrmu.ps}
}}
\vspace{0.08in}
\caption[]{
$\mu$-dependence of the theoretical prediction for the ratio $R$ for
$\mt=170\gev$ and $\Lms^{(5)}=225\mev$. The solid line corresponds to the
leading order prediction. The dashed lines describe possible
next-to-leading results.
\label{fig:bsg:rmu}}
\end{figure}

One can see that when $\mu$ is varied by a factor of 2 in both
directions around $\mb \simeq 5\gev$, the ratio (\ref{main}) changes
by around $\pm 25\%$, i.e. the ratios $R$ obtained for $\mu=2.5\gev$
and $\mu=10\gev$ differ by a factor of 1.6 \cite{aligreub:93}. 

The dashed lines in fig.\ \ref{fig:bsg:rmu} show the expected
$\mu$-dependence of the ratio (\ref{main}) once a complete
next-to-leading calculation is performed. The $\mu$-dependence is then
much weaker, but until one performs the calculation explicitly one
cannot say which of the dashed curves is the proper one. The way the
dashed lines are obtained is described in \cite{BMMP:94}.
\item	
Finally, there exists a $\pm 2.4\%$ error in determining $B(B \to
X_s \gamma)$ from eq. (\ref{ratios}), which is due to the error in the
experimental measurement of $B(B \to X_c e \bar{\nu}_e) = (10.43 \pm
0.24)\%$ \cite{particledata:94}.
\item
The uncertainty due to the value of $\mt$ is small as is shown
explicitly below.
\end{itemize} 


Fig.\ \ref{fig:bsg:brmt} based on \cite{BMMP:94} presents the Standard
Model prediction for the inclusive $\Bsg$ branching ratio including the
errors listed above as a function of $\mt$ together with the CLEO
result.

\begin{figure}[hbt]
\vspace{0.10in}
\centerline{
\epsfysize=5in
\rotate[r]{
\epsffile{ps/bsgbrmt.ps}
}}
\vspace{0.08in}
\caption[]{
Predictions for $B \to X_s \gamma$ in the SM as a function of the top
quark mass with the theoretical uncertainties taken into account.
\label{fig:bsg:brmt}}
\end{figure}

We stress that the theoretical curves have been obtained prior to the
experimental result. Since the theoretical error is dominated by scale
ambiguities a complete NLO analysis is very desirable.

\subsection { Looking at $\Bsg$ Beyond Leading Logarithms}
         \label{sec:Heff:Bsgamma:nlo}
In this section we describe briefly a complete next-to-leading
calculation of $\Bsg$ in general terms. 
This section collects the most important findings of section 4 of
\cite{BMMP:94}.

Let us first enumerate what has been already calculated in the
literature and which calculations are still required in order to
complete the next-to-leading calculation of $B(B \to X_s \gamma)$.

\noindent
The present status is as follows:
\begin{itemize}
\item
The $6 \times 6$ submatrix of
$\gamma^{(1)}$ describing the two-loop mixing of $(Q_1,\ldots,
Q_6)$ and the corresponding ${\cal{O}}(\as)$ corrections in
$\vec{C}(\mw)$ have been already calculated. They are given in section
\ref{sec:HeffdF1:66}.
\item
The two-loop mixing in the $(Q_{7\gamma},Q_{8G})$ sector of
$\gamma^{(1)}$ is known \cite{misiakmuenz:95} and given in
section \ref{sec:Heff:BXsgamma:RGE}.
\item
The ${\cal{O}}(\as)$ corrections to the matrix element
of the operators $Q_{7\gamma}$ and $Q_{8G}$ have 
been calculated \cite{aligreub:91a}, \cite{aligreub:91b}.
They have been recently confirmed by \cite{pott:95} who also presents
the results for the matrix elements of the remaining operators.
\end{itemize}
The remaining ingredients of a next-to-leading analysis of
$B(B \to X_s \gamma)$ are:
\begin{itemize}
\item
The three-loop mixing between the sectors $(Q_1,\ldots,Q_6)$ and
$(Q_{7\gamma},Q_{8G})$ which, with our normalizations, contributes to
$\gamma^{(1)}$.\\
\item
The ${\cal O}(\as)$ corrections to $C_{7\gamma}(\mw)$ and $C_{8G}(\mw)$
in (\ref{c7}) and (\ref{c8}). This requires evaluation of two-loop
penguin diagrams with internal W and top quark masses and a proper
matching with the effective five-quark theory. An attempt to calculate
the necessary two-loop Standard Model diagrams has been made in \cite{Yao2}.
\item
The finite parts of the effective theory two-loop diagrams with the
insertions of the four-quark operators .
\end{itemize}
All these calculations are very involved, and the necessary three-loop
calculation is a truly formidable task! Yet, as stressed in
\cite{BMMP:94} all these calculations have to be done if we want to
reduce the theoretical uncertainties in $\bsg$ to around 10\%.

As demonstrated formally in \cite{BMMP:94} the cancellation of the
dominant $\mu$-dependence in the leading order can be achieved by
calculating the relevant two-loop matrix element of the dominant
four-quark operator $Q_2$.  This matrix element is however
renormalization-scheme dependent and moreover mixing with other
operators takes place.  This scheme dependence can only be canceled by
calculating $\gamma^{(1)}$ in the same renormalization scheme.
This point has been extensively discussed in this review
and  we will not repeat this discussion here. However,
it is clear from these remarks, that in order to address the
$\mu$-dependence and the renormalization-scheme dependence as well as
their cancellations, it is necessary to perform a complete
next-to-leading order analysis of $\vec{C}(\mu)$ and of the corresponding
matrix elements.

In this context we would like to comment on an analysis of
\cite{Ciu:94} in which the known two-loop mixing of $Q_1,\ldots,Q_6$
has been added to the leading order analysis of $\Bsg$.  Strong
renormalization scheme dependence of the resulting branching ratio has
been found, giving the branching ratio $(1.7 \pm 0.2)\times 10^{-4}$
and $(2.3 \pm 0.2)\times 10^{-4}$ at $\mu=5\gev$ for HV and NDR
schemes, respectively. It has also been observed that whereas in the HV
scheme the $\mu$ dependence has been weakened, it remained still strong
in the NDR scheme. In our opinion this partial cancellation of the
$\mu$-dependence in the HV scheme is rather accidental and has nothing
to do with the cancellation of the $\mu$-dependence discussed above. The
latter requires the evaluation of finite parts in two-loop matrix
elements of the four-quark operators $Q_1,\ldots,Q_6$.  On the other
hand the strong scheme dependence in the partial NLO analysis presented
in \cite{Ciu:94} demonstrates very clearly the need for a full
analysis.  In view of this discussion we think that the decrease of the
branching ratio for $\Bsg$ relative to the LO prediction, found in
\cite{Ciu:94}, and given by $B(B \to X_s \gamma) = (1.9 \pm 0.2 \pm
0.5) \cdot 10^{-4}$ is still premature and one should wait until the
full NLO analysis has been done.

\section{The Decay $B\to X_{\lowercase{s}} \lowercase{e}^+\lowercase{e}^-$}
         \label{sec:Heff:BXsee:nlo}
\subsection{General Remarks} 
         \label{sec:Heff:BXsee:nlo:rem}
The rare decay $\Bsee$ has been the subject of many theoretical studies
in the framework of the Standard Model and its extensions such as the
two Higgs doublet models and models involving supersymmetry
\cite{HWS:87}, \cite{grinstein:89a}, \cite{JW:90}, \cite{BBMR:91},
\cite{AMM:91}, \cite{DPT:93}, \cite{AGM:94}, \cite{GIW:94}.  In
particular the strong dependence of $\Bsee$ on $\mt$ has been stressed
in \cite{HWS:87}. It is clear that once $\Bsee$ has been observed, it
will offer a useful test of the Standard Model and its extensions.
To this end the relevant branching ratio, the dilepton invariant mass
distribution and other distributions of interest should be calculated
with sufficient precision. In particular the QCD effects should be
properly taken into account.

The central element in any analysis of $\Bsee$ is the effective
hamiltonian for this decay given in section \ref{sec:Heff:BXsee} where a
detailed analysis of the Wilson coefficients has been presented.
However, the actual calculation of $\Bsee$ involves not only the
evaluation of Wilson coefficients of the relevant local operators but
also the calculation of the corresponding matrix elements of these
operators relevant for $\Bsee$. The latter part of the analysis can be
done in the spectator model, which, as indicated by the heavy quark
expansion should offer a good approximation to QCD for B-decays. One can
also include the non-perturbative ${\cal O}(1/\mb^2)$ corrections to
the spectator model which enhance the rate for $\Bsee$ by roughly 10\%
\cite{falketal:94}. A realistic phenomenological analysis should also
include the long-distance contributions which are mainly due to the
$J/\psi$ and $\psi'$ resonances \cite{LMS:89}, \cite{DTP:89},
\cite{DT:91}. Since in this review we are mainly interested in the
next-to-leading short-distance QCD effects we will not include these
complications in what follows. This section closely follows
\cite{burasmuenz:95} execpt that the numerical results in figs.\
\ref{fig:bsee:rs}--\ref{fig:bsee:rs2} have been slightly changed in
accordance with the input parameters of appendix \ref{app:numinput}.

We stress again that in a consistent NLO analysis of the decay $\Bsee$,
one should on one hand calculate the Wilson coefficient of the operator
$Q_{9V} = (\bar s b)_{V-A} (\bar e e)_V$ including leading and
next-to-leading logarithms, but on the other hand only leading
logarithms should be kept in the remaining Wilson coefficients. Only
then a scheme independent amplitude can be obtained. As already
discussed in section \ref{sec:Heff:BXsee}, this special treatment of
$Q_9$ is related to the fact that strictly speaking in the leading
logarithmic approximation only this operator contributes to $\Bsee$.
The contributions of the usual current-current operators, QCD penguin
operators, magnetic penguin operators and of $Q_{10A} = (\bar s
b)_{V-A}(\bar e e)_A$ enter only at the NLO level and to be consistent
only the leading contributions to the corresponding Wilson coefficients
should be included.

\subsection{The Differential Decay Rate}
         \label{sec:Heff:BXsee:nlo:rate}
Introducing
\begin{equation} \label{invleptmass}
\hat s = \frac{(p_{e^+} + p_{e^-})^2}{\mb^2}, \qquad z =
\frac{\mc}{\mb}
\end{equation}
and calculating the one-loop matrix elements of $Q_i$ using the
spectator model in the NDR scheme one finds \cite{misiak:94},
\cite{burasmuenz:95}
\begin{eqnarray} \label{rate}
& &
R(\hat s) \equiv \frac{{d}/{d\hat s} \, \Gamma (\bsee)}{\Gamma
(\bcenu)} = \frac{\aem^2}{4\pi^2}
\left|\frac{V_{ts}}{V_{cb}}\right|^2 \frac{(1-\hat s)^2}{f(z)\kappa(z)}
\times \\ 
& &
\biggl[(1+2\hat s)\left(|\Ctilde_9^{eff}|^2 + |\Ctilde_{10}|^2\right) + 
4 \left( 1 + \frac{2}{\hat s}\right) |C_{7\gamma}^{(0)eff}|^2 + 12
C_{7\gamma}^{(0)eff} \ \RE\,\Ctilde_9^{eff}  \biggr]
\nn
\end{eqnarray}
where
\begin{eqnarray} \label{C9eff}
\Ctilde_9^{eff} & = & \Ctilde_9^{NDR} \tilde\eta(\hat s) + h(z, \hat
s)\left( 3 C_1^{(0)} + C_2^{(0)} + 3 C_3^{(0)} + C_4^{(0)} + 3
C_5^{(0)} + C_6^{(0)} \right) - \nn \\
& & \frac{1}{2} h(1, \hat s) \left( 4 C_3^{(0)} + 4 C_4^{(0)} + 3
C_5^{(0)} + C_6^{(0)} \right) - \\
& & \frac{1}{2} h(0, \hat s) \left( C_3^{(0)} + 3 C_4^{(0)} \right) +
\frac{2}{9} \left( 3 C_3^{(0)} + C_4^{(0)} + 3 C_5^{(0)} + C_6^{(0)}
\right) \, .
\nn
\end{eqnarray}
The general expression (\ref{rate}) with $\kappa(z)=1$ has been first
presented by \cite{grinstein:89a} who in their approximate leading
order renormalization group analysis kept only the operators $Q_1, Q_2,
Q_{7\gamma},Q_{9V}, Q_{10A}$.

The various entries in (\ref{rate}) are given as follows
\begin{eqnarray} \label{phasespace}
h(z, \hat s) & = & -\frac{8}{9}\ln\frac{\mb}{\mu} - \frac{8}{9}\ln z +
\frac{8}{27} + \frac{4}{9} x - \\
& & \frac{2}{9} (2+x) |1-x|^{1/2} \left\{
\begin{array}{ll}
\left( \ln\left| \frac{\sqrt{1-x} + 1}{\sqrt{1-x} - 1}\right| - i\pi \right),
 & \mbox{for } x \equiv 4 z^2/\hat s < 1 \nn \\
2 \arctan \frac{1}{\sqrt{x-1}}, & \mbox{for } x \equiv 4 z^2/\hat s
 > 1,
\end{array}
\right. \\
h(0, \hat s) & = &
\frac{8}{27} -\frac{8}{9} \ln\frac{\mb}{\mu} - \frac{4}{9} \ln
\hat s + \frac{4}{9} i\pi. \\ 
f(z) & = & 1 - 8 z^2 + 8 z^6 - z^8 - 24 z^4 \ln z, \\
\kappa(z)  & = & 1 - \frac{2 \as(\mu)}{3\pi}\left[(\pi^2 -
\frac{31}{4})(1-z)^2 + \frac{3}{2} \right]  \label{eq:kappaz} \\
\tilde\eta(\hat s) & = & 1 + \frac{\as(\mu)}{\pi}\, \omega(\hat s)
\end{eqnarray}
with
\begin{eqnarray} \label{omega}
\omega(\hat s) & = & - \frac{2}{9} \pi^2 - \frac{4}{3}\mbox{Li}_2(s) -
\frac{2}{3}
\ln s \ln(1-s) - \frac{5+4s}{3(1+2s)}\ln(1-s) - \nn \\
& &  \frac{2 s (1+s) (1-2s)}{3(1-s)^2 (1+2s)} \ln s + \frac{5+9s-6s^2}{6
(1-s) (1+2s)}.
\end{eqnarray}
Here $f(z)$ is the phase-space factor for $b \to c e \bar\nu$.
$\kappa(z)$ is the corresponding single
gluon QCD correction \cite{CM:78} in the approximation of \cite{KM:89}.
$\tilde\eta$ on the other hand represents single gluon
corrections to the matrix element of $Q_9$ with $\ms = 0$ \cite{JK:89},
\cite{misiak:94}. For consistency reasons this correction should only
multiply the leading logarithmic term in $\tilde{C}_9^{\rm NDR}$.

In the HV scheme the one-loop matrix elements are different and one
finds an additional explicit contribution to (\ref{C9eff}) given by
\cite{burasmuenz:95}
\begin{equation} \label{MEHV}
- \xi^{HV} \frac{4}{9} \left( 3 C_1^{(0)} + C_2^{(0)} - C_3^{(0)} - 3
C_4^{(0)} \right).
\end{equation}
However $\Ctilde_9^{NDR}$ has to be replaced by $\Ctilde_9^{HV}$ given in
(\ref{C9tilde}) and (\ref{P0HV}) and consequently $\Ctilde_9^{eff}$ is the
same in both schemes.

The first term in the function $h(z, \hat s)$ in (\ref{phasespace})
represents the leading $\mu$-dependence in the matrix elements. It is
canceled by the $\mu$-dependence present in the leading logarithm in
$\tilde C_{9}$. This is precisely the type of cancellation of the
$\mu$-dependence which one would like to achieve in the case of $B \to
X_s \gamma$. The $\mu$-dependence present in the coefficients of
the other operators can only be canceled by going to still higher order
in the renormalization group improved perturbation theory. To this end
the matrix elements of four-quark operators should be evaluated at
two-loop level. Also certain unknown three-loop anomalous dimensions
should be included in the evaluation of $C_{7\gamma}^{eff}$ and
$C_{9V}$. Certainly this is beyond the scope of this review and we will
only investigate the left-over $\mu$-dependence below.

\subsection{Numerical Analysis}
         \label{sec:Heff:BXsee:nlo:num}
A detailed numerical analysis of the formulae above has been presented
in \cite{burasmuenz:95}. We give here a brief account of this work.  We
set first $|V_{ts}/V_{cb}|  = 1$ which in view of (\ref{KM}) is a good
approximation.  We keep in mind that for $\hat s \approx m_\psi^2 /
\mb^2$, $\hat s \approx m_{\psi'}^2 / \mb^2$ etc.~the spectator model
cannot be the full story and additional long-distance contributions
discussed in \cite{LMS:89}, \cite{DTP:89}, \cite{DT:91} have to be
taken into account in a phenomenological analysis. Similarly we do not
include $1/\mb^2$ corrections calculated in \cite{falketal:94} which
typically enhance the differential rate by about 10\%.

\begin{figure}[hbt]
\vspace{0.10in}
\centerline{
\epsfysize=7in
\rotate[r]{
\epsffile{ps/bseers.ps}
}}
\vspace{0.08in}
\caption[]{
(a) $R(\hat{s})$ for $\mt=170\gev$, $\Lms^{(5)}=225\mev$ and differents
values of $\mu$. 
\\
\phantom{xxxxxxxxxx}
(b) $R(\hat{s})$ for $\mu=5\gev$, $\Lms^{(5)}=225\mev$ and various
values of $\mt$. 
\label{fig:bsee:rs}}
\end{figure}

In fig.\ \ref{fig:bsee:rs}\,(a) we show $R(\hat s)$ for $\mt = 170 \gev$,
$\Lms = 225 \mev$ and different values of $\mu$. In fig.\
\ref{fig:bsee:rs}\,(b) we set $\mu = 5 \gev$ and vary $\mt$ from $150 \gev$
to $190 \gev$. The remaining $\mu$ dependence is rather weak and
amounts to at most $\pm 8\%$ in the full range of parameters
considered. The $\mt$ dependence of $R(\hat s)$ is sizeable. Varying
$\mt$ between $150\gev$ and $190\gev$ changes $R(\hat s)$ by typically
60--65\% which in this range of $\mt$ corresponds to $R(\hat s) \sim
\mt^2$. It is easy to verify that this strong $\mt$ dependence
originates in the coefficient $\Ctilde_{10}$ given in (\ref{C10}) as
already stressed by several authors in the past \cite{HWS:87},
\cite{grinstein:89a}, \cite{BBMR:91}, \cite{DPT:93}, \cite{GIW:94},
\cite{AGM:94}, \cite{AMM:91}, \cite{JW:90}.

We do not show the $\Lms$ dependence as it is very weak. Typically,
changing $\Lms$ from $140\mev$ to $310\mev$ decreases $R(\hat s)$ by
about 5\%.

\begin{figure}[hbt]
\vspace{0.10in}
\centerline{
\epsfysize=4in
\rotate[r]{
\epsffile{ps/bseerscmp.ps}
}}
\vspace{0.08in}
\caption[]{
Comparison of the four different contributions to $R(\hat{s})$ according
to eq.\ \eqn{rate}.
\label{fig:bsee:rscmp}}
\end{figure}

$R(\hat s)$ is governed by three coefficients, $\Ctilde_9^{eff}$,
$\Ctilde_{10}$ and $C_{7\gamma}^{(0)eff}$.  The importance of various
contributions has been investigated in \cite{burasmuenz:95}. To this
end one sets $\Lms^{(5)}=225\gev$, $\mt=170\gev$ and $\mu = 5 \gev$. In
fig.\ \ref{fig:bsee:rscmp} we show $R(\hat s)$ keeping only
$\Ctilde_9^{eff}$, $\Ctilde_{10}$, $C_{7\gamma}^{(0)eff}$ and the
$C_{7\gamma}^{(0)eff}$--$\Ctilde_9^{eff}$ interference term,
respectively.  Denoting these contributions by $R_9$, $R_{10}$, $R_7$
and $R_{7/9}$ we observe that the term $R_7$ plays only a minor role in
$R(\hat s)$. On the other hand the presence of $C_{7\gamma}^{(0)eff}$
cannot be ignored because the interference term $R_{7/9}$ is
significant. In fact the presence of this large interference term could
be used to measure experimentally the relative sign of
$C_{7\gamma}^{(0)eff}$ and $\mbox{Re}\,\Ctilde_9^{eff}$
\cite{grinstein:89a}, \cite{JW:90}, \cite{AMM:91}, \cite{GIW:94},
\cite{AGM:94} which as seen in fig.\ \ref{fig:bsee:rscmp} is negative
in the Standard Model. However, the most important contributions are
$R_9$ and $R_{10}$ in the full range of $\hat s$ considered. For $\mt
\approx 170 \gev$ these two contributions are roughly of the same size.
Due to a strong $\mt$ dependence of $R_{10}$, this contribution
dominates for higher values of $\mt$ and is less important than $R_9$
for $\mt < 170 \gev$.

Next, in fig.\ \ref{fig:bsee:rs2} we show $R(\hat s)$ for $\mu = 5
\gev$, $\mt = 170 \gev$ and $\Lms = 225 \mev$ compared to the case of
no QCD corrections and to the results \cite{grinstein:89a} would obtain
for our set of parameters using their approximate leading order
formulae.

\begin{figure}[hbt]
\vspace{0.10in}
\centerline{
\epsfysize=4in
\rotate[r]{
\epsffile{ps/bseers2.ps}
}}
\vspace{0.08in}
\caption[]{
$R(\hat{s})$ for $\mt=170\gev$, $\Lms^{(5)}=225\mev$ and $\mu=5\gev$. 
\label{fig:bsee:rs2}}
\end{figure}

The discussion of the definition of $\mt$ used here is identical to the
one in the case of $K_L \to \pi^0 e^+ e^-$ and will not be repeated here.
On the basis of the arguments given there we believe that if $\mt =
\overline{m}_{\rm t}(\mt)$ is chosen, the additional short-distance QCD
corrections to $B(\Bsee)$ should be small.

Our discussion has been restricted to $B(B \to X_s \gamma)$. Also the
photon spectrum has been the subject of several papers. We just refer
to the most recent articles \cite{neubert:94c}, \cite{shifmanetal:94},
\cite{dikemanetal:94}, \cite{kapustinligeti:95}, \cite{kapustinetal:95},
\cite{aligreub:95}, \cite{pott:95} where further references can be found.
