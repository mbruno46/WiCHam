\skipevenpage

{\Huge\bf
\noindent
Part Two --

\bigskip
\bigskip
\bigskip

\noindent
The Effective Hamiltonians
}

% \vfil
\bigskip
\bigskip
\bigskip

\noindent
The second part constitutes a compendium of effective hamiltonians for
weak decays.  We will deal with all decays for which NLO corrections
have been calculated in the literature and whose list is given in table
\ref{tab:processes}. This includes a listing of the initial conditions
$C_i (\mw)$, a listing of all one-loop and two-loop anomalous dimension
matrices and finally tables of numerical values of the relevant Wilson
coefficients as functions of $\Lms$, $\mt$ and the renormalization
schemes considered.  In certain cases we are able to give analytic
formulae for $C_i$.

We will discuss all effective hamiltonians one by one. With the help of the
master formulae and the procedure of section \ref{sec:basicform} it is
easy to see similarities and differences between various cases.  Our
compendium includes also the $b \to s\gamma$ and $b\to s g$ transitions
which although known only in the leading logarithmic approximation
deserve special attention.

Finally, as a preparation for the third part we give a brief
description of the ``Penguin-Box Expansion'' (PBE), which can be
regarded as a version of OPE particularly suited for the study of the
$m_t$ dependence in weak decays.

In addition we have also included a section on NLO QCD calculations in
the context of HQET. This chapter lies somewhat outside our main line of
presentation. Also a comprehensive discussion of HQET is clearly beyond
the scope of the present paper. However, we would like to illustrate the
application of the general formalism for short distance QCD corrections
within this framework and summarize a few important NLO results that
have been obtained in HQET.
