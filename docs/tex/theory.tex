
By integrating out the weak bosons from the standard model we obtain an 
effective theory, called weak effective hamiltonian. 
As an effective theory it loses the property of being renormalizable,
meaning that at each order in perturbation theory a new (finite) set of couplings and operators
is needed to properly cancel the remaining divergences. Nevertheless, if we work at a fixed 
order the theory is finite and in the following we will concentrate only 
on the leading order effective theory.

Composite operators in the effective theory are expected to mix under renormalization, unless
there is a symmetry protecting them. This mixing can be generically expressed through a 
$Z$-factor matrix, which can be perturbatively expanded in terms of the so-called 
anomalous dimension matrices. Once they are known, the operators or alternatively the Wilson Coefficients
can be run to low energies and the implementation provided in this library includes effects
up to $O(\as)$ and $O(\aem)$.

The Wilson Coefficients are obtained by comparing 
(order by order in perturbation theory) amplitudes computed in the standard model
against amplitudes computed in the effective theory. Obviously amplitudes are scheme-independent
but the two parts of the calculation, the Wilson Coefficients and the low-energy matrix elements, 
individually depend on the scheme. 

\subsection{The $\Delta F=1$ Hamiltonian}

We are interested here in weak decays where either the strange or the bottom quantum numbers 
are violated by one unit. In general these processes can be described by a single effective
hamiltonian

\begin{equation}
H_W = \frac{G_F}{\sqrt{2}}V_{us}^*V_{ud}\sum_{i=1}^{10} \bigl[z_i(\mu) + \tau y_i(\mu)\bigr] Q_i(\mu).
\label{eq:H_W}
\end{equation}

The 10-operator basis is given below. The scope of this library is to provide the user
with a few easy-to-use Mathematica functions to immediately compute the (renormalized) 
coefficients $z_i$ and $y_i$ at a given renormalization scale $\mu$. The matrix elements 
$Q_i$ must be computed in the same scheme to obtain consistent result at the end.

\leftline{\bf Current-Current Operators}

\begin{eqnarray}
Q_{1} = \left( \bar s_{i} u_{j}  \right)_{\rm V-A}
            \left( \bar u_{j}  d_{i} \right)_{\rm V-A}
&\qquad&
Q_{2} = \left( \bar s u \right)_{\rm V-A}
            \left( \bar u d \right)_{\rm V-A}
\label{eq:Q12}
\end{eqnarray}

\leftline{\bf QCD-Penguins Operators}

\begin{eqnarray}
Q_{3} = \left( \bar s d \right)_{\rm V-A}
   \sum_{q} \left( \bar q q \right)_{\rm V-A}
&\qquad&
Q_{4} = \left( \bar s_{i} d_{j}  \right)_{\rm V-A}
   \sum_{q} \left( \bar q_{j}  q_{i} \right)_{\rm V-A}
\label{eq:Q34} \\
Q_{5} = \left( \bar s d \right)_{\rm V-A}
   \sum_{q} \left( \bar q q \right)_{\rm V+A}
&\qquad&
Q_{6} = \left( \bar s_{i} d_{j}  \right)_{\rm V-A}
   \sum_{q} \left( \bar q_{j}  q_{i} \right)_{\rm V+A}
\label{eq:Q56}
\end{eqnarray}

\leftline{\bf Electroweak-Penguins Operators}

\begin{eqnarray}
Q_{7} = \frac{3}{2} \left( \bar s d \right)_{\rm V-A}
         \sum_{q} e_{q} \left( \bar q q \right)_{\rm V+A}
&\qquad&
Q_{8} = \frac{3}{2} \left( \bar s_{i} d_{j} \right)_{\rm V-A}
         \sum_{q} e_{q} \left( \bar q_{j}  q_{i}\right)_{\rm V+A}
\label{eq:Q78} \\
Q_{9} = \frac{3}{2} \left( \bar s d \right)_{\rm V-A}
         \sum_{q} e_{q} \left( \bar q q \right)_{\rm V-A}
&\qquad&
Q_{10} = \frac{3}{2} \left( \bar s_{i} d_{j} \right)_{\rm V-A}
         \sum_{q} e_{q} \left( \bar q_{j}  q_{i}\right)_{\rm V-A}
\label{eq:Q910}
\end{eqnarray}

