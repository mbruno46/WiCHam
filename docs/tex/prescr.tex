
Here we give a detailed explanation of the various 
subtleties which one encounters in the implementation
of the running of the Wilson Coefficients, which are 
not described nor mentioned in the original paper 
by Buras et al.

\subsection{LO prescription}

\begin{itemize}
\item $C_1 \,, C_2$ \\

\item $C_3 \ldots C_6$ \\

\item $C_7 \ldots C_{10}$ \\
\end{itemize}

Let us consider now the RG evolution matrices.

\begin{equation}
U(\mu,m)= \overbrace{U^{(0)}(\mu,m)}^\text{LO} + 
\overbrace{ {1 \over 4\pi} \Big[ \as(\mu) J U^{(0)}(\mu,m) - \as(m) U^{(0)}(\mu,m) J \Big] }^\text{NLO}
\end{equation}
 
\begin{equation}
\hR(m_1,m_2) \equiv -\frac{2\pi}{\beta_0} \; \hV \; \left(
\overbrace{\hK^{(0)}(m_1,m_2)}^\text{LO} +
\overbrace{\frac{1}{4\pi} \sum_{i=1}^{3} \hK_i^{(1)}(m_1,m_2)}^\text{NLO} \right) \hV^{-1}
\end{equation}

The quark threshold matching matrices at leading order correspond to the 
identity matrix only, while at NLO to the full formula given in the previous sections.
However note that the prescription for the $z_i$ coefficients at the charm threshold 
(recall the functions \code{Fs} and \code{Fse}) is at leading order.

\subsection{Apparent divergence in the $10 \times 10$ $R$ matrix}

The $10 \times 10$ case, with full QCD+QED, leads to an apparent
divergence in the matrix $K_1^{(1)}$ (off-diagonal components)

\begin{equation}
\left( \hK_1^{(1)}(m_1,m_2) \right)_{ij} =
\Bigg\lbrace
\begin{array}{ll}
\frac{M^{(1)}_{ij}}{a_i - a_j}
\left[ \left( \frac{\as(m_2)}{\as(m_1)} \right)^{a_j} -
       \left( \frac{\as(m_2)}{\as(m_1)} \right)^{a_i} \right] & i \not= j \\
M^{(1)}_{ii} \left( \frac{\as(m_2)}{\as(m_1)} \right)^{a_i}
             \ln\frac{\as(m_1)}{\as(m_2)}             & i=j
\end{array} 
\end{equation}

due to the presence of two couples of identical eigenvalues of $\gs$. 
However one can easily verify that the divergence cancels 
by substituting $a^i = a^j + \varepsilon$ in both numerator and denominator, 
and by expanding around $\varepsilon =0$
\begin{equation}
\frac{x^{a_j} - x^{a_i}}{a_i - a_j} = x^{a_j} \frac{1 - x^\varepsilon}{\varepsilon} 
 = x^{a_j} \frac{1 - (1 + \varepsilon \log x + O(\varepsilon^2))}{\varepsilon} = 
- x^{a_j} \log x
\end{equation}
In fact, the same type of calculation leads to the diagonal element of $K_1^{(1)}$,
which can be re-expressed as
\begin{equation}
\left( \hK_1^{(1)}(m_1,m_2) \right)_{ij} =
\Bigg\lbrace
\begin{array}{ll}
\frac{M^{(1)}_{ij}}{a_i - a_j}
\left[ \left( \frac{\as(m_2)}{\as(m_1)} \right)^{a_j} -
       \left( \frac{\as(m_2)}{\as(m_1)} \right)^{a_i} \right] & \mathbf{a_i \not= a_j} \\
M^{(1)}_{ii} \left( \frac{\as(m_2)}{\as(m_1)} \right)^{a_i}
             \ln\frac{\as(m_1)}{\as(m_2)}             & \mathbf{ a_i=a_j}
\end{array} 
\end{equation}


\subsection{Apparent divergence in the $N_f=3$ case}

The divergence appearing in the limit $N_f=3$ looks similar to the case above, since
it arises from the fact that two eigenvalues differ exactly by 1 and therefore
the combination $a_i -a_j -1$, present in various denominators, vanishes.

By performing the same type of expansions described above one can check that for pure QCD
such a divergence cancels in the combination
\[ \as(\mu) J U^{(0)}(\mu,m) - \as(m) U^{(0)}(\mu,m) J \]
while in the QED part it cancels automatically inside the matrix $K^{(0)}$
\begin{equation}
(\hK^{(0)}(m_1,m_2))_{ij} = \frac{\hM^{(0)}_{ij}}{a_i - a_j - 1}
\left[
\left( \frac{\as(m_2)}{\as(m_1)} \right)^{a_j} \frac{1}{\as(m_1)} -
\left( \frac{\as(m_2)}{\as(m_1)} \right)^{a_i} \frac{1}{\as(m_2)}
\right] 
\label{eq:K0}
\end{equation}
and in the difference
\begin{equation}
K_1^{(1)} - \as(m_2) \; \hK^{(0)} \; H + \,\as(m_1) \; H \; \hK^{(0)}
\end{equation}

In the code we implemented the substitution $N_f \to N_f + \varepsilon$ both in the 
$U_0$ and $J$ matrices and the limit $\varepsilon \to 0$ is taken by calling the functions 
\code{U} and \code{R}, or alternatively \code{FullU}. 
Only in the definition of $K^{(0)}$ we hard-coded the solution to the entire expansion

