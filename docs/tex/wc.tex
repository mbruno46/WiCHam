
In this section we describe the most important functions of this library:
those to compute the Wilson Coefficients at any given energy scale $\mu$ 
between $1 \gev$ and $\mz$.

\subsection{Definition of the Wilson Coefficients}

\code{C1[a]}\\
\code{C2[a,ae]}\\
\code{C3[a,mt,MW,ae]}\\
\code{C4[a,mt,MW]}\\
\code{C5[a,mt,MW]}\\
\code{C6[a,mt,MW]}\\
\code{C7[mt,MW,ae]}\\
\code{C8}\\
\code{C9[mt,MW,ae]}\\
\code{C10}

\begin{itemize}
\item \code{a}: value of $\as$
\item \code{mt}: value of the top quark mass
\item \code{MW}: value of the W boson mass
\item \code{ae}: value of $\aem$
\end{itemize}

The Wilson Coefficients at the weak scale are
\begin{eqnarray}
C_1(\mw) &=&     \frac{11}{2} \; \frac{\as(\mw)}{4\pi} \, ,
\label{eq:CMw1} \\
C_2(\mw) &=& 1 - \frac{11}{6} \; \frac{\as(\mw)}{4\pi}
               - \frac{35}{18} \; \frac{\aem}{4\pi} \, ,
\label{eq:CMw2} \\
C_3(\mw) &=& -\frac{\as(\mw)}{24\pi} \widetilde{E}_0(x_t)
             +\frac{\aem}{6\pi} \frac{1}{\sin^2\theta_W}
             \left[ 2 B_0(x_t) + C_0(x_t) \right] \, , 
\label{eq:CMw3} \\
C_4(\mw) &=& \frac{\as(\mw)}{8\pi} \widetilde{E}_0(x_t) \, ,
\label{eq:CMw4} \\
C_5(\mw) &=& -\frac{\as(\mw)}{24\pi} \widetilde{E}_0(x_t) \, ,
\label{eq:CMw5} \\
C_6(\mw) &=& \frac{\as(\mw)}{8\pi} \widetilde{E}_0(x_t) \, ,
\label{eq:CMw6} \\
C_7(\mw) &=& \frac{\aem}{6\pi} \left[ 4 C_0(x_t) + \widetilde{D}_0(x_t)
\right]\, ,
\label{eq:CMw7} \\
C_8(\mw) &=& 0 \, 
\label{eq:CMw8} \\
C_9(\mw) &=& \frac{\aem}{6\pi} \left[ 4 C_0(x_t) + \widetilde{D}_0(x_t) +
             \frac{1}{\sin^2\theta_W} (10 B_0(x_t) - 4 C_0(x_t)) \right] \, ,
\label{eq:CMw9} \\
C_{10}(\mw) &=& 0 \, ,
\label{eq:CMw10} \\
x_t = \frac{m_t^2}{\mw^2} \, .
\label{eq:xt}
\end{eqnarray}
with the QED-related auxiliary functions 
\begin{eqnarray}
B_0(x) &=& \frac{1}{4} \left[ \frac{x}{1-x} + \frac{x \ln x}{(x-1)^2}
\right]\, , \label{eq:Bxt} \\
C_0(x) &=& \frac{x}{8} \left[ \frac{x-6}{x-1} + \frac{3 x + 2}{(x-1)^2}
\ln x \right]\, ,
\label{eq:Cxt} \\
D_0(x) &=& -\frac{4}{9} \ln x + \frac{-19 x^3 + 25 x^2}{36 (x-1)^3} +
         \frac{x^2 (5 x^2 - 2 x - 6)}{18 (x-1)^4} \ln x \, ,
\label{eq:Dxt} \\
\widetilde{D}_0(x_t) &=& D_0(x_t) - \frac{4}{9} \, .
\label{eq:Dxttilde} 
\end{eqnarray}
and the QCD-related ones
\begin{eqnarray}
E_0(x) &=& -\frac{2}{3} \ln x + \frac{x (18 -11 x - x^2)}{12 (1-x)^3} +
          \frac{x^2 (15 - 16 x  + 4 x^2)}{6 (1-x)^4} \ln x \, ,
\label{eq:Ext} \\
\widetilde{E}_0(x_t) &=& E_0(x_t) - \frac{2}{3}
\label{eq:Exttilde}
\end{eqnarray}

\subsection{Evolution of the Wilson Coefficients}

\code{ComputeZ[mu,initAlphaMZ,loop,MZ,aem,init12]}

\begin{itemize}
\item \code{mu}: mass scale at which the coefficients $v_i$ are computed
\item \code{initAlphaMZ}: initial value of $\as$ at the electroweak scale
\item \code{loop}: number of loops of $\as$. Accepted values are 1,2,3,4.
\item \code{MZ}: energy scale corresponding to \code{initAlphaMZ}. Default is \code{MZ}.
\item \code{ae}: value of $\aem$. Default is 1/129.
\item \code{init12}: bi-dimensional vector containing the values of the Wilson Coefficients $C_1$ and $C_2$ 
at the weak scale. Default corresponds to the results presented in the previous subsection.
\end{itemize}

The function returns the vector $\vec z(\mu)$ by taking into account
possible quark thresholds. If $\mu>\mb$

\begin{equation}
z_1(\mw) = C_1(\mw) \, ,
\qquad
z_2(\mw) = C_2(\mw) \, .
\label{eq:zMw12}
\end{equation}

\begin{equation}
\left( \begin{array}{ll} z_1(\mc) \\ z_2(\mc) \end{array} \right) =
\hU_5(\mu,\mw) \;
\left( \begin{array}{ll} z_1(\mw) \\ z_2(\mw) \end{array} \right) \, ,
\label{eq:zmc12}
\end{equation}

whereas if $\mu=\mb$

\begin{equation}
\left( \begin{array}{ll} z_1(\mc) \\ z_2(\mc) \end{array} \right) =
\hM(\mb) \; \hU_5(\mb,\mw) \;
\left( \begin{array}{ll} z_1(\mw) \\ z_2(\mw) \end{array} \right) \, ,
\label{eq:zmc12}
\end{equation}

If $\mu<\mb$ 

\begin{equation}
\left( \begin{array}{ll} z_1(\mc) \\ z_2(\mc) \end{array} \right) =
\hU_4(\mu,\mb) \; \hM(\mb) \; \hU_5(\mb,\mw) \;
\left( \begin{array}{ll} z_1(\mw) \\ z_2(\mw) \end{array} \right) \, ,
\label{eq:zmc12}
\end{equation}

whereas if $\mu=\mc$ 

\begin{equation}
\left( \begin{array}{ll} z_1(\mc) \\ z_2(\mc) \end{array} \right) =
\hU_4(\mc,\mb) \; \hM(\mb) \; \hU_5(\mb,\mw) \;
\left( \begin{array}{ll} z_1(\mw) \\ z_2(\mw) \end{array} \right) \, ,
\label{eq:zmc12}
\end{equation}

\begin{equation}
\vec{z}^{\rm }(\mc) =
\left( \begin{array}{c}
z_1^{\rm }(\mc) \\ z_2^{\rm }(\mc) \\
\as/(36\pi) z_2^{\rm }(\mc) \\
-\as/(12 \pi) z_2^{\rm }(\mc) \\
\as/(36\pi) z_2^{\rm }(\mc) \\
-\as/(12 \pi) z_2^{\rm }(\mc) \\
\aem/(6\pi) F_{\rm e}^{\rm }(\mc) \\  0  \\
\aem/(6\pi) F_{\rm e}^{\rm }(\mc) \\  0
\end{array} \right) \, ,
\label{eq:zmc:1010}
\end{equation}
with
\begin{equation}
F_{\rm e}^{\rm }(\mc) =
-\frac{4}{9} \; \left( 3 z_1(\mc) + z_2(\mc) \right) \, .
\label{eq:Femc}
\end{equation}

Finally if $\mu<\mc$

\begin{equation}
\vec{z}(\mu) = \hU_3(\mu,\mc) \vec{z}(\mc) \, .
\label{eq:WCz}
\end{equation}

\hrulefill

\code{ComputeY[z,mu,initAlphaMZ,loop,MZ,aem,init]}

\begin{itemize}
\item \code{z}: 10 dimensional vector containg the values of the Wilson coefficients $z_i(\mu)$
\item \code{mu}: mass scale at which the coefficients $v_i$ are computed
\item \code{initAlphaMZ}: initial value of $\as$ at the electroweak scale
\item \code{loop}: number of loops of $\as$. Accepted values are 1,2,3,4.
\item \code{MZ}: energy scale corresponding to \code{initAlphaMZ}. Default is \code{MZ}.
\item \code{ae}: value of $\aem$. Default is 1/129.
\item \code{init}: ten dimensional vector containing the values of the Wilson Coefficients $C_i$ at the
weak scale. Default corresponds to the results presented in the previous subsection.
\end{itemize}

\begin{equation}
y_i(\mu) = v_i(\mu) - z_i(\mu) \, .
\label{eq:WCy}
\end{equation}

\hrulefill

\code{ComputeV[mu,initAlphaMZ,loop,MZ,aem,init]}

\begin{itemize}
\item \code{mu}: mass scale at which the coefficients $v_i$ are computed
\item \code{initAlphaMZ}: initial value of $\as$ at the electroweak scale
\item \code{loop}: number of loops of $\as$. Accepted values are 1,2,3,4.
\item \code{MZ}: energy scale corresponding to \code{initAlphaMZ}. Default is \code{MZ}.
\item \code{ae}: value of $\aem$. Default is 1/129.
\item \code{init}: ten dimensional vector containing the values of the Wilson Coefficients $C_i$ at the
weak scale. Default corresponds to the results presented in the previous subsection.
\end{itemize}

The functions return the vector $\vec v(\mu)$ by taking into account 
possible quark thresholds. If $\mu>\mb$

\begin{equation}
\vec{v}(\mu) =
\hU_5(\mu,\mw) \vec{C}(\mw) \, ,
\label{eq:WCv}
\end{equation}

whereas if $\mu=\mb$

\begin{equation}
\vec{v}(\mu) =
\hM(\mb) \hU_5(\mb,\mw) \vec{C}(\mw) \, ,
\label{eq:WCv}
\end{equation}

If $\mu<\mb$
\begin{equation}
\vec{v}(\mu) =
\hU_4(\mu,\mb) \hM(\mb) \hU_5(\mb,\mw) \vec{C}(\mw) \, ,
\label{eq:WCv}
\end{equation}

whereas if $\mu=\mc$

\begin{equation}
\vec{v}(\mu) =
\hM(\mc) \hU_4(\mc,\mb) \hM(\mb) \hU_5(\mb,\mw) \vec{C}(\mw) \, ,
\label{eq:WCv}
\end{equation}

Finally if $\mu<\mc$

\begin{equation}
\vec{v}(\mu) =
\hU_3(\mu,\mc) \hM(\mc) \hU_4(\mc,\mb) \hM(\mb) \hU_5(\mb,\mw)
\vec{C}(\mw) \, ,
\label{eq:WCv}
\end{equation}

\subsection{LO and NLO prescriptions}

\code{ReduceOrder[expr,order]}

\begin{itemize}
\item \code{expr}: expression of the Wilson Coefficients $y_i$ and $z_i$
\item \code{order}: accepted values are LO and NLO.
\end{itemize}

This function takes as input the full expression of the Wilson Coefficients 
$y_i$ and $z_i$ and reduce it to the desired order.

