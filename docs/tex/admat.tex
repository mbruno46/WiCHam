
In thi section we describe the functions to compute the anomalous dimension
matrices used in the evolution of the Wilson Coefficients.

\subsection{QCD anomalous dimension matrices}

\code{gammas0[Nc,Nf,size,Nu]}

\begin{itemize}
\item \code{Nc}: number of colors
\item \code{Nf}: number of flavors
\item \code{size}: size of returned matrix. Default is 10
\item \code{Nu}: number of up quarks. Default is 2
\end{itemize}

\begin{equation}
\gs
\end{equation}

\hrulefill

\code{gammas1[Nf,size,Nu]}

\begin{itemize}
\item \code{Nf}: number of flavors
\item \code{size}: size of returned matrix. Default is 10
\item \code{Nu}: number of up quarks. Default is 2
\end{itemize}

\begin{equation}
\gss
\end{equation}

\subsection{QED anomalous dimension matrices}

\code{gammae0[Nc,Nf,size,Nu]}

\begin{itemize}
\item \code{Nc}: number of colors
\item \code{Nf}: number of flavors
\item \code{size}: size of returned matrix. Default is 10
\item \code{Nu}: number of up quarks. Default is 2
\end{itemize}

\begin{equation}
\gem
\end{equation}

\hrulefill

\code{gammase1[Nf,size,Nu]}

\begin{itemize}
\item \code{Nf}: number of flavors
\item \code{size}: size of returned matrix. Default is 10
\item \code{Nu}: number of up quarks. Default is 2
\end{itemize}

\begin{equation}
\gse
\end{equation}

