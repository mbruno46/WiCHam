\documentclass[a4paper]{scrartcl}
\usepackage{amsmath, amscd}
\usepackage{booktabs}
\usepackage[dvipsnames]{xcolor}

\title{\textcolor{red}{Wi}lson \textcolor{red}{C}oefficient for the effective weak \textcolor{red}{Ham}iltonian}
\author{Mattia Bruno}


\newcommand{\gev}{\mathrm{GeV}}
\newcommand{\mb}{m_{\mathrm b}}
\newcommand{\mc}{m_{\mathrm c}}
\newcommand{\mt}{m_{\mathrm t}}
\newcommand{\mw}{M_{\mathrm W}}
\newcommand{\mz}{M_{\mathrm Z}}

\newcommand{\as}{\alpha_{\rm s}}
\newcommand{\aem}{\alpha_{\rm e}}

\newcommand{\hU}{U}
\newcommand{\hR}{R}
\newcommand{\hJ}{J}
\newcommand{\hV}{V}
\newcommand{\hK}{K}
\newcommand{\hM}{M}

\newcommand{\gs}{\gamma_{\rm s}^{(0)}}
\newcommand{\gem}{\gamma_{\rm e}^{(0)}}
\newcommand{\gss}{\gamma_{\rm s}^{(1)}}
\newcommand{\gse}{\gamma_{\rm se}^{(1)}}

\newcommand{\gst}{\gamma_{\rm s}^{(0) T}}
\newcommand{\gemt}{\gamma_{\rm e}^{(0) T}}
\newcommand{\gsst}{\gamma_{\rm s}^{(1) T}}
\newcommand{\gset}{\gamma_{\rm se}^{(1) T}}

\newcommand{\hJe}{J_{\rm e}}
\newcommand{\hJs}{J_{\rm s}}
\newcommand{\hJse}{J_{\rm se}}
\newcommand{\vardre}{\delta r_{\rm e}}
\newcommand{\vardrs}{\delta r_{\rm s}}

\newcommand{\code}[1]{\noindent\textcolor{RedOrange}{\texttt{#1}}}

\begin{document}
\maketitle

This library is a Mathematica implementation of the Wilson Coefficients 
for the $\Delta F=1$ Weak Hamiltonian computed in \textit{Buras et al.} 
\texttt{arXiv:hep-ph/95123801}.
The goal for this library is to provide the user with a simple-to-use 
series of functions which automatically evaluate the Wilson Coefficients 
at a given mass scale $\mu$ and order in $\as$.

This library contains also all the basic blocks used in the evolution of the 
Wilson Coefficients, from the strong coupling constant coded up to 4 loops to the 
evolution matrices $U$. Starting from this basic functions it is possible to 
evaluate the Wilson Coefficients for all transitions related to the effective
hamiltonian written below.\\

The documentation is organized as follows. The colored text represents 
Mathematica code and each function is described according to this structure:

\hrulefill

\code{function[variable1,variable2, \ldots]}\\
\begin{itemize}
\item \code{variable1}: explanation
\item \code{variable2}: explanation
\end{itemize}

Additional comments and explanations go here together with
equations to define and describe what \code{function} does.

\newpage
\section{A brief theoretical introduction}
\def\VmA{\mathrm{V-A}}
\def\VpA{\mathrm{V+A}}


By integrating out the weak bosons from the standard model we obtain an 
effective theory, called weak effective hamiltonian. 
As an effective theory it loses the property of being renormalizable,
meaning that at each order in perturbation theory a new (finite) set of couplings and operators
is needed to properly cancel the remaining divergences. Nevertheless, if we work at a fixed 
order the theory is finite and in the following we will concentrate only 
on the leading order effective theory.

Composite operators in the effective theory are expected to mix under renormalization, unless
there is a symmetry protecting them. This mixing can be generically expressed through a 
$Z$-factor matrix, which can be perturbatively expanded in terms of the so-called 
anomalous dimension matrices. Once they are known, the operators or alternatively the Wilson Coefficients
can be run to low energies and the implementation provided in this library includes effects
up to $O(\as)$ and $O(\aem)$.

The Wilson Coefficients are obtained by comparing 
(order by order in perturbation theory) amplitudes computed in the standard model
against amplitudes computed in the effective theory. Obviously amplitudes are scheme-independent
but the two parts of the calculation, the Wilson Coefficients and the low-energy matrix elements, 
individually depend on the scheme. 

\subsection{The $\Delta F=1$ Hamiltonian}

We are interested here in weak decays where either the strange or the bottom quantum numbers 
are violated by one unit. In general these processes can be described by a single effective
hamiltonian

\begin{equation}
H_W = \frac{G_F}{\sqrt{2}}V_{us}^*V_{ud}\sum_{i=1}^{10} \bigl[z_i(\mu) + \tau y_i(\mu)\bigr] Q_i(\mu).
\label{eq:H_W}
\end{equation}

The 10-operator basis is given below. The scope of this library is to provide the user
with a few easy-to-use Mathematica functions to immediately compute the (renormalized) 
coefficients $z_i$ and $y_i$ at a given renormalization scale $\mu$. The matrix elements 
$Q_i$ must be computed in the same scheme to obtain consistent result at the end.

\leftline{\textbf Current-Current Operators}

\begin{eqnarray}
Q_{1} = \left( \bar s_{i} u_{j}  \right)_{\VmA}
            \left( \bar u_{j}  d_{i} \right)_{\VmA}
&\qquad&
Q_{2} = \left( \bar s u \right)_{\VmA}
            \left( \bar u d \right)_{\VmA}
\label{eq:Q12}
\end{eqnarray}

\leftline{\textbf QCD-Penguins Operators}

\begin{eqnarray}
Q_{3} = \left( \bar s d \right)_{\VmA}
   \sum_{q} \left( \bar q q \right)_{\VmA}
&\qquad&
Q_{4} = \left( \bar s_{i} d_{j}  \right)_{\VmA}
   \sum_{q} \left( \bar q_{j}  q_{i} \right)_{\VmA}
\label{eq:Q34} \\
Q_{5} = \left( \bar s d \right)_{\VmA}
   \sum_{q} \left( \bar q q \right)_{\VpA}
&\qquad&
Q_{6} = \left( \bar s_{i} d_{j}  \right)_{\VmA}
   \sum_{q} \left( \bar q_{j}  q_{i} \right)_{\VpA}
\label{eq:Q56}
\end{eqnarray}

\leftline{\textbf Electroweak-Penguins Operators}

\begin{eqnarray}
Q_{7} = \frac{3}{2} \left( \bar s d \right)_{\VmA}
         \sum_{q} e_{q} \left( \bar q q \right)_{\VpA}
&\qquad&
Q_{8} = \frac{3}{2} \left( \bar s_{i} d_{j} \right)_{\VmA}
         \sum_{q} e_{q} \left( \bar q_{j}  q_{i}\right)_{\VpA}
\label{eq:Q78} \\
Q_{9} = \frac{3}{2} \left( \bar s d \right)_{\VmA}
         \sum_{q} e_{q} \left( \bar q q \right)_{\VmA}
&\qquad&
Q_{10} = \frac{3}{2} \left( \bar s_{i} d_{j} \right)_{\VmA}
         \sum_{q} e_{q} \left( \bar q_{j}  q_{i}\right)_{\VmA}
\label{eq:Q910}
\end{eqnarray}



\newpage
\section{Computation of the Wilson Coefficients}

In this section we describe the most important functions of this library:
those to compute the Wilson Coefficients at any given energy scale $\mu$ 
between $1 \gev$ and $\mz$.

\subsection{Definition of the Wilson Coefficients}

\code{C1[a]}\\
\code{C2[a,ae]}\\
\code{C3[a,mt,MW,ae]}\\
\code{C4[a,mt,MW]}\\
\code{C5[a,mt,MW]}\\
\code{C6[a,mt,MW]}\\
\code{C7[mt,MW,ae]}\\
\code{C8}\\
\code{C9[mt,MW,ae]}\\
\code{C10}

\begin{itemize}
\item \code{a}: value of $\as$
\item \code{mt}: value of the top quark mass
\item \code{MW}: value of the W boson mass
\item \code{ae}: value of $\aem$
\end{itemize}

The Wilson Coefficients at the weak scale are
\begin{eqnarray}
C_1(\mw) &=&     \frac{11}{2} \; \frac{\as(\mw)}{4\pi} \, ,
\label{eq:CMw1} \\
C_2(\mw) &=& 1 - \frac{11}{6} \; \frac{\as(\mw)}{4\pi}
               - \frac{35}{18} \; \frac{\aem}{4\pi} \, ,
\label{eq:CMw2} \\
C_3(\mw) &=& -\frac{\as(\mw)}{24\pi} \widetilde{E}_0(x_t)
             +\frac{\aem}{6\pi} \frac{1}{\sin^2\theta_W}
             \left[ 2 B_0(x_t) + C_0(x_t) \right] \, , 
\label{eq:CMw3} \\
C_4(\mw) &=& \frac{\as(\mw)}{8\pi} \widetilde{E}_0(x_t) \, ,
\label{eq:CMw4} \\
C_5(\mw) &=& -\frac{\as(\mw)}{24\pi} \widetilde{E}_0(x_t) \, ,
\label{eq:CMw5} \\
C_6(\mw) &=& \frac{\as(\mw)}{8\pi} \widetilde{E}_0(x_t) \, ,
\label{eq:CMw6} \\
C_7(\mw) &=& \frac{\aem}{6\pi} \left[ 4 C_0(x_t) + \widetilde{D}_0(x_t)
\right]\, ,
\label{eq:CMw7} \\
C_8(\mw) &=& 0 \, 
\label{eq:CMw8} \\
C_9(\mw) &=& \frac{\aem}{6\pi} \left[ 4 C_0(x_t) + \widetilde{D}_0(x_t) +
             \frac{1}{\sin^2\theta_W} (10 B_0(x_t) - 4 C_0(x_t)) \right] \, ,
\label{eq:CMw9} \\
C_{10}(\mw) &=& 0 \, ,
\label{eq:CMw10} \\
x_t = \frac{m_t^2}{\mw^2} \, .
\label{eq:xt}
\end{eqnarray}
with the QED-related auxiliary functions 
\begin{eqnarray}
B_0(x) &=& \frac{1}{4} \left[ \frac{x}{1-x} + \frac{x \ln x}{(x-1)^2}
\right]\, , \label{eq:Bxt} \\
C_0(x) &=& \frac{x}{8} \left[ \frac{x-6}{x-1} + \frac{3 x + 2}{(x-1)^2}
\ln x \right]\, ,
\label{eq:Cxt} \\
D_0(x) &=& -\frac{4}{9} \ln x + \frac{-19 x^3 + 25 x^2}{36 (x-1)^3} +
         \frac{x^2 (5 x^2 - 2 x - 6)}{18 (x-1)^4} \ln x \, ,
\label{eq:Dxt} \\
\widetilde{D}_0(x_t) &=& D_0(x_t) - \frac{4}{9} \, .
\label{eq:Dxttilde} 
\end{eqnarray}
and the QCD-related ones
\begin{eqnarray}
E_0(x) &=& -\frac{2}{3} \ln x + \frac{x (18 -11 x - x^2)}{12 (1-x)^3} +
          \frac{x^2 (15 - 16 x  + 4 x^2)}{6 (1-x)^4} \ln x \, ,
\label{eq:Ext} \\
\widetilde{E}_0(x_t) &=& E_0(x_t) - \frac{2}{3}
\label{eq:Exttilde}
\end{eqnarray}

\subsection{Evolution of the Wilson Coefficients}

\code{ComputeZ[mu,initAlphaMZ,loop,MZ,aem,init12]}

\begin{itemize}
\item \code{mu}: mass scale at which the coefficients $v_i$ are computed
\item \code{initAlphaMZ}: initial value of $\as$ at the electroweak scale
\item \code{loop}: number of loops of $\as$. Accepted values are 1,2,3,4.
\item \code{MZ}: energy scale corresponding to \code{initAlphaMZ}. Default is \code{MZ}.
\item \code{ae}: value of $\aem$. Default is 1/129.
\item \code{init12}: bi-dimensional vector containing the values of the Wilson Coefficients $C_1$ and $C_2$ 
at the weak scale. Default corresponds to the results presented in the previous subsection.
\end{itemize}

The function returns the vector $\vec z(\mu)$ by taking into account
possible quark thresholds. If $\mu>\mb$

\begin{equation}
z_1(\mw) = C_1(\mw) \, ,
\qquad
z_2(\mw) = C_2(\mw) \, .
\label{eq:zMw12}
\end{equation}

\begin{equation}
\left( \begin{array}{ll} z_1(\mc) \\ z_2(\mc) \end{array} \right) =
\hU_5(\mu,\mw) \;
\left( \begin{array}{ll} z_1(\mw) \\ z_2(\mw) \end{array} \right) \, ,
\label{eq:zmc12}
\end{equation}

whereas if $\mu=\mb$

\begin{equation}
\left( \begin{array}{ll} z_1(\mc) \\ z_2(\mc) \end{array} \right) =
\hM(\mb) \; \hU_5(\mb,\mw) \;
\left( \begin{array}{ll} z_1(\mw) \\ z_2(\mw) \end{array} \right) \, ,
\label{eq:zmc12}
\end{equation}

If $\mu<\mb$ 

\begin{equation}
\left( \begin{array}{ll} z_1(\mc) \\ z_2(\mc) \end{array} \right) =
\hU_4(\mu,\mb) \; \hM(\mb) \; \hU_5(\mb,\mw) \;
\left( \begin{array}{ll} z_1(\mw) \\ z_2(\mw) \end{array} \right) \, ,
\label{eq:zmc12}
\end{equation}

whereas if $\mu=\mc$ 

\begin{equation}
\left( \begin{array}{ll} z_1(\mc) \\ z_2(\mc) \end{array} \right) =
\hU_4(\mc,\mb) \; \hM(\mb) \; \hU_5(\mb,\mw) \;
\left( \begin{array}{ll} z_1(\mw) \\ z_2(\mw) \end{array} \right) \, ,
\label{eq:zmc12}
\end{equation}

\begin{equation}
\vec{z}^{\rm }(\mc) =
\left( \begin{array}{c}
z_1^{\rm }(\mc) \\ z_2^{\rm }(\mc) \\
\as/(36\pi) z_2^{\rm }(\mc) \\
-\as/(12 \pi) z_2^{\rm }(\mc) \\
\as/(36\pi) z_2^{\rm }(\mc) \\
-\as/(12 \pi) z_2^{\rm }(\mc) \\
\aem/(6\pi) F_{\rm e}^{\rm }(\mc) \\  0  \\
\aem/(6\pi) F_{\rm e}^{\rm }(\mc) \\  0
\end{array} \right) \, ,
\label{eq:zmc:1010}
\end{equation}
with
\begin{equation}
F_{\rm e}^{\rm }(\mc) =
-\frac{4}{9} \; \left( 3 z_1(\mc) + z_2(\mc) \right) \, .
\label{eq:Femc}
\end{equation}

Finally if $\mu<\mc$

\begin{equation}
\vec{z}(\mu) = \hU_3(\mu,\mc) \vec{z}(\mc) \, .
\label{eq:WCz}
\end{equation}

\hrulefill

\code{ComputeY[z,mu,initAlphaMZ,loop,MZ,aem,init]}

\begin{itemize}
\item \code{z}: 10 dimensional vector containg the values of the Wilson coefficients $z_i(\mu)$
\item \code{mu}: mass scale at which the coefficients $v_i$ are computed
\item \code{initAlphaMZ}: initial value of $\as$ at the electroweak scale
\item \code{loop}: number of loops of $\as$. Accepted values are 1,2,3,4.
\item \code{MZ}: energy scale corresponding to \code{initAlphaMZ}. Default is \code{MZ}.
\item \code{ae}: value of $\aem$. Default is 1/129.
\item \code{init}: ten dimensional vector containing the values of the Wilson Coefficients $C_i$ at the
weak scale. Default corresponds to the results presented in the previous subsection.
\end{itemize}

\begin{equation}
y_i(\mu) = v_i(\mu) - z_i(\mu) \, .
\label{eq:WCy}
\end{equation}

\hrulefill

\code{ComputeV[mu,initAlphaMZ,loop,MZ,aem,init]}

\begin{itemize}
\item \code{mu}: mass scale at which the coefficients $v_i$ are computed
\item \code{initAlphaMZ}: initial value of $\as$ at the electroweak scale
\item \code{loop}: number of loops of $\as$. Accepted values are 1,2,3,4.
\item \code{MZ}: energy scale corresponding to \code{initAlphaMZ}. Default is \code{MZ}.
\item \code{ae}: value of $\aem$. Default is 1/129.
\item \code{init}: ten dimensional vector containing the values of the Wilson Coefficients $C_i$ at the
weak scale. Default corresponds to the results presented in the previous subsection.
\end{itemize}

The functions return the vector $\vec v(\mu)$ by taking into account 
possible quark thresholds. If $\mu>\mb$

\begin{equation}
\vec{v}(\mu) =
\hU_5(\mu,\mw) \vec{C}(\mw) \, ,
\label{eq:WCv}
\end{equation}

whereas if $\mu=\mb$

\begin{equation}
\vec{v}(\mu) =
\hM(\mb) \hU_5(\mb,\mw) \vec{C}(\mw) \, ,
\label{eq:WCv}
\end{equation}

If $\mu<\mb$
\begin{equation}
\vec{v}(\mu) =
\hU_4(\mu,\mb) \hM(\mb) \hU_5(\mb,\mw) \vec{C}(\mw) \, ,
\label{eq:WCv}
\end{equation}

whereas if $\mu=\mc$

\begin{equation}
\vec{v}(\mu) =
\hM(\mc) \hU_4(\mc,\mb) \hM(\mb) \hU_5(\mb,\mw) \vec{C}(\mw) \, ,
\label{eq:WCv}
\end{equation}

Finally if $\mu<\mc$

\begin{equation}
\vec{v}(\mu) =
\hU_3(\mu,\mc) \hM(\mc) \hU_4(\mc,\mb) \hM(\mb) \hU_5(\mb,\mw)
\vec{C}(\mw) \, ,
\label{eq:WCv}
\end{equation}

\subsection{LO and NLO prescriptions}

\code{ReduceOrder[expr,order]}

\begin{itemize}
\item \code{expr}: expression of the Wilson Coefficients $y_i$ and $z_i$
\item \code{order}: accepted values are LO and NLO.
\end{itemize}

This function takes as input the full expression of the Wilson Coefficients 
$y_i$ and $z_i$ and reduce it to the desired order.



\newpage
\section{Basic QCD Functions and input parameters}

In this section we describe the functions related to the strong coupling constant
and the physical constants which are automatically loaded with this library.

\subsection{Strong coupling constant}

\code{beta0[Nc,Nf]}\\
\code{beta1[Nc,Nf]}

\begin{itemize}
\item \code{Nc}: number of colors
\item \code{Nf}: number of flavors
\end{itemize}

First coefficients $b_0$ and $b_1$ of the QCD $\beta$-function.\\

\hrulefill

\code{alphas[mu,L,Nc,Nf,loop]}

\begin{itemize}
\item \code{mu}: energy scale at which $\as$ is computed
\item \code{L}: value of the $\Lambda$ parameter
\item \code{Nc}: number of colors
\item \code{Nf}: number of flavors
\item \code{loop}: number of loops. Accepted values are 1,2,3,4. Default is 2.
\end{itemize}

\hrulefill

\code{FindLambda[a,mu,Nc,Nf,loop]}

\begin{itemize}
\item \code{a}: input value of $\as$
\item \code{mu}: energy scale at which $\as$ is computed
\item \code{Nc}: number of colors
\item \code{Nf}: number of flavors
\item \code{loop}: number of loops. Accepted values are 1,2,3,4. Default is 2.
\end{itemize}

The function returns the value of the $\Lambda$ parameter which solves the matching equation
\begin{equation}
\as^\mathrm{input}  = \as^\mathrm{(loop)}(\mu,\Lambda,N_{\mathrm c},N_{\mathrm f})
\end{equation}

\subsection{Constants}

\code{MW,MZ,mtop,mbottom,mcharm,alphasMZ}

\begin{eqnarray}
\mw = 80.2 \gev \, &,& \mz = \mw/\sqrt{1-0.23^2} \,, \\ 
\mt = 170 \gev \, &,& \mb = 4.4 \gev \,, \mc = 1.3 \gev \,, \\
\as(\mz) = 0.117
\end{eqnarray}


\newpage
\section{Examples}

\hrulefill

\textbf{Example 1} : computation of the LO Wilson Coefficients at $\mu=4 \gev$ using $\Lambda^4 = 0.325 \gev$ 
and $\as$ in LO. Note that to obtain the Wilson Coefficients in the LO approximation 
we must use the one-loop running of $\as$, while for the NLO approximation the two-loop running.\\

Since we want to start from the $\Lambda$ parameter in the 4-flavor theory we have to first 
find the $\Lambda$ parameter in the 5-flavor theory where we match the standard model with 
effective weak hamiltonian. Hence we start by computing $\as$ at the bottom threshold\\
\code{amb = alphas[mbottom,0.325,3,4,1];}\\
and then we compute $\Lambda^5$ by matching the 4 and 5 flavors theories\\
\code{L = FindLambda[amb,mbottom,3,5,1];}\\
\code{aMW = alphas[mMW,L,3,5,1];}\\

Now we can evolve the Wilson Coefficients down to $4 \gev$\\
\code{z = ComputeZ[4,aMW,1,MW];}\\
\code{y = ComputeY[z,4,aMW,1,MW];}\\

To obtain the LO results we simply use\\
\code{ReduceOrder[z,LO]}\\
\code{ReduceOrder[y,LO]}\\

\hrulefill

\textbf{Example 2}: computation of the NLO Wilson Coefficients at $\mu=1 \gev$ using $\as(\mz)$\\
\code{z = ComputeZ[4,alphasMZ,2];}\\
\code{y = ComputeY[z,4,alphasMZ,2];}\\

To obtain the NLO results we simply use\\
\code{ReduceOrder[z,NLO]}\\
\code{ReduceOrder[y,NLO]}



\newpage
\section{Anomalous Dimension Matrices}

In thi section we describe the functions to compute the anomalous dimension
matrices used in the evolution of the Wilson Coefficients.

\subsection{QCD anomalous dimension matrices}

\code{gammas0[Nc,Nf,size,Nu]}

\begin{itemize}
\item \code{Nc}: number of colors
\item \code{Nf}: number of flavors
\item \code{size}: size of returned matrix. Default is 10
\item \code{Nu}: number of up quarks. Default is 2
\end{itemize}

\begin{equation}
\gs
\end{equation}

\hrulefill

\code{gammas1[Nf,size,Nu]}

\begin{itemize}
\item \code{Nf}: number of flavors
\item \code{size}: size of returned matrix. Default is 10
\item \code{Nu}: number of up quarks. Default is 2
\end{itemize}

\begin{equation}
\gss
\end{equation}

\subsection{QED anomalous dimension matrices}

\code{gammae0[Nc,Nf,size,Nu]}

\begin{itemize}
\item \code{Nc}: number of colors
\item \code{Nf}: number of flavors
\item \code{size}: size of returned matrix. Default is 10
\item \code{Nu}: number of up quarks. Default is 2
\end{itemize}

\begin{equation}
\gem
\end{equation}

\hrulefill

\code{gammase1[Nf,size,Nu]}

\begin{itemize}
\item \code{Nf}: number of flavors
\item \code{size}: size of returned matrix. Default is 10
\item \code{Nu}: number of up quarks. Default is 2
\end{itemize}

\begin{equation}
\gse
\end{equation}



\section{Renormalization Group Functions}

In this section we describe the core functions used to implement the RG evolution of the 
Wilson Coefficients.

\subsection{QCD RG functions}

\code{U0[a1,a2,b0,g0]}

\begin{itemize}
\item \code{a1,a2}: values of $\as$ at two different scales $\mu_1>\mu_2$
\item \code{b0}: coefficient $b_0$ of the QCD $\beta$-function
\item \code{g0}: LO anomalous dimension matrix $\gs$ 
\end{itemize}

\begin{equation}
\label{u0vd} U^{(0)}(\mu,m)= V
\left({\left[{\as(m)\over\as(\mu)}\right]}^{{\vec\gamma^{(0)}\over 2\beta_0}}
   \right)_D V^{-1}   
\end{equation}
with
\begin{equation}
\label{ga0d} \gamma^{(0)}_D=V^{-1} {\gamma^{(0)T}} V  
\end{equation}

\code{J[b0,b1,g0,g1]}

\begin{itemize}
\item \code{b0}: coefficient $b_0$ of the QCD $\beta$-function
\item \code{b1}: coefficient $b_1$ of the QCD $\beta$-function
\item \code{g0}: LO anomalous dimension matrix $\gs$
\item \code{g1}: NLO anomalous dimension matrix $\gss$
\end{itemize}

\begin{equation}
\label{jvs} J=V H V^{-1}   
\end{equation}
\begin{equation}
\label{sij} H_{ij}=\delta_{ij}\gamma^{(0)}_i{\beta_1\over 2\beta^2_0}-
    {G_{ij}\over 2\beta_0+\gamma^{(0)}_i-\gamma^{(0)}_j}  
\end{equation}
\begin{equation}
\label{gvg1} G=V^{-1} {\gamma^{(1)T}} V   
\end{equation}
\begin{equation}
\label{ga0d} \gamma^{(0)}_D=V^{-1} {\gamma^{(0)T}} V  
\end{equation}

\code{U[a1,a2,b0,g0,J]}

\begin{itemize}
\item \code{a1,a2}: values of $\as$ at two different scales $\mu_1>\mu_2$
\item \code{b0}: coefficient $b_0$ of the QCD $\beta$-function
\item \code{g0}: LO anomalous dimension matrix $\gs$ 
\item \code{J}: $J$ matrix obtained from \code{J[b0,b1,g0,g1]}
\end{itemize}

\begin{equation}\label{u0jj}
U(\mu,m)= U^{(0)}(\mu,m) + {1 \over 4\pi} \Big[ \as(\mu) J U^{(0)}(\mu,m) - \as(m) U^{(0)}(\mu,m) J \Big]
\end{equation}

\subsection{QED RG functions}

\code{M1[b0,b1,ge0,gse1,J]}

\begin{itemize}
\item \code{b0}: coefficient $b_0$ of the QCD $\beta$-function
\item \code{b1}: coefficient $b_1$ of the QCD $\beta$-function
\item \code{ge0}: LO anomalous dimension matrix $\ge$
\item \code{gse1}: NLO anomalous dimension matrix $\gse$
\item \code{J}: $J$ matrix obtained from \code{J[b0,b1,g0,g1]}
\end{itemize}

\begin{equation}
\hM^{(1)} = \hV^{-1} \left( \gset - \frac{\beta_1}{\beta_0} \gemt +
                \left[ \gemt, \hJ \right] \right) \hV \, .
\label{eq:M1}
\end{equation}

\code{R[a1,a2,b0,g0,ge0,M1,J]}

\begin{itemize}
\item \code{a1,a2}: values of $\as$ at two different scales $\mu_1>\mu_2$
\item \code{b0}: coefficient $b_0$ of the QCD $\beta$-function
\item \code{g0}: LO anomalous dimension matrix $\gs$
\item \code{ge0}: LO anomalous dimension matrix $\ge$
\item \code{M1}: $M_1$ matrix obtained from \code{M1[b0,b1,ge0,gse1,J]}
\item \code{J}: $J$ matrix obtained from \code{J[b0,b1,g0,g1]}
\end{itemize}

\begin{equation}
\hR(m_1,m_2) \equiv -\frac{2\pi}{\beta_0} \; \hV \; \left(
                 \hK^{(0)}(m_1,m_2) +
                 \frac{1}{4\pi} \sum_{i=1}^{3} \hK_i^{(1)}(m_1,m_2)
                 \right) \hV^{-1}
\end{equation}

\begin{equation}
(\hK^{(0)}(m_1,m_2))_{ij} = \frac{\hM^{(0)}_{ij}}{a_i - a_j - 1}
\left[
\left( \frac{\as(m_2)}{\as(m_1)} \right)^{a_j} \frac{1}{\as(m_1)} -
\left( \frac{\as(m_2)}{\as(m_1)} \right)^{a_i} \frac{1}{\as(m_2)}
\right] 
\label{eq:K0}
\end{equation}

\begin{equation}
\left( \hK_1^{(1)}(m_1,m_2) \right)_{ij} =
\Bigg\lbrace
\begin{array}{ll}
\frac{M^{(1)}_{ij}}{a_i - a_j}
\left[ \left( \frac{\as(m_2)}{\as(m_1)} \right)^{a_j} -
       \left( \frac{\as(m_2)}{\as(m_1)} \right)^{a_i} \right] & i \not= j \\
M^{(1)}_{ii} \left( \frac{\as(m_2)}{\as(m_1)} \right)^{a_i}
             \ln\frac{\as(m_1)}{\as(m_2)}             & i=j
\end{array} 
\label{eq:K11}
\end{equation}

\begin{eqnarray}
\hK_2^{(1)}(m_1,m_2) & = &
-\,\as(m_2) \; \hK^{(0)}(m_1,m_2) \; H \, ,
\label{eq:K12} \\
\hK_3^{(1)}(m_1,m_2) & = &
\phantom{-}\,\as(m_1) \; H \; \hK^{(0)}(m_1,m_2)
\label{eq:K13}
\end{eqnarray}
with
\begin{equation}
\hM^{(0)} = \hV^{-1} \; \gemt \; \hV
\end{equation}

\subsection{Quark threshold matching functions}

\code{Fs[a,z12]}

\begin{itemize}
\item \code{a}: value of $\as$ at the mass threshold $\mc$.
\item \code{z12}: vector with the two Wilson Coefficients $z_1$ and $z_2$ computed at the
quark threshold $\mc$
\end{itemize}

The function \code{Fs} returns the vector below

\begin{equation}
\vec{z}^{\rm }(\mc) =
\left( \begin{array}{c}
z_1^{\rm }(\mc) \\ z_2^{\rm }(\mc) \\
\as/(36\pi) z_2^{\rm }(\mc) \\
-\as/(12 \pi) z_2^{\rm }(\mc) \\
\as/(36\pi) z_2^{\rm }(\mc) \\
-\as/(12 \pi) z_2^{\rm }(\mc) 
\end{array} \right) \, ,
\label{eq:zmc:1010}
\end{equation}

\code{Fse[a,z12,ae]}

\begin{itemize}
\item \code{a}: value of $\as$ at the mass thresholds $\mb$ and $\mc$.
\item \code{z12}: vector with the two Wilson Coefficients $z_1$ and $z_2$ computed at the
quark thresholds $\mc$ and $\mb$
\item \code{ae}: value of $\aem$. Default is 0.
\end{itemize}

The function \code{Fse} returns the vector below 

\begin{equation}
\vec{z}^{\rm }(\mc) =
\left( \begin{array}{c}
z_1^{\rm }(\mc) \\ z_2^{\rm }(\mc) \\
\as/(36\pi) z_2^{\rm }(\mc) \\
-\as/(12 \pi) z_2^{\rm }(\mc) \\
\as/(36\pi) z_2^{\rm }(\mc) \\
-\as/(12 \pi) z_2^{\rm }(\mc) \\
\aem/(6\pi) F_{\rm e}^{\rm }(\mc) \\  0  \\
\aem/(6\pi) F_{\rm e}^{\rm }(\mc) \\  0
\end{array} \right) \, ,
\label{eq:zmc:1010}
\end{equation}
with 
\begin{equation}
F_{\rm e}^{\rm }(\mc) =
-\frac{4}{9} \; \left( 3 z_1(\mc) + z_2(\mc) \right) \, .
\label{eq:Femc}
\end{equation}


\code{M[mu,a,ae,size]}

\begin{itemize}
\item \code{mu}: energy scale of \code{a}. Values accepted are $\mc$ and $\mb$.
\item \code{a}: value of $\as$ at the mass thresholds $\mb$ and $\mc$.
\item \code{ae}: value of $\aem$. Default is 0.
\item \code{size}: size of the returned matrix. Default is 10.
\end{itemize}

\begin{equation}
\hM(m) = 1 + \frac{\as(m)}{4\pi} \; \vardrs^T +
         \frac{\aem}{4\pi} \; \vardre^T \, .
\label{eq:MmdF1:1010}
\end{equation}
The routine automatically uses
\begin{equation}\nonumber
\vardrs^T=\frac{5}{18} P \, (0,0,0,-2,0,-2,0,1,0,1)
\end{equation}
\begin{equation}
\vardre^T=\frac{10}{81} \bar{P} \, (0,0,6,2,6,2,-3,-1,-3,-1)
\label{eq:drsembdF1:1010}
\end{equation}
if $\mu=\mb$, whereas if  $\mu=\mc$
\begin{equation}\nonumber
\vardrs^T=-\frac{5}{9} P \, (0,0,0,1,0,1,0,1,0,1)
\end{equation}
\begin{equation}
\vardre^T=-\frac{40}{81} \bar{P} \, (0,0,3,1,3,1,3,1,3,1)
\label{eq:drsemcdF1:1010}
\end{equation}
with 
\begin{eqnarray}
P^T &=& (0,0,-\frac{1}{3},1,-\frac{1}{3},1,0,0,0,0) \, ,
\label{eq:Pdrs:10} \\
\bar{P}^T &=& (0,0,0,0,0,0,1,0,1,0) \, .
\label{eq:Pbardre:10}
\end{eqnarray}


\subsection{QCD+QED RG functions}

\code{FullU[a1,a2,ae,b0,g0,ge0,M1,J]}

\begin{itemize}
\item \code{a1,a2}: values of $\as$ at two different scales $\mu_1>\mu_2$
\item \code{ae}: value of $\aem$
\item \code{b0}: coefficient $b_0$ of the QCD $\beta$-function
\item \code{g0}: LO anomalous dimension matrix $\gs$
\item \code{ge0}: LO anomalous dimension matrix $\ge$
\item \code{M1}: $M_1$ matrix obtained from \code{M1[b0,b1,ge0,gse1,J]}
\item \code{J}: $J$ matrix obtained from \code{J[b0,b1,g0,g1]}
\end{itemize}

\begin{equation}
\hU(m_1,m_2,\aem) =
\hU(m_1,m_2) + \frac{\aem}{4\pi} \hR(m_1,m_2) \, ,
\label{eq:UdF1:1010}
\end{equation}



\section{Prescriptions}

Here we give a detailed explanation of the various 
subtleties which one encounters in the implementation
of the running of the Wilson Coefficients, which are 
not described nor mentioned in the original paper 
by Buras et al.

\subsection{LO prescription}

\begin{itemize}
\item $C_1 \,, C_2$ \\

\item $C_3 \ldots C_6$ \\

\item $C_7 \ldots C_{10}$ \\
\end{itemize}

Let us consider now the RG evolution matrices.

\begin{equation}
U(\mu,m)= \overbrace{U^{(0)}(\mu,m)}^\text{LO} + 
\overbrace{ {1 \over 4\pi} \Big[ \as(\mu) J U^{(0)}(\mu,m) - \as(m) U^{(0)}(\mu,m) J \Big] }^\text{NLO}
\end{equation}
 
\begin{equation}
\hR(m_1,m_2) \equiv -\frac{2\pi}{\beta_0} \; \hV \; \left(
\overbrace{\hK^{(0)}(m_1,m_2)}^\text{LO} +
\overbrace{\frac{1}{4\pi} \sum_{i=1}^{3} \hK_i^{(1)}(m_1,m_2)}^\text{NLO} \right) \hV^{-1}
\end{equation}

The quark threshold matching matrices at leading order correspond to the 
identity matrix only, while at NLO to the full formula given in the previous sections.
However note that the prescription for the $z_i$ coefficients at the charm threshold 
(recall the functions \code{Fs} and \code{Fse}) is at leading order.

\subsection{Apparent divergence in the $10 \times 10$ $R$ matrix}

The $10 \times 10$ case, with full QCD+QED, leads to an apparent
divergence in the matrix $K_1^{(1)}$ (off-diagonal components)

\begin{equation}
\left( \hK_1^{(1)}(m_1,m_2) \right)_{ij} =
\Bigg\lbrace
\begin{array}{ll}
\frac{M^{(1)}_{ij}}{a_i - a_j}
\left[ \left( \frac{\as(m_2)}{\as(m_1)} \right)^{a_j} -
       \left( \frac{\as(m_2)}{\as(m_1)} \right)^{a_i} \right] & i \not= j \\
M^{(1)}_{ii} \left( \frac{\as(m_2)}{\as(m_1)} \right)^{a_i}
             \ln\frac{\as(m_1)}{\as(m_2)}             & i=j
\end{array} 
\end{equation}

due to the presence of two couples of identical eigenvalues of $\gs$. 
However one can easily verify that the divergence cancels 
by substituting $a^i = a^j + \varepsilon$ in both numerator and denominator, 
and by expanding around $\varepsilon =0$
\begin{equation}
\frac{x^{a_j} - x^{a_i}}{a_i - a_j} = x^{a_j} \frac{1 - x^\varepsilon}{\varepsilon} 
 = x^{a_j} \frac{1 - (1 + \varepsilon \log x + O(\varepsilon^2))}{\varepsilon} = 
- x^{a_j} \log x
\end{equation}
In fact, the same type of calculation leads to the diagonal element of $K_1^{(1)}$,
which can be re-expressed as
\begin{equation}
\left( \hK_1^{(1)}(m_1,m_2) \right)_{ij} =
\Bigg\lbrace
\begin{array}{ll}
\frac{M^{(1)}_{ij}}{a_i - a_j}
\left[ \left( \frac{\as(m_2)}{\as(m_1)} \right)^{a_j} -
       \left( \frac{\as(m_2)}{\as(m_1)} \right)^{a_i} \right] & \mathbf{a_i \not= a_j} \\
M^{(1)}_{ii} \left( \frac{\as(m_2)}{\as(m_1)} \right)^{a_i}
             \ln\frac{\as(m_1)}{\as(m_2)}             & \mathbf{ a_i=a_j}
\end{array} 
\end{equation}


\subsection{Apparent divergence in the $N_f=3$ case}

The divergence appearing in the limit $N_f=3$ looks similar to the case above, since
it arises from the fact that two eigenvalues differ exactly by 1 and therefore
the combination $a_i -a_j -1$, present in various denominators, vanishes.

By performing the same type of expansions described above one can check that for pure QCD
such a divergence cancels in the combination
\[ \as(\mu) J U^{(0)}(\mu,m) - \as(m) U^{(0)}(\mu,m) J \]
while in the QED part it cancels automatically inside the matrix $K^{(0)}$
\begin{equation}
(\hK^{(0)}(m_1,m_2))_{ij} = \frac{\hM^{(0)}_{ij}}{a_i - a_j - 1}
\left[
\left( \frac{\as(m_2)}{\as(m_1)} \right)^{a_j} \frac{1}{\as(m_1)} -
\left( \frac{\as(m_2)}{\as(m_1)} \right)^{a_i} \frac{1}{\as(m_2)}
\right] 
\label{eq:K0}
\end{equation}
and in the difference
\begin{equation}
K_1^{(1)} - \as(m_2) \; \hK^{(0)} \; H + \,\as(m_1) \; H \; \hK^{(0)}
\end{equation}

In the code we implemented the substitution $N_f \to N_f + \varepsilon$ both in the 
$U_0$ and $J$ matrices and the limit $\varepsilon \to 0$ is taken by calling the functions 
\code{U} and \code{R}, or alternatively \code{FullU}. 
Only in the definition of $K^{(0)}$ we hard-coded the solution to the entire expansion



\end{document}
